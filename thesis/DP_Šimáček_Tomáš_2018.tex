% options:
% thesis=B bachelor's thesis
% thesis=M master's thesis
% czech thesis in Czech language
% slovak thesis in Slovak language
% english thesis in English language
% hidelinks remove colour boxes around hyperlinks

\documentclass[thesis=M,czech]{FITthesis}[2012/06/26]

\usepackage[utf8]{inputenc} % LaTeX source encoded as UTF-8

\usepackage{graphicx} %graphics files inclusion
% \usepackage{amsmath} %advanced maths
% \usepackage{amssymb} %additional math symbols

 \usepackage{dirtree} %directory tree visualisation

% % list of acronyms
% \usepackage[acronym,nonumberlist,toc,numberedsection=autolabel]{glossaries}
% \iflanguage{czech}{\renewcommand*{\acronymname}{Seznam pou{\v z}it{\' y}ch zkratek}}{}
% \makeglossaries

\newcommand{\tg}{\mathop{\mathrm{tg}}} %cesky tangens
\newcommand{\cotg}{\mathop{\mathrm{cotg}}} %cesky cotangens

% % % % % % % % % % % % % % % % % % % % % % % % % % % % % % 
% ODTUD DAL VSE ZMENTE
% % % % % % % % % % % % % % % % % % % % % % % % % % % % % % 

\department{Katedra počítačových systémů}
\title{Podpora automatické správy virtualizačního kontejneru Solaris Zones na platformě Solaris}
\authorGN{Tomáš} %(křestní) jméno (jména) autora
\authorFN{Šimáček} %příjmení autora
\authorWithDegrees{Bc. Tomáš Šimáček} %jméno autora včetně současných akademických titulů
\author{Tomáš Šimáček} %jméno autora bez akademických titulů
\supervisor{Ing. Michal Šoch, Ph.D.}
\acknowledgements{Doplňte, máte-li komu a za co děkovat. V~opačném případě úplně odstraňte tento příkaz.}
\abstractCS{% Abstract závěrečné práce v češtině
Tato diplomová práce se zabývá problematikou automatické správy virtualizačního kontejneru Solaris Zones na platformě Solaris.
Její součástí je podrobný popis tohoto virtualizačního kontejneru a také porovnání běžně využívaných virtualizačních technik.
Praktická část této práce je zaměřena na~návrh a implementaci nástroje, který podporuje automatickou správu Solaris Zones.
Nástroj klade důraz na možnost automatického vytváření zón pomocí šablon, které umožňují předem definovat jejich vlastnosti.
Součástí implementace jsou také automatizované procesy zálohy, obnovy nebo migrace zón, které je možné provádět na lokálních
i vzdálených serverech.


}
\abstractEN{% Abstract of the master thesis in english
Abstract in english
}
\placeForDeclarationOfAuthenticity{V~Praze}
\declarationOfAuthenticityOption{4} %volba Prohlášení (číslo 1-6)
\keywordsCS{% Klíčová slova v češtině
Solaris, Solaris Zones, virtualizace, automatická správa, šablony zón, vzdálená správa, administrace
}
\keywordsEN{% Keywords in english
Solaris, Solaris Zones, virtualization, automatic management
}
% \website{http://site.example/thesis} %volitelná URL práce, objeví se v tiráži - úplně odstraňte, nemáte-li URL práce

\begin{document}

% \newacronym{CVUT}{{\v C}VUT}{{\v C}esk{\' e} vysok{\' e} u{\v c}en{\' i} technick{\' e} v Praze}
% \newacronym{FIT}{FIT}{Fakulta informa{\v c}n{\' i}ch technologi{\' i}}

\begin{introduction}
  % Chapter: Introduction
% Author: Tomáš Šimáček
\label{chapter:introduction}
Virtualizace je technika, se kterou se dnes v~IT můžeme setkat v~mnoha podobách. Jednou z~hlavních oblastí jejího využití,
je virtualizace serverů, ale objevuje se také v oblastech komunikačních sítí nebo desktopů. Tato technologie umožňuje vytvářet
virtuální prostředky a poskytovat tak kompletní virtuální prostředí. Toto prostředí umožňuje provozovat systémy na~jiných
fyzických architekturách, než pro jaké jsou určeny.

Hlavním tématem této práce je virtualizace serverů, která umožňuje rozdělit jeden fyzický systém do několika nezávislých virtuálních
prostředí, ve~který jsou spouštěny virtuální počítače. Možnost vytváření virtuálních počítačů v~rámci jednoho fyzického systému
značně snižuje náklady na pořizování a provoz fyzických serverů. Díky virtualizaci již není třeba pořizovat dedikovaný server pro
každou instanci operačního systému, který chce společnost provozovat. Správným rozdělením virtuálních počítačů na fyzické servery může být
docíleno ideální rozdělení zátěže a tím mohou být dostupné fyzické prostředky efektivně využity.

Rostoucí počet virtualizovaných serverů může mít za následek obtížnější správu. Automatizované nasazení, instalace nebo
zálohování virtuálních počítačů může být značným ulehčením správy počítačové infrastruktury, která využívá virtualizačních technik.
Díky tomuto ulehčení lze jednoduše vytvářet předem definovaná virtuální prostředí, která mohou složit pro~vývoj software, testování
nebo nasazení aplikací do produkčního prostředí. 

V~dnešní době existuje mnoho operačních systémů, které nějakým způsobem poskytují virtualizaci v~rámci svých služeb. Jedním ze zástupců
takovýchto systémů je operační systém Solaris. Exkluzivně pro tento operační systém byla vytvořena virtualizační technika Solaris
Zones, která umožňuje v~rámci jedné instance operačního systému Solaris vytvářet virtuální počítače nazývané zóny. Nástroje pro správu
této virtualizační technologie umožňují spravovat zóny pouze v~rámci lokálního serveru. Tato diplomová práce se věnuje právě možnosti
správy zón nacházejících se na vzdálených serverech a automatizaci administrátorských procesů vytváření, zálohy, obnovy a migrace 
těchto zón.
\section{Cíle práce}
\label{chapter:introduction:goals}
Cílem této diplomové práce je seznámení s~operačním systémem Solaris a~jeho aktuální stabilní verzí 11.3. Součástí popisu tohoto
operačního systému je i představení podporovaných architektur a jeho základních služeb pro správu softwarových balíčku nebo souborového
systému ZFS. Především jde však o~popis základních principů virtualizační techniky Solaris Zones, která umožňuje běh více zón v~rámci 
jedné instance operačního systému Solaris. Nebude chybět ani porovnání běžně používaných virtualizačních technik.

Diplomové práce také souvisí se~správou virtualizační techniky Solaris Zones a má za úkol detailně popsat možnosti konfigurace a instalace
zón. Součástí tohoto popisu bude i popis administrátorských procesů pro zálohování, obnovu a migraci. Popis se také věnuje integraci
techniky Solaris Zones s~ostatními službami operačního systému Solaris.

Hlavním cílem této diplomové práce je návrh a implementace nástroje pro podporu automatické správy virtualizačního kontejneru
Solaris Zones na platformě Solaris. Jelikož základní nástroje pro správu této virtualizační technologie neumožňují správu vzdálených
zón, implementovaný nástroj bude tuto funkcionalitu podporovat. Dále tento nástroj bude umožňovat provádění základních administrátorských
procesů pro větší množství lokálních i~vzdálených zón. Mezi těmito procesy bude zahrnuta automatická konfigurace, instalace, náhrada,
záloha, obnova a migrace zón v~rámci několika virtualizačních serverů. Nástroj bude umožňovat definici softwarových balíků, které mají být
při instalaci do zóny zahrnuty a to pomocí šablon nebo interaktivně pomocí uživatelského rozhraní.

Posledním cílem této diplomové práce je otestovat implementovaný nástroj. Testovány budou hlavní scénáře využití výsledného nástroje
a bude změřena doba běhu pro určité funkce nástroje.
\section{Struktura práce}
\label{chapter:introduction:structure}
Struktura této diplomové práce se skládá ze šesti hlavním kapitol a příloh. První kapitola práce se zabývá obecným popisem virtualizace
a jejím využitím v~informačních technologiích. Dále tato kapitola definuje pojem virtuálního stroje a představuje jednotlivé druhy
virtualizačních technik. V~závěru první kapitoly je zmíněno několik hlavních scénářů pro nasazení virtuální infrastruktury. Druhá kapitola
stručně představuje operační systém Solaris. Důraz je kladen především na podporované platformy a služby, které
tento operační systém poskytuje. Třetí kapitola obsahuje podrobný popis virtualizační techniky Solaris Zones. Úvod této kapitoly
popisuje základní principy a typy zón, které je možné v~rámci této technologie vytvářet. Ve zbytku kapitoly jsou představeny
konkrétní způsoby správy této virtualizační techniky, které se zaměřují na popis konfigurace, instalace, zálohy a~migrace zón. 
Čtvrtá kapitola se zabývá návrhem nástroje pro podporu automatické správy virtualizačního kontejneru Solaris Zones. Hlavním obsahem
této kapitoly je stanovení požadavků na výsledný nástroj a navržení jeho architektury. V~závěru této kapitoly jsou popsány některé
bezpečnostní aspekty, které by měl uživatel nástroje splňovat. Pátá kapitola popisuje způsob implementace nástroje pro podporu automatické
správy virtualizačního kontejneru Solaris Zones. V~úvodu této kapitoly je popsána volba programovacího jazyka, pomocí kterého byl výsledný
nástroj implementován. Zbytek kapitoly popisuje implementaci šablon, uživatelského rozhraní a ostatních funkčních komponent výsledného
nástroje. Poslední kapitola je věnována testování implementovaného nástroje. Hlavním obsahem kapitoly je popis testování hlavních
scénářů využití nástroje pro správu Solaris Zones. Praktické ukázky testování jsou obsaženy v~příloze. Závěr této kapitoly se zabývá měřením
doby běhu některých funkcí výsledného nástroje a rozebírá výsledky měření. 

Zdrojové kódy celé diplomové práce a implementovaného nástroje pro podporu automatické správy virtualizačního kontejneru Solaris Zones
jsou dostupné na~přiloženém médiu.





\end{introduction}

\chapter{Virtualizace}
  % Virtualization chapter about concept, types, examples and implementation
V dnešní době existuje mnoho různých typů virtualizace, jak již bylo zmíněno v úvodu. Následující kapitola zběžně představuje několik nejznámější typů virtualizace v IT a dále se věnuje hlavně
tématu virtualizace serverů. V kapitole jsou popsány některé scénáře pro přechod z klasické infrastruktury na virtuální. Ve zbytku kapitoly jsou popsány obecné principy virtualizace a základy 
virtualizace CPU, paměti a I/O zařízení.

\section{Využití virtualizace v sítích}
\section{Virtualizace desktopu}
\section{Virtualizace serverů}

\section{Definice pojmů}
Pro potřeby popisu virtualizačních technik si definujeme základní pojmy a entity, které se ve virtuální infrastruktuře vyskytují.

\textit{Fyzické prostředky} TODO

\textit{Virtualizační monitor}, \textit{Virtual Machine Monitor - VMM} nebo také \textit{Hypervisor} je softwarová vrstva, která virtualizuje HW prostředky fyzického počítače a přiděluje je
virtuálním strojům.

\textit{Host} je HW a SW platforma, která poskytuje virtuálním strojům výpočetní výkon, paměť, úložiště, síťové připojení a další fyzické prostředky. Softwarové vybavení hosta obsahuje virtualizační
monitor a v některých případech může obsahovat i operační systém hosta tzv. \textit{Host OS}

\textit{Virtuální stroj} nebo \textit{Virtual Machine - VM} je virtualizované prostředí vytvořené virtualizačním monitorem, ve kterém běží operační systém virtuálního počítače nazývaný \textit{Guest OS}.

\section{Nasazení virtuální infrastruktury}
Pro přechod k virtuální infrastruktuře serverů existuje v dnešní době několik dobrých důvodů. Jedním z hlavních benefitů virtualizace pro dnešní firmy a organizace je značná finanční úspora. Tato úspora se
projevuje především ve snížení nákladů organizace na pořizování a provoz fyzických zařízení. 

Mezi další benefity virtualizace patří především efektivní využití výpočetních zdrojů, vysoká dostupnost běžících aplikací nebo vytvoření oddělených a nezávislých prostředí pro vývoj, testování a nasazení software.

Výhody zavedení virtuální infrastruktury jsou podrobněji popsány v následujících podkapitolách, které se zabývají základními scénáři pro nasazení virtuální infrastruktury.

\subsection{Konsolidace}
\label{consolidation}

Konsolidace serverů je proces sjednocování více fyzických serverů na jeden fyzický server, který pro tyto servery poskytne virtuální prostředí pro jejich běh. Vstupem tohoto procesu je tedy několik fyzických serverů,
na kterých běží různé aplikace. Vstup procesu je naznačen na obrázku \ref{consolidation_img} vlevo. Výstupem konsolidace je jeden fyzický server s dostatečnými prostředky, na kterém konsolidované servery běží jako virtuální počítače.
Výstup můžeme vidět na obrázku \ref{consolidation_img} vpravo.

\begin{figure}
    \centering    
    \caption{Konsolidace serverů}
    \label{consolidation_img}
\end{figure}

\subsubsection*{Využití scénáře}

Dnes je zcela běžnou praktikou provozovat jednu aplikaci na jednom dedikovaném serveru. Pokud aplikace využívá jen malé procento výpočetních zdrojů daného serveru, může administrátor sjednotit více takovýchto serverů
do jednoho. Pro organizaci, která vlastní tisíce takovýchto serverů může konsolidace výrazně zmenšit požadavky na prostor, spotřebu energie a provoz fyzických serverů. Správnou konsolidací serverů může společnost docílit
efektivního využití dostupných prostředků a tím výrazně snížit vynaložené finanční prostředky \cite{reasons}.

Rychlý vývoj technologií v oblasti hadware zapříčiňuje rychlé stárnutí některých systémů a přechod ze staršího na nový může být složitý. Obzvláště v případě, kdy systém potřebuje ke svému běhu speciální hadware.
Aby bylo možné provozovat služby poskytované těmito zastaralými systémy, můžeme je spustit jako virtuální počítač na modernějším hadware. Systém se bude chovat stejně jako kdyby běžel na zastaralém hadware, zatímco
výkonost služby může těžit z novější a výkonnější hadwarové vrstvy  \cite{reasons}.

\subsection{Izolace}

Dalším ze scénářů využití virtualizované infrastruktury je izolace aplikací. Proces izolace aplikací spočívá v oddělení dvou a více kritických aplikací běžících na jednom systému do nezávislých virtuálních prostředí.
Vstupem je jeden systém s aplikacemi, které se mohou negativně ovlivňovat. Vstup izolačního scénáře je naznačen na obrázku \ref{izolation} vlevo. Výstupem je několik nezávislých virtuálních počítačů, ve kterých běží
jednotlivé aplikace. Výstup izolace aplikací je ukázán na obrázku \ref{izolation} vpravo.

\begin{figure}
    \centering    
    \caption{Izolace aplikací}
    \label{izolation}
\end{figure}

\subsubsection*{Využití scénáře}

V dnešní době jsou útoky na aplikace vystavené do internetu běžnou záležitostí. Pokud útočník využije nějaké zranitelnosti aplikace, může v některých případech získat kontrolu nad celým systémem. V takovém případě jsou
ohroženy všechny data a aplikace, které na daném systému běží. Vhodným krokem v tomto případě je proto využití virtualizace a rozdělení aplikací do nezávislých prostředí.

Jedním z příkladů ohrožení systému může být útok na výpočetní zdroje. Podstatou útoku je vyčerpání fyzických zdrojů systému, což má za následek nedostupnost jeho služeb a v některých případech i pád celého systému.
Ve virtualizovaném prostředí lze přidělit každému VM pouze určitou část prostředků a tím chránit celý systém před jejich vyčerpáním. V případě napadení jednoho VM sice dojde k jeho vyřazení, ale ostatní VM a jejich služby
mohou dále pokračovat v běhu.

\subsection{Migrace}

Posledním diskutovaným scénářem nasazení virtualizované infrastruktury je migrace. Jedná se o proces přesunutí systému z jednoho počítače na druhý. V rámci virtualizace se budeme bavit o přesouvání systému na počítač s běžícím VMM,
který zprostředkovává virtuální prostředí. Výstupem procesu je systém, který do něj zároveň vstupuje. Rozdíl je v tom, že daný systém na konci procesu běží ve virtuálním prostředí nějakého VMM. Dle typu migrovaného systému můžeme
rozdělit scénář na následující typy.

\subsubsection*{Migrace VM}

Migrací virtuálního stroje se rozumí přesun VM mezi dvěma různými fyzickými stroji s VMM. Tento přesun byl dříve možný pouze v případě když oba stroje měli stejný HW, operační systém a procesor \cite{reasons}.
Tato možnost administrátorovi umožňuje přesouvat virtualizované systémy na výkonnější hosty a tím umožňuje dynamicky regulovat využití fyzických prostředků v závislosti na aktuální zátěži systému.

Další výhodou zavedení virtualizované architektury je zajištění vysoké dostupnosti služeb. Virtualizace umožňuje zajistit redundanci ve smyslu spuštění služby na více serverech najednou. Ve virtualizované architektuře můžou
nastat dva typy selhání. Prvním typem je selhání VM uvnitř VMM. Pokud dojde k selhání některé VM, jiná VM převezme obsluhu požadavků a v minimálním čase dojde k obnovení služby. Druhým typem je selhání celého VMM nebo hosta.
V tomto případě je nutné provozovat více redundantních hostů pro VM, které v případě HW chyby převezmou obsluhu služby.

Proces migrace VM je představen na obrázku \ref{migration1}, kde můžeme vidět konfiguraci před migrací (vlevo) a po provedení migrace VM (vpravo).

\begin{figure}
    \centering    
    \caption{Migrace virtuálního stroje}
    \label{migration1}
\end{figure}

\subsubsection*{Migrace fyzického stoje na VM}

Migrace fyzického stroje na virtuální stroj je proces, kdy dochází k přesunu a virtualizaci systému z fyzického stroje. Vstupem je tedy systém běžící na stroji bez VMM, jak je naznačeno na obrázku \ref{migration2} vlevo. 
Výstupem tohoto procesu je opět virtuální stroj běžící ve virtuálním prostředí VMM.

Virtualizací serverů dochází k uvolňování fyzického HW a stejně jako v případě konsolidace tak společnost může značně ušetřit na nákladech nutných k provozu a správě fyzických serverů. Obecně lze říct, že tento scénář přináší 
podobné benefity jako v případě konsolidace popsané v kapitole \ref{consolidation}.

\begin{figure}
    \centering    
    \caption{Migrace fyzického na virtuální stroj}
    \label{migration2}
\end{figure}








\section{Virtualizační monitor}
\section{Techniky virtualizace}
\section{Virtualizace CPU}
\section{Virtualizace Paměti}
\section{Virtualizace I/O zařízení}



  
\chapter{Solaris}
  % Brief description of Solaris operating system
\label{chapter:solaris}
Solaris nebo dříve SunOS je operační systém původně vytvořený firmou Sun Microsystems. V~současné době je vyvíjený a podporovaný
firmou Oracle. Je to komplexní unixový operační systém, který v~sobě integruje pokročilé technologie pro virtualizaci, moderní
souborový systém ZFS, vlastní systém pro~instalaci a správu SW a v~neposlední řadě také podporu cloudu. Díky integraci těchto
technologií poskytuje Solaris stabilní a rychlé prostředí pro různé scénáře nasazení aplikací a navíc tato integrace vytváří
pohodlné rozhraní pro~správu tohoto OS. 
\section{Verze Solarisu}
\label{chapter:solaris:version}
Nejaktuálnější stabilní verze operačního systému Solaris je verze s~označením 11.3. Ke dni 30. ledna 2018 byla uvolněna beta
verze 11.4 \cite{oracle:solaris:beta}. Pro účely popisu operačního systému Solaris a jeho služeb, zejména služby Solaris Zones,
bude použita stabilní verze 11.3. Existují i starší verze 11.2 a 11.1, které ale nebudou předmětem zkoumání.
\section{Podporované architektury}
\label{chapter:solaris:support}
Operační systém Solaris v~současné době podporuje následující dvě HW architektury počítačových systémů:
\begin{itemize}
 \item x86,
 \item SPARC.
\end{itemize}
Jelikož architektura SPARC není běžně dostupná, bude pro účely této diplomové práce použita architektura x86, přesněji
její 64 bitová verze.
\subsection{Architektura x86}
\label{chapter:solaris:support:x86}
První počítačovou architekturou podporovanou operačním systémem Solaris je x86. Tato architektura je v~dnešní době velmi rozšířená
především v~oblasti osobních počítačů a je podporována nejznámějšími OS jako Windows, Linux a Mac. Solaris tuto architekturu
podporuje jak v~32 bitové verzi \textit{x86-32} tak i~v~64 bitové verzi \textit{x86-64}.
\subsection{Architektura SPARC}
\label{chapter:solaris:support:sparc}
Scalable Processor Architecture neboli SPARC je z~pohledu operačního systému Solaris domovská architektura. Architektura SPARC
byla stejně jako Solaris původně navržena společností Sun Microsystems a nyní ji spravuje společnost Oracle. Tato architektura
je tedy od začátku své existence úzce spojena s~operačním systémem Solaris, který se snaží využívat všechny její výhody.
Uplatnění této architektury je především v~komerčním sektoru, který klade vysoké nároky na přizpůsobivost a možnosti škálování
potřebného řešení.
\section{Služby}
\label{chapter:solaris:services}
Hlavní předností operačního systému je kvantita a kvalita jeho služeb. Tyto služby umožňují nasazení tohoto OS i ve scénářích,
kdy by ostatní OS selhaly nebo by nemohly být vůbec použity.  
\subsection{Service Management Facility}
\label{chapter:solaris:smf}
Service Management Facility neboli SMF je systém, který v~operačním systému Solaris spravuje systémové služby. Nahrazuje tím
tradiční způsob spravování služeb pomocí tzv. \textit{init} skriptů, který byl běžný v~ostatních unixových operačních systémech
a dokonce i v~dřívějších verzích OS Solaris. Hlavním rozdílem oproti staršímu způsobu je možnost u~služby definovat závislosti
na~ostatních službách. Na~rozdíl od sériového spouštění \textit{init} skriptů z~adresáře je díky tomuto zlepšení možné při startu
systému paralelně spouštět více nezávislých služeb najednou, a tím urychlit start systému \cite{cvut:biadu:sysstart}. Pro účel
startu jsou v~systému definovány speciální služby tzv. \textit{milestone}. Tyto služby mají definovaný
pouze seznam závislostí, který určuje jaké služby se mají spustit. Při startu se určí, do kterého \textit{milestone} má systém
nastartovat a~tím je přesně určeno, které služby se mají spustit.
\subsection{Souborový systém ZettaByte}
\label{chapter:solaris:zfs}
Pro ukládání na disk používá Solaris souborový systém ZettaByte neboli ZFS. Je to pokročilý systém, který byl vyvinut společností
Sun Microsystems a integrován do operačního systému Solaris. ZFS dokáže spravovat velké množství dat díky své 128-bitové 
architektuře \cite{cvut:thesis:mythesis}. Mezi hlavní funkce ZFS patří ověřování integrity dat, vlastní softwarový RAID nebo
šifrování dat. Díky principu \textit{copy on write} dokáže udržet data neustále konzistentní, což některé souborové systémy
nedokážou nebo tento problém řeší složitě. Architektura tohoto souborového systému umožňuje klonování jednotlivých svazků nebo
rychlou a elegantní tvorbu obrazů disku tzv. \textit{snapshot}, které z~počátku zabírají minimální místo na disku. Datové bloky
jsou totiž zduplikovány až v~okamžiku, kdy se zdrojový blok nebo jeho klon změní. Tento způsob uchovávání dat společně s~možností
\textit{deduplikace} stejných datových bloků značně snižuje nároky na diskové místo.

Principů a funkcí ZFS hojně využívají další služby operačního systému Solaris. Příkladem může být uvedena virtualizační technika
Solaris Zones, která je hlavním tématem této diplomové práce.
\subsection{Virtualizace}
\label{chapter:solaris:virtualization}
Dle specifikace \cite{oracle:solaris:virtualization} nabízí operační systém Solaris ve verzi 11.3 následující techniky virtualizace:
\begin{itemize}
 \item Virtualizace na úrovni OS,
 \item Virtuální počítače,
 \item Hardware partitions.
\end{itemize}
Tyto modely se liší zejména ve způsobu izolace virtualizovaných prostředí a~ve~flexibilitě přidělování prostředků těmto
prostředím. Čím více model izoluje prostředí od sebe, tím nabízí menší flexibilitu v~přidělování prostředků.
\subsubsection{Solaris Zones}
\label{chapter:solaris:virtualization:szones}
Jedním z~modelů virtualizace nabízené operačním systémem Solaris je \textit{virtualizace na úrovni OS}. Tento model umožňuje
vytvořit jedno nebo více izolovaných prostředí (zón) pro běh programů v~rámci jedné instance OS. Takto vytvořená prostředí
jsou izolována na úrovni procesů, souborového systému a síťových rozhraní. Každá zóna má vlastní lokální pohled na systémové
prostředky, které mohou být dále virtualizované operačním systémem. Virtualizace na úrovni operačního systému poskytuje vysoký
výkon a flexibilitu, protože nezanechává tak velkou stopu na disku, paměti nebo CPU na rozdíl od ostatních dvou modelů virtualizace. 

Operační systém Solaris poskytuje tento model virtualizace skrze službu Solaris Zones, která je přímo integrována to jádra OS.
\subsubsection{Virtuální počítače}
\label{chapter:solaris:virtualization:vm}
Model \textit{virtuálních počítačů} popsaný v~kapitole \ref{chapter:virtualization:clasification:system_vm:virtual_computer} 
umožňuje souběžný běh více instancí operačního systému na jednom fyzickém stroji. Každý virtuální počítač má svojí instanci
operačního systému, který nemusí být stejný ve všech virtuálních strojích. Tento typ virtualizace je umožněn díky virtualizačnímu
monitoru, který vytváří pro operační systémy iluzi, která izoluje jednotlivé virtuální počítače. Virtuální počítače poskytují na rozdíl od
virtualizace na úrovni OS menší flexibilitu rozdělování prostředků, ale naopak poskytuje větší úroveň izolace.

Tento typ virtualizace je v~OS Solaris 11.3 podporován produkty Oracle VM Server for x86, Oracle VM Server for SPARC a Oracle
VM VirtualBox \cite{oracle:virtualization:technologies}. Každá z~těchto implementací se zaměřuje na jinou architekturu nebo
používá jiný typ virtualizačního monitoru.
\subsubsection{Hardware partitions}
\label{chapter:solaris:virtualization:hw_partition}
Posledním modelem, který je nepřímo podporovaný operačním systémem Solaris, jsou tzv. \textit{hardware partitions}. Je to technika,
která fyzicky odděluje běh OS na oddělených částech fyzických prostředků. Tohoto způsobu virtualizace je dosaženo
bez pomocí virtualizačního monitoru, a proto tato technika poskytuje reálný výkon systému. \textit{Hardware partitions} je technika
poskytující běžícím operačním systémům největší izolaci, ale není tak flexibilní v~konfiguraci prostředků jako výše
zmíněné modely.

Tento model virtualizace není z~logických důvodů poskytován operačním systémem Solaris, jelikož se jedná o~virtualizaci na HW úrovni.
Pro~účely nasazení tohoto OS s~touto virtualizační technikou používá Oracle speciální servery SPARC M-Series 
\cite{oracle:virtualization:technologies}.

\chapter{Solaris Zones}
  % Detailed description of Solaris Zones configuration, installation, backup etc.
V úvodu následující kapitoly jsou popsány základní principy a struktura virtualizační techniky Solaris Zones od
firmy Oracle. Dále se tato kapitola zabývá popisem datových struktur, postupů a nástrojů, které slouží ke správě a manipulaci
se zónami. Detailnější popis je věnován především administrátorským rutinám pro konfiguraci, instalaci a zálohování zón.


\section{Virtualizační technika}
Oracle Solaris Zones je virtualizační technika, která umožňuje běh více virtuálních strojů na jednom fyzickém stroji. Jak již
bylo zmíněno v kapitole \ref{chapter:solaris}, tato technika je standardní součástí operačního systému Oracle Solaris od 
verze systému Solaris 10. Aktuální verze je Solaris 11.3 a přináší novou funkcionalitu v podobě podpory starších verzí 
operačního systému Solaris.

Pokud bychom chtěli zařadit Solaris Zones do klasifikace virtuálních strojů popsané v kapitole \ref{section:clasification}, 
pak bychom jí zařadili do sekce \textit{virtualizace na úrovni OS}. Jinak řečeno, tato technika rozděluje zdroje hostitelského
operačního systému jako CPU, paměť nebo I/O zařízení mezi běžící virtuální stroje a zajišťuje izolaci na úrovni procesů,
souborového systému a sítě. Zónu pak můžeme definovat jako virtuální kontejner běžící v hostitelském operačním systému, který
využívá zdrojů hostitelského operačního systému a je izolovaný od ostatních zón.

Standardně se po instalaci operačního systému Solaris nachází v systému jedna zóna. Je to vlastní instance operačního systému
a nazývá se \textbf{globální} zóna. Jinými slovy, je to zóna, která běží přímo na hardwaru počítače nebo ve virtualizovaném
prostředí. Tato zóna má dvě hlavní funkce. Za prvé tato zóna plní funkci hlavního operačního systému a přebírá kontrolu
nad fyzickými prostředky po startu systému. Za druhé je hlavním centrálním prvkem pro administraci celého systému a ostatních
zón. Globální zóna poskytuje globální pohled na celý systém a má přehled o všech systémových zdrojích a aktivitách ostatních
zón. Její role v rámci Solaris Zones je zásadní a její chyba může zapříčinit pád ostatních zón. Z tohoto důvodu je doporučené
používat globální zónu pouze pro účely administrace systému a managementu ostatních zón.

Zóny, které jsou spouštěny v rámci globální zóny nazýváme \textbf{neglobální} zóny. Tyto zóny jsou navzájem izolované na
několika úrovních. První úroveň izolace je izolace na úrovni sítě. Každá neglobální zóna může mít svůj vlastní logický síťový
adaptér, který je vytvořený nad nějakým fyzickým síťovým rozhraním a je dostupný pouze pro konkrétní zónu. Z jiné neglobální
zóny tento adaptér není přístupný. Takto vytvořený síťový adaptér může být spravován pouze z globální zóny a ze zóny, ke které
byl přiřazen v průběhu jejího vytváření.

Druhou úrovní izolace zón je souborový systém. Globální zóna spravuje svůj vlastní souborový systém ve kterém se nachází
standardní adresářová struktura operačních systému typu UNIX. Můžeme v něm nalézt adresář \textit{/etc} sloužící pro globální
systémovou konfiguraci, adresář \textit{/bin} obsahující uživatelsky spustitelné programy nebo například adresář \textit{/sbin},
který uchovává programy spustitelné pod privilegovaným uživatelem. Každá neglobální zóna potřebuje pro svůj běh velmi podobné
prostředí, a proto má svůj vlastní souborový systém, který se nachází někde v hierarchii souborového systému globální zóny.
Podle typu zón popsaných v kapitole \ref{chapter:zones:types} může být tento souborový systém částečně sdílený se souborovým
systémem globální zóny a nebo může obsahovat kompletně nezávislý obraz zóny. Při spouštění zóny pak dojde pomocí příkazu
\verb|chroot(1)| k přepnutí kořenu souborového systému a zóna pracuje pouze se svojí částí souborového systému. V případě 
sdílené části souborového systému jsou tyto části připojeny v režimu read-only \cite{virt1}. Tím je zajištěno, že souborové
systémy jednotlivých zón jsou vzájemně izolované a nemohou se vzájemně ovlivnit. Správu těchto souborových systémů je opět
možné provádět pouze z konkrétní zóny a nebo ze zóny globální.



Druhou funkcí globální zóny je centrální administrace celého systému a ostatních zón. Zónám, které jsou spouštěny v rámci
globální zóny se říká \textbf{neglobální} zóny a nemají žádný přístup do globální ani jiné neglobální zóny. Konfigurace, 
instalace a další správa neglobálních zón jako přihlašování do konzole, se dá provádět pouze ze zóny globální. 

ve kterém můžeme spouštět procesy izolovaně od ostatních virtuálních kontejnerů
.

Tento kontejner tedy stanovuje
softwarové hranice a umožňuje procesům v něm spuštěných přímo komunikovat pouze s procesy ve stejném kontejneru (zóně). Pokud
chce proces komunikovat s procesem z jiné zóny, musí k tomu využít počítačovou síť \cite{oracle:solaris:zones:introduction}.
Z pohledu softwarového vývojáře i uživatele se tedy jedná o dva oddělené systémy, i když reálně běží na jednom fyzickém
systému.







Neglobální zóny se chovají jako nezávislé operační systémy a izolují procesy spuštěné v nich. Každá neglobální zóna má svůj
vlastní souborový systém, který má svůj kořenový adresář umístěný v souborovém systému globální zóny. Díky úzké spolupráci 
zón se souborovým systémem ZFS je správa těchto souborových systému jednoduchá a umožňuje využívat přednosti tohoto
souborového systému i v rámci Solaris Zones.

zóny mají svůj vlastní souborový systém, vlastní síťové rozhraní, vlastní plánovač procesů a vlastní softwarové
balíčky. Všechny
tyto zdroje si neglobální zóny samy spravují. Nicméně všechny tyto prostředky jsou vytvořeny v globální zóně a delegovány 
jednotlivým neglobálním zónám.


\subsection{Typy zón}
\label{chapter:zones:types}



\subsection{Virtuální síť}

\section{Nástroje pro správu}

\section{Konfigurace}

\section{Instalace}

\section{Zálohování}

\begin{conclusion}
  % Chapter: Conclusion
% Author: Tomáš Šimáček
\label{chapte:conclusion}
Virtualizace se stala běžnou a možná i nezbytnou součástí dnešního počítačového světa. Teto technika se využívá v~mnoha oblastech
informačních technologií, kde přináší různé benefity. Virtualizace umožňuje společnostem efektivně využívat dostupné 
fyzické prostředky a tím výrazně ušetřit náklady na provoz fyzických zařízení. Tématem této diplomové práce byla virtualizace serverů,
která umožňuje současný běh několika virtuálních počítačů v~rámci jednoho fyzického systému.

Jednou z~technik virtualizace serverů je technologie Solaris Zones, která je součástí operačního systému Solaris vyvíjeného firmou Oracle.
Tato virtualizační technika umožňuje běh mnoha virtualizačních kontejnerů v~rámci jedné instance operačního systému Solaris. Tyto
kontejnery se nazývají zóny a jsou izolovány na úrovni počítačové sítě, souborového systému a spuštěných procesů. Standardně hlavní
operační systém sdílí svoje jádro s~ostatními zónami, ale tato technika umožňuje vytvářet i zóny s~vlastním jádrem.

Cílem teoretické části této diplomové práce bylo tuto virtualizační techniku popsat a porovnat s~ostatními běžně využívanými technikami
virtualizace. Dále se práce zaměřila na popis konfigurace, instalace a základních administrátorských rutin pro správu zón.
Předmětem byly také pokročilé rutiny vyžadují provedení několika kroků, jejichž postupným vykonáváním lze dosáhnout požadovaného výsledku. Z~tohoto
důvodu bylo hlavním cílem práce vytvořit nástroj, který by tyto procesy automatizoval a umožnil je provádět i pro vzdálené zóny.
Hlavním důvodem pro vytvoření tohoto nástroje je fakt, že standardní nástroje pro správu Solaris Zones tuto funkcionalitu neposkytují.
Jelikož Solaris Zones poskytují uživateli pouze rozhraní na příkazové řádce, bylo nutné implementovaný nástroj postavit právě nad tímto
rozhraním a využívat tak standardní nástroje pro správu zón.

Výsledná implementace nástroje se skládá z několika částí. Jednou z nich je modul Solaris Zones, který poskytuje funkcionalitu ostatním
částem nástroje. Jeho architektura se skládá z~několika hierarchicky uspořádaných vrstev. Nejníže v~hierarchii se nachází vrstva,
která umožňuje lokální i vzdálené spouštění nástrojů \textit{zonecfg}, \textit{zoneadm}, \textit{archiveadm} a v~neposlední řadě 
také \textit{zfs}. Ostatní vrstvy modulu využívají těchto příkazů a vytvářejí z~nich sekvence, které reprezentují jednotlivé 
administrátorské rutiny pro vytváření, mazání, zálohu, obnovu a migraci zón. Jednotlivé rutiny jsou implementované jako transakce. V~rámci
transakce je zajištěno, že všechny dočasně vytvořené soubory budou smazány a že všechny provedené změny budou navráceny
do původního stavu v~případě neúspěchu transakce. Takto implementované rutiny jsou poskytovány ostatním částem nástroje pomocí
předem definovaného rozhraní. 

Jelikož v~dnešní době existuje mnoho virtualizačních technik s~podobnými vlastnostmi jako Solaris Zones, byla architektura nástroje
navržena tak, aby se dala v~budoucnu lehce rozšířit o~další moduly. Tuto funkcionalitu zajišťuje v~rámci nástroje knihovna, která
jasně definuje rozhraní jednotlivých modulů. Pomocí tohoto rozhraní knihovna zprostředkovává funkcionalitu modulů klientským 
aplikacím. Ve výsledném nástroji je implementován pouze výše zmíněný modul pro podporu správy Solaris Zones. Tato knihovna také
definuje generickou šablonu, která slouží pro popis konkrétních virtuálních strojů. Výše popsaný modul implementuje rozšíření této
generické šablony, které umožňuje uživateli specifikovat konfiguraci, softwarové vybavení a systémové nastavení zón. Tyto šablony
je možné využít pro vytváření většího množství zón s~danými vlastnostmi.

Ovládání nástroje je zajištěno pomocí uživatelského rozhraní, které je uživateli prezentováno pomocí příkazové řádky. Jednotlivé
příkazy umožňují vyvolat odpovídající funkce modulu pro vytváření, mazání, zálohu, obnovu nebo migraci zón. Uživatel může jednoduše 
specifikovat pro jaké zóny chce danou akci provést. Tyto zóny se mohou nacházet na lokálním i vzdáleném serveru a nástroj
pro každou specifikovanou zónu vykoná konkrétní akci pokud možno paralelně. Uživatelské rozhraní dále implementuje 
systém pro správu vzdálených hostů, který umožňuje dané hosty registrovat a specifikovat parametry připojení. Hromadné akce
poskytované uživatelským rozhraním jsou prováděny právě pro všechny registrované hosty. Jelikož se zónami může manipulovat každý
privilegovaný uživatel, implementuje tato část nástroje také uživatelský žurnál. V~žurnálu jsou udržovány jednotlivé stavy zón, aby
bylo možné poznat, zda v~době nepřítomnosti uživatele nedošlo ke změně stavu zón. Další součástí uživatelského rozhraní je
grafický editor šablon, který uživateli poskytuje možnost interaktivní tvorby a editace šablon. Hlavním smyslem tohoto editoru
je odstínění uživatele od implementačních detailů šablon. Tohoto grafického rozhraní je využito i v~případě interaktivní instalace
zón, která uživateli nabízí možnost specifikace vlastností v~průběhu vytváření zón. 

Implementovaný nástroj umožňuje uživateli jednoduchou správu většího množství zón pomocí automatizovaných rutin pro vytváření,
mazání, zálohu, obnovu a migraci těchto zón. Součástí nástroje jsou šablony, které umožňují přesně specifikovat vlastnosti 
vytvářených zón. Tato funkcionalita se dá dobře využít v~procesu vývoje software pro definici produkčních, testovacích a 
vývojářských prostředí. Implementované uživatelské rozhraní prezentuje uživateli funkce nástroje v~jednoduché a přehledné formě.
Grafické součásti nástroje slouží především pro komfortní vytváření a editaci šablon.

Implementovaný nástroj pro podporu automatické správy virtualizačního
kontejneru Solaris Zones poskytuje očekávanou funkcionalitu a splňuje stanovené požadavky. Tato skutečnost byla ověřena v~rámci testování
tohoto nástroje, které je součástí této diplomové práce. Z těchto důvodů je možné konstatovat, že všechny stanovené cíle této diplomové práce 
byly naplněny. Instalační balík a zdrojové kódy implementovaného nástroje i zdrojové kódy celé diplomové práce jsou dostupné na přiloženém médiu.
\end{conclusion}

\bibliographystyle{bibliography/csn690}
\bibliography{bibliography/references}

\appendix

\chapter{Seznam použitých zkratek}
% \printglossaries
\begin{description}
        \item[DNS] Domain Name System
        \item[HW]  Hardware
	\item[IT]  Information Technology
	\item[OS]  Operating System
	\item[SW]  Software
	\item[SMF] Solaris Management Facility
	\item[VM]  Virtual Machine
	\item[VMM] Virtual Machine Monitor
	\item[ZFS] Zettabyte File System
\end{description}

\chapter{Obsah přiloženého CD}

%upravte podle skutecnosti

\begin{figure}
	\dirtree{%
		.1 readme.txt\DTcomment{stručný popis obsahu CD}.
		.1 exe\DTcomment{adresář se spustitelnou formou implementace}.
		.1 src.
		.2 impl\DTcomment{zdrojové kódy implementace}.
		.2 thesis\DTcomment{zdrojová forma práce ve formátu \LaTeX{}}.
		.1 text\DTcomment{text práce}.
		.2 thesis.pdf\DTcomment{text práce ve formátu PDF}.
		.2 thesis.ps\DTcomment{text práce ve formátu PS}.
	}
\end{figure}

\end{document}
