% options:
% thesis=B bachelor's thesis
% thesis=M master's thesis
% czech thesis in Czech language
% slovak thesis in Slovak language
% english thesis in English language
% hidelinks remove colour boxes around hyperlinks

\documentclass[thesis=M,czech]{FITthesis}[2012/06/26]

\usepackage[utf8]{inputenc} % LaTeX source encoded as UTF-8

\usepackage{graphicx} %graphics files inclusion
% \usepackage{amsmath} %advanced maths
% \usepackage{amssymb} %additional math symbols

\usepackage{listings}

\usepackage{courier}
\usepackage{listings}
\usepackage{color}
\usepackage{xcolor}

\definecolor{redbg}{RGB}{254, 210, 210}
\definecolor{redtext}{RGB}{182, 49, 39}
%\lstset{basicstyle=\footnotesize\ttfamily,breaklines=true}
\lstset{basicstyle=\ttfamily\color{black},
commentstyle = \ttfamily\color{red},
keywordstyle=\ttfamily\color{blue},
stringstyle=\color{orange}}
\lstloadlanguages{Ruby}
\colorlet{punct}{red!60!black}
\definecolor{background}{HTML}{EEEEEE}
\definecolor{delim}{RGB}{20,105,176}
\colorlet{numb}{magenta!60!black}

\lstdefinelanguage{json}{
    basicstyle=\normalfont\ttfamily,
    %numbers=left,
    numberstyle=\scriptsize,
    stepnumber=1,
    numbersep=8pt,
    showstringspaces=false,
    breaklines=true,
    %frame=lines,
    %backgroundcolor=\color{background},
    literate=
     *{0}{{{\color{numb}0}}}{1}
      {1}{{{\color{numb}1}}}{1}
      {2}{{{\color{numb}2}}}{1}
      {3}{{{\color{numb}3}}}{1}
      {4}{{{\color{numb}4}}}{1}
      {5}{{{\color{numb}5}}}{1}
      {6}{{{\color{numb}6}}}{1}
      {7}{{{\color{numb}7}}}{1}
      {8}{{{\color{numb}8}}}{1}
      {9}{{{\color{numb}9}}}{1}
      {:}{{{\color{punct}{:}}}}{1}
      {,}{{{\color{punct}{,}}}}{1}
      {\{}{{{\color{delim}{\{}}}}{1}
      {\}}{{{\color{delim}{\}}}}}{1}
      {[}{{{\color{delim}{[}}}}{1}
      {]}{{{\color{delim}{]}}}}{1},
}
%\lstset{framextopmargin=50pt,frame=bottomline}
\renewcommand*{\lstlistingname}{Kód}% Listing -> Kód
\renewcommand*{\lstlistlistingname}{Seznam ukázek kódů}% List of Listings -> List of Algorithms

\usepackage{tabularx}

 \usepackage{dirtree} %directory tree visualisation

% % list of acronyms
% \usepackage[acronym,nonumberlist,toc,numberedsection=autolabel]{glossaries}
% \iflanguage{czech}{\renewcommand*{\acronymname}{Seznam pou{\v z}it{\' y}ch zkratek}}{}
% \makeglossaries

\newcommand{\tg}{\mathop{\mathrm{tg}}} %cesky tangens
\newcommand{\cotg}{\mathop{\mathrm{cotg}}} %cesky cotangens

% % % % % % % % % % % % % % % % % % % % % % % % % % % % % % 
% ODTUD DAL VSE ZMENTE
% % % % % % % % % % % % % % % % % % % % % % % % % % % % % % 

\department{Katedra počítačových systémů}
\title{Podpora automatické správy virtualizačního kontejneru Solaris Zones na~platformě Solaris}
\authorGN{Tomáš} %(křestní) jméno (jména) autora
\authorFN{Šimáček} %příjmení autora
\authorWithDegrees{Bc. Tomáš Šimáček} %jméno autora včetně současných akademických titulů
\author{Tomáš Šimáček} %jméno autora bez akademických titulů
\supervisor{Ing. Michal Šoch, Ph.D.}
\acknowledgements{Doplňte, máte-li komu a za co děkovat. V~opačném případě úplně odstraňte tento příkaz.}
\abstractCS{% Abstract závěrečné práce v češtině
Tato diplomová práce se zabývá problematikou automatické správy virtualizačního kontejneru Solaris Zones na platformě Solaris.
Její součástí je podrobný popis tohoto virtualizačního kontejneru a také porovnání běžně využívaných virtualizačních technik.
Praktická část této práce je zaměřena na~návrh a implementaci nástroje, který podporuje automatickou správu Solaris Zones.
Nástroj klade důraz na možnost automatického vytváření zón pomocí šablon, které umožňují předem definovat jejich vlastnosti.
Součástí implementace jsou také automatizované procesy zálohy, obnovy nebo migrace zón, které je možné provádět na lokálních
i vzdálených serverech.


}
\abstractEN{% Abstract of the master thesis in english
Abstract in english
}
\placeForDeclarationOfAuthenticity{V~Praze}
\declarationOfAuthenticityOption{4} %volba Prohlášení (číslo 1-6)
\keywordsCS{% Klíčová slova v češtině
Solaris Zones
}
\keywordsEN{% Keywords in english
Solaris, Solaris Zones, virtualization, automatic management
}
% \website{http://site.example/thesis} %volitelná URL práce, objeví se v tiráži - úplně odstraňte, nemáte-li URL práce

\newtheorem{definition}{Definice}

\begin{document}

% \newacronym{CVUT}{{\v C}VUT}{{\v C}esk{\' e} vysok{\' e} u{\v c}en{\' i} technick{\' e} v Praze}
% \newacronym{FIT}{FIT}{Fakulta informa{\v c}n{\' i}ch technologi{\' i}}

\begin{introduction}
  % Chapter: Introduction
% Author: Tomáš Šimáček
Virtualizace je technika, se kterou se dnes v IT můžeme setkat v mnoha podobách. Jednou z hlavních oblastí využití virtualizace je virtualizace serverů a mimo jiné se objevuje i v oblasti
komunikačních sítí a desktopů. Tato technologie umožňuje vytvářet virtuální prostředí nebo prostředky na fyzickým hardware. Speciální softwarová vrstva zvaná virtualizační monitor (VMM)
zajišťuje efektivní rozdělování prostředků fyzického systému mezi virtualizované subjekty.

Hlavním tématem této práce je virtualizace serverů, která umožňuje rozdělit jeden fyzický systém na několik nezávislých virtuálních prostředí zvaných virtuální počítač (VM).
Možnost vytváření VM značně snižuje náklady na pořizování a provoz fyzických strojů, jelikož už není třeba dedikovaný server pro každou instanci OS. A konečně správným rozdělením
VMs na fyzické servery můžeme docílit ideálního rozdělení zátěže a tím efektivně využít dostupné fyzické prostředky.

Rostoucí počet virtualizovaných serverů může mít za následek obtížnější správu. Automatizované nasazování, instalace nebo zálohování VMs může být značným ulehčením vývoje software,
testování nebo nasazování aplikací do produkčního prostředí. Tomuto tématu se tato práce věnuje v souvislosti s virtualizačním kontejnerem Solaris Zones.

\section{Cíle práce}

Prvním cílem této práce je seznámení se s operačním systémem Solaris a jeho funkcemi. Především jde o popis virtualizační techniky Solaris Zones. Důraz je kladen na popis
základních principů, které umožňují běh více zón v rámci jednoho sdíleného jádra OS.

Dalším cílem je detailní popis možností konfigurace zón, jejich instalace, zálohování a v neposlední řadě také integrace Solaris Zones s ostatními službami operačního systému
Solaris.

Třetí cíl této práce je porovnat Solaris Zones s ostatními virtualizačními technologemi.

Posledním cílem této práce je implementace nástroje, který umí spravovat větší množství Solaris Zones. Tento nástroj bude umožňovat (dávkovou a interaktivní) instalaci zón na
lokální i vzdálené servery, náhradu existujících zón a jejich zálohování. Dále bude umožňovat automatické přidání předem definovaných softwarových balíčků po instalaci zóny.

\section{Struktura práce}

TODO


\end{introduction}

\chapter{Virtualizace}
  % Virtualization chapter about concept, types, examples and implementation
V dnešní době existuje mnoho různých typů virtualizace, jak již bylo zmíněno v úvodu. Následující kapitola zběžně představuje několik nejznámější typů virtualizace v IT a dále se věnuje hlavně
tématu virtualizace serverů. V kapitole jsou popsány některé scénáře pro přechod z klasické infrastruktury na virtuální. Ve zbytku kapitoly jsou popsány obecné principy virtualizace a základy 
virtualizace CPU, paměti a I/O zařízení.

\section{Definice pojmů}
Pro potřeby popisu virtualizačních technik si definujeme základní pojmy a entity, které se ve virtuální infrastruktuře vyskytují.

\textit{Fyzické prostředky} počítače jsou TODO

\textit{Virtualizační monitor}, \textit{Virtual Machine Monitor - VMM} nebo také \textit{Hypervisor} je softwarová vrstva, která virtualizuje HW prostředky fyzického počítače a přiděluje je
virtuálním strojům.

\textit{Host} je HW a SW platforma, která poskytuje virtuálním strojům své fyzické prostředky. Softwarové vybavení hosta obsahuje virtualizační
monitor a v některých případech může obsahovat i operační systém hosta tzv. \textit{Host OS}

\textit{Virtuální stroj} nebo \textit{Virtual Machine - VM} je virtualizované prostředí vytvořené virtualizačním monitorem, ve kterém běží operační systém virtuálního počítače nazývaný \textit{Guest OS}.

\section{Virtualizace serverové infrastruktury}

První a pravděpodobně nejvýznamnější oblastí využití virtualizace, je oblast výpočetních serverů. Výpočetní server je fyzický počítač s operačním systémem a výpočetními prostředky, který může poskytovat nějaké služby svému okolí.
V dnešní době se s těmito servery setkáváme každý den a to v podobě webových serverů, serverů poskytujících službu DNS a mnohých dalších.

Proces virtualizace serverů spočívá v přenesení systémů a jejich služeb z fyzických serverů do virtuálního prostředí. Toto virtuální prostředí je vytvořeno virtualizačním monitorem běžícím na jiném fyzickém stroji a následně poskytováno virtuálním strojům.
Virtualizační monitor nebo také VMM je detailněji popsán v kapitole \ref{vmm}, kde jsou představeny jeho typy a funkce. Serverová architektura využívající virtualizace se skládá především z tzv. virtualizačních serverů, které slouží jako zdroj fyzických
prostředků pro virtuální počítače. Tyto servery se vyznačují především velkým množstvím operační paměti a vysokým výpočetním výkonem, který je díky VMM rozdělován mezi hostované virtuální stroje.

Tato práce se zaměřuje právě na techniky virtualizace, které jsou v dnešní době aktuální a využívají se k virtualizaci serverů. Práce podrobně představuje virtualizační techniku Solaris Zones of firmy Oracle, která slouží pro
vyváření virtuálních strojů (zón), které sdílejí jedno jádro OS.

\section{Využití virtualizace v sítích}



\section{Virtualizace desktopu}



\section{Nasazení virtuální infrastruktury}

Pro přechod k virtuální infrastruktuře serverů existuje v dnešní době několik dobrých důvodů. Jedním z hlavních benefitů virtualizace pro dnešní firmy a organizace je značná finanční úspora. Tato úspora se
projevuje především ve snížení nákladů organizace na pořizování a provoz fyzických zařízení. 

Mezi další benefity virtualizace patří především efektivní využití výpočetních zdrojů, vysoká dostupnost běžících aplikací nebo vytvoření oddělených a nezávislých prostředí pro vývoj, testování a nasazení software.

Výhody zavedení virtuální infrastruktury jsou podrobněji popsány v následujících podkapitolách, které se zabývají základními scénáři pro nasazení virtuální infrastruktury.

\subsection{Konsolidace}
\label{consolidation}

Konsolidace serverů je proces sjednocování systémů z více fyzických serverů na jeden fyzický server, který pro tyto systémy poskytne virtuální prostředí pro jejich běh. Vstupem tohoto procesu je tedy několik systémů na fyzických serverech,
na kterých běží různé aplikace. Vstup procesu je naznačen na obrázku \ref{consolidation_img} vlevo. Výstupem konsolidace je jeden fyzický server s dostatečnými prostředky, na kterém konsolidované systémy běží jako virtuální počítače.
Výstup můžeme vidět na obrázku \ref{consolidation_img} vpravo.

\begin{figure}
    \centering    
    \caption{Konsolidace serverů}
    \label{consolidation_img}
\end{figure}

\subsubsection*{Využití scénáře}

Dnes je zcela běžnou praktikou provozovat jednu aplikaci na jednom dedikovaném serveru. Pokud aplikace využívá jen malé procento výpočetních zdrojů daného serveru, může administrátor sjednotit více takovýchto serverů
do jednoho. Pro organizaci, která vlastní tisíce takovýchto serverů může konsolidace výrazně zmenšit požadavky na prostor, spotřebu energie a provoz fyzických serverů. Správnou konsolidací serverů může společnost docílit
efektivního využití dostupných prostředků a tím výrazně snížit vynaložené finanční prostředky \cite{reasons}.

Rychlý vývoj technologií v oblasti hadware zapříčiňuje rychlé stárnutí některých systémů a přechod ze staršího na nový může být složitý. Obzvláště v případě, kdy systém potřebuje ke svému běhu speciální hadware.
Aby bylo možné provozovat služby poskytované těmito zastaralými systémy, můžeme je spustit jako virtuální počítač na modernějším hadware. Systém se bude chovat stejně jako kdyby běžel na zastaralém hadware, zatímco
výkonost služby může těžit z novější a výkonnější hadwarové vrstvy  \cite{reasons}.

\subsection{Izolace}

Dalším ze scénářů využití virtualizované infrastruktury je izolace aplikací. Proces izolace aplikací spočívá v oddělení dvou a více kritických aplikací běžících na jednom systému do nezávislých virtuálních prostředí.
Vstupem je jeden systém s aplikacemi, které se mohou negativně ovlivňovat. Vstup izolačního scénáře je naznačen na obrázku \ref{izolation} vlevo. Výstupem je několik nezávislých virtuálních počítačů, ve kterých běží
jednotlivé aplikace. Výstup izolace aplikací je ukázán na obrázku \ref{izolation} vpravo.

\begin{figure}
    \centering    
    \caption{Izolace aplikací}
    \label{izolation}
\end{figure}

\subsubsection*{Využití scénáře}

V dnešní době jsou útoky na aplikace vystavené do internetu běžnou záležitostí. Pokud útočník využije nějaké zranitelnosti aplikace, může v některých případech získat kontrolu nad celým systémem. V takovém případě jsou
ohroženy všechny data a aplikace, které na daném systému běží. Vhodným krokem v tomto případě je proto využití virtualizace a rozdělení aplikací do nezávislých prostředí.

Jedním z příkladů ohrožení systému může být útok na výpočetní zdroje. Podstatou útoku je vyčerpání fyzických zdrojů systému, což má za následek nedostupnost jeho služeb a v některých případech i pád celého systému.
Ve virtualizovaném prostředí lze přidělit každému VM pouze určitou část prostředků a tím chránit celý systém před jejich vyčerpáním. V případě napadení jednoho VM sice dojde k jeho vyřazení, ale ostatní VM a jejich služby
mohou dále pokračovat v běhu.

\subsection{Migrace}

Posledním diskutovaným scénářem nasazení virtualizované infrastruktury je migrace. Jedná se o proces přesunutí systému z jednoho počítače na druhý. V rámci virtualizace se budeme bavit o přesouvání systému na počítač s běžícím VMM,
který zprostředkovává virtuální prostředí. Výstupem procesu je systém, který do něj zároveň vstupuje. Rozdíl je v tom, že daný systém na konci procesu běží ve virtuálním prostředí nějakého VMM. Dle typu migrovaného systému můžeme
rozdělit scénář na následující typy.

\subsubsection*{Migrace VM}

Migrací virtuálního stroje se rozumí přesun VM mezi dvěma různými fyzickými stroji s VMM. Tento přesun byl dříve možný pouze v případě když oba stroje měli stejný HW, operační systém a procesor \cite{reasons}.
Tato možnost administrátorovi umožňuje přesouvat virtualizované systémy na výkonnější hosty a tím umožňuje dynamicky regulovat využití fyzických prostředků v závislosti na aktuální zátěži systému.

Další výhodou zavedení virtualizované architektury je zajištění vysoké dostupnosti služeb. Virtualizace umožňuje zajistit redundanci ve smyslu spuštění služby na více serverech najednou. Ve virtualizované architektuře můžou
nastat dva typy selhání. Prvním typem je selhání VM uvnitř VMM. Pokud dojde k selhání některé VM, jiná VM převezme obsluhu požadavků a v minimálním čase dojde k obnovení služby. Druhým typem je selhání celého VMM nebo hosta.
V tomto případě je nutné provozovat více redundantních hostů pro VM, které v případě HW chyby převezmou obsluhu služby.

Proces migrace VM je představen na obrázku \ref{migration1}, kde můžeme vidět konfiguraci před migrací (vlevo) a po provedení migrace VM (vpravo).

\begin{figure}
    \centering    
    \caption{Migrace virtuálního stroje}
    \label{migration1}
\end{figure}

\subsubsection*{Migrace fyzického stoje na VM}

Migrace fyzického stroje na virtuální stroj je proces, kdy dochází k přesunu a virtualizaci systému z fyzického stroje. Vstupem je tedy systém běžící na stroji bez VMM, jak je naznačeno na obrázku \ref{migration2} vlevo. 
Výstupem tohoto procesu je opět virtuální stroj běžící ve virtuálním prostředí VMM.

Virtualizací serverů dochází k uvolňování fyzického HW a stejně jako v případě konsolidace tak společnost může značně ušetřit na nákladech nutných k provozu a správě fyzických serverů. Obecně lze říct, že tento scénář přináší 
podobné benefity jako v případě konsolidace popsané v kapitole \ref{consolidation}.

\begin{figure}
    \centering    
    \caption{Migrace fyzického na virtuální stroj}
    \label{migration2}
\end{figure}

\section{Virtualizační monitor}
\label{vmm}



\section{Techniky virtualizace}



\section{Virtualizace CPU}
\section{Virtualizace Paměti}
\section{Virtualizace I/O zařízení}



  
\chapter{Solaris}
  % Brief description of Solaris operating system


\chapter{Solaris Zones}
  % Detailed description of Solaris Zones configuration, installation, backup etc.
V úvodu následující kapitoly jsou popsány základní principy a struktura virtualizační techniky Solaris Zones od
firmy Oracle. Dále se tato kapitola zabývá popisem datových struktur, postupů a nástrojů, které slouží ke správě a manipulaci
se zónami. Detailnější popis je věnován především administrátorským rutinám pro konfiguraci, instalaci a zálohování zón.
\section{Virtualizační technika}
Oracle Solaris Zones je virtualizační technika, která umožňuje běh více virtuálních strojů na jednom fyzickém stroji. Jak již
bylo zmíněno v kapitole \ref{chapter:solaris}, tato technika je standardní součástí operačního systému Oracle Solaris od 
verze systému Solaris 10. Aktuální verze je Solaris 11.3 a přináší novou funkcionalitu v podobě podpory starších verzí 
operačního systému Solaris.

Pokud bychom chtěli zařadit Solaris Zones do klasifikace virtuálních strojů popsané v kapitole \ref{section:clasification}, 
pak bychom jí zařadili do sekce \textit{virtualizace na úrovni OS}. Jinak řečeno, tato technika rozděluje zdroje hostitelského
operačního systému jako CPU, paměť nebo I/O zařízení mezi běžící virtuální stroje a zajišťuje izolaci na úrovni procesů,
souborového systému a sítě. Zónu pak můžeme definovat jako virtuální kontejner běžící v hostitelském operačním systému, který
využívá zdrojů hostitelského operačního systému a je izolovaný od ostatních zón.

Standardně se po instalaci operačního systému Solaris nachází v systému jedna zóna. Je to vlastní instance operačního systému
a nazývá se \textbf{globální} zóna. Jinými slovy, je to zóna, která běží přímo na hardwaru počítače nebo ve virtualizovaném
prostředí. Tato zóna má dvě hlavní funkce. Za prvé tato zóna plní funkci hlavního operačního systému a přebírá kontrolu
nad fyzickými prostředky po startu systému. Za druhé je hlavním centrálním prvkem pro administraci celého systému a ostatních
zón. Globální zóna poskytuje globální pohled na celý systém a má přehled o všech systémových zdrojích a aktivitách ostatních
zón. Její role v rámci Solaris Zones je zásadní a její chyba může zapříčinit pád ostatních zón. Z tohoto důvodu je doporučené
používat globální zónu pouze pro účely administrace systému a managementu ostatních zón.

Zóny, které jsou spouštěny v rámci globální zóny nazýváme \textbf{neglobální} zóny. Tyto zóny jsou navzájem izolované na
několika úrovních. První úroveň izolace je izolace na úrovni sítě. Každá neglobální zóna může mít svůj vlastní logický síťový
adaptér, který je vytvořený nad nějakým fyzickým síťovým rozhraním a je dostupný pouze pro konkrétní zónu. Z jiné neglobální
zóny tento adaptér není přístupný. Takto vytvořený síťový adaptér může být spravován pouze z globální zóny a ze zóny, ke které
byl přiřazen v průběhu jejího vytváření.

Druhou úrovní izolace zón je souborový systém. Globální zóna spravuje svůj vlastní souborový systém ve kterém se nachází
standardní adresářová struktura operačních systému typu UNIX. Můžeme v něm nalézt adresář \textit{/etc} sloužící pro globální
systémovou konfiguraci, adresář \textit{/bin} obsahující uživatelsky spustitelné programy nebo například adresář \textit{/sbin},
který uchovává programy spustitelné pod privilegovaným uživatelem. Každá neglobální zóna potřebuje pro svůj běh velmi podobné
prostředí, a proto má svůj vlastní souborový systém, který se nachází někde v hierarchii souborového systému globální zóny.
Podle typu zón popsaných v kapitole \ref{chapter:zones:types} může být tento souborový systém částečně sdílený se souborovým
systémem globální zóny a nebo může obsahovat kompletně nezávislý obraz zóny. Při spouštění zóny pak dojde pomocí příkazu
\verb|chroot(1)| k přepnutí kořenu souborového systému a zóna pracuje pouze se svojí částí souborového systému. V případě 
sdílené části souborového systému jsou tyto části připojeny v režimu read-only \cite{virt1}. Tím je zajištěno, že souborové
systémy jednotlivých zón jsou vzájemně izolované a nemohou se vzájemně ovlivnit. Správu těchto souborových systémů je opět
možné provádět pouze z konkrétní zóny a nebo ze zóny globální.

Mimo izolace na úrovni sítě a souborového systému Solaris Zones implementuje ještě izolaci na úrovni procesů. Každá neglobální
zóna má svůj plánovač a může spouštět svoje vlastní procesy. Procesy běžící v jedné neglobální zóně nejsou žádným způsobem 
viditelné ani přímo ovlivnitelné z jiných neglobálních zón. Pokud chce proces z jedné zóny komunikovat s procesem z jiné zóny,
nemůže k tomu využít mezi procesovou komunikaci, ale musí použít počítačové sítě. Naopak procesy, které běží v rámci jedné
zóny spolu mohou komunikovat pomocí signálů, sdílené paměti a tak podobně. Všechny procesy běžící v systému mohou být
spravovány z globální zóny. Globální zóna tedy nabízí globální přehled všech procesů, které jsou spuštěny ve všech běžících
zónách v systému. Výstup příkazu \verb|ps(1)| v globální zóně zobrazí všechny procesy, zatímco v neglobální zóně budou zobrazeny
pouze procesy příslušící dané zóně.
\subsection{Typy zón}
\label{chapter:zones:types}
Jak bylo zmíněno výše, virtualizační technika Solaris Zones umožňuje v rámci jedné globální zóny spouštět mnoho neglobálních
zón. Každá neglobální zóna má vlastnost zvanou \textit{brand}, která určuje typ neglobální zóny. Tato vlastnost se specifikuje
při konfiguraci zóny v globální zóny a dle přehledu \cite{oracle:solaris:zones:brands} může typ zóny mít následující hodnoty:

\begin{itemize}
 \item \textit{solaris}
 \item \textit{solaris-kz}
 \item \textit{solaris10}
\end{itemize}

\textit{Brand} neboli typ zóny určuje jakým způsobem se zóna bude po spuštění chovat. Implicitním typem zóny v Solaris Zones
je \textit{solaris}, kterému se také jinak přezdívá nativní zóna nebo také tenká zóna. Dalším typem zóny je \textit{solaris-kz},
kde zkratka za pomlčkou v názvu odpovídá slovnímu spojením kernel zone. Jak název napovídá, tato zóna má vlastní jádro
operačního systému a někdy se jí také přezdívá plná nebo tlustá zóna. Posledním typem zón, kterou Solaris Zones umí vytvářet
je \textit{solaris10}. Hlavním úkolem této zóny je zajišťovat zpětnou kompatibilitu s operačním systémem Solaris 10 a umožňuje
uvnitř této zóny spouštět aplikace určené pro tento systém.

V následujících podkapitolách jsou podrobněji popsány výše zmíněné typy zón.
\subsubsection{Nativní zóna}
\label{chapter:zones:native}
Nativní neboli tenká zóna umožňuje administrátorovi vytvořit zónu, která má sdílené jádro operačního systému s globální zónou.
Jinými slovy verze jádra operačního systému musí být stejná jako v globální zóně. Tento typ zóny je izolovaný pouze nad svým
souborovým systémem a nemá standardně nemá k dispozici informace o žádném fyzickém zařízení systému. Souborové systémy ostatních
zón jsou nedostupné a konkrétní neglobální zóna o nich nemá žádné informace. Z jejího pohledu existuje pouze její kořenový
souborový systém. Jak již bylo popsáno výše, kořenový sytém nativní zóny může být sdílený se souborovým systémem globální
zóny a sdílet tak základní systémové nástroje. Pokud chceme této zóně delegovat nějaký typ zařízení, musíme tak učinit při
konfiguraci zóny v globální zóně. Tímto způsobem můžeme nativní zóně zpřístupnit souborové systémy, ZFS pool nebo
ZFS dataset. Takto definované prostředky pak po instalaci zóny můžeme z této zóny využívat.

Dále tento typ zóny má svoji vlastní databázi produktů, která obsahuje informace o všech nainstalovaných softwarových
komponentech v konkrétní globální zóně. Opět platí, že konkrétní neglobální zóna vidí pouze své balíčky. Díky tomu je možné
instalovat dodatečné softwarové balíčky do neglobálních zón, které nemusí být nainstalované v globální zóně
\cite{oracle:solaris:zones:brands}. Některé softwarové balíčky jsou ale společné s globální zónou (kernel) a nelze tedy
provádět kompletní aktualizaci bez zásahu do globální zóny. 

Nativní zóna podporuje dva typy síťových rozhraní, které mohou být zóně při konfiguraci přiřazeny. Prvním typem je sdílená 
adresa neboli \textit{shared-ip}. Tento typ síťového rozhraní sdílí ip adresu s nějakým fyzickým rozhraním globální zóny. 
Pokud chce neglobální zóna komunikovat s okolím, bude v hlavičce paketu ip adresa globální zóny a při obdržení odpovědi
globální zóna přesměruje paket na virtuální síťové rozhraní konkrétní globální zóny. Zde můžeme pozorovat podobnost s
technikou NAT v sítích. Jako druhý typ rozhraní můžeme použít exkluzivní rozhraní nebo také \textit{exclusive-ip}, které
nesdílí ip adresu s globální zónou, ale má svoji vlastní. V tomto případě veškerý síťový provoz generovaný touto zónou bude
mít v hlavičce jinou ip adresu než zóna globální.

Tento typ zóny nepodporuje vytváření další neglobálních zón. Jinými slovy se nativní neglobální zóna nemůže chovat jako
globální zóna a vytvářet nové zóny uvnitř sebe. Stejně tak z nativní zóny nemůžeme vytvářet ani spravovat jiné neglobální zóny.

Nativní zóna je implicitní typ zóny v Solaris Zones a pokud administrátor nespecifikuje jinak při vytváření zóny, bude nově
vytvořená zóna právě typu \textit{solaris}. Tento typ zóny může být provozován na všech systémech, které podporují operační
systém Oracle Solaris 11.3 \cite{oracle:solaris:zones:brands}.
\subsubsection{Kernel zóna}
\label{chapter:zones:kernel}
Druhým typem zóny, který virtualizační technika Solaris Zones umožňuje vytvářet, je kernel zóna nebo také tlustá zóna. Tento
typ zóny obsahuje vlastní jádro operačního systému a na rozdíl od nativní zóny ho nesdílí s globální zónou. Kernel zóna
tedy může být provozována na jiné verzi jádra než globální zóna. V důsledku toho kernel zóna podporuje funkcionalitu,
které nelze pomocí nativní zóny dosáhnout.

Stejně jako v případě nativní zóny i kernel zóna obsahuje vlastní databázi instalovaných softwarových balíčků. Jelikož i 
kernel zóna je neglobální, nelze z ní žádným způsobem vidět balíčky ostatních zón. Na rozdíl od nativní zóny administrátor
může provádět aktualizaci všech balíčků, protože kernel zóna nesdílí nic s globální zónou. Žádné balíčky tedy nejsou závislé
na balíčkách globální zóny.

V případě síťových rozhraní kernel zóny podporují pouze rozhraní typu \textit{exclusive-ip} a neumožňuje sdílet síťovou adresu
s globální zónou.

Na rozdíl od nativní zóny se kernel zóny mohou chovat jako globální zóny uvnitř hostitelské globální zóny. Jinými slovy je
možné uvnitř kernel zóny vytvářet další neglobální zóny a vytvářet tak hierarchickou strukturu virtuální strojů. Je na zvážení
administrátora, jestli daný scénář vyžaduje tuto strukturu.
\subsubsection*{Požadavky}
Jelikož provoz kernel zóny se liší od provozu standardní nativní zóny, liší se i požadavky na hostitelský systéme. Požadavky
se liší v závislosti na platformě. Pro jednoduchost si uvedeme pouze požadavky pro systémy s architekturou x86 a procesorem 
intel. Podle specifikace \cite{oracle:solaris:zones:kernel_zones_requiremets} se na hostitelský systém kladou následující
požadavky.

\begin{itemize}
 \item Procesor musí být typu Nehalem nebo novější
 \item Virtualizace CPU (VT-x) 
 \item Podpora virtualizace paměti (RVI, EPT)
 \item Ochrana paměti
\end{itemize}

Výše zmíněné požadavky kladou nároky na HW vybavení hostitelského systému. Spolu s těmito požadavky musí být v globální zóně
nainstalovaný balíček \textit{brand/brand-solaris-kz}, který umožňuje vytváření kernel zón. Administrátor může pomocí příkazu
\verb|virtinfo(1)| zjistit, jaký typ virtualizace je v globální zóně podporován. Výpis programu na virtualizované platformě
VMvare, kde jsou splněné výše zmíněné požadavky může vypadat následovně.

\begin{verbatim}
zadmin@shost:~$ virtinfo
NAME            CLASS     
vmware          current   
non-global-zone supported
kernel-zone     supported
\end{verbatim}


\subsubsection{Branded zóna}
\label{chapter:zones:branded}
Branded zóny byly vytvořeny pro zpětné zajištění kompatibility se staršími verzemi operačního sytému Solaris. Díky technologii
\textit{BrandZ} \cite{oracle:solaris:zones:brands} umožňují spouštění aplikací určených pro operační systém Solaris 10 na 
systému s OS Solaris 11. Aplikace mohou běžet v nezměněné formě v bezpečném prostředí, které je zajištěno neglobální zónou.

Z pohledu administrátora se tento typ zóny chová stejně jako nativní zóna a má stejné vlastnosti, které jsou popsané v
podkapitole \ref{chapter:zones:native}.
\subsubsection{Shrnutí}
\label{chapter:zones:summary}
Virtualizační technologie Solaris Zones umožňuje vytvářet neglobální zóny uvnitř primární globální zóny. Neglobální zóny
poskytují izolované prostředí pro nezávislý a bezpečný běh aplikací. Zóny jsou izolovány na úrovni počítačové sítě,
souborového systému a běžících procesů, čímž je zajištěno, že se vzájemně nemohou přímo ovlivňovat. Jediný způsob komunikace
procesů z jiných zón je pomocí počítačové sítě. 

Neglobální zóna může být několika druhů, které poskytují různé vlastnosti. Tabulka \ref{table:zone_comparison} poskytuje stručný
přehled základních vlastností jednotlivých druhů zón.
\begin{table}
  \centering  
  \caption[Porovnání typů zón a jejich vlastností]{Porovnání typů zón a jejich vlastností}
  \label{table:zone_comparison}
\end{table}
\section{Administrace}
\label{chapter:zones:administration}
Globální zóna slouží hlavně pro účely správy hostitelského systému a všech neglobálních zón. Poskytuje nainstalovaným zónám
prostředky pro jejich běh a spravuje informace o jejich stavu. Je to klíčové místo pro správu celého systému a správných chod
neglobálních zón vyžaduje bezproblémový chod globální zóny. Administrátor takového systému by tento fakt vzít v úvahu a
nepoužívat globální zónu jako zdroj pro spouštění uživatelských aplikací.

Proces vytváření zón se skládá ze dvou částí. První částí je konfigurace neglobální zóny, kdy administrátor specifikuje jaké
parametry má zóna mít a jaké prostředky bude moci využívat. Tento proces zavede zónu do databáze globální zóny a od této
chvíle je registrovaná v systému. K tomuto účelu nabízí Solaris Zones nástroj \verb|zonecfg(1)|. Více o tomto nástroji a
možnostech konfigurace zón je popsáno v kapitole \ref{chapter:zones:configuration}.

Z předchozího procesu vznikne jakýsi recept na to, jak danou neglobální zónu vytvořit. Nyní je třeba vytvořit souborový systém
zóny se všemi balíčky, které zóna bude potřebovat ke svému běhu. Této fázi se říká instalace zóny a v jejím průběhu se do
kořenového souborového systému nahrávají určené balíčky a nastavuje se konfigurační profil zón. Pro tento účel slouží nástroj
\verb|zoneadm(1)|. Výstupem tohoto procesu je nainstalovaná zóna připravená ke spuštění. Detailněji nástroj pro instalaci zón
a proces instalace popisuje kapitola \ref{chapter:zones:instalation}.

Čerstvě nainstalovaná zóna může být opět pomocí nástroje \verb|zoneadm(1)| spuštěna a následně se do ní může privilegovaný 
uživatel přihlásit pomocí nástroje \verb|zlogin(1)|.
\subsection{Administrátor}
\label{chapter:zones:administration:administrator}
Jelikož neglobálních zón může být v systému provozováno velké množství, může být pro administrátora globální zóny spravovat 
celou globální zónu a přitom se starat o všechny neglobální zóny. Pro tento účel může být vytvořen uživatel, který se bude
starat výhradně jenom o neglobální zóny. Tento uživatel bude mít práva na správu všech neglobálních zón.

Pokud by i tak bylo neglobálních zón mnoho, je možné jednotlivým neglobálním zónám přiřadit vlastního administrátora. Ten
se bude moc do zóny přihlásit pomocí příkazu \verb|zlogin(1)| a provádět údržbu a správu systému.
\subsection{Stavový model zón}
\label{chapter:zones:administration:states}
V Solaris Zones má neglobální zóna definovaný stavový model. Jsou to stavy, ve kterých se zóna během jejího životního cyklu
může nacházet. Zónu můžeme převést z jednoho stavu do druhého pouze nějakou kombinací nástrojů \verb|zonecfg(1)| a
\verb|zoneadm(1)|. Podle specifikace \cite{oracle:solaris:zones:states} se neglobální zóna může nacházet v jednom z
následujících sedmi stavů.
\begin{itemize}
 \item \textit{Configured}
 \item \textit{Incomplete}
 \item \textit{Unavailable}
 \item \textit{Installed}
 \item \textit{Ready}
 \item \textit{Running}
 \item \textit{Shutting down/Down}
\end{itemize}
V každém z těchto stavů může administrátor používat pouze nějakou podmnožinu příkazů, které zónu ovládají. Pro příklad
můžeme uvést, že nelze zónu spustit pokud se nachází například ve stavu \textit{configured}. Pro spuštění se zóna musí nacházet
ve stavu \textit{installed}. V následujících odstavcích je stručně shrnut význam jednotlivých stavů.
\subsubsection{Configured}
\label{chapter:zones:administration:states:configured}
Stav \textit{configured} značí, že konfigurace zóny je hotová a uložená na perzistentním úložišti. V tuto chvíli se zónou
ještě nebyl asociován žádný diskový obraz a tedy nemá připojený kořenový souborový systém. Tento stav je v pořadí první stav,
ve kterém se zóna může od svého vzniku nacházet. Nachází se v něm buď bezprostředně po vytvoření konfigurace pomocí \verb|zonecfg(1)|
nebo když je zóna odinstalována nebo odpojena.
\subsubsection{Incomplete}
\label{chapter:zones:administration:states:incomplete}
Zóna se nachází ve stavu \textit{incomplete} během procesu instalace a odinstalace. Je to přechodný stav, ale v případě poškození
nainstalované zóny může být v tomto stavu stále. V případě úspěchu procesu instalace pomocí nástroje \verb|zonecfg(1)| je 
stav zóny změněn na stav \textit{installed}. Pokud uspěje proces odinstalování zóny pomocí stejného nástroje, je stav
změněn na stav \textit{configured}.
\subsubsection{Unavailable}
\label{chapter:zones:administration:states:unavailable}
Ve stavu \textit{unavailable} se zóna nachází v případě, kdy zóna byla v minulosti nainstalována, ale momentálně nemůže být
spuštěna, přesunuta nebo její validace vrací chybu. Tento stav může mít několik příčin. Jednou z nich může být nedostupnost
zdrojového souborového systému zóny. Souborový systém může být nedostupný chybou administrátora nebo například chybou diskového
zařízení. Další příčinou může být nekompatibilita softwarového vybavení globální zóny a neglobální zóny. To se může stát například
ve chvíli kdy migrujeme neglobální nativní zónu z jednoho systému na druhý a tyto dva systémy mají odlišnou verzi jádra.
\subsubsection{Installed}
\label{chapter:zones:administration:states:installed}
Stav \textit{installed} signalizuje, že zóna s danou konfigurací je nainstalovaná ve svém kořenovém souborovém systému,
ale nemá alokovanou žádnou virtuální platformu pro svůj běh. Jinými slovy ještě nemůže být přímo spuštěna. Zóna ve stavu
\textit{installed} již může být zálohována nebo migrována mezi různými hosty. 
\subsubsection{Ready}
\label{chapter:zones:administration:states:ready}
Zóna se nachází ve stavu \textit{ready}, právě když je pro ni alokována virtuální platforma pro její běh. To znamená, že jádro
hostitelského operačního systému Solaris vytvořilo proces \verb|zsched|, vytvořilo virtuální síťové rozhraní a zpřístupnilo
je neglobální zóně. Obecně jádro inicializovalo všechny prostředky specifikované v konfiguraci zóny a zpřístupnilo je dané
neglobální zóně. V tomto stavu ještě nebyl spuštěn žádný uživatelský proces asociovaný s konkrétní zónou \cite{oracle:solaris:zones:states}.
Tento stav je tranzitní a nastává v okamžiku kdy zahajujeme boot zóny pomocí příkazu \verb|zoneadm(1)|.
\subsubsection{Running}
\label{chapter:zones:administration:states:running}
Ve stavu \textit{running} se zóna nachází když je spuštěn první uživatelský proces. Většinou se jedná o proces \verb|init(1)|,
který inicializuje celou zónu a umožňuje spouštění procesů uvnitř dané neglobální zóny. Zóna ve stavu \textit{running} má tedy
alokovanou virtuální platformu v jádru hostitelského operačního systému, inicializované všechny zařízení a spuštěné uživatelské
procesy.
\subsubsection{Shutting down/Down}
\label{chapter:zones:administration:states:down}
Posledním stavem respektive dvojicí stavů, ve kterých se může neglobální zóna nacházet jsou stavy \textit{shutting down} resp. 
\textit{down}. Tyto stavy jsou tranzitní a nastávají ve chvíli, kdy daná zóna zastavuje svůj běh. V případě, kdy nelze zónu
z nějakého důvodu zastavit, může daná zóna setrvat v některém z těchto dvou stavů \cite{oracle:solaris:zones:states}. 
\subsubsection{Doplňkové stavy kernel zón}
\label{chapter:zones:administration:states:kernel_zones}
Výše zmíněné stavy jsou společné pro všechny typy zón. Nastávají tedy u nativních zón, kernel zón i branded zón. Pro kernel
zóny existují ještě další stavy, které zmíníme jenom v rychlosti, jelikož nejsou tolik podstatné. Jedná se o stavy
\textit{suspended}, \textit{debugging}, \textit{panicked}, \textit{migrating-out} a \textit{migrating-in}. Jména stavů jsou
samovysvětlující, a proto nemá smysl je podrobněji popisovat.
\section{Konfigurace}
\label{chapter:zones:configuration}
\section{Instalace}
\label{chapter:zones:instalation}
\section{Zálohování}
\label{chapter:zones:backup}
  
\chapter{Návrh aplikace}
  Tato část diplomové práce popisuje návrh aplikace pro podporu automatické správy virtualizačního kontejneru Solaris Zones na
platformě Solaris. Zaměřuje se především na popis funkcionality a požadavků, které aplikace musí splňovat. V~závěrečné části
této kapitoly je rozebrána bezpečnost a požadavky na~uživatele, který aplikaci bude moci využívat.
\section{Požadavky na aplikaci}
\label{chapter:design:demands}
Hlavním cílem této práce je vytvořit aplikaci, která bude administrátorovi operačního systému Solaris ulehčovat správu
většího množství neglobálních zón. Na základě účelu aplikace je nutné vytvořit požadavky, které bude muset výsledná implementace
aplikace splňovat. Pokud vytvořená aplikace splní stanovené požadavky, bude moci být cíl práce považován za splněný.

Jak bylo uvedeno výše, virtualizační technika Solaris Zones je exkluzivním produktem pro operační
systém Solaris. Tomuto faktu musí být přizpůsoben výběr technologií, které budou použity při implementaci výsledné aplikace.
Operační systém Solaris není standardní platformou, i když je v~dnešní době podporován na více platformách platformě. Hlavní důraz
musí být kladen na kompatibilitu programovacího jazyka a jeho knihoven s~operačním systémem Solaris. Z~výše uvedených důvodů
je možné vyvodit první požadavek na administrační nástroj, kterým je podpora na operačním systému \textbf{Solaris}.

Účelem nástroje má být podpora automatické správy neglobálních zón. Pod pojmem správa je myšlena podpora základních administračních
postupů a technik, které jsou z~velké části popsány v~kapitole \ref{chapter:zones}. Mezi tyto postupy patří 
vytváření neglobálních zón, ale také podpora jejich správy, zálohování nebo migrace. Automatickou správou je myšlena hlavně
automatizace procesů vytváření zóny, zálohy nebo migrace, které se skládají z~několika kroků. Aplikace by měla administrátorovi
poskytovat funkce, které umožní provedení výše zmíněných procesů pomocí jednoho příkazu. Požadavky na aplikaci vyplývající
z~účelu nástroje je možné specifikovat jako \textbf{podpora správy} Solaris Zones a~\textbf{automatizace procesů} administrace.

Virtualizační technika Solaris Zones poskytuje administrátorovi skrze příkazy \verb|zonecfg(1)| a \verb|zoneadm(1)| způsob,
jak spravovat lokální neglobální zóny. Zcela zde však chybí podpora pro správu zón na vzdálených serverech. V~dnešních infrastrukturách
počítačových systémů využívajících virtualizace se nachází mnoho serverů. Tyto servery poskytují své výpočetní prostředky
virtuálním strojům. Z~tohoto pohledu je tedy žádoucí, aby implementovaná aplikace umožňovala správu neglobálních zón, které
se nacházejí na~\textbf{vzdálených} serverech.

Následující požadavek se vztahuje k~automatizaci administračních procesů. Jelikož definice zóny se skládá z~její
konfigurace, softwarového vybavení a systémového nastavení, aplikace by měla umožňovat specifikaci této definice 
jednotným způsobem. Aplikace tedy musí poskytovat administrátorovi systém pro vytváření definic zón, které bude možné
používat pro jejich vytváření. Tento požadavek lze specifikovat jako podpora vytváření \textbf{šablon}.

Uživatel musí mít možnost ovládat nástroj pro podporu správy neglobálních zón. To znamená, že aplikace bude
poskytovat uživateli své funkce pomocí \textbf{uživatelského rozhraní}. Toto uživatelské rozhraní musí být přehledné a poskytovat
uživateli všechny informace potřebné pro využívání jeho funkcí. Pomocí tohoto rozhraní bude uživatel zadávat příkazy, které
aplikace bude vykonávat. Rozhraní by mělo nabízet izolovaný pohled pro každého uživatele, který bude aplikaci využívat.

Posledním požadavkem, který musí aplikace splňovat, je \textbf{bezpečnost}. Na~bezpečnost používání aplikace se musí dbát především
proto, že nesprávným a~neopatrným používáním virtualizační techniky Solaris Zones může dojít k~nestabilitě celého systému.
K~takovým případům dochází především ve chvílích, kdy neglobální zóny vyčerpají všechny fyzické prostředky systému a tím
znemožní správný běh globální zóny.

Kompletní požadavky na aplikaci pro podporu automatické správy Solaris Zones můžeme shrnout do následujících bodů:
\begin{itemize}
 \item operační systém Solaris,
 \item lokální a vzdálená správa,
 \item automatizace administračních procesů,
 \item šablony,
 \item uživatelské rozhraní,
 \item bezpečnost.
\end{itemize}
Splnění těchto požadavků by mělo vést k~značnému zjednodušení správy virtualizačního kontejneru Solaris Zones. Výsledná
aplikace by měla zajistit přehled o~neglobálních zónách, které se nacházejí na lokálním serveru a~vzdálených serverech. Aplikace
by také měla umožňovat správu těchto zón. 
\section{Architektura aplikace}
\label{chapter:design:architecture}
Prvním krokem v~návrhu aplikace je její architektura. Architektura aplikace popisuje její strukturu a určuje jakým způsobem
spolu jednotlivé funkční bloky budou komunikovat. Jelikož virtualizační technika Solaris Zones neposkytuje žádné aplikační rozhraní pro konkrétní
programovací jazyk, bude nutné postavit aplikaci nad nástroji \verb|zonecfg(1)| a \verb|zoneadm(1)|. Pokud uživatel bude chtít
provádět akci se zónou, aplikace sestaví z~těchto nástrojů potřebný příkaz nebo jejich sekvenci a vykoná je. Mimo výše
uvedených nástrojů je pro~efektivní administraci Solaris Zones nutné používat příkaz \verb|zfs(1)| umožňující práci se souborovým
systémem ZFS. Pro vytváření záloh pomocí techniky popsané v~kapitole \ref{chapter:zones:backup:uar} je nutné, aby aplikace uměla
používat příkaz \verb|archvieadm(1)|. Aplikace bude simulovat práci administrátora při vykonávání základních administračních
rutin tím, že bude vykonávat výše zmíněné příkazy na příkazové řádce.

Pro usnadnění vývoje bude aplikace rozdělena do funkčních bloků, které budou mít na starost konkrétní funkcionalitu aplikace. Prvním
funkčním blokem aplikace bude knihovna, která bude zprostředkovávat komunikaci mezi klientskou aplikací a hlavní částí aplikace.
Další částí aplikace bude modul, který se bude starat o~sestavování a provádění příkazů pro správu Solaris Zones.
Tato vrstva bude poskytovat základní funkce pro vytvoření, správu, zálohu, obnovu a migraci neglobálních zón. Poslední částí aplikace
bude klientská aplikace, která bude skrze knihovnu využívat funkce modulu. Na obrázku \ref{image:architecture} jsou znázorněny 
jednotlivé funkční bloky aplikace a jejich vzájemná interakce.
\begin{figure}
    \centering    
    \caption{Funkční bloky aplikace}
    \label{image:architecture}
\end{figure}
\subsection{Knihovna}
\label{chapter:design:architecture:library}
Knihovna je kontejner, který se bude starat o~zprostředkování komunikace mezi klientem a modulem knihovny. Součástí
knihovny budou jednotlivé moduly, které budou zajišťovat konkrétní funkcionalitu. V~tomto případě se bude jednat o~modul, který
bude implementovat základní funkce pro administraci Solaris Zones. Knihovna pak bude tyto administrační funkce poskytovat klientům.
Celá knihovna bude navržená tak, aby se dala jednoduše rozšířit o~další modul. Tento modul bude muset implementovat určité
rozhraní, pomocí kterého s~ním bude knihovna komunikovat. Díky této architektuře bude v~budoucnu jednoduché implementovat
další modul, který bude zprostředkovávat administraci jiného virtualizačního nástroje. Dalším kandidátem může být například
modul využívající rozhraní aplikace VirtualBox, která také nabízí rozhraní na příkazové řádce.

Mimo funkcí implementovaných v~modulech bude knihovna poskytovat i~některé vlastní funkce. Většina virtualizačních technik
má společnou jednu věc. Tím je specifikace virtuálního stroje, který chce uživatel ve virtualizovaném prostředí spouštět. Tato specifikace
by měla obsahovat základní vlastnost a prostředky, které bude moci daný virtuální stroj využívat. Proto bude knihovna implementovat
generickou šablonu, která bude sloužit pro specifikaci virtuálních strojů. V~podstatě se bude jednat o~hlavičku, která bude určovat
název a typ specifikace. Podle typu této specifikace pak bude knihovna přesměrovávat požadavky na konkrétní modul.

Architektura knihovny bude typu \textit{standalone} a nebude tedy poskytovat žádné rozhraní, které by bylo dostupné z~počítačové
sítě. Veškerou funkcionalitu knihovny bude možné využívat pouze ve stejném systému. Tento krok minimalizuje rizika spojená
s~napadením aplikace prostřednictvím počítačové sítě.
\subsection{Modul Solaris Zones}
\label{chapter:design:architecture:szones}
Jednou z~hlavních částí aplikace bude modul Solaris Zones, který bude součástí výše zmíněné knihovny. Tento modul se bude skládat
z~několika hierarchicky uspořádaných vrstev, které se budou navzájem využívat. Nejníže v~hierarchii se bude nacházet vrstva,
která bude poskytovat spouštění základních nástrojů sloužících pro správu Solaris Zones. Pro poskytnutí základní funkcionality
musí tato vrstva poskytovat následující nástroje:
\begin{itemize}
 \item \verb|zonecfg(1)|,
 \item \verb|zoneadm(1)|,
 \item \verb|zfs(1)|,
 \item \verb|archvieadm(1)|.
\end{itemize}
Tato vrstva bude tedy umožňovat vyšším vrstvám spouštět tyto nástroje s~různým nastavením a různými příkazy.

Vyšší vrstvy budou implementovat základní administrátorské rutiny, které budou složené z~funkcí nižších vrstev. Tento modul
bude zajišťovat smazání všech dočasných souborů, které byly v~průběhu rutiny vytvořeny. K~tomu bude také
zajišťovat konzistenci ve smyslu navrácení všech změn, které byly v~průběhu rutiny provedeny. Toto chování bude nastávat
pouze v~případě, kdy v~průběhu rutiny dojde k~chybě.

Jelikož bude tento modul součástí výše zmíněné knihovny, bude od něj očekávána implementace daného rozhraní. Mimo jiné 
bude muset modul implementovat funkce pro validaci a zpracování šablon, které budou sloužit pro~specifikaci vlastností
neglobálních zón.
\subsection{Klientská aplikace}
\label{chapter:design:architecture:client}
Poslední neméně důležitou částí nástroje pro správu virtualizačního kontejneru Solaris Zones bude klientská aplikace. Tento
funkční blok bude mít za úkol zprostředkovat uživateli funkce modulu Solaris Zones. Samostatná knihovna bude pouze prostředkem,
jakým způsobem přímo spravovat konkrétní zónu. Z~tohoto důvodu bude na klientské aplikaci, aby zařídila možnost správy většího
množství neglobálních zón.

Takto vylepšené možnosti správy Solaris Zones bude klientská aplikace nabízet uživateli pomocí uživatelského rozhraní. Toto
uživatelské rozhraní by mělo být jednoduché a přehledné, ale přitom nabízet nejdůležitější funkce pro správu zón. Klientská
aplikace by si měla udržovat seznam hostů, které chce daný uživatel spravovat. Na základě tohoto seznamu by měla například
zobrazovat všechny neglobální zóny, které se na těchto hostech nachází.

Jelikož aplikaci bude moc využívat více uživatelů, bude každému z~nich poskytovat nezávislý pohled. K~tomuto účelu bude aplikace
využívat uživatelův domovský adresář, kde si bude potřebné informace ukládat. Bude se jednat například o~zóny, se kterými
uživatel nějak manipuloval nebo je vytvářel. Na~základě těchto dat pak klientská aplikace může zobrazovat uživateli změny,
které nastaly v~době jeho nepřítomnosti.
\section{Uživatelské rozhraní}
\label{chapter:design:ui}
Požadavky v~kapitole \ref{chapter:design:demands} stanovují, že hlavním ovládacím prvkem implementované aplikace bude
uživatelské rozhraní. Uživatelské rozhraní je prvek, který uživateli umožňuje konkrétním způsobem interagovat s~danou aplikací.
Tento prvek je možné implementovat pomocí mnoha technologií. Ne všechny typy uživatelského rozhraní jsou však vhodné pro 
konkrétní typy aplikací.

Hlavní podstatou navrhovaného nástroje je podpora automatizace správy. Nástroj má uživateli umožnit zvládnout
hodně práce s~malým množství příkazů. Příkladem může být vytvoření desítek neglobálních zón pomocí jedné akce v~uživatelském 
rozhraní. Dalším požadavkem na uživatelské rozhraní může být i možnost uživatelské rozhraní skriptovat a využívat ho například
v~automatických zálohovacích rutinách. Uživatelské rozhraní tedy musí především umět pracovat v~dávkovém režimu. Musí však
umožňovat i interaktivní instalaci neglobálních zón.

V~dnešní době je pravděpodobně nejpopulárnější webové uživatelské rozhraní. Tento způsob definuje uživatelské rozhraní pomocí
HTML stránek a~pomocí webového serveru nebo Javascriptu, umožňuje reagovat na akce uživatele. Standardně je tento typ uživatelského
rozhraní zprostředkováván pomocí webového serveru bežícího na portech 80 v~případě HTTP nebo 443 v~případě HTTPS. Aplikace
využívající webové rozhraní jsou většinou typu klient-server, kde aplikace běží na serveru. Výhodou je, že klient se
k~serveru připojuje vzdáleně pomocí webového prohlížeče a danou aplikaci nemusí mít nainstalovanou. Nevýhodou je nutnost
uživatelské interakce a špatná možnost skriptování. Z~tohoto důvodu je tento typ uživatelského rozhraní nevhodný pro navrhovaný
nástroj.

Klientská aplikace bude jako uživatelské rozhraní využívat kombinaci CLI a grafického rozhraní. Situace a důvody pro použití
konkrétního typu uživatelského rozhraní jsou vysvětleny níže.
\subsection{CLI}
\label{chapter:design:ui:cli}
Pro umožnění jednoduchého skriptování aplikace bude většina její funkcionality prezentována pomocí rozhraní na příkazové
řádce. Celé rozhraní by mělo mít jednotný tvar a syntaxe jednotlivých příkazů by se neměla moc lišit. Uživatel si jednoduše
bude moci specifikovat cílové zóny a akci, kterou na nich chce provést. Program potom bez dalšího zásahu uživatele provede
danou akci a informuje ho o~jejím výsledku pomocí informačního výpisu.

Po zadání příkazu na příkazové řádce bude běh programu ve většině případů neinteraktivní a nebude tedy vyžadovat žádný 
zásah uživatele. Jedinou výjimku bude tvořit interaktivní instalace zóny, kdy bude použito grafického rozhraní. Po ukončení
běhu aplikace může uživatel opět zadávat další příkazy pomocí příkazové řádky.
\subsection{Grafické rozhraní}
\label{chapter:design:ui:gui}
Grafické rozhraní bude v~aplikaci použito ve dvou případech. Prvním případem je již zmiňovaná interaktivní instalace zón. Při
tomto typu instalace bude uživateli zobrazeno dialogové okno, které slouží jako formulář pro vyplnění atributů zóny popsaných
v~kapitole \ref{chapter:zones:configuration}. Po vyplnění formuláře bude aplikace pokračovat standardním neinteraktivním způsobem.

Výsledný nástroj bude využívat grafické rozhraní ještě v~okamžiku, kdy bude uživatel chtít vytvářet nebo upravovat šablony pro
specifikaci virtuálního stroje. Pro tento účel bude aplikace poskytovat editor, který uživateli pomůže s~vytvořením šablony.
Tento editor se bude opět spouštět pomocí příkazové řádky.

Při výběru grafického rozhraní bude nutné brát ohled na fakt, že aplikace je určená pro platformu Solaris. Jelikož Solaris
není standardní platformou, nemusí být všechny grafické knihovny na této platformě podporovány.
\section{Šablony}
\label{chapter:design:templates}
Podle požadavků specifikovaných v~kapitole \ref{chapter:design:demands}, má aplikace umožňovat vytváření šablon. Šablona slouží
jako předpis pro vytvoření virtuálního stroje. Knihovna popsaná v~kapitole\ref{chapter:design:architecture:library} bude definovat 
generickou šablonu, kterou bude aplikace umět zpracovávat. Knihovna bude muset umět načítat šablony ze souborového systému
a provádět základní validaci. Hlavním obsahem šablony bude její jméno a typ. Podle typu se pak knihovna rozhodne
jakému modulu šablonu předá na zpracování.
\subsection{Šablona Solaris Zones}
\label{chapter:design:templates:zones}
Ve výsledném nástroji bude obsažený pouze modul pro administraci virtualizační techniky Solaris Zones. Tento modul bude
definovat typ šablony, který bude specifikovat neglobální zónu. Opět bude poskytovat funkce pro její validaci, ale oproti
knihovně tuto šablonu bude umět využívat pro tvorbu virtuálních kontejnerů Solaris Zones.

Jak již bylo zmíněno, pro úspěšnou instalaci zóny je třeba provést konfiguraci, instalaci 
softwarových balíků a volitelně i konfiguraci systémových služeb. Aby mohla být zóna ze šablony vytvořena, musí šablona obsahovat
právě tyto části.
\subsubsection{Konfigurace zóny}
\label{chapter:design:templates:zones:configuration}
První část šablony bude obsahovat informace o~konfiguraci zóny. Bude zde tedy specifikováno o~jaký typ zóny se jedná,
jaký typ IP adresy má mít, jaké zdroje mají být zóně delegovány a podobně. Obecně lze říct, že v~této části budou specifikovány
globální atributy zóny popsané v~kapitole \ref{chapter:zones:configuration:global_attributes} a zdroje zóny popsané v~kapitole
\ref{chapter:zones:configuration:resources}. Šablona nebude umožňovat specifikovat všechny atributy a zdroje popsané
v~manuálových stránkách \cite{oracle:manpages:zonecfg}, ale jejich podmnožinu, která je využitelná ve většině případů použití virtualizační
techniky Solaris Zones.
\subsubsection{Softwarové balíky}
\label{chapter:design:templates:zones:manifest}
Další podstatnou částí šablony bude sekce se softwarovými balíky. Balíky, které zde uživatel definuje, budou při instalaci
zóny z~této šablony nainstalovány do kořenového souborového systému zóny. Ihned po prvním startu zóny je uživatel bude moci
využívat. Tato část šablony umožňuje uživateli jasně definovat softwarové vybavení a umožňuje tak vytváření konzistentního
prostředí.

Přehled softwarových balíků dostupných pro konkrétní verzi operačního systému Solaris může uživatel získat pomocí nástroje
pro správu softwarových balíků \verb|pkg(1)| nebo z~oficiálního repozitáře \cite{oracle:solaris:desing:pkg_repository}.
\subsubsection{Systémový profil}
\label{chapter:design:templates:zones:profile}
Poslední sekcí šablony pro definici neglobální zóny je konfigurace systémového profilu. V~této části šablony může uživatel
definovat konfiguraci pro základní systémové služby. Touto cestou může uživatel například nastavovat konfiguraci síťových
adaptérů definovaných v~sekci s~konfigurací. Uživatel může také nastavit časovou zónu, jazyk systému nebo uživatelské účty.
Tato sekce nebude opět umožňovat konfiguraci všech systémových služeb, ale jejich část nutnou ke správné a funkční konfiguraci
systému.

Minimální konfigurace obsahuje definici hesla uživatele \textit{root} a definici počátečního systémového uživatele. Pokud uživatel
nespecifikuje konfiguraci systémových služeb v~šabloně, nebude instalovaná zóna vůbec nakonfigurována. Při prvním spuštění
zóny bude uživatel vyzván k~interaktivní konfiguraci systému.
\section{Automatizace}
\label{chapter:design:automation}
Automatizaci administračních procesů bude aplikace zajišťovat na úrovni modulu, který slouží pro správu Solaris Zones. Pokročilejší
administrační rutiny se skládají ze sekvence několika příkazů, které musí být provedeny po sobě. V~případě selhání, některého
z~nich dojde k~selhání celé rutiny. Často je potřeba při provádění těchto činností vytvořit dočasné soubory nebo dočasně 
vypnout konkrétní zónu. Tyto situace nastávají hlavně v~zálohovacích a migračních rutinách. Modul Solaris Zones bude implementovat
tyto pokročilejší rutiny a poskytovat je skrze knihovnu klientským aplikacím. Dále bude zajišťovat, že všechny dočasné soubory
budou na konci rutiny odstraněny a dočasné akce vráceny v~případě neúspěchu. Toto chování se dá přirovnat k~chování transakcí.

Dále bude příkazová řádka aplikace umožňovat zadávání většího počtu neglobálních zón, pro které se má daný příkaz vykonat. Aplikace
potom paralelně (pokud je to možné) vykoná daný příkaz pro všechny specifikované zóny.
\section{Vzdálená správa}
\label{chapter:design:remote}
Jelikož nástroje pro správu virtualizační techniky Solaris Zones neumožňují správu neglobálních zón na vzdálených serverech,
bude modul zmíněný v~kapitole \ref{chapter:design:architecture:szones} implementovat i funkce pro vzdálenou správu. Ideálním
nástrojem pro ovládání vzdálených serverů je program \verb|ssh(1)|. Tento nástroj umožňuje používat příkazovou řádku na 
vzdáleném serveru a s~použitím veřejného a privátního klíče umožňuje i neinteraktivní přihlášení. Tyto dvě vlastnosti jsou
pro požadovanou funkcionalitu nástroje pro automatickou správu Solaris Zones klíčové.

Prostředí, do kterého je navrhovaná aplikace směřována, se skládá z~několika virtualizačních serverů, které používají operační
systém Solaris. V~rámci těchto serverů je provozována virtualizační technologie Solaris Zones a běží na nich mnoho neglobálních
zón. Klientská aplikace zmíněná v~kapitole \ref{chapter:design:architecture:client} musí implementovat způsob identifikace
těchto zón v~rámci většího počtu serverů. Pomocí tohoto identifikátoru bude uživatel schopný danou zónu specifikovat na příkazové
řádce a provádět s~ní konkrétní akce.

Aplikace bude umožňovat registraci jednotlivých hostů v~rámci dané infrastruktury. Ke každému hostu si aplikace
bude držet přístupové údaje, které má použít při připojování. Tyto údaje budou zahrnovat především uživatelské jméno a 
privátní klíč, který má být použit pro šifrování spojení. Hromadné akce nabízené uživatelským rozhraním se pak budou vztahovat
právě k~registrovaným hostů.
\section{Bezpečnost}
\label{chapter:design:security}
Důležitou součástí návrhu aplikace je její bezpečnost. V~případě nástroje pro~automatickou správu Solaris Zones je nutné věnovat
zabezpečení velkou pozornost. Nástroje specifikované v~kapitole \ref{chapter:design:architecture:szones} totiž při nesprávném
použití mohou způsobit pád systému. V~případě příkazu \verb|zfs(1)| je možné kompletně zničit souborový systém všech neglobálních zón
a způsobit tak chybu systému. Naopak pomocí příkazů \verb|zoneadm(1)| je možné vytvořit takové množství neglobálních zón, že
dojde k~vyčerpání fyzických prostředků globální zóny a následné nefunkčnosti celého systému. Autoři těchto nástrojů na tento
problém mysleli, a proto jsou všechny tyto příkazy přístupné pouze privilegovaným uživatelům. Pro aplikaci to znamená, že ji 
bude moci spouštět pouze uživatel s~konkrétními právy. Operační systém Solaris poskytuje službu RBAC,
která administrátorovi umožňuje jemněji rozdělit práva mezi uživatele \cite{oracle:solaris:desing:rbac}. Pomocí této služby
je možné vytvořit uživatele, který bude
přímo určený pro~správu zón a bude moci používat všechny nástroje stanovené v~\ref{chapter:design:architecture:szones}. Pro~spouštění
aplikace může být použit jeden z~následujících dvou uživatelů:
\begin{itemize}
 \item Uživatel \textbf{root},
 \item Privilegovaný uživatel (\textbf{RBAC}).
\end{itemize}
V~následujících kapitolách jsou popsány důvody, proč není vhodné pro spouštění aplikace používat uživatele \textit{root} a 
jaké výhody přináší RBAC.
\subsection{Uživatel root}
\label{chapter:design:security:root}
Uživatel \textit{root} je plně privilegovaný uživatel v~rámci operačního sytému Solaris a má potřebná práva pro spouštění všech
potřebných nástrojů. Existuje však několik důvodů, proč není vhodné uživatele \textit{root} používat pro spouštění navrhovaného
nástroje pro automatickou správu Solaris Zones. Jedním z~důvodů je ctění principu nejnižších privilegií 
\cite{cvut:presentations:least_user_privilege}. Tento princip říká, že aplikace má být spuštěna pouze s~nejmenší možnou množinou
práv, se kterými je ještě schopná plnit svůj účel. Pokud by byla aplikace nějakým způsobem zkompromitována, útočník může využít
pouze těchto práv. Pokud by nebyl dodržen princip nejnižších oprávnění a aplikace by byla spouštěna pod uživatelem \textit{root},
útočník by mohl využívat privilegovaného přístupu v~celém systému.

Dalším důvodem proč nepoužívat uživatele \textit{root} je standardní systémové nastavení operačního systému Solaris. Standardně
je totiž uživatel \textit{root} v~systému zaregistrovaný jako role. Role je funkcionalita RBAC \cite{oracle:solaris:desing:rbac},
která se chová téměř jako uživatel. Je možné ji přiřazovat práva na provádění privilegovaných operací nebo se na ní přepínat pomocí
nástroje \verb|su(1)|. Uživatel může mít v~systému přiřazeny role, které může používat. Samotná role však nemá v~systému žádnou
funkci a nedá se na ni dokonce ani přihlásit. Z~tohoto důvodu by nebylo možné přihlašovat se na vzdálených systémech jako uživatel
\textit{root} a~aplikace by nesplňovala požadavek vzdálené správy.

Poslední důvod souvisí s~předchozím důvodem a opět se týká standardního nastavení. Tentokrát se však týká standardního nastavení nástroje
\verb|ssh(1)|, které nepovoluje vzdálené přihlašování uživatele \textit{root}.

V~důsledku používání uživatele \textit{root} by došlo k~porušení principu nejnižších privilegií 
\cite{cvut:presentations:least_user_privilege} a navíc by muselo dojít k~vypnutí některých standardních bezpečnostních opatření.
Z~výše uvedených důvodů není vhodné tohoto uživatele používat pro spouštění navrhovaného nástroje pro automatickou správu Solaris Zones.
\subsection{RBAC}
\label{chapter:design:security:rbac}
Správnou volbou je vytvořit uživatele, kterému pomocí RBAC přiřadíme potřebná oprávnění pro používání potřebných nástrojů.
Aplikace bude spouštěna v~souladu s~principem nejnižších oprávnění a nebude nutné vypínat bezpečnostní
opatření nástroje \verb|ssh(1)| ani jiné systémové nastavení.
  
\chapter{Implementace}
  \label{chapter:implementation}
Následující část diplomové práce představuje implementaci nástroje pro automatickou správu Solaris Zones. V~úvodu
kapitoly je představen použitý programovací jazyk a~důvody pro jeho použití. Hlavní částí
je popis knihovny, modulu Solaris Zones a~klientské aplikace. Důraz je kladen na popis funkcionality jednotlivých 
částí aplikace a~jejich vzájemné komunikace.
\section{Programovací jazyk}
\label{chapter:implementation:language}
Prvním krokem  při implementaci bylo zvolení vhodného programovacího jazyka. Požadavky stanovené v~kapitole 
\ref{chapter:design:demands} vyžadují od zvoleného programovacího jazyka následující dvě podmínky:
\begin{itemize}
 \item Operační systém Solaris,
 \item Možnost tvorby grafického rozhraní.
\end{itemize}
Nástroj pro automatickou správu virtualizačního kontejneru Solaris Zones využívá
nástrojů na příkazové řádce a~zpracovává jejich výstup. Pro tento účel bylo vhodné zvolit interpretovaný
programovací jazyk, který umožnil jednoduše spustit nástroje a~následně snadno zpracovat jejich výstup. Na základě
výstupu se pak nástroj rozhodne o~dalším průběhu zpracování uživatelského příkazu. První podmínka není pro volbu jazyka
tolik omezující. Pro operační systém Solaris existuje implementace standardního kompilátoru \verb|gcc(1)|
pro jazyk C a~stejně tak implementace virtuálního stroje JVM pro jazyk Java. Většina interpretovaných jazyků
staví svůj překladač právě nad jedním z~těchto základních programovacích jazyků.

Z~výše uvedených důvodů bylo nutné při volbě jazyka dbát hlavně na~dostupnost grafických knihoven pro operační systém
Solaris. \verb|Shell| je standardním skriptovacím jazykem pro většinu operačních systému typu UNIX. Tento program
interpretuje uživatelské příkazy na příkazové řádce a~následně je provádí. Tato volba by splňovala podmínku platformy,
ale těžko by se s~pomocí tohoto jazyka vytvářelo grafické uživatelské rozhraní. Z~tohoto důvodu byl zvolen programovací
jazyk Ruby \cite{ruby}, který umožnil splnit obě stanovené podmínky.
\subsection{Ruby}
\label{chapter:implementation:language:ruby}
Ruby je objektově orientovaný programovací jazyk, který má mnoho možností využití. Jedním ze scénářů využití může být
právě spouštění příkazů na~příkazové řádce a~tvorba uživatelského rozhraní. Objektová povaha tohoto jazyka umožňuje
programátorovi využívat všech výhod objektově orientovaného programování. Podle dokumentu \cite{ruby:implementation}
existuje několik implementací interpretu jazyka Ruby, z~nichž nejpoužívanější jsou YARV \cite{ruby:implementation:yarv} 
a~JRuby \cite{ruby:implementation:jruby}. Obě tyto implementace jsou dostupné i pro operační systém Solaris.

Pokud chce programátor využívat grafické rozhraní pomocí programovacího jazyka Ruby, je nutné, aby byly v~systému
nainstalované potřebné grafické knihovny. Standardní knihovny pro programovací jazyk Ruby však nejsou na~operačním
systému Solaris podporované. Z~toho důvodu bylo nutné využít grafické rozhraní, které nabízí implementace JRuby.
Tato implementace je postavená nad virtuálním strojem JVM a~může využívat grafické knihovny v~něm implementované.
Navrhovaný nástroj pro automatickou správu virtualizačního kontejneru Solaris Zones tedy využívá programovacího
jazyka Ruby. Pokud bude chtít uživatel nástroje využívat grafického rozhraní, musí nástroj spouštět pomocí
interpretu JRuby. Zbytek nástroje je nezávislý na použitém interpretu programovacího jazyka Ruby.
\section{Knihovna}
\label{chapter:implementation:library}
Hlavním centrálním prvkem implementace je knihovna, která zprostředkovává komunikaci mezi implementovanými moduly
a~klientskými aplikacemi. Knihovna je navržená tak, aby se v~budoucnosti dala lehce rozšířit o~další moduly, které budou
poskytovat funkce pro správu jiných virtualizačních technologií. Jedním z~takových rozšíření by mohl být například modul
pro podporu virtualizační technologie Oracle VirtualBox. Výsledná implementace obsahuje pouze modul pro podporu automatické
správy virtualizačního kontejneru Solaris Zones, který bude popsán v~kapitole \ref{chapter:implementation:szones}.

Knihovna poskytuje hlavní rozhraní, pomocí kterého může klient využívat funkcí jednotlivých modulů. Jednotlivé
moduly tedy slouží jako hlavní zdroj funkcionality pro knihovnu.

Mimo zprostředkovávání komunikace mezi moduly a~klientem slouží kni\-ho\-vna k~validaci šablon, které mají specifikovat
konkrétní virtuální stroj. V~případě šablon knihovna funguje jako vstupní bod, který umí šablonu načíst a~provést
prvotní validaci. Spouštění těchto operací a~jejich výsledky knihovna zprostředkovává klientovi.

Jelikož moduly mohou implementovat různé typy operací pomocí různých technologií, je nutné ponechat vývojářům velkou volnost
v~možnostech jejich implementace. Pro účel zajištění jednotné komunikace s~moduly je nutné, aby každý implementovaný modul
splňoval určité rozhraní. Toto rozhraní zajistí, aby všechny moduly mohly jednotně komunikovat s~knihovnou a~také, aby knihovna
mohla zprostředkovávat jejich funkce klientovi.

Poslední funkcí knihovny je udržování hlavní konfigurace. V~této konfiguraci je například uchováván seznam implementovaných
modulů, kořenový adresář knihovny nebo například jméno knihovny. Klientská aplikace má možnost tuto konfiguraci změnit
a~docílit tak jiného chování knihovny.
\subsection{Rozhraní modulu}
\label{chapter:implementation:library:interface}
Povinné rozhraní modulu slouží především ke komunikaci mezi knihovnou a~samotným modulem. Funkcionalitu, kterou modul
musí poskytovat, je možné shrnout do následujících bodů:
\begin{itemize}
 \item Inicializační rutina,
 \item Rozhraní poskytované klientům,
 \item Funkce pro validaci šablon.
\end{itemize}
Prvním požadavkem na rozhraní modulu je existence inicializační rutiny. Pomocí této rutiny je do modulu předána hlavní
konfigurace knihovny, která umožňuje modulu zjistit kořenový adresář aplikace a~další parametry. Hlavním smyslem
této rutiny je inicializace daného modulu. Hlavní knihovna v~rámci inicializační smyčky spustí tuto rutinu pro každý
registrovaný modul. Uvnitř této rutiny může modul provádět inicializaci vlastních datových struktur nebo vytvoření potřebné
adresářové struktury. Dále může modul využít hlavní konfiguraci knihovny k~doplnění vlastní lokální konfigurace. Tímto způsobem
je zajištěno, že všechny registrované moduly knihovny obdrží globální konfiguraci a~dojde k~jejich inicializaci.

Další nutnou částí rozhraní modulu jsou funkce, které mají být poskytovány klientovi. K~tomuto účelu musí modul poskytovat
třídu, která bude tyto funkce implementovat nebo je bude pouze zprostředkovávat pomocí jiných tříd modulu. Tato třída je tedy
hlavním funkčním rozhraním modulu, které klientské aplikace mohou využívat. V~rámci inicializace celé knihovny dojde nejprve
k~inicializaci jednotlivých modulů. Po této akci knihovna provede registraci těchto tříd a~v~udržuje si jejich seznam.

Posledním požadavkem na rozhraní modulu je existence funkcí pro validaci šablon. Tyto funkce musí umožňovat validovat šablony,
které se týkají konkrétního modulu knihovny. Modul, který podporuje správu virtualizačního kontejneru Solaris Zones, musí poskytovat
funkce pro validaci šablon specifikující neglobální zóny.

Vlastní implementace modulu není nijak jinak omezena. Jediným logickým omezením je fakt, že tento modul musí být napsaný
v~programovacím jazyku Ruby. Pokud modul splní výše zmíněné požadavky, může být jednoduše registrován do knihovny a~klientské
aplikace ho můžou bezprostředně po~inicializaci knihovny využívat. Na obrázku \ref{figure:module:interface} je názorně zobrazeno, 
jakým způsobem knihovna využívá rozhraní modulu a~jakým způsobem je modul poskytován klientské aplikaci.
\begin{figure}
    \centering    
    \caption{Rozhraní modulu}
    \label{figure:module:interface}
\end{figure}
\subsection{Přesměrování požadavků}
\label{chapter:implementation:library:routing}
Mimo inicializace modulů je hlavní funkcí knihovny přesměrovávat požadavky klientských aplikací na funkční rozhraní 
implementovaných modulů. K~tomuto účelu obsahuje knihovna hlavní třídu, která virtuálně reprezentuje rozhraní všech modulů
knihovny. Tato třída se nazývá hlavní rozhraní. Jak již bylo zmíněno,
knihovna si udržuje odkazy na hlavní třídy modulů, které reprezentují jejich funkční rozhraní.

V~okamžiku, kdy klientská aplikace vznese požadavek na zavolání konkrétní funkce, knihovna za běhu zjistí jakému modulu daná
funkce přísluší a~vyvolá ji. Pokud neexistuje žádný modul, který umí danou funkci provést, dojde k~vyvolání výjimky a~aplikace
se ukončí. Díky tomuto chování může dojít ke kolizi jmen funkcí. V~takovém případě by knihovna použila takovou funkci, kterou
by našla jako první v~pořadí. Z~tohoto důvodu je nutné se vyvarovat opakování jmen funkcí a~nejlépe používat pro funkce určitého
modulu prefix, který daný modul jasně identifikuje, například jeho jméno.

Toto směrování za běhu aplikace je umožněno díky programovacímu jazyku Ruby a~jeho možnosti dynamického volání funkcí za běhu
programu. Směrování požadavků ke konkrétním modulům knihovny je demonstrováno na obrázku \ref{figure:module:interface}.
\subsection{Generická šablona}
\label{chapter:implementation:library:generic}
Poslední funkcionalitou knihovny je definice generické šablony, která má za~úkol specifikovat virtuální stroj. Hlavním úkolem
generické šablony je specifikovat typ virtuálního stroje. Tento atribut určuje, který modul knihovny je zodpovědný
za zpracování a~validaci. Generická šablona by dále měla obsahovat jméno, které bude nějakým způsobem vystihovat a~popisovat
specifikovaný virtuální stroj. Knihovna tedy zajišťuje validaci těchto dvou atributů a~v~případě úspěchu předá
šablonu zodpovědnému modulu.

Knihovna poskytuje klientským aplikacím funkce pro načítání a~validaci šablon. V~případě úspěšného načtení šablony
knihovna vrátí objekt, který je možné použít v~rámci konkrétního modulu. Validace šablony je prováděna ve~dvou
krocích. Knihovna nejprve zjistí, zdali daná šablona obsahuje atribut jména a~typu. Podle typu šablony se knihovna rozhodne
jakému modulu ji předá na druhý krok validace.
\subsubsection{Struktura šablony}
\label{chapter:implementation:library:generic:structure}
Aby mohla být šablona opakovaně používána pro tvorbu virtuálních strojů, musí být perzistentně uložena v~souborovém systému.
Pro tento účel je použitý datový formát JSON \cite{json}, který slouží pro reprezentaci šablony. Hlavním důvodem využití tohoto
formátu je relativně dobrá uživatelská čitelnost a~především snadné zpracování pomocí programovacího jazyka Ruby. Uživatel
může pro konstrukci šablony použít jednoduchý textový editor nebo grafický editor, který je součástí uživatelského rozhraní
klientské aplikace. 

Struktura šablony se skládá ze dvou částí. První částí je název a~definice typu šablony. Tato hlavička určuje
způsob zacházení s~danou šablonou. V~ukázce kódu \ref{code:generic_template} je naznačeno, jakým způsobem by mohla taková šablona vypadat.
Atribut \textit{type} určuje o~jaký typ virtuálního stroje se jedná a~k~jakému modulu knihovny přísluší. Druhou povinnou položkou
v~šabloně je atribut \textit{name}, který má za~úkol popsat funkcionalitu virtuálního stroje. Tečky v~ukázce
\ref{code:generic_template} reprezentují atributy specifické pro konkrétní typ šablony. Tyto atributy jsou z~ukázky vynechány.
\begin{lstlisting}[language=json, caption={Generická šablona}, float,label={code:generic_template}]  
{
  "name": "template_webserver",
  "type": "szones",
  ...
}
\end{lstlisting}
\subsubsection{Validace šablony}
\label{chapter:implementation:library:generic:validation}
Šablona musí poskytovat validní definici virtuálního stroje, aby z~ní bylo možné konkrétní virtuální stroj zkonstruovat.
Pro tento účel je nutné zavést validaci šablon a~jejich atributů. Jelikož je pro ukládání šablon použit datový formát JSON,
je pro validaci šablony použité tvz. JSON schéma definované ve~specifikaci \cite{json:schema}. Tento dokument je opět ve formátu
JSON, ale neslouží pro ukládání dat. Jeho funkcí je definovat formát jiného dokumentu JSON. Pomocí tohoto schématu
je možné specifikovat atributy a~typy jejich hodnot, které má konkrétní typ dokumentu obsahovat.

Tento nástroj umožňuje definovat, jaké atributy může konkrétní typ virtuálního stroje mít. Pokud uživatel sestrojí
nevalidní šablonu virtuálního stroje, knihovna skrze validaci JSON dokumentu pozná, že se jedná o~neplatnou konfiguraci. Knihovna
implementuje základní schéma, které slouží pro základní validaci šablon. Jak je vidět v~ukázce \ref{code:generic_tmplate:validation},
toto schéma vyžaduje, aby v~dokumentu byly přítomné atributy \textit{name} a~\textit{type}. Moduly aplikace musí implementovat
podrobnější schéma, které má definovat konkrétní typ virtuálního stroje.

Aby bylo možné v~programovacím jazyku Ruby validovat JSON dokumenty, je nutné použít knihovnu, která bude implementovat
JSON schéma. Pro tento účel aplikace využívá volně dostupné řešení \textit{json-schema} \cite{json:schema:ruby}, které implementuje
funkce validace dokumentů typu JSON pomocí schémat.
\begin{lstlisting}[language=json, caption={Schéma generické šablony}, float, label={code:generic_tmplate:validation}]  
{
  {
  "title": "general-vm-template",
  "description": "Used for general template distinction",
  "type": "object",
  "properties": {
    "name": {
      "type": "string",
      "description": "Name of the vm template"
    },
    "type": {
      "enum": [ "szones", "vbox" ],
      "description": "Type of the vm template"
    }

  },
  "required": [ "name", "type" ],
  "additionalProperties": true
}
\end{lstlisting}
\section{Modul Solaris Zones}
\label{chapter:implementation:szones}
Modul Solaris Zones je hlavním stavebním kamenem celé implementace výsledného nástroje. Tento modul je zařazen do knihovny
popsané v~kapitole \ref{chapter:implementation:library} a~mimo jiné poskytuje základní rutiny pro správu virtualizačního
kontejneru Solaris Zones. Aby mohl být tento modul využíván knihovnou, musí implementovat rozhraní definované v~kapitole \ref{chapter:implementation:library:interface}.
Takto podmínka zahrnuje především implementaci tříd pro zpracovávání šablon virtuálních strojů. V~rámci tohoto modulu
se jedná o~šablony, které specifikují vlastnosti neglobálních zón. 

Funkcionalita modulu je rozdělena do několika vrstev, které se vzájemně využívají. Názvy jednotlivých vrstev jsou následující:
\begin{itemize}
 \item Management šablon,
 \item Nástroje pro správu Solaris Zones,
 \item Administrátorské rutiny,
 \item Funkční rozhraní modulu.
\end{itemize}
Architekturu vrstev modulu je možné pozorovat na obrázku \ref{image:implemetation:szones}. V~následujících kapitolách
bude podrobně popsána funkce jednotlivých vrstev a~jejich vzájemná interakce.
\begin{figure}
    \centering    
    \caption{Modul Solaris Zones}
    \label{image:implemetation:szones}
\end{figure}
\subsection{Šablona Solaris Zones}
\label{chapter:implementation:szones:template}
Nástroje pro správu Solaris Zones neposkytují možnost vytvářet zóny pomocí jednoho předpisu. Jak bylo popsáno v~kapitole
\ref{chapter:zones}, k~úspěšnému vytvoření neglobální zóny se využívají tři soubory. Prvním a~povinným parametrem instalace zóny
je její konfigurace. Není podmínkou, aby konfigurace zóny byla v~podobě souboru, ale pro automatizaci tohoto procesu je to výhodné.
Dále je nutné instalátoru předat definici softwarových balíků, které má nainstalovat. Posledním nepovinným parametrem instalace
je konfigurace systémových služeb. Aby bylo možné nainstalovat zónu pomocí jednoho souboru, musí tato šablona kombinovat
vlastnosti výše zmíněných souborů. 

Kostra šablony pro neglobální zónu je demonstrována v~ukázce kódu \ref{code:szones_template}. Z~ukázky je patrné, že šablona obsahuje
tři sekce, které korespondují s~jednotlivými soubory vyžadovanými při instalaci. Z~ukázky jsou vynechány konkrétní atributy zón.
Modul implementuje JSON schéma, které využívá pro validaci šablony a~pro mechanizmy jejího zpracování. Úkolem modulu při zpracování
šablony je reprodukce souborů s~konfigurací, manifestem a~systémovým profilem. Tyto soubory nástroj využívá pro instalaci 
neglobální zóny s~parametry, které jsou v~dané šabloně specifikovány.
\begin{lstlisting}[language=json, caption={Kostra šablony neglobální zóny}, float,label={code:szones_template}]  
{  
  "name": "template_webserver",
  "type": "szones",
  "configuration": {...},
  "manifest": {
    "packages": [...]
   },
   "profile": {...} 
}
\end{lstlisting}
Pokud modul zpracovává rutinu, která využívá šablonu, je v~prvním kroku šablona rozdělena do třech zmíněných částí.
V~následujícím kroku jsou tyto části převedeny do vnitřní reprezentace a~následně zpracovány.
\subsubsection{Zpracování konfigurace}
\label{chapter:implementation:szones:template:configuration}
Konfigurační sekce šablony se skládá z~definice globálních atributů zóny popsaných v~kapitole \ref{chapter:zones:configuration:global_attributes}
a~z~definice zdrojů zóny, které jsou popsané v~kapitole \ref{chapter:zones:configuration:resources}. Jednoduché globální atributy
jsou v~šabloně specifikovány přímo pomocí jejich jména a~hodnoty. Pro globální atribut typu zóny může definice vypadat 
následovně \lstinline[language=json]{"brand" : "solaris"}.

Dále tato sekce šablony obsahuje definici zdrojů zóny, které mají složitější strukturu a~obsahují několik atributů. Z~tohoto
důvodu šablona obsahuje speciální atribut \lstinline[language=json]{resources}, který je typu pole a~obsahuje definici
všech zdrojů zóny. V~rámci jednotlivého zdroje je použita stejná technika definice atributu jako ve výše zmíněném případě
globálních atributů.

Cílem zpracování této části šablony je vygenerovat soubor s~konfigurací, který má přesně definovanou syntaxi. Pomocí načtené
šablony uchované v~asociativním poli jsou globální atributy přetransformované do podoby, kterou vyžaduje nástroj \verb|zonecfg(1)|.
V~případě zdrojů je proveden stejný postup s~tím rozdílem, že se před každý zdroj přidá příkaz \verb|add| a~jeho definice
se ukončí příkazem \verb|end|. Takto přetransformovaná konfigurace je připravena k~použití v~nástroji \verb|zonecfg(1)|.
\subsubsection{Zpracování manifestu}
\label{chapter:implementation:szones:template:manifest}
V~rámci sekce šablony s~názvem \lstinline[language=json]{manifest} může uživatel specifikovat softwarové balíky, které má zóna obsahovat.
Pro tento účel obsahuje tato část šablony atribut \lstinline[language=json]{packages}, který je typu pole. Hodnotou každého
prvku pole je obyčejný textový řetězec, který obsahuje jméno softwarového balíku. V~tomto poli může uživatel specifikovat libovolné
množství balíků. 

Cílem zpracování této sekce šablony je vytvořit manifest popsaný v~kapitole \ref{chapter:zones:instalation:repozitory:manifest}.
K~tomuto účelu si modul drží kopii tohoto souboru, která nese název \textit{manifest\_template.xml} a~nachází se v~kořenovém 
adresáři aplikace ve~složce \textit{szones/manifest}. Při zpracovávání manifestu jsou jednotlivé balíky načteny ze šablony
a~ve správném formátu vloženy do kopie tohoto souboru. Výsledný soubor je možné použít pro instalaci zóny.
\subsubsection{Zpracování systémového profilu}
\label{chapter:implementation:szones:template:profile}
Poslední částí zpracovávání šablony je transformace systémového profilu. Tato sekce slouží k~nastavení systémových služeb
neglobální zóny a~má stejnou strukturu jako sekce s~konfigurací zóny. Na rozdíl od způsobu zpracování konfigurační sekce je
však v~tomto případě vyžadován jiný výstup. Cílem této transformace má být soubor v~XML formátu popsaný v~kapitole \ref{chapter:zones:instalation:profile}.

Pro každou službu existuje korespondující soubor obsahující potřebnou část výsledného XML souboru. Tyto soubory jsou
uloženy v~kořenovém adresáři aplikace ve složce \textit{szones/profile}. V~průběhu zpracovávání šablony jsou jednotlivé soubory
načítány a~vyplňovány hodnotami ze šablony. Tímto způsobem se zkonstruuje celý soubor, který může být předán instalátoru.
\subsection{Nástroje}
\label{chapter:implementation:szones:commands}
Základním stavebním kamenem modulu je vrstva, která zajišťuje vykonávání potřebných příkazů na příkazové řádce.
Tato vrstva poskytuje vyšším vrstvám modulu možnost vykonávání základní administračních příkazů pro správu Solaris Zones
a~souborového systému ZFS.

Pro účel vykonávání příkazů tato vrstva implementuje třídu, která umožňuje provádění příkazů jak na lokálním tak i
na vzdáleném serveru. Tato třída nevykonává daný příkaz okamžitě, ale umožňuje vyšším vrstvám aplikace vykonání odložit.
Tento požadavek je na třídu kladen zejména z~důvodu efektivity využívání vytvořených SSH spojení. Dále třída umožňuje
definovat automatické chování v~případě chyby prováděného příkazu. Pro každý nástroj využívaný aplikací je definována sada
pravidel, které jsou uplatňovány na chybové výstupy nástrojů a~následně jsou vyvolávány příslušné výjimky. Vyšší vrstvy
aplikace musí tyto výjimky odchytávat a~adekvátně na ně reagovat.

Pomocí výše zmíněné třídy tato vrstva modulu umožňuje volat jednotlivé nástroje s~požadovanými parametry a~argumenty.
Například umožňuje spouštět nástroj \verb|zonecfg(1)| se jménem konfigurované zóny a~cestou k~souboru s~konfigurací. Vyšším
vrstvám je poskytnutý standardní výstup a~standardní chybový výstup daného nástroje.
\subsection{Administrátorské rutiny}
\label{chapter:implementation:szones:routines}
Hlavní částí modulu je vrstva tříd, které implementují požadované rutiny pro~správu Solaris Zones. Rutiny využívají
nástrojů implementovaných v~nižší vrstvě a~pomocí nich vytvářejí sekvence příkazů nutné k~vykonání požadované administrátorské činnosti.
Tato vrstva poskytuje pokročilejší rutiny pro~vytváření neglobálních zón, zálohu, obnovu, ovládání a~migraci.
\subsubsection{Transakce}
\label{chapter:implementation:szones:routines:transaction}
Jednotlivé rutiny jsou implementované jako transakce. Transakce se skládá z~jednotlivých příkazů, které tvoří
celek. V~případě úspěchu všech příkazů v~transakci je celá transakce označena za úspěšnou. Jestliže jakýkoli příkaz v~průběhu
transakce selže, selže i celá transakce.

V~průběhu některých rutin dochází k~vytváření dočasných souborů. Tyto soubory slouží například pro přenos konfigurace zóny
na vzdálený server nebo pro dočasné uložení diskového obrazu zóny. Soubory, které jsou dočasně vytvořené v~průběhu transakce, jsou
zaznamenávány a~na konci transakce dojde k~jejich smazaní. Status transakce nemá na vykonání tohoto procesu vliv.
Dále některé transakce vytvářejí nové konfigurace zón, snapshoty konkrétních souborových systémů nebo celé diskové obrazy zón. 
Některé z~těchto entit mají být výstupem transakce. Příkladem může být rutina pro vytváření zóny. V~tomto případě 
má být konfigurace zóny a~její diskový obraz výsledkem transakce. Tento případ je nutné rozlišit a~tyto entity smazat pouze 
v~případě neúspěchu transakce. Tuto funkcionalitu v~rámci modulu zajišťuje třída \verb|Cleanuper|.

Během zálohovacích a~migračních rutin je třeba provádět akce, které konkrétním způsobem ovlivní existující zóny. Tyto akce se opět
zaznamenávají a~v~případě neúspěchu transakce se ovlivněné zóny vrátí do původního stavu. Příkladem může být zastavení běžící
zóny z~důvodu její zálohy nebo migrace. Tuto funkcionalitu v~rámci knihovního modulu zajišťuje třída \verb|Rollbacker|.

Rutiny modulu tedy zachovávají stav existujících zón v~případě neúspěchu transakce a~zajišťují tak konzistentní stav systému. Stejně
tak zajišťují, že všechny dočasně vytvořené entity budou po ukončení transakce smazány.
\subsubsection{Vytváření zón}
\label{chapter:implementation:szones:routines:creation}
Hlavní součástí administrátorských rutin jsou funkce pro vytváření zón. Tyto rutiny umožňují vytváření vzdálených i
lokálních neglobálních zón z~následujících zdrojů:
\begin{itemize}
 \item Ze standardní souborů (konfigurace, manifest, profil),
 \item Ze šablon virtuálních strojů,
 \item Z~jiné existující zóny (vzdálené i lokální),
 \item Z~archivu zóny.
\end{itemize}
Prvním způsobem je vytvoření zóny pomocí tří standardních souborů obsahující konfiguraci zóny, manifest a~nastavení systémových
služeb. V~případě vytváření zóny na vzdáleném serveru dojde v~první řadě ke~zkopírování zdrojových souborů na~cílový server.
Dále se vytvoří konfigurace zóny pomocí nástroje \verb|zonecfg(1)| ze zdrojového konfiguračního souboru. Následuje spuštění
instalace zóny z~repozitáře popsané v~kapitole \ref{chapter:zones:instalation:repozitory}, kde se jako parametry předají
cesty k~souborům s~manifestem a~systémovým profilem. Tato sekvence příkazů se vykoná lokálně nebo v~rámci jednoho SSH spojení
s~cílovým serverem.

Dále je v~rámci modulu poskytnuta podpora pro vytváření zón ze šablon popsaných v~kapitole \ref{chapter:implementation:szones:template}. 
V~průběhu této rutiny nejprve dojde k~transformaci šablony, popsané ve stejné kapitole, na tři standardní soubory. Tato transformace
proběhne vždy na lokálním serveru. Dále rutina pokračuje stejně jako v~případě instalace ze standardních souborů popsané výše.

Výše zmíněné rutiny pro vytváření zóny neměly k~dispozici diskový obraz zón a~instalace zóny musela vždy probíhat z~repozitáře.
Následující dva způsoby tvorby neglobálních zón využívají jako zdroj již existující diskový obraz. První z~těchto dvou postupů
využívá diskový obraz již existující zóny. Kroky této rutiny je možné rozdělit na části získání diskového obrazu a~samotné instalace
zóny. Obě tyto části mohou být prováděny buď na lokálním serveru nebo na vzdáleném. Server, odkud je zóna získávána, se označuje
jako \textbf{zdrojový} a~server, kde je vytvářená nová zóna, se nazývá \textbf{cílový}. Před získáním diskového obrazu se 
nástroj nejprve musí ujistit, jestli je zóna v~konzistentním stavu a~zda není spuštěná. Pokud ano, nástroj ji dočasně zastaví.
Následuje vytvoření archivu zdrojové zóny pomocí techniky ZFS popsané v~kapitole \ref{chapter:zones:backup:zfs}. Vytvořený archiv je
následně společně s~konfigurací zdrojové zóny přesunut na~cílový server a~následuje druhá část instalace zóny. Tato část probíhá na
cílovém serveru a~zde se nejprve nakonfiguruje cílová zóna s~pomocí konfigurace zóny zdrojové. Následuje samotná část připojení
diskového obrazu z~poskytnutého archivu. Po tomto procesu je cílová zóna nainstalována na cílový server. Technika klonování
se použije pouze v~případě, že zdrojový a~cílový server jsou identické stroje.

Poslední podporovanou rutinou pro instalaci neglobálních zón je vytváření z~archivu. Tato rutina předpokládá existenci archivu,
který může být typu ZFS nebo UAR. V~případě tvorby
zóny na vzdáleném serveru se nejprve zkopíruje archiv do dočasného adresáře na cílovém serveru. Na cílovém serveru se z~archivu
extrahuje konfigurační soubor a~pomocí něj se zóna nakonfiguruje. Následuje proces vytvoření kořenového souborového systému cílové
zóny z~archivu. Po dokončení tohoto procesu je cílová zóna úspěšně nainstalována na cílovém serveru.

Všechny výše zmíněné typy rutin mají několik společných parametrů, které specifikují chování v~krajních situacích. Prvním
takovým parametrem je \textit{force}. Tento parametr ovlivní rutiny v~případě, kdy již existuje zóna se jménem zóny, kterou
chce rutina vytvořit. V~případě, že je tento parametr zapnutý, rutina danou existující zónu smaže a~místo ní nainstaluje
zónu novou. V~opačném případě skončí rutina s~chybou, že se nepodařilo zónu nainstalovat. Druhým parametrem rutin je \textit{boot}.
Tento parametr určuje, zda se vytvořená zóna má rovnou spustit. Implicitní hodnota obou parametrů je nastavená na \textit{false}.
Všechny rutiny pro vytváření neglobálních zón jsou implementované v~rámci třídy \verb|DeploymentRoutines|.
\subsubsection{Záloha a~obnova zón}
\label{chapter:implementation:szones:routines:backup}
Záloha zón může podle kapitoly \ref{chapter:zones:backup} probíhat dvojím způsobem. Tyto dva způsoby se liší především v~technologii,
která vytváří danou zálohu. V~obou případech se jedná o~vytvoření archivu kořenového souborového systému neglobální zóny. Jeden
způsob používá standardní techniku systémové archivace pomocí UAR a~druhý způsob používá přímo nástroje souborového
systému ZFS. Rutiny pro zálohu podporují oba tyto způsoby a~liší se pouze v~technice vytvoření daného archivu.

Oba typy zálohovacích rutin je možné provádět na lokálních i vzdálených serverech. V~prvním kroku je nutné uvést danou neglobální
zónu do konzistentního stavu a~případně ji zastavit. Následuje proces vytváření archivu, který se liší v~závislosti na
použité technice. Po dokončení archivace je záloha hotová. 

V~obou případech zálohovacích rutin je možné použít volitelný parametr \textit{archive\_destination} určující, na který server 
se má záloha zkopírovat. Implicitně se záloha vytváří na serveru, kde se daná zóna nachází.

Obnova zóny předpokládá existenci její zálohy. Jelikož se jedná o~vytvoření zóny z~archivu, je tento proces stejný
s~procesem vytváření zón z~archivu popsaným v~předchozí kapitole. Všechny zálohovací rutiny jsou v~modulu implementované ve třídě
\verb|BackupRoutines|.
\subsubsection{Migrace zón}
\label{chapter:implementation:szones:routines:migration}
Migrace v~rámci Solaris Zones je přesun neglobální zóny z~jednoho virtualizačního serveru na druhý. Jedná se jak o~přesun
konfigurace zóny tak i o~přesun diskového obrazu. Tato administrační rutina se v~mnoha ohledech podobá rutině pro vytvoření
neglobální zóny z~jiné zóny, která již existuje. Podstatou je vytvoření archivu zóny na~zdrojovém serveru a~přesun tohoto archivu
na~server cílový. Vytvoření archivu je tedy možné provést dvojím způsobem. Migrační rutiny
implementovaného nástroje budou umožňovat migraci zón jak s~použitím archivu ZFS tak i s~použitím archivu UAR.

Hlavním rozdílem migrace oproti vytváření zóny z~již existující zóny je v~tom, že původní zóna se v~případě úspěchu transakce smaže.
Zóna na zdrojovém serveru se smaže až v~případě, kdy je zóna kompletně nakonfigurována a~zdárně nainstalována na
cílovém serveru. Dřívější smazání zdrojové zóny by mohlo vést ke ztrátě dat.

Dalším rozdílem migrace oproti vytváření zóny z~již existující zóny je typ používaných příkazů. Před procesem vytváření archivu
jakéhokoli typu dojde na zdrojovém serveru
k~použití nástroje \verb|zoneadm(1)| a~jeho příkazu \verb|detach|, který bezpečně odpojí diskový obraz zóny od její konfigurace.
Po tomto kroku následuje vytvoření archivu a~jeho přesun na cílový server. Jakmile se dokončí přenos archivu je zóna na cílovém
serveru nakonfigurována a~následně je její obraz připojen pomocí příkazu \verb|attach|, který má jako argument vytvořený archiv.
V~případě úspěchu může být konfigurace zóny i jejího obrazu smazána ze zdrojového serveru.

Migrační rutiny poskytují ještě jeden typ přenosu diskového obrazu zóny a~to přímo pomocí příkazu \verb|zfs send|
a~\verb|zfs recv|. Spuštění prvního příkazu na zdrojovém server a~druhého příkazu na druhém serveru v~rámci SSH spojení
zajistí přenos zdrojového souborového systému z~jednoho serveru na druhý. Příkaz \verb|attach| po tomto přenosu nemá téměř žádnou práci,
protože nemusí souborový systém extrahovat z~archivu.

Všechny výše zmíněné funkce umožňují migraci mezi všemi hosty dané infrastruktury. Těmito funkcemi je umožněno migrovat 
lokální zónu na vzdálený server a~naopak. Také je umožněna migrace zóny mezi dvěma vzdálenými hosty. V~rámci migračních rutin je
možné specifikovat, že zóna na cílovém serveru se může jmenovat jinak než na zdrojovém. Migrační rutiny jsou v~implementovaném modulu zahrnuty
ve třídě s~názvem \verb|MigrationRoutines|.
\subsubsection{Rutiny pro správu zón}
\label{chapter:implementation:szones:routines:management}
Modul Solaris Zones mimo výše zmíněných pokročilejších rutin poskytuje i~základní rutiny pro manipulaci a~správu 
neglobálních zón. V~těchto rutinách je zahrnuto spouštění, násilné i nenásilné vypnutí, restart nebo úplné odstranění zóny
ze systému. Všechny tyto akce je možné provádět jak na lokálním tak i~na~vzdáleném serveru. Třída poskytující tyto
rutiny v~rámci implementovaného modulu se nazývá \verb|BasicRoutines|.
\subsection{Funkční rozhraní modulu}
\label{chapter:implementation:szones:api}
Poslední částí modulu pro správu virtualizačního kontejneru Solaris Zones je rozhraní, které je nabízeno klientským aplikacím
prostřednictvím knihovny. Pro tyto účely je vytvořena speciální třída, která zprostředkovává administrátorské rutiny popsané 
výše. Princip rozhraní funguje stejně jako v~případě knihovny. Toto rozhraní má zaregistrované všechny 
třídy, jejichž metody chce veřejně poskytovat knihovně a~klientským aplikacím. Pokud knihovna obdrží požadavek na volání konkrétní
rutiny, přesměruje tento požadavek právě na toto rozhraní. Rozhraní vyhledá v~rámci zaregistrovaných tříd, zad umí obsloužit
konkrétní požadavek. Pokud nějaká z~tříd modulu umí danou rutinu provést, rozhraní vrátí její návratovou hodnotu. V~opačném případě
je vyvolána výjimka, kterou knihovna odchytí a~případně bude vyhledávat v~ostatních implementovaných modulech.

Ve výsledné implementaci je rozhraní reprezentováno třídou \verb|SZONESAPI|, která má zaregistrované třídy korespondující
s~rutinami popsanými v~kapitole \ref{chapter:implementation:szones:routines}.
\section{Klientská aplikace}
\label{chapter:implementation:client}
Další významnou částí implementovaného nástroje je klientská aplikace. Hlavním úkolem tohoto funkčního bloku nástroje je 
skrývat implementační detaily knihovny a~jejích modulů. Funkcionalitu knihovny tato klientská aplikace zprostředkovává
uživateli pomocí následujících komponent:
\begin{itemize}
 \item Uživatelské rozhraní (CLI),
 \item Správa vzdálených hostů,
 \item Uživatelský žurnál.
\end{itemize}
Klientská aplikace poskytuje rutiny pro správu Solaris Zones popsané v~kapitole \ref{chapter:implementation:szones}
Tyto rutiny může uživatel využívat pomocí uživatelského rozhraní aplikace. Hlavním cílem tohoto
rozhraní je přehlednost a~možnost vykonávat rutiny pro větší množství neglobálních zón. Uživatel má možnost jednoduše
specifikovat konkrétní zóny v~uživatelském rozhraní a~klientská aplikace se postará o~vykonání daného příkazu pro všechny
specifikované neglobální zóny.

Jak bylo uvedeno, nástroj musí umožňovat správu vzdálených zón. Tato funkcionalita
je již implementována v~modulu Solaris Zones. Klientská aplikace musí tedy implementovat pouze způsob, jakým způsobem jednoznačně identifikovat
neglobální zónu v~rámci infrastruktury. Dále klientská aplikace spravuje databázi virtualizačních hostů. Pomocí této databáze
je možné vykonávat některé rutiny hromadně napříč všemi registrovanými hosty.

Poslední funkcionalitou klientské aplikace je udržování tzv. uživatelského žurnálu. Tento žurnál slouží uživateli pro uchovávání
stavů jednotlivých neglobálních zón. V~případě změny stavu registrované zóny jiným uživatelem, je aplikace schopna dohledat,
že došlo ke změně.
\subsection{Rozhraní na příkazové řádce}
\label{chapter:implementation:client:cli}
Hlavní ovládací prvek klientské aplikace, ale i celého nástroje, je rozhraní na~příkazové řádce. Důvody pro výběr tohoto rozhraní
byly popsány v~kapitole \ref{chapter:design:ui:cli}. Uživatel může toto rozhraní ovládat pomocí příkazů, které určují typ 
prováděné operace. Příkazy je možné rozdělit do dvou skupin. První skupina příkazů pouze mění vnitřní stav klientské aplikace.
Druhá skupina příkazů slouží ke správě a~manipulaci s~neglobálními zónami. Téměř všechny tyto příkazy vyžadují jako argument jednu
nebo více neglobálních zón, nad kterými se má provést požadovaná akce. Pro tento účel je nutné zavést unikátní identifikátor,
který přesně specifikuje neglobální zónu v~rámci několika hostů. Jelikož doménové jméno stroje musí být v~rámci sítě
unikátní, bude v~identifikátoru figurovat. Druhou částí identifikátoru bude jméno neglobální zóny v~rámci jednoho hosta.
Toto jméno je v~prostředí jedné globální zóny také unikátní. Kombinací těchto dvou identifikátorů je možné sestrojit název
pro neglobální zónu, který bude unikátní v~rámci celé infrastruktury. Pro identifikaci zóny na příkazové řádce bude uživatel
používat název zóny a~doménové jméno hosta spojené dvojtečkou. Globální identifikátor může vypadat následovně \textit{z1:host1}.

V~následujících kapitolách je popsána funkcionalita a~parametry jednotlivých příkazů uživatelského rozhraní, které slouží
pro správu virtualizačního kontejneru Solaris Zones.
\subsubsection{Deploy}
\label{chapter:implementation:client:cli:deploy}
Prvním zástupcem příkazů pro práci s~neglobálními zónami je \verb|deploy|. Úkolem tohoto příkazu je tvorba zón ze zadaných parametrů.
Jako zdroj pro~vytvoření zóny umí tento příkaz využívat všechny způsoby, které poskytuje implementovaný modul Solaris Zones.
Konkrétně se jedná o~vytvoření pomocí souborů, šablony nebo existující zóny v~rámci infrastruktury. Dále tento příkaz poskytuje
interaktivní instalaci, která využívá grafického rozhraní. Tento způsob je podrobněji popsán v~kapitole
\ref{chapter:implementation:gui:interactive}. Pokud uživatel nespecifikuje jinak, použije se pro vytvoření zóny právě interaktivní
instalace.

Jako argumenty příkazu musí uživatel specifikovat minimálně jednu neglobální zónu pomocí výše zmíněného identifikátoru. Jméno zóny 
se použije při~vytváření a~doménové jméno se použije pro připojení ke vzdálenému hostu. Pro všechny takto specifikované zóny
je použit stejný zdroj a~příkaz je vykonán paralelně pro každou z~nich.

Volitelnými parametry jsou \textit{boot} a~\textit{force}, jejichž funkčnost byla popsána v~kapitole \ref{chapter:implementation:szones:routines:creation}.
\subsubsection{Backup}
\label{chapter:implementation:client:cli:backup}
Příkaz \verb|backup| slouží pro zálohování zón. Tento příkaz umožňuje zálohovat zóny dvěma způsoby. Jedná o~archivaci pomocí archivů ZFS nebo UAR.
Struktura tohoto příkazu je velmi podobná předchozímu. Uživatel nejprve musí definovat seznam zón pomocí globálních
identifikátorů. Aplikace se následně paralelně připojí ke každému specifikovanému hostu a~začne vytvářet zálohy konkrétních zón. 
V~případě zálohy pomocí ZFS archivu se záloha všech zón na konkrétním serveru provádí paralelně. V~druhém případě se záloha provádí v~rámci
hosta sériově, jelikož archivace pomocí UAR neumožňuje paralelní tvorbu archivů na jednom serveru.

Implicitně se záloha vytváří na serveru korespondujícím se zálohovanou zónou. Pomocí volitelných parametrů \textit{destination}
a~\textit{path} se toto chování dá změnit a~zkopírovat zálohu na vzdálený server.
\subsubsection{Recovery}
\label{chapter:implementation:client:cli:recovery}
Opačným příkazem k~příkazu \verb|backup| je \verb|recovery|. Tento příkaz umožňuje obnovovat zóny z~dříve vytvořených záloh.
Podle typu dané zálohy nástroj rozhodne, jakou techniku obnovy použije. Uživatel tedy nemusí specifikovat, že používá zálohu
typu UAR nebo ZFS. Syntaxe tohoto příkazu je stejná jako v~případě zálohy. Uživatel specifikuje identifikátory zón, které
chce obnovit a~pomocí parametru \textit{archives} specifikuje dané zálohy. Pořadí identifikátorů musí odpovídat pořadí archivů, jinak
dojde k~záměně diskových obrazů zón.

Volitelné parametry tohoto příkazu jsou \textit{boot} a~\textit{force}, které umožňují obnovované zóny rovnou spustit nebo
přepsat existující zóny.
\subsubsection{Migrate}
\label{chapter:implementation:client:cli:migrate}
Příkaz \verb|migrate| zajišťuje přesun neglobálních zón mezi dvěma servery. Tento příkaz umožňuje přenášet větší množství zón
libovolně rozložených v~infrastruktuře na jeden konkrétní server. Cílový server může být buď lokální nebo vzdálený. Migrace
je implementována v~několika způsobech. Jedná se o~migraci pomocí přímé metody nebo pomocí archivu ZFS a~UAR.

Argumenty tohoto příkazu jsou identifikátory neglobálních zón, které chce uživatel přesunout. Pomocí parametru \textit{destination}
je možné určit na jaký server mají být specifikované zóny přesunuty. Pokud uživatel nespecifikuje jinak, jsou zóny migrovány
na lokální server. Typ migrace se určuje parametrem \textit{type}. Pokud není využíván způsob migrace pomocí UAR, jsou všechny
migrace prováděny paralelně.
\subsubsection{Template}
\label{chapter:implementation:client:cli:template}
Speciálním příkazem v~rámci uživatelského rozhraní je \verb|template|. Tento příkaz umožňuje vytvářet v~infrastruktuře instance
šablon neglobálních zón. V~kombinaci s~příkazem \verb|deploy| a~jeho parametrem \textit{template} je možné vytvářet zóny
dané konfigurace opravdu rychle.

Příkaz \verb|template| obsahuje dva podpříkazy. Prvním z~nich je \verb|create|, který umožňuje vytvořit instanci šablony na 
jakémkoli vzdáleném serveru. Implicitní chování tohoto příkazu vytvoří instanci specifikované šablony na~každém serveru, který
je v~aplikaci registrovaný.
Chování tohoto příkazu je podobné jako vytváření zóny ze šablony. Následně může uživatel použít příkaz \verb|deploy|
s~parametrem \textit{template} a~vytvořit specifikované zóny pomocí funkce klonování. Tento způsob vytváření zóny je výrazně
rychlejší než všechny ostatní a~navíc šetří místo na disku.

Druhým podpříkazem je \verb|destroy|, který implementuje opačnou funkcionalitu k~příkazu \verb|create|. Tento příkaz tedy 
smaže specifikovanou instanci šablony ze specifikovaných serverů. Implicitně je šablona smazána ze všech registrovaných serverů.

Pomocí parametrů \textit{hosts} je možné specifikovat, na kterých serverech se má šablona vytvořit nebo smazat. Parametr
\textit{force} potom slouží k~přepsání existujících instancí šablon se stejným jménem.
\subsubsection{Manage}
\label{chapter:implementation:client:cli:manage}
Poslední příkaz související se správou neglobálních zón je \verb|manage|. Tento příkaz obsahuje několik podpříkazů, které
jako argumenty vyžadují globální identifikátory zón. Názvy jednotlivých podpříkazů korespondují s~akcí, která se má se specifikovanými
zónami provést. Jedná se o~příkazy spuštění, nenásilného vypnutí, zastavení, restartu a~kompletní odinstalace zóny ze systému.
Tyto podpříkazy nemají žádné parametry a~pro každou specifikovanou zónu se provádí paralelně.
\subsubsection{Logování}
\label{chapter:implementation:client:cli:logs}
Uživatelské rozhraní vytváří ve~složce \textit{\textasciitilde/.szmgmt/logs}, která se nachází v domovském adresáři uživatele, soubory s~průběhem
jednotlivých transakcí. Pro každou akci s~konkrétní zónou je vytvořený samostatný soubor, kde jsou uloženy informace o~provádění
konkrétní příkazů. V~informačním výpisu uživatelského rozhraní se uživatel dozví, jaký soubor souvisí s~konkrétní transakcí.
V~případě chyby transakce zde uživatel může dohledat co danou chybu způsobilo. 
\subsection{Správa vzdálených hostů}
\label{chapter:implementation:client:hosts}
Další komponentou klientské aplikace je správa vzdálených hostů. Tato komponenta umožňuje uživateli registrovat vzdálené 
hosty do vnitřního stavu aplikace. Hromadné akce nabízené uživatelským rozhraním jsou vykonávány právě pomocí této databáze
hostů. Dalším důvodem, proč je nutné uchovávat hosty v~aplikaci, je specifikace připojení. Jelikož se aplikace připojuje ke
vzdáleným hostům pomocí SSH, je nutné specifikovat uživatelské údaje. Pro tyto účely je vytvořena databáze hostů uložená
v~domovském adresáři uživatele ve složce \textit{\textasciitilde/.szmgmt/hosts/}.

Tato komponenta se ovládá pomocí uživatelského rozhraní a~konkrétně pomocí příkazu \verb|host|. Tento příkaz má následující
tři podpříkazy:
\begin{itemize}
 \item \verb|add|,
 \item \verb|delete|,
 \item \verb|list|.
\end{itemize}
Příkaz \verb|add| slouží pro registrování hostů do aplikace. Jako argument požaduje doménové jméno vzdáleného
serveru, které je následně uloženo do databáze. Volitelné parametry tohoto příkazu umožňují specifikovat uživatele a~jeho
privátní klíče, které mají být použity pro připojování k~tomuto serveru. Implicitní hodnoty těchto parametrů jsou demonstrovány
v~ukázce kódu \ref{code:host_spec}.

Druhý příkaz \verb|delete| slouží k~odstranění daného hosta z~databáze aplikace. Při této akci dojde i ke smazání specifikace
SSH připojení k~tomuto serveru. Poslední příkaz \verb|list| vypíše na standardní výstup seznam všech registrovaných hostů.

Funkcionalita této komponenty značně souvisí se všemi příkazy uživatelského rozhraní, které se připojují ke vzdáleným hostům.
Uživatel specifikuje neglobální zóny pomocí globálních identifikátorů, které obsahují doménová jména. V~okamžiku připojování
ke konkrétnímu vzdálenému serveru je použito právě nastavení uložené v~databázi. Pokud pro daného hosta neexistuje záznam
v~databázi, jsou použity implicitní hodnoty.

Pro každého uživatele tato komponenta udržuje jeho vlastní databázi v~domovském adresáři. Hlavním důvodem pro toto umístění
je izolace pohledů jednotlivých uživatelů aplikace.
\subsubsection{Struktura databáze}
\label{chapter:implementation:client:hosts:database}
Databáze hostů je uložena v~několika souborech, které se nacházejí v~adresáři \textit{\textasciitilde/.szmgmt/hosts} konkrétního uživatele.
Hlavní soubor databáze se nazývá \textit{hosts.json} a~obsahuje seznam doménových jmen jednotlivých hostů. Pro~každý záznam 
v~tomto souboru je vytvořen další soubor, který obsahuje specifikaci SSH připojení. Název tohoto souboru odpovídá doménovému
jménu konkrétního serveru. Všechny soubory databáze jsou uloženy v~datovém formátu JSON. V~ukázce kódu \ref{code:host_spec}
je zobrazeno implicitní nastavení pro uživatele \textit{zadmin} a~server s~doménovým jménem \textit{shost1}.
\begin{lstlisting}[language=json, caption={Implicitní nastavení parametrů SSH připojení}, float,label={code:host_spec}]  
{
  "host_name": "shost1",
  "user": "zadmin",
  "keys": ["~/.ssh/id_rsa"],
  "timeout": 1
}
\end{lstlisting}
\subsection{Uživatelský žurnál}
\label{chapter:implementation:client:journal}
Poslední komponentou klientské aplikace je uživatelský žurnál. Solaris Zones neposkytují žádný nástroj pro přidělování zón 
jednotlivým uživatelům. Každý privilegovaný uživatel má možnost jakýmkoli způsobem ovlivnit kteroukoli zónu v~rámci dané globální
zóny. Z~tohoto důvodu se může stát, že neglobální zóna vytvořená jedním uživatelem bude smazána nebo změněna druhým uživatelem.
Standardně Solaris Zones na tuto skutečnost uživatele nijak neupozorní. 

Uživatelský žurnál slouží pro sledování stavů jednotlivých neglobálních zón, se kterými uživatel nějakým způsobem manipuloval.
Tyto stavy budou sloužit k~informování uživatele o~tom, zda se daná zóna od předchozí manipulace nějakým způsobem změnila.
Konkrétně, zda se změnil její stav nebo diskový obraz. Žurnál umožňuje také detekci nových zón, které se v~infrastruktuře
od posledního spuštění aplikace objevily.

Práce této komponenty je z~větší části automatická, ale také je umožněno její přímé ovládání pomocí uživatelského rozhraní. Automatická
práce žurnálu spočívá v~aktualizování stavů jednotlivých zón vždy, když uživatel vyvolá nějakou administrátorskou rutinu. Při vytvoření
zóny dojde k~přidání konkrétního stavu do uživatelského žurnálu. V~případě odstranění zóny je odstraněn i stav z~databáze. Pro 
manuální správu žurnálu poskytuje uživatelské rozhraní příkaz \verb|journal|, který má následující podpříkazy:
\begin{itemize}
 \item \verb|track|,
 \item \verb|detrack|,
 \item \verb|update|,
 \item \verb|clear|,
 \item \verb|status|,
 \item \verb|list|.
\end{itemize}
Příkaz \verb|track| slouží pro zaregistrování dané zóny. Díky tomuto příkazu není nutné, aby uživatel s~danou zónou manipuloval.
Pokud uživatel spustí tento příkaz a~předá mu jako argument globální identifikátor zóny, dojde k~získání aktuálního stavu zóny
a~zaregistrování do žurnálu. Od této chvíle se konkrétní zóna zařadila do seznamu sledovaných. Opačnou akci provádí příkaz
\verb|detrack|, který ze žurnálu odstraní záznam o~dané zóně.

Pokud chce uživatel znát stavy všech zón v~rámci registrovaných hostů, může použít příkaz \verb|update|. Tento příkaz se pomocí
SSH přihlásí na všechny registrované hosty a~stáhne informace o~všech neglobálních zónách. Tyto informace jsou následně uloženy
do žurnálu a~všechny zóny se stávají sledovanými. Pro opačnou funkcionalitu slouží příkaz \verb|clear|, který kompletně vymaže
databázi sledovaných zón.

Poslední dvojicí podporovaných příkazů jsou \verb|status| a~\verb|clear|. Oba tyto příkazy mají za úkol informovat uživatele
o~stavech sledovaných zón. Příkaz \verb|list| jednoduše vypíše všechny zaregistrované zóny a~jejich stavy. Tento příkaz se nikam
nepřipojuje a~vypisuje informace přímo ze žurnálu. Příkaz \verb|status| slouží pro porovnání stavů uložených v~žurnálu
s~aktuálním stavem všech zón v~rámci registrovaných hostů. Pokud se některý uložený stav neshoduje se stavem aktuálním, je 
o~tom uživatel prostřednictvím standardního výstupu informován. Program také označí zóny, které se nově objevily v~infrastruktuře. 
Stejně jako v~případě správy vzdálených hostů, je žurnál udržován pro každého uživatele zvlášť. 
\subsubsection{Struktura databáze}
\label{chapter:implementation:client:journal:database}
Uživatelský žurnál je uložen v~souboru \textit{\textasciitilde/.szmgmt/journal/tracked\_zones.json} v~domovském adresáři
konkrétního uživatele. V~tomto souboru jsou uložené atributy sledovaných zón, které slouží pro definici konkrétního stavu zóny.
Kombinace hodnot těchto atributů by měla zajistit detekci změn, které nemusí být na první pohled patrné. Příkladem takové změny
může být například přeinstalování zóny a~navození stejného stavu. Při této akci zůstane jméno, konfigurace a~stav zóny stejné,
ale změní se diskový obraz. Pro tento účel udržuje žurnál u~každé sledované zóny atribut \textit{UUID}, který jednoznačně
definuje diskový obraz zóny. Porovnáním uložených atributů s~aktuálním stavem zóny je možné zjistit, zda bylo se zónou 
nějakým způsobem manipulováno. Ukládané atributy zóny a~struktura záznamu je patrná z~ukázky kódu \ref{code:journal}.
V~uživatelském žurnálu je podobný záznam udržován pro každou sledovanou zónu.
\begin{lstlisting}[language=json, caption={Záznam stavu zóny v~žurnálu}, float,label={code:journal}]  
"zweb1b-clone:localhost": {
  "zone_name": "zweb1b-clone:localhost",
  "zone_state": "running",
  "zone_path": "/system/zones/zweb1b-clone",
  "zone_uuid": "a000b847-7854-4939-8ac1-8ae5d9013072",
  "zone_brand": "solaris",
  "zone_ip": "excl"
}
\end{lstlisting}
\section{Grafické rozhraní}
\label{chapter:implementation:gui}
Poslední součástí nástroje pro podporu automatické správy virtualizačního kontejneru Solaris Zones je grafické uživatelské 
rozhraní. Jeho hlavní účel je zvyšování uživatelského komfortu a~odstínění uživatele od implementačních detailů. 
Implementované grafické rozhraní slouží jako nadstavba nad tvorbou šablon specifikovaných v~kapitole \ref{chapter:implementation:szones:template}
a~umožňuje uživateli jejich tvorbu a~editaci. Pro tyto účely byl vytvořen graficky orientovaný editor, které pomocí grafických
elementů umožňuje vyplňování příslušných atributů neglobálních zón. Stejný princip je využitý i pro interaktivní instalaci 
zón.

Implementované grafické rozhraní využívá grafické knihovny Swing, která je podporována pouze na platformě JVM. Z~důvodů uvedených
v~kapitole \ref{chapter:implementation:language:ruby} je použití grafického rozhraní podmíněno využitím interpretu JRuby.
V~případě, že uživatel spustí grafické rozhraní na jiné platformě, aplikace se ukončí a~grafické rozhraní se nezobrazí. Ostatní
funkce nástroje nejsou závislé na použitém interpretu.
\subsection{Editor šablon}
\label{chapter:implementation:gui:editor}
Prvním využitím grafického rozhraní je editor pro manipulaci se šablonami zón. Tento editor se spouští pomocí příkazu \verb|editor|,
který je součástí uživatelského rozhraní klientské aplikace. Po spuštění tohoto příkazu je uživateli zobrazeno grafické okno,
které slouží pro editaci šablon.

Horní část editoru obsahuje ovládací panel, který umožňuje uživateli načítat šablony nebo vytvářet nové. Součástí tohoto panelu
je také tlačítko pro~ukončení editoru. Pokud uživatel zvolí možnost načtení šablony, je pomocí několika dialogových oken dotázán
na cestu k~dané šabloně. Po úspěšném zadání cesty se uživateli v~prostřední části editoru zobrazí formulář vyplněný pomocí
dat z~načtené šablony. Pokud uživatel vybere druhou možnost, kterou je vytvoření šablony, je v~prostřední části editoru zobrazen 
ten samý formulář, ale vyplněný implicitními hodnotami. Ovládací menu společně s~úvodním oknem editoru je zobrazeno na 
obrázku \ref{image:implemetation:gui:menu}.
\begin{figure}
    \centering    
    \caption{Ovládací menu editoru šablon}
    \includegraphics[scale=0.9]{assets/pdfs/implemetation_gui_menu.png}
    \label{image:implemetation:gui:menu}
\end{figure}
Formulář pro vyplňování atributu šablony je umístěn ve středu editoru a~tvoří jeho nejpodstatnější část. Jelikož má struktura 
šablony neglobální zóny tři části, skládá se i hlavní okno editoru ze tří oddělených částí. Tyto části přímo odpovídají jednotlivým
sekcím šablony a~nazývají se konfigurace, manifest a~nastavení. Jak je patrné z obrázku \ref{image:implemetation:gui:form}, každá
sekce editoru má ve spodní části ovládací prvky. Tyto prvky slouží k~přidávání a~odebírání konfiguračních prvků ze šablony. Část s~konfigurací
umožňuje vybírat z~několika různých typů zdrojů, které lze v~editoru použít. V~druhé části editoru lze přidávat a~odebírat
softwarové balíky, které mají být v~instancích dané šablony nainstalované. Poslední část umožňuje přidávat a~odebírat nastavení
pro definované síťové adaptéry.

Poslední částí editoru je spodní ovládací panel, který obsahuje dvě funkční tlačítka. V~případě použití prvního tlačítka
pro validaci se pomocí knihovny vyvolá příslušná funkce a~šablona se ověří. Uživatel je o~výsledku informován pomocí dialogového
okna. Druhé tlačítko slouží k~uložení šablony do souboru a~uživatel je vyzván k~zadání jména šablony a~adresáře, kam se šablona má uložit.
V~obou případech dojde k~rekonstrukci datového formátu JSON popsaného v~kapitole \ref{chapter:implementation:szones:template},
který je vyplněn pomocí hodnot specifikovaných uživatelem v~editoru. Tato konstrukce šablon je pro uživatele jednodušší, protože
vždy vygeneruje validní šablonu. V~případě manuální tvorby musí uživatel zajistit její validitu.
\begin{figure}
    \centering    
    \caption{Formulář editoru šablon}
    \includegraphics[scale=0.35]{assets/pdfs/implemetation_gui_form.png}
    \label{image:implemetation:gui:form}
\end{figure}
\subsection{Interaktivní instalace}
\label{chapter:implementation:gui:interactive}
Grafické rozhraní je v~nástroji využito ještě v~případě, že uživatel spustí příkaz \verb|deploy| s~parametrem \textit{interactive}.
V~tomto případě je spuštěna interaktivní instalace a~uživateli je opět zobrazeno grafické okno. Toto okno je stejného charakteru
jako výše popsaný editor a~slouží ke specifikaci vlastností vytvářených zón. Uživatel nyní nebude mít na výběr z~načtení šablon, ale bude muset
zadat hodnoty manuálně. Některé atributy jsou vyplněny implicitními hodnotami a~uživatel musí zadat minimálně počáteční heslo 
uživatele \textit{root} a~konfiguraci počátečního systémového uživatele.
  
\chapter{Testování a měření}
  \label{chapter:testing}
Poslední kapitola této diplomové práce popisuje testování funkcionality nástroje pro podporu automatické správy virtualizačního
kontejneru Solaris Zones, jehož implementace je popsána v kapitole \ref{chapter:implementation}. Zaměřuje se především testování
jednotlivých scénářů použití nástroje a zkoumá jeho chování. Na začátku této kapitoly je definováno prostředí, ve kterém byly
testy prováděny. Následuje série testů, které zkoumají funkčnost nástroje v konkrétních případech použití. Kapitola je zakončena
měřením, které zkoumá dobu trvání některých funkcí nástroje.
\section{Definice testovacího prostředí}
\label{chapter:testing:environment}
Pro účely testování výše zmíněného nástroje bylo nutné vytvořit prostředí odpovídající jeho cílové platformě. Toto prostředí
obsahuje několik virtualizačních serverů s operačním systémem Solaris, který bude poskytovat své prostředky neglobálním zónám.
Tyto servery jsou propojeny počítačovou sítí, pomocí které je lze ovládat. Tato infrastruktura virtualizovaně vytvořena na 
fyzickém systému s následujícími parametry.
\begin{itemize}
 \item Procesor Intel(R) Xeon(R) CPU E3-1230 v3 (3.30Ghz)
 \item RAM 16GB
 \item Operační systém Windows 10 (64-bit)
\end{itemize}
Virtualizace architektury byla docílena pomocí virtualizační technologie Virtualbox, která umožňuje spouštění virtuálních 
počítačů v rámci jiného operačního systému. Pomocí této technologie byly vytvořeny tři virtuální stroje s operačním systémem
Solaris ve verzi 11.3. Tyto stroje byly propojeny pomocí virtuální počítačové sítě a nakonfigurovány tak, aby se na ně dalo 
připojovat pomocí SSH. Dále byly jednotlivým strojů přiřazeny doménové jména \textit{shost}, \textit{shost1} a \textit{shost2},
a přidány korespondující řádky do souboru \textit{/etc/hosts}. Toto nastavení umožňuje používat specifikované doménové jména
místo IP adres a zjednoduší tak identifikaci strojů v testovacích ukázkách. 

Jelikož provozování virtualizační technologie Solaris Zones vyžaduje nemalé množství výpočetní prostředků, bylo nutné
dostupné prostředky fyzického systému rozdělit mezi virtuální stroje. Z tohoto důvodu byly každému virtuálnímu počítači
přiřazeny následující výpočetní prostředky.
\begin{itemize}
 \item Jedno jádro fyzického procesoru
 \item RAM 3 GB 
 \item Virtuální disk 50 GB (HDD)
\end{itemize}

Na virtuální počítač s doménovým jménem \textit{shost} byl nainstalován interpret programovacího jazyka Ruby ve verzi 2.4.2.
Dále byla na stejný počítač nainstalována Java ve verzi 1.8.0\_60 a následně druhý interpret programovacího jazyka Ruby
tentokrát ve verzi 2.3.3 a implementaci JRuby. Pokud nebude uvedeno jinak, testovaný nástroj bude vždy spouštěn z virtuálního
počítače s doménovým jménem \textit{shost}.

Posledním učiněným krokem byla konfigurace uživatele \textit{zadmin}, který má práva na vykonávání příkazů nutných k správnému
chodu implementovaného nástroje. Tyto nástroje byly vyjmenovány v kapitole \ref{chapter:design:architecture:szones}. Uživatel 
byl pomocí nástroje RBAC vytvořen a nakonfigurován na všech vytvořených virtuálních počítačích.
\section{Testování scénářů použití}
\label{chapter:testing:scenario}
V následujících kapitolách je popsáno akceptační testování některých scénářů použití nástroje pro podporu automatické správy
Solaris Zones.  Pro testování aplikace bylo vždy použito popsané prostředí, pokud není uvedeno jinak. Na začátku každého scénáře
je stanoven cíl, který se by uživatel chtěl pomocí implementovaného nástroje dosáhnout. Následně je popsán stav prostředí, ve
kterém se sytém nachází před provedením konkrétní akce. Dále proveden korespondující příkaz v uživatelském rozhraní nástroje,
který má splnit stanovený cíl. Výsledek tohoto kroku je ověřen pomocí systémových příkazů a v závěru je rozhodnuto, zda bylo
dosaženo stanoveného cíle.
\subsection{Vytvoření neglobálních ze šablony}
\label{chapter:testing:scenario:deploy_template}
Pro komplexní otestování funkcionality implementovaného nástroje byl zvolen scénář vytvoření několika neglobálních zón pomocí
šablony. Hlavní důvod pro výběr tohoto scénáře je, že se do tohoto procesu se zapojují téměř všechny části nástroje.

Cílem tohoto scénáře je vytvoření několika neglobálních zón na různých hostech v rámci dané infrastruktury. Jako zdroj byla 
použita šablona popsaná v kapitole \ref{chapter:implementation:szones:template}. Z šablony byly vybrány některé důležité 
vlastnosti, které mají vytvořené zóny mít. Typ zóny byl stanoven jako \textit{solaris} s exkluzivní IP adresou. Dále má 
nainstalovaná zóna obsahovat softwarové balíčky pro správu zdrojového kódu. Tyto balíčky obsahují nástroje \verb|hg| a 
\verb|git|. Šablona také definuje počáteční heslo uživatele \textit{root} a nastavuje typ tohoto uživatelského účtu na roli.
Vedle uživatele \textit{root} je v šabloně definován počáteční systémový uživatel, který má být zároveň systémový administrátor.
Vytvářené neglobální zóny mají mít jedno síťové rozhraní se jménem \textit{net0}, které bude konfigurováno automaticky pomocí
DHCP. Takto definovaná šablona je uložena na severu s doménovým jménem \textit{shost}.

Příkaz pro vytvoření čtyř zón pomocí uživatelského rozhraní nástroje je v ukázce kódu \ref{code:test:deployment} na první řádce.
Tento příkaz říká, že mají být vytvořeny zóny \textit{zdev} a \textit{zdev1} na lokálním serveru a zóna \textit{zdev} na vzdálených
serverech \textit{shost1} a \textit{shost2}. Jako parametr \textit{specification} je udána cesta k šabloně s výše popsanými 
vlastnostmi. Dále byl příkazu předán parametr \textit{boot}, který rovnou spustí vytvořené zóny.
\begin{lstlisting}[basicstyle=\scriptsize\ttfamily, caption={Vytvoření neglobálních zón ze šablony}, float,label={code:test:deployment}]  
# szmgmt_cli deploy -b zdev zdev zdev:shost1 zdev:shost2 -s ~/zdev.json 
Solaris zones deployment from virtual machine specification initialized.
  ---------------------------------------------------------
  Options:
                Boot zones: enable
    Rewrite existing zones: disable
                    Source: specification </export/home/zadmin/zdev.json>
  ---------------------------------------------------------
  Loading virtual machine specification.
  Virtual machine specification loaded.
  ---------------------------------------------------------
  Connecting concurrently to hosts 'localhost, shost1, shost2'.
  Processing zone 'zdev' deployment on host 'localhost'.
  Processing zone 'zdev1' deployment on host 'localhost'.
  Processing zone 'zdev' deployment on host 'shost1'.
  Processing zone 'zdev' deployment on host 'shost2'.
  ---------------------------------------------------------
  Deployment finished.
    Status:
      localhost:
        zdev: success
        zdev1: success
      shost1:
        zdev: success
      shost2:
        zdev: success
\end{lstlisting}

Z konce standardního výstupu nástroje v ukázce kódu \ref{code:test:deployment} je patrné, že vytvoření všech neglobálních zón
proběhlo v pořádku. Jelikož se pro vytváření zón používá stejná rutina a stejná šablona, musí mít všechny stejné parametry.
Pro otestování korektnosti práce nástroje byla použita neglobální zóna \textit{zdev} na vzdáleném serveru \textit{shost2}.
Korektní vytvoření a spuštění zóny bylo ověřeno pomocí nástroje \verb|zlogin(1)|, který umožňuje připojení ke konzoly dané
zóny. Úspěšné přihlášení v ukázce kódu \ref{code:test:deployment:result} signalizovalo hned několik věcí. Za prvé se zdárně
podařilo vytvořit a spustit danou zónu a za druhé byly správně nakonfigurovány uživatelské systémové služby pomocí atributů ze šablony.
Dále je ukázky \ref{code:test:deployment:result} patrné, že došlo k vytvoření síťového adaptérů \textit{net0} a jeho automatické
konfigurace pomocí služby DHCP. Daná neglobální zóna tak byla okamžitě po vytvoření dostupná ze sítě. Přítomnost softwarových
balíčků byla otestována pomocí jejich rozhraní na příkazové řádce jak je vidět v ukázce \ref{code:test:deployment:result}.
Posledním kritériem úspěchu bylo správné nakonfigurování počátečního systémového uživatele. Jméno a heslo bylo ověřeno již
při přihlašování do zóny. Zbývalo tedy ověřit jestli uživatel má práva systémového administrátora, což bylo provedeno pomocí
příkazu \verb|profile|. Výpis na ukázce \ref{code:test:deployment:result} je zkrácený, ale obsahuje profil \textit{System Administrator}.
\begin{lstlisting}[basicstyle=\scriptsize\ttfamily, caption={Ověření správného vytvoření zóny}, float,label={code:test:deployment:result}]  
zadmin@shost2:~$ zlogin -C  zdev
[Connected to zone 'zweb' console]
Hostname: solaris
solaris console login: admin
Password:
admin@solaris:~$ ifconfig net0
net0: flags=100001000843<UP,BROADCAST,RUNNING,MULTICAST,IPv4,PHYSRUNNING>
    inet 10.164.85.13 netmask ff000000 broadcast 10.255.255.255
admin@solaris:~$ git --version
git version 1.7.9.2
admin@solaris:~$ hg --version
Mercurial Distributed SCM (version 3.4)
simactom@solaris:~$ profiles
System Administrator
...
\end{lstlisting}

Pomocí výše zmíněných testů bylo ověřeno, že se daná zóna vytvořila, spustila a že měla vlastnosti specifikované v použité šabloně.
Stejným způsobem byly ověřeny i ostatní vytvářené zóny. Jelikož tyto zóny vykazovaly stejné chování a vlastnosti, byl tento
test uzavřen a konstatován jako splněný.
\subsection{Využití uživatelského žurnálu}
\label{chapter:testing:scenario:journal}
Dalším využitím implementovaného nástroje může být využití uživatelského žurnálu. Cílem následujícího scénáře je kontrola 
funkcionality uživatelského žurnálu a jeho schopnosti informovat uživatele o změnách neglobálních zón v rámci infrastruktury.
K tomuto účelu byl využitý stav, ve kterém se systém nacházel po testování předchozího scénáře popsaného v kapitole
\ref{chapter:testing:scenario:deploy_template}. Součástí předchozího scénáře bylo vytvoření čtyř zón pomocí implementovaného
nástroje. Před tímto vytvořením se v systému nenacházely žádné jiné neglobální zóny. V tomto stavu by měl uživatelský žurnál 
obsahovat čtyři zóny ve stavu \textit{running}. Jak je vidět z ukážky kódu \ref{code:test:journal}, součástí uživatelského
žurnálu byly opravdu čtyři zóny ve stavu \textit{running} a výpis neobsahoval žádné jiné nesledované neglobální zóny.
Toto zjištění indikovalo, že nástroj opravdu aktualizuje uživatelský nástroj provedenou akcí.
\begin{lstlisting}[basicstyle=\scriptsize\ttfamily, caption={Uživatelské žurnál po vytvoření zón}, float,label={code:test:journal}]  
zadmin@shost:~$ szmgm_cli journal status
Tracked zones:
  Host localhost
      zdev1:localhost
           Zone type: solaris
          Zone state: running
           Zone path: /system/zones/zdev1
      zdev:localhost
           Zone type: solaris
          Zone state: running
           Zone path: /system/zones/zdev
  Host shost2
      zdev:shost2
           Zone type: solaris
          Zone state: running                      
           Zone path: /system/zones/zdev
  Host shost1
      zdev:shost1
           Zone type: solaris
          Zone state: running
           Zone path: /system/zones/zdev
\end{lstlisting}
Následně byla simulována situace, kdy jiný uživatel změní nějakým způsobem stav sledované zóny. Konkrétně byla bez pomoci 
implementovaného nástroje přeinstalována zóna \textit{zdev} na vzdáleném serveru \textit{shost2} a její stav byl změněn z původního
\textit{running} na \textit{installed}. Dále byla vytvořena nová zóna \textit{zdev-clone} na stejném vzdáleném počítači. V tomto
případě by měl uživatel při dalším vypsání uživatelského žurnálu zjistit, že se změnil stav a diskový obraz dané zóny 
\textit{zdev} změnil. Součástí výpisu by měla být i informace o nově vytvořené zóny v rámci infrastruktury. Z ukázky kódu 
\ref{code:test:journal:change} je vidět, že uživatelský žurnál opravdu informuje uživatele o změně sledované zóny a na konci
výpisu je zobrazena informace o nově vytvořené zóně. Pro úsporu místa byly ostatní zóny z výpisu vynechány.
\begin{lstlisting}[basicstyle=\scriptsize\ttfamily, caption={Uživatelské žurnál po změně}, float,label={code:test:journal:change}]  
zadmin@shost:~$ szmgm_cli journal status
Tracked zones:
  ...
  Host shost2
      zweb:shost2
           Zone type: solaris
          Zone state: running
                      MISMATCH - Fresh zone state property is installed.
           Zone path: /system/zones/zweb
                      MISMATCH - Fresh zone UUID mismatch.
Untracked zones:
  Host shost2
      zdev-colne:shost2
          Zone type: solaris
         Zone state: installed
          Zone path: /system/zones/zdev-colne
\end{lstlisting}

Pomocí implementovaného nástroje může uživatel neglobální zóny vytvářet, mazat nebo měnit jejich stav. Jak bylo zjištěno
na začátku této kapitoly, nástroj aktualizuje uživatelský žurnál a jeho konkrétní záznam pokud danou zónu vytváří. Podobným
způsobem bylo ověřeno, že uživatelský žurnál je aktualizován i při mazání a změně stavu. Dále tento scénář ověřil funkcionalitu
žurnálu, která má informovat uživatele v případě, kdy dojde ke změně stavu sledované zóny nebo vytvoření nové zóny v rámci
infrastruktury. Z výše uvedených důvodů bylo testování využití uživatelského žurnálu úspěšné.
\subsection{Migrace neglobálních zón}
\label{chapter:testing:scenario:migration}
\subsection{Záloha a obnova zón}
\label{chapter:testing:scenario:migration}

\section{Měření}
\label{chapter:measurement}

\begin{conclusion}
  % Chapter: Conclusion
% Author: Tomáš Šimáček
Virtualizace se stala běžnou a možná i nezbytnou součástí dnešního počítačového světa.
\end{conclusion}

\bibliographystyle{csn690}
\bibliography{references}

\appendix

\chapter{Seznam použitých zkratek}
% \printglossaries
\begin{description}
        \item[BIOS] Basic Input Output System
        \item[CLI] Command Line Interface
        \item[DNS] Domain Name System
        \item[EPT] Extended Page Tables
        \item[HLL]  High Level Langugage
        \item[HW]  Hardware
        \item[I/O] Input/Output
        \item[ISA]  Instruction Set Architecture
	\item[IT]  Information Technology
	\item[JSON] JavaScript Object Notation
	\item[NAT]  Network Address Transaltion
	\item[OS]  Operating System
	\item[PC]  Program Counter/Personal Computer
	\item[RVI]  Rapid Virtualization Indexing
	\item[SW]  Software
	\item[SMF] Solaris Management Facility
	\item[SSH] Secure Shell
	\item[TLB] Translation Lookaside Buffer
	\item[YARV] Yet Another Ruby VM
	\item[VM]  Virtual Machine
	\item[VMM] Virtual Machine Monitor
	\item[XML] Extesible Markup Language
	\item[ZFS] Zettabyte File System
\end{description}

\chapter{Obsah přiloženého CD}

%upravte podle skutecnosti

\begin{figure}
	\dirtree{%
		.1 readme.txt\DTcomment{stručný popis obsahu CD}.
		.1 exe\DTcomment{adresář se spustitelnou formou implementace}.
		.1 src.
		.2 impl\DTcomment{zdrojové kódy implementace}.
		.2 thesis\DTcomment{zdrojová forma práce ve formátu \LaTeX{}}.
		.1 text\DTcomment{text práce}.
		.2 thesis.pdf\DTcomment{text práce ve formátu PDF}.
		.2 thesis.ps\DTcomment{text práce ve formátu PS}.
	}
\end{figure}

\chapter{Testování}
  \begin{listing}[ht]
  \caption{Výstup příkazu pro vytvoření neglobálních zón ze šablony}
  \begin{minted}[fontsize=\small]{bash}
zadmin@shost:~$ szmgmt_cli deploy -b zdev zdev \
                                     zdev:shost1 \
                                     zdev:shost2 \
                                     -s ~/zdev.json 
Solaris zones deployment initialized.
  ---------------------------------------------------------
  Options:
      Boot zones: enable
   Rewrite zones: disable
          Source: </export/home/zadmin/zdev.json>
  ---------------------------------------------------------
  Loading virtual machine specification.
  Virtual machine specification loaded.
  ---------------------------------------------------------
  Connecting concurrently to hosts 'localhost,shost1,shost2'.
  Processing zone 'zdev' deployment on host 'localhost'.
  Processing zone 'zdev1' deployment on host 'localhost'.
  Processing zone 'zdev' deployment on host 'shost1'.
  Processing zone 'zdev' deployment on host 'shost2'.
  ---------------------------------------------------------
  Deployment finished.
    Status:
      localhost:
        zdev:  success
        zdev1: success
      shost1:
        zdev:  success
      shost2:
        zdev:  success
  \end{minted}
  \label{code:test:deployment}
\end{listing}

\begin{listing}[ht]
  \caption{Výpis uživatelského žurnálu po vytvoření zón}  
  \begin{minted}[fontsize=\small]{bash}
zadmin@shost:~$ szmgm_cli journal status
Tracked zones:
  Host localhost
      zdev1:localhost
           Zone type: solaris
          Zone state: running
           Zone path: /system/zones/zdev1
      zdev:localhost
           Zone type: solaris
          Zone state: running
           Zone path: /system/zones/zdev
  Host shost2
      zdev:shost2
           Zone type: solaris
          Zone state: running                      
           Zone path: /system/zones/zdev
  Host shost1
      zdev:shost1
           Zone type: solaris
          Zone state: running
           Zone path: /system/zones/zdev
  \end{minted}
  \label{code:test:journal}
\end{listing}

\begin{listing}[ht]
  \caption{Výpis uživatelského žurnálu po změně původní zóny}  
  \begin{minted}[fontsize=\small]{bash}
zadmin@shost:~$ szmgm_cli journal status
Tracked zones:
  ...
  Host shost2
      zweb:shost2
           Zone type: solaris
          Zone state: running
                      MISMATCH - Current state installed.
           Zone path: /system/zones/zweb
                      MISMATCH - UUID mismatch.
Untracked zones:
  Host shost2
      zdev-colne:shost2
          Zone type: solaris
         Zone state: installed
          Zone path: /system/zones/zdev-colne
  \end{minted}
  \label{code:test:journal:change}
\end{listing}

\begin{listing}[ht]
  \caption{Sekvence příkazů pro ověření správnosti vytvoření zóny}  
  \begin{minted}[fontsize=\small]{bash} 
zadmin@shost2:~$ zlogin -C  zdev
[Connected to zone 'zweb' console]
Hostname: solaris
solaris console login: admin
Password:
admin@solaris:~$ ifconfig net0
net0: flags=100001000843<UP,BROADCAST,RUNNING,MULTICAST,IPv4>
    inet 10.164.85.13 netmask ff000000
admin@solaris:~$ git --version
git version 1.7.9.2
admin@solaris:~$ hg --version
Mercurial Distributed SCM (version 3.4)
simactom@solaris:~$ profiles
System Administrator
...
  \end{minted}
  \label{code:test:deployment:result}
\end{listing}

\begin{listing}[ht]
  \caption{Výpis uživatelského žurnálu před migrací zón}  
  \begin{minted}[fontsize=\small]{bash}
zadmin@shost:~$ szmgmt_cli journal status
Geting fresh information about zones on all registered hosts...
Tracked zones:
  Host localhost
      zmigr1:localhost
           Zone type: solaris
          Zone state: running
           Zone path: /system/zones/zmigr1
      zmigr:localhost
           Zone type: solaris
          Zone state: running
           Zone path: /system/zones/zmigr                 
  Host shost1
      zmigr3:shost1
           Zone type: solaris
          Zone state: running
           Zone path: /system/zones/zmigr3
      zmigr2:shost1
           Zone type: solaris
          Zone state: running
           Zone path: /system/zones/zmigr2      
  \end{minted}
  \label{code:test:migration:before}
\end{listing}

\begin{listing}[ht]
  \caption{Výpis příkazu pro migraci neglobálních zón}  
  \begin{minted}[fontsize=\small]{bash}
zadmin@shost:~$ szmgm_cli migrate -t d zmigr \
                                       zmigr1 \
                                       zmigr1:shost1 \
                                       zmigr1:shost1 \
                                       -d shost2
Solaris zones migration initialized.
  ---------------------------------------------------------
  Options:
                Boot zones: disable
    Rewrite existing zones: disable
  ---------------------------------------------------------
  Connecting concurrently to hosts 'localhost shost1'.
  Processing migration of zone 'zmigr1:localhost'. 
    See log '~/.szmgmt/log/zmigr1_migration_ph5o7o.log'. 
  Processing migration of zone 'zmigr2:shost1'. 
    See log '~/.szmgmt/log/zmigr2_migration_uy1frm.log'.
  Processing migration of zone 'zmigr3:shost1'. 
    See log '~/.szmgmt/log/zmigr3_migration_lhc24y.log'.
  Processing migration of zone 'zmigr:localhost'. 
    See log '~/.szmgmt/log/zmigr_migration_h2rdat.log'.
  ---------------------------------------------------------
  Migration finished.
    Status:
      localhost:
        zmigr:  success
        zmigr1: success
      shost1:
        zmigr2: success
        zmigr3: success
  \end{minted}
  \label{code:test:migration}
\end{listing}

\begin{listing}[ht]
  \caption{Výpis zón na jednotlivých serverech po migraci}  
  \begin{minted}[fontsize=\small]{bash}
zadmin@shost:~$ zoneadm list -vic
  ID NAME     STATUS            BRAND      IP    
   0 global   running           solaris    shared

zadmin@shost1:~$ zoneadm list -vic
  ID NAME     STATUS            BRAND      IP    
   0 global   running           solaris    shared
   
zadmin@shost2:~$ zoneadm list -vic
  ID NAME     STATUS            BRAND      IP    
   0 global   running           solaris    shared
   - zmigr1   installed         solaris    excl  
   - zmigr2   installed         solaris    excl
   - zmigr3   installed         solaris    excl  
   - zmigr    installed         solaris    excl  
  \end{minted}
  \label{code:test:migration:list}
\end{listing}

\begin{listing}[ht]
  \caption{Výpis uživatelského žurnálu po migraci zón}  
  \begin{minted}[fontsize=\small]{bash}
zadmin@shost:~$ szmgmt_cli journal status
Geting fresh information about zones on all registered hosts...
Tracked zones:  
  Host shost2
      zmigr1:shost2
           Zone type: solaris
          Zone state: installed
           Zone path: /system/zones/zmigr1
      zmigr2:shost2
           Zone type: solaris
          Zone state: installed
           Zone path: /system/zones/zmigr2
      zmigr:shost2
           Zone type: solaris
          Zone state: installed
           Zone path: /system/zones/zmigr  
      zmigr3:shost1
           Zone type: solaris
          Zone state: running
           Zone path: /system/zones/zmigr3
  \end{minted}
  \label{code:test:migration:after}
\end{listing}


\begin{listing}[ht]
  \caption{Výpis příkazu pro vytvoření zálohy zón pomocí archivu UAR}  
  \begin{minted}[fontsize=\small]{bash}
zadmin@shost:~$ szmgmt_cli backup zback:shost2 \
                                  zback1:shost2 \
                                  zback2:shost1 \
                                  zback3:shost1
                -d shost -p /zonepool/backup -t uar
Solaris zones backup initialized (UAR).
  ---------------------------------------------------------
  Options:
          Backup directory: /zonepool/backup
          Destination host: shost
  ---------------------------------------------------------
  Connecting concurrently to hosts 'shost2, shost1'.
  Processing zone backup of 'zback2:shost1'. 
    See log '~/.szmgmt/log/zback2_backup_syikav.log'.
  Processing zone backup of 'zback:shost2'. 
    See log '~/.szmgmt/log/zback_backup_71blh4.log'.
  Processing zone backup of 'zback1:shost2'. 
    See log '~/.szmgmt/log/zback1_backup_5rks8b.log'.
  Processing zone backup of 'zback3:shost1'. 
    See log '~/.szmgmt/log/zback3_backup_shmtgw.log'.
  ---------------------------------------------------------
  Backup finished.
    Status:
      shost2:
        zback: success
        zback1: success
      shost1:
        zback2: success
        zback3: success
  \end{minted}
  \label{code:test:backup}
\end{listing}

\begin{listing}[ht]
  \caption{Výpis příkazu pro obnovení zón ze zálohy typu UAR}  
  \begin{minted}[fontsize=\small]{bash}
zadmin@shost:~$ szmgmt_cli recover zback:shost2 \
                                   zback1:shost2 \
                                   zback2:shost1 \
                                   zback3:shost1 
               -a \ 
               /zonepool/backup/zback_backup_1525020859.uar \
               /zonepool/backup/zback1_backup_1525020859.uar \
               /zonepool/backup/zback2_backup_1525020859.uar \
               /zonepool/backup/zback3_backup_1525020859.uar
Solaris zones recovery initialized.
  ---------------------------------------------------------
  Options:
                Boot zones: disable
    Rewrite existing zones: disable
  ---------------------------------------------------------
  Connecting concurrently to hosts 'shost2, shost1' .
  Processing zone 'zback1' recovery on host 'shost2'.
    Aarchive: /zonepool/backup/zback1_backup_1525020859.uar.
    See log '~/.szmgmt/log/zback1_recovery_vai7x.log'.
  Processing zone 'zback2' recovery on host 'shost1'.
    Archive: /zonepool/backup/zback2_backup_1525020859.uar.
    See log '~/.szmgmt/log/zback2_recovery_lpl0c1.log'.
  Processing zone 'zback' recovery on host 'shost2'.
    Archive: /zonepool/backup/zback_backup_1525020859.uar.
    See log '~/.szmgmt/log/zback_recovery_7ecl60.log'.
  Processing zone 'zback3' recovery on host 'shost1'.
    Archive: /zonepool/backup/zback3_backup_1525020859.uar.
    See log '~/.szmgmt/log/zback3_recovery_7ta1fr.log'.
tes  ---------------------------------------------------------
  Recovery finished.
    Status:
      shost2:
        zback: success
        zback1: success
      shost1:
        zback2: success
        zback3: success
  \end{minted}
  \label{code:test:recovery}
\end{listing}

\end{document}