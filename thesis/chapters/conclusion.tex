% Chapter: Conclusion
% Author: Tomáš Šimáček
\label{chapte:conclusion}
Virtualizace se stala běžnou a možná i nezbytnou součástí dnešního počítačového světa. Teto technika se využívá v~mnoha oblastech
informačních technologií, kde přináší různé benefity. Virtualizace umožňuje společnostem efektivně využívat dostupné 
fyzické prostředky a tím výrazně ušetřit náklady na provoz fyzických zařízení. Tématem této diplomové práce byla virtualizace serverů,
která umožňuje současný běh několika virtuálních počítačů v~rámci jednoho fyzického systému.

Jednou z~technik virtualizace serverů je technologie Solaris Zones, která je součástí operačního systému Solaris vyvíjeného firmou Oracle.
Tato virtualizační technika umožňuje běh mnoha virtualizačních kontejnerů v~rámci jedné instance operačního systému Solaris. Tyto
kontejnery se nazývají zóny a jsou izolovány na úrovni počítačové sítě, souborového systému a spuštěných procesů. Standardně hlavní
operační systém sdílí svoje jádro s~ostatními zónami, ale tato technika umožňuje vytvářet i zóny s~vlastním jádrem.

Cílem teoretické části této diplomové práce bylo tuto virtualizační techniku popsat a porovnat s~ostatními běžně využívanými technikami
virtualizace. Dále se práce zaměřila na popis konfigurace, instalace a základních administrátorských rutin pro správu zón.
Předmětem byly také pokročilé rutiny vyžadují provedení několika kroků, jejichž postupným vykonáváním lze dosáhnout požadovaného výsledku. Z~tohoto
důvodu bylo hlavním cílem práce vytvořit nástroj, který by tyto procesy automatizoval a umožnil je provádět i pro vzdálené zóny.
Hlavním důvodem pro vytvoření tohoto nástroje je fakt, že standardní nástroje pro správu Solaris Zones tuto funkcionalitu neposkytují.
Jelikož Solaris Zones poskytují uživateli pouze rozhraní na příkazové řádce, bylo nutné implementovaný nástroj postavit právě nad tímto
rozhraním a využívat tak standardní nástroje pro správu zón.

Výsledná implementace nástroje se skládá z několika částí. Jednou z nich je modul Solaris Zones, který poskytuje funkcionalitu ostatním
částem nástroje. Jeho architektura se skládá z~několika hierarchicky uspořádaných vrstev. Nejníže v~hierarchii se nachází vrstva,
která umožňuje lokální i vzdálené spouštění nástrojů \textit{zonecfg}, \textit{zoneadm}, \textit{archiveadm} a v~neposlední řadě 
také \textit{zfs}. Ostatní vrstvy modulu využívají těchto příkazů a vytvářejí z~nich sekvence, které reprezentují jednotlivé 
administrátorské rutiny pro vytváření, mazání, zálohu, obnovu a migraci zón. Jednotlivé rutiny jsou implementované jako transakce. V~rámci
transakce je zajištěno, že všechny dočasně vytvořené soubory budou smazány a že všechny provedené změny budou navráceny
do původního stavu v~případě neúspěchu transakce. Takto implementované rutiny jsou poskytovány ostatním částem nástroje pomocí
předem definovaného rozhraní. 

Jelikož v~dnešní době existuje mnoho virtualizačních technik s~podobnými vlastnostmi jako Solaris Zones, byla architektura nástroje
navržena tak, aby se dala v~budoucnu lehce rozšířit o~další moduly. Tuto funkcionalitu zajišťuje v~rámci nástroje knihovna, která
jasně definuje rozhraní jednotlivých modulů. Pomocí tohoto rozhraní knihovna zprostředkovává funkcionalitu modulů klientským 
aplikacím. Ve výsledném nástroji je implementován pouze výše zmíněný modul pro podporu správy Solaris Zones. Tato knihovna také
definuje generickou šablonu, která slouží pro popis konkrétních virtuálních strojů. Výše popsaný modul implementuje rozšíření této
generické šablony, které umožňuje uživateli specifikovat konfiguraci, softwarové vybavení a systémové nastavení zón. Tyto šablony
je možné využít pro vytváření většího množství zón s~danými vlastnostmi.

Ovládání nástroje je zajištěno pomocí uživatelského rozhraní, které je uživateli prezentováno pomocí příkazové řádky. Jednotlivé
příkazy umožňují vyvolat odpovídající funkce modulu pro vytváření, mazání, zálohu, obnovu nebo migraci zón. Uživatel může jednoduše 
specifikovat pro jaké zóny chce danou akci provést. Tyto zóny se mohou nacházet na lokálním i vzdáleném serveru a nástroj
pro každou specifikovanou zónu vykoná konkrétní akci pokud možno paralelně. Uživatelské rozhraní dále implementuje 
systém pro správu vzdálených hostů, který umožňuje dané hosty registrovat a specifikovat parametry připojení. Hromadné akce
poskytované uživatelským rozhraním jsou prováděny právě pro všechny registrované hosty. Jelikož se zónami může manipulovat každý
privilegovaný uživatel, implementuje tato část nástroje také uživatelský žurnál. V~žurnálu jsou udržovány jednotlivé stavy zón, aby
bylo možné poznat, zda v~době nepřítomnosti uživatele nedošlo ke změně stavu zón. Další součástí uživatelského rozhraní je
grafický editor šablon, který uživateli poskytuje možnost interaktivní tvorby a editace šablon. Hlavním smyslem tohoto editoru
je odstínění uživatele od implementačních detailů šablon. Tohoto grafického rozhraní je využito i v~případě interaktivní instalace
zón, která uživateli nabízí možnost specifikace vlastností v~průběhu vytváření zón. 

Implementovaný nástroj umožňuje uživateli jednoduchou správu většího množství zón pomocí automatizovaných rutin pro vytváření,
mazání, zálohu, obnovu a migraci těchto zón. Součástí nástroje jsou šablony, které umožňují přesně specifikovat vlastnosti 
vytvářených zón. Tato funkcionalita se dá dobře využít v~procesu vývoje software pro definici produkčních, testovacích a 
vývojářských prostředí. Implementované uživatelské rozhraní prezentuje uživateli funkce nástroje v~jednoduché a přehledné formě.
Grafické součásti nástroje slouží především pro komfortní vytváření a editaci šablon.

Implementovaný nástroj pro podporu automatické správy virtualizačního
kontejneru Solaris Zones poskytuje očekávanou funkcionalitu a splňuje stanovené požadavky. Tato skutečnost byla ověřena v~rámci testování
tohoto nástroje, které je součástí této diplomové práce. Z těchto důvodů je možné konstatovat, že všechny stanovené cíle této diplomové práce 
byly naplněny. Instalační balík a zdrojové kódy implementovaného nástroje i zdrojové kódy celé diplomové práce jsou dostupné na přiloženém médiu.