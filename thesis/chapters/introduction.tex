% Chapter: Introduction
% Author: Tomáš Šimáček
\label{chapter:introduction}
Virtualizace je technika, se kterou se dnes v~IT můžeme setkat v~mnoha podobách. Jednou z~hlavních oblastí jejího využití,
je virtualizace serverů, ale objevuje se také v oblastech komunikačních sítí nebo desktopů. Tato technika umožňuje vytvářet
virtuální prostředky a poskytovat tak kompletní virtuální prostředí. Toto prostředí umožňuje provozovat systémy na~jiných
fyzických architekturách, než pro jaké jsou určeny.

Hlavním tématem této práce je virtualizace serverů, která umožňuje rozdělit jeden fyzický systém do několika nezávislých virtuálních
prostředí, ve~který jsou spouštěny virtuální počítače. Možnost vytváření virtuálních počítačů v~rámci jednoho fyzického systému
značně snižuje náklady na pořizování a~provoz fyzických serverů. Díky virtualizaci již není třeba pořizovat dedikovaný server pro
každou instanci operačního systému, který chce společnost provozovat. Správným rozdělením virtuálních počítačů na fyzické servery může být
docíleno ideální rozdělení zátěže a tím mohou být dostupné fyzické prostředky efektivně využity.

Rostoucí počet virtualizovaných serverů může mít za následek obtížnější správu. Automatizované nasazení, instalace nebo
zálohování virtuálních počítačů může být značným ulehčením správy počítačové infrastruktury, která využívá virtualizačních technik.
Díky tomuto ulehčení lze jednoduše vytvářet předem definovaná virtuální prostředí, která mohou složit pro~vývoj software, testování
nebo nasazení aplikací do produkčního prostředí. 

V~dnešní době existuje mnoho operačních systémů, které nějakým způsobem poskytují virtualizaci v~rámci svých služeb. Jedním ze zástupců
takovýchto systémů je operační systém Solaris. Exkluzivně pro tento operační systém byla vytvořena virtualizační technika Solaris
Zones, která umožňuje v~rámci jedné instance operačního systému Solaris vytvářet virtuální počítače nazývané zóny. Nástroje pro správu
této virtualizační technologie umožňují spravovat zóny pouze v~rámci lokálního serveru. Tato diplomová práce se věnuje právě možnosti
správy zón nacházejících se na vzdálených serverech a~automatizaci administrátorských procesů vytváření, zálohy, obnovy a migrace.
\section{Cíle práce}
\label{chapter:introduction:goals}
Cílem této diplomové práce je seznámení s~operačním systémem Solaris a~jeho aktuální stabilní verzí 11.3. Součástí popisu tohoto
operačního systému je i představení podporovaných architektur a jeho základních služeb pro správu softwarových balíčku nebo souborového
systému ZFS. Především jde však o~popis základních principů virtualizační techniky Solaris Zones, která umožňuje běh více zón v~rámci 
jedné instance operačního systému Solaris. Nebude chybět ani porovnání běžně používaných virtualizačních technik.

Diplomová práce také souvisí se~správou virtualizační techniky Solaris Zones a má za úkol detailně popsat možnosti konfigurace a instalace
zón. Součástí tohoto popisu bude i popis administrátorských procesů pro zálohování, obnovu a migraci. Popis se také věnuje integraci
techniky Solaris Zones s~ostatními službami operačního systému Solaris.

Hlavním cílem této diplomové práce je návrh a implementace nástroje pro podporu automatické správy virtualizačního kontejneru
Solaris Zones na platformě Solaris. Jelikož základní nástroje pro správu této virtualizační technologie neumožňují správu vzdálených
zón, implementovaný nástroj bude tuto funkcionalitu podporovat. Dále tento nástroj bude umožňovat provádění základních administrátorských
procesů pro větší množství lokálních i~vzdálených zón. Mezi těmito procesy bude zahrnuta automatická konfigurace, instalace, náhrada,
záloha, obnova a migrace zón v~rámci několika virtualizačních serverů. Nástroj bude umožňovat definici softwarových balíků, které mají být
při instalaci do zóny zahrnuty a to pomocí šablon nebo interaktivně pomocí uživatelského rozhraní.

Posledním cílem této diplomové práce je otestovat implementovaný nástroj. Testovány budou hlavní scénáře využití výsledného nástroje
a bude změřena doba běhu pro určité funkce nástroje.
\section{Struktura práce}
\label{chapter:introduction:structure}
Struktura této diplomové práce se skládá ze šesti hlavním kapitol a příloh. První kapitola práce se zabývá obecným popisem virtualizace
a jejím využitím v~informačních technologiích. Dále tato kapitola definuje pojem virtuálního stroje a představuje jednotlivé druhy
virtualizačních technik. V~závěru první kapitoly je zmíněno několik hlavních scénářů pro nasazení virtuální infrastruktury. Druhá kapitola
stručně představuje operační systém Solaris. Důraz je kladen především na podporované platformy a služby, které
tento operační systém poskytuje. Třetí kapitola obsahuje podrobný popis virtualizační techniky Solaris Zones. Úvod této kapitoly
popisuje základní principy a typy zón, které je možné v~rámci této technologie vytvářet. Ve zbytku kapitoly jsou představeny
konkrétní způsoby správy této virtualizační techniky, které se zaměřují na popis konfigurace, instalace, zálohy a~migrace zón. 
Čtvrtá kapitola se zabývá návrhem nástroje pro podporu automatické správy virtualizačního kontejneru Solaris Zones. Hlavním obsahem
této kapitoly je stanovení požadavků na výsledný nástroj a navržení jeho architektury. V~závěru této kapitoly jsou popsány některé
bezpečnostní aspekty, které by měl uživatel nástroje splňovat. Pátá kapitola popisuje způsob implementace nástroje pro podporu automatické
správy virtualizačního kontejneru Solaris Zones. V~úvodu této kapitoly je popsána volba programovacího jazyka, pomocí kterého byl výsledný
nástroj implementován. Zbytek kapitoly popisuje implementaci šablon, uživatelského rozhraní a ostatních funkčních komponent výsledného
nástroje. Poslední kapitola je věnována testování implementovaného nástroje. Hlavním obsahem kapitoly je popis testování hlavních
scénářů využití nástroje pro správu Solaris Zones. Praktické ukázky testování jsou obsaženy v~příloze. Závěr této kapitoly se zabývá měřením
doby běhu některých funkcí výsledného nástroje a rozebírá výsledky měření. 

Zdrojové kódy celé diplomové práce a implementovaného nástroje pro podporu automatické správy virtualizačního kontejneru Solaris Zones
jsou dostupné na~přiloženém médiu.




