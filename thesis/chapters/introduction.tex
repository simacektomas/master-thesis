% Chapter: Introduction
% Author: Tomáš Šimáček
Virtualizace je technika, se kterou se dnes v IT můžeme setkat v mnoha podobách. Jednou z hlavních oblastí využití virtualizace je virtualizace serverů a mimo jiné se objevuje i v oblasti
komunikačních sítí a desktopů. Tato technika umožňuje vytvořit virtuální prostředí nebo prostředky, které se mohou chovat jinak nebo k nim může být přistupováno jiným způsobem, než
k těm fyzickým. 

Virtualizace serverů  virtualizaci serverů můžeme jednoduše vytvořit více virtuálních počítačů na jednom fyzickém.
\section{Cíle práce}

Prvním cílem této práce je popis virtualizační techniky Solaris Zones od firmy Oracle. Důraz je kladen na popis základních principů, které umožňují běh více zón v rámci jednoho
sdíleného jádra OS.

Dalším cílem je detailní popis možností konfigurace zón, jejich instalace, zálohování a v neposlední řadě také integrace Solaris Zones s ostatními službami operačního systému
Solaris.

Třetí cíl této práce je porovnat Solaris Zones s ostatními virtualizačními technologemi.

Posledním cílem této práce je implementace nástroje, který umí spravovat větší množství Solaris Zones. Tento nástroj bude umožňovat (dávkovou a interaktivní) instalaci zón na
lokální i vzdálené servery, náhradu existujících zón a jejich zálohování. Dále bude umožňovat automatické přidání předem definovaných softwarových balíčků po instalaci zóny.

\section{Struktura práce}


