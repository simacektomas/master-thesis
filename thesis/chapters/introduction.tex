% Chapter: Introduction
% Author: Tomáš Šimáček
Virtualizace je technika, se kterou se dnes v IT můžeme setkat v mnoha podobách. Jednou z hlavních oblastí využití virtualizace je virtualizace serverů a mimo jiné se objevuje i v oblasti
komunikačních sítí a desktopů. Tato technologie umožňuje vytvářet virtuální prostředí nebo prostředky na fyzickým hardware. Speciální softwarová vrstva zvaná virtualizační monitor (VMM)
zajišťuje efektivní rozdělování prostředků fyzického systému mezi virtualizované subjekty.

Hlavním tématem této práce je virtualizace serverů, která umožňuje rozdělit jeden fyzický systém na několik nezávislých virtuálních prostředí zvaných virtuální počítač (VM).
Možnost vytváření VM značně snižuje náklady na pořizování a provoz fyzických strojů, jelikož už není třeba dedikovaný server pro každou instanci OS. A konečně správným rozdělením
VMs na fyzické servery můžeme docílit ideálního rozdělení zátěže a tím efektivně využít dostupné fyzické prostředky.

Rostoucí počet virtualizovaných serverů může mít za následek obtížnější správu. Automatizované nasazování, instalace nebo zálohování VMs může být značným ulehčením vývoje software,
testování nebo nasazování aplikací do produkčního prostředí. Tomuto tématu se tato práce věnuje v souvislosti s virtualizačním kontejnerem Solaris Zones.

\section{Cíle práce}

Prvním cílem této práce je seznámení se s operačním systémem Solaris a jeho funkcemi. Především jde o popis virtualizační techniky Solaris Zones. Důraz je kladen na popis
základních principů, které umožňují běh více zón v rámci jednoho sdíleného jádra OS.

Dalším cílem je detailní popis možností konfigurace zón, jejich instalace, zálohování a v neposlední řadě také integrace Solaris Zones s ostatními službami operačního systému
Solaris.

Třetí cíl této práce je porovnat Solaris Zones s ostatními virtualizačními technologemi.

Posledním cílem této práce je implementace nástroje, který umí spravovat větší množství Solaris Zones. Tento nástroj bude umožňovat (dávkovou a interaktivní) instalaci zón na
lokální i vzdálené servery, náhradu existujících zón a jejich zálohování. Dále bude umožňovat automatické přidání předem definovaných softwarových balíčků po instalaci zóny.

\section{Struktura práce}

TODO

