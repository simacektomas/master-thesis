% Detailed description of Solaris Zones configuration, installation, backup etc.
V úvodu následující kapitoly jsou popsány základní principy a struktura virtualizační techniky Solaris Zones od
firmy Oracle. Dále se tato kapitola zabývá popisem datových struktur, postupů a nástrojů, které slouží ke správě a manipulaci
se zónami. Detailnější popis je věnován především administrátorským rutinám pro konfiguraci, instalaci a zálohování zón.


\section{Virtualizační technika}
Oracle Solaris Zones je virtualizační technika, která umožňuje běh více virtuálních strojů na jednom fyzickém stroji. Jak již
bylo zmíněno v kapitole \ref{chapter:solaris}, tato technika je standardní součástí operačního systému Oracle Solaris od 
verze systému Solaris 10. Aktuální verze je Solaris 11.3 a přináší novou funkcionalitu v podobě podpory starších verzí 
operačního systému Solaris.

Pokud bychom chtěli zařadit Solaris Zones do klasifikace virtuálních strojů popsané v kapitole \ref{section:clasification}, 
pak bychom jí zařadili do sekce \textit{virtualizace na úrovni OS}. Jinak řečeno, tato technika rozděluje zdroje hostitelského
operačního systému jako CPU, paměť nebo I/O zařízení mezi běžící virtuální stroje a zajišťuje izolaci na úrovni procesů,
souborového systému a sítě. Zónu pak můžeme definovat jako virtuální kontejner běžící v hostitelském operačním systému, který
využívá zdrojů hostitelského operačního systému a je izolovaný od ostatních zón.

Standardně se po instalaci operačního systému Solaris nachází v systému jedna zóna. Je to vlastní instance operačního systému
a nazývá se \textbf{globální} zóna. Jinými slovy, je to zóna, která běží přímo na hardwaru počítače nebo ve virtualizovaném
prostředí. Tato zóna má dvě hlavní funkce. Za prvé tato zóna plní funkci hlavního operačního systému a přebírá kontrolu
nad fyzickými prostředky po startu systému. Za druhé je hlavním centrálním prvkem pro administraci celého systému a ostatních
zón. Globální zóna poskytuje globální pohled na celý systém a má přehled o všech systémových zdrojích a aktivitách ostatních
zón. Její role v rámci Solaris Zones je zásadní a její chyba může zapříčinit pád ostatních zón. Z tohoto důvodu je doporučené
používat globální zónu pouze pro účely administrace systému a managementu ostatních zón.

Zóny, které jsou spouštěny v rámci globální zóny nazýváme \textbf{neglobální} zóny. Tyto zóny jsou navzájem izolované na
několika úrovních. První úroveň izolace je izolace na úrovni sítě. Každá neglobální zóna může mít svůj vlastní logický síťový
adaptér, který je vytvořený nad nějakým fyzickým síťovým rozhraním a je dostupný pouze pro konkrétní zónu. Z jiné neglobální
zóny tento adaptér není přístupný. Takto vytvořený síťový adaptér může být spravován pouze z globální zóny a ze zóny, ke které
byl přiřazen v průběhu jejího vytváření.

Druhou úrovní izolace zón je souborový systém. Globální zóna spravuje svůj vlastní souborový systém ve kterém se nachází
standardní adresářová struktura operačních systému typu UNIX. Můžeme v něm nalézt adresář \textit{/etc} sloužící pro globální
systémovou konfiguraci, adresář \textit{/bin} obsahující uživatelsky spustitelné programy nebo například adresář \textit{/sbin},
který uchovává programy spustitelné pod privilegovaným uživatelem. Každá neglobální zóna potřebuje pro svůj běh velmi podobné
prostředí, a proto má svůj vlastní souborový systém, který se nachází někde v hierarchii souborového systému globální zóny.
Podle typu zón popsaných v kapitole \ref{chapter:zones:types} může být tento souborový systém částečně sdílený se souborovým
systémem globální zóny a nebo může obsahovat kompletně nezávislý obraz zóny. Při spouštění zóny pak dojde pomocí příkazu
\verb|chroot(1)| k přepnutí kořenu souborového systému a zóna pracuje pouze se svojí částí souborového systému. V případě 
sdílené části souborového systému jsou tyto části připojeny v režimu read-only \cite{virt1}. Tím je zajištěno, že souborové
systémy jednotlivých zón jsou vzájemně izolované a nemohou se vzájemně ovlivnit. Správu těchto souborových systémů je opět
možné provádět pouze z konkrétní zóny a nebo ze zóny globální.



Druhou funkcí globální zóny je centrální administrace celého systému a ostatních zón. Zónám, které jsou spouštěny v rámci
globální zóny se říká \textbf{neglobální} zóny a nemají žádný přístup do globální ani jiné neglobální zóny. Konfigurace, 
instalace a další správa neglobálních zón jako přihlašování do konzole, se dá provádět pouze ze zóny globální. 

ve kterém můžeme spouštět procesy izolovaně od ostatních virtuálních kontejnerů
.

Tento kontejner tedy stanovuje
softwarové hranice a umožňuje procesům v něm spuštěných přímo komunikovat pouze s procesy ve stejném kontejneru (zóně). Pokud
chce proces komunikovat s procesem z jiné zóny, musí k tomu využít počítačovou síť \cite{oracle:solaris:zones:introduction}.
Z pohledu softwarového vývojáře i uživatele se tedy jedná o dva oddělené systémy, i když reálně běží na jednom fyzickém
systému.







Neglobální zóny se chovají jako nezávislé operační systémy a izolují procesy spuštěné v nich. Každá neglobální zóna má svůj
vlastní souborový systém, který má svůj kořenový adresář umístěný v souborovém systému globální zóny. Díky úzké spolupráci 
zón se souborovým systémem ZFS je správa těchto souborových systému jednoduchá a umožňuje využívat přednosti tohoto
souborového systému i v rámci Solaris Zones.

zóny mají svůj vlastní souborový systém, vlastní síťové rozhraní, vlastní plánovač procesů a vlastní softwarové
balíčky. Všechny
tyto zdroje si neglobální zóny samy spravují. Nicméně všechny tyto prostředky jsou vytvořeny v globální zóně a delegovány 
jednotlivým neglobálním zónám.


\subsection{Typy zón}
\label{chapter:zones:types}



\subsection{Virtuální síť}

\section{Nástroje pro správu}

\section{Konfigurace}

\section{Instalace}

\section{Zálohování}