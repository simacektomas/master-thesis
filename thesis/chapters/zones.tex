% Detailed description of Solaris Zones configuration, installation, backup etc.
V úvodu následující kapitoly jsou popsány základní principy a struktura virtualizační techniky Solaris Zones od
firmy Oracle. Dále se tato kapitola zabývá popisem datových struktur, postupů a nástrojů, které slouží ke správě a manipulaci
se zónami. Detailnější popis je věnován především administrátorským rutinám pro konfiguraci, instalaci a zálohování zón.
\section{Virtualizační technika}
Oracle Solaris Zones je virtualizační technika, která umožňuje běh více virtuálních strojů na jednom fyzickém stroji. Jak již
bylo zmíněno v kapitole \ref{chapter:solaris}, tato technika je standardní součástí operačního systému Oracle Solaris od 
verze systému Solaris 10. Aktuální verze je Solaris 11.3 a přináší novou funkcionalitu v podobě podpory starších verzí 
operačního systému Solaris.

Pokud bychom chtěli zařadit Solaris Zones do klasifikace virtuálních strojů popsané v kapitole \ref{section:clasification}, 
pak bychom jí zařadili do sekce \textit{virtualizace na úrovni OS}. Jinak řečeno, tato technika rozděluje zdroje hostitelského
operačního systému jako CPU, paměť nebo I/O zařízení mezi běžící virtuální stroje a zajišťuje izolaci na úrovni procesů,
souborového systému a sítě. Zónu pak můžeme definovat jako virtuální kontejner běžící v hostitelském operačním systému, který
využívá zdrojů hostitelského operačního systému a je izolovaný od ostatních zón.

Standardně se po instalaci operačního systému Solaris nachází v systému jedna zóna. Je to vlastní instance operačního systému
a nazývá se \textbf{globální} zóna. Jinými slovy, je to zóna, která běží přímo na hardwaru počítače nebo ve virtualizovaném
prostředí. Tato zóna má dvě hlavní funkce. Za prvé tato zóna plní funkci hlavního operačního systému a přebírá kontrolu
nad fyzickými prostředky po startu systému. Za druhé je hlavním centrálním prvkem pro administraci celého systému a ostatních
zón. Globální zóna poskytuje globální pohled na celý systém a má přehled o všech systémových zdrojích a aktivitách ostatních
zón. Její role v rámci Solaris Zones je zásadní a její chyba může zapříčinit pád ostatních zón. Z tohoto důvodu je doporučené
používat globální zónu pouze pro účely administrace systému a managementu ostatních zón.

Zóny, které jsou spouštěny v rámci globální zóny nazýváme \textbf{neglobální} zóny. Tyto zóny jsou navzájem izolované na
několika úrovních. První úroveň izolace je izolace na úrovni sítě. Každá neglobální zóna může mít svůj vlastní logický síťový
adaptér, který je vytvořený nad nějakým fyzickým síťovým rozhraním a je dostupný pouze pro konkrétní zónu. Z jiné neglobální
zóny tento adaptér není přístupný. Takto vytvořený síťový adaptér může být spravován pouze z globální zóny a ze zóny, ke které
byl přiřazen v průběhu jejího vytváření.

Druhou úrovní izolace zón je souborový systém. Globální zóna spravuje svůj vlastní souborový systém ve kterém se nachází
standardní adresářová struktura operačních systému typu UNIX. Můžeme v něm nalézt adresář \textit{/etc} sloužící pro globální
systémovou konfiguraci, adresář \textit{/bin} obsahující uživatelsky spustitelné programy nebo například adresář \textit{/sbin},
který uchovává programy spustitelné pod privilegovaným uživatelem. Každá neglobální zóna potřebuje pro svůj běh velmi podobné
prostředí, a proto má svůj vlastní souborový systém, který se nachází někde v hierarchii souborového systému globální zóny.
Podle typu zón popsaných v kapitole \ref{chapter:zones:types} může být tento souborový systém částečně sdílený se souborovým
systémem globální zóny a nebo může obsahovat kompletně nezávislý obraz zóny. Při spouštění zóny pak dojde pomocí příkazu
\verb|chroot(1)| k přepnutí kořenu souborového systému a zóna pracuje pouze se svojí částí souborového systému. V případě 
sdílené části souborového systému jsou tyto části připojeny v režimu read-only \cite{virt1}. Tím je zajištěno, že souborové
systémy jednotlivých zón jsou vzájemně izolované a nemohou se vzájemně ovlivnit. Správu těchto souborových systémů je opět
možné provádět pouze z konkrétní zóny a nebo ze zóny globální.

Mimo izolace na úrovni sítě a souborového systému Solaris Zones implementuje ještě izolaci na úrovni procesů. Každá neglobální
zóna má svůj plánovač a může spouštět svoje vlastní procesy. Procesy běžící v jedné neglobální zóně nejsou žádným způsobem 
viditelné ani přímo ovlivnitelné z jiných neglobálních zón. Pokud chce proces z jedné zóny komunikovat s procesem z jiné zóny,
nemůže k tomu využít mezi procesovou komunikaci, ale musí použít počítačové sítě. Naopak procesy, které běží v rámci jedné
zóny spolu mohou komunikovat pomocí signálů, sdílené paměti a tak podobně. Všechny procesy běžící v systému mohou být
spravovány z globální zóny. Globální zóna tedy nabízí globální přehled všech procesů, které jsou spuštěny ve všech běžících
zónách v systému. Výstup příkazu \verb|ps(1)| v globální zóně zobrazí všechny procesy, zatímco v neglobální zóně budou zobrazeny
pouze procesy příslušící dané zóně.
\subsection{Typy zón}
\label{chapter:zones:types}
Jak bylo zmíněno výše, virtualizační technika Solaris Zones umožňuje v rámci jedné globální zóny spouštět mnoho neglobálních
zón. Každá neglobální zóna má vlastnost zvanou \textit{brand}, která určuje typ neglobální zóny. Tato vlastnost se specifikuje
při konfiguraci zóny v globální zóny a dle přehledu \cite{oracle:solaris:zones:brands} může typ zóny mít následující hodnoty:

\begin{itemize}
 \item \textit{solaris}
 \item \textit{solaris-kz}
 \item \textit{solaris10}
\end{itemize}

\textit{Brand} neboli typ zóny určuje jakým způsobem se zóna bude po spuštění chovat. Implicitním typem zóny v Solaris Zones
je \textit{solaris}, kterému se také jinak přezdívá nativní zóna nebo také tenká zóna. Dalším typem zóny je \textit{solaris-kz},
kde zkratka za pomlčkou v názvu odpovídá slovnímu spojením kernel zone. Jak název napovídá, tato zóna má vlastní jádro
operačního systému a někdy se jí také přezdívá plná nebo tlustá zóna. Posledním typem zón, kterou Solaris Zones umí vytvářet
je \textit{solaris10}. Hlavním úkolem této zóny je zajišťovat zpětnou kompatibilitu s operačním systémem Solaris 10 a umožňuje
uvnitř této zóny spouštět aplikace určené pro tento systém.

V následujících podkapitolách jsou podrobněji popsány výše zmíněné typy zón.
\subsubsection{Nativní zóna}
\label{chapter:zones:native}
Nativní neboli tenká zóna umožňuje administrátorovi vytvořit zónu, která má sdílené jádro operačního systému s globální zónou.
Jinými slovy verze jádra operačního systému musí být stejná jako v globální zóně. Tento typ zóny je izolovaný pouze nad svým
souborovým systémem a nemá standardně nemá k dispozici informace o žádném fyzickém zařízení systému. Souborové systémy ostatních
zón jsou nedostupné a konkrétní neglobální zóna o nich nemá žádné informace. Z jejího pohledu existuje pouze její kořenový
souborový systém. Jak již bylo popsáno výše, kořenový sytém nativní zóny může být sdílený se souborovým systémem globální
zóny a sdílet tak základní systémové nástroje. Pokud chceme této zóně delegovat nějaký typ zařízení, musíme tak učinit při
konfiguraci zóny v globální zóně. Tímto způsobem můžeme nativní zóně zpřístupnit souborové systémy, ZFS pool nebo
ZFS dataset. Takto definované prostředky pak po instalaci zóny můžeme z této zóny využívat.

Dále tento typ zóny má svoji vlastní databázi produktů, která obsahuje informace o všech nainstalovaných softwarových
komponentech v konkrétní globální zóně. Opět platí, že konkrétní neglobální zóna vidí pouze své balíčky. Díky tomu je možné
instalovat dodatečné softwarové balíčky do neglobálních zón, které nemusí být nainstalované v globální zóně
\cite{oracle:solaris:zones:brands}. Některé softwarové balíčky jsou ale společné s globální zónou (kernel) a nelze tedy
provádět kompletní aktualizaci bez zásahu do globální zóny. 

Nativní zóna podporuje dva typy síťových rozhraní, které mohou být zóně při konfiguraci přiřazeny. Prvním typem je sdílená 
adresa neboli \textit{shared-ip}. Tento typ síťového rozhraní sdílí ip adresu s nějakým fyzickým rozhraním globální zóny. 
Pokud chce neglobální zóna komunikovat s okolím, bude v hlavičce paketu ip adresa globální zóny a při obdržení odpovědi
globální zóna přesměruje paket na virtuální síťové rozhraní konkrétní globální zóny. Zde můžeme pozorovat podobnost s
technikou NAT v sítích. Jako druhý typ rozhraní můžeme použít exkluzivní rozhraní nebo také \textit{exclusive-ip}, které
nesdílí ip adresu s globální zónou, ale má svoji vlastní. V tomto případě veškerý síťový provoz generovaný touto zónou bude
mít v hlavičce jinou ip adresu než zóna globální.

Tento typ zóny nepodporuje vytváření další neglobálních zón. Jinými slovy se nativní neglobální zóna nemůže chovat jako
globální zóna a vytvářet nové zóny uvnitř sebe. Stejně tak z nativní zóny nemůžeme vytvářet ani spravovat jiné neglobální zóny.

Nativní zóna je implicitní typ zóny v Solaris Zones a pokud administrátor nespecifikuje jinak při vytváření zóny, bude nově
vytvořená zóna právě typu \textit{solaris}. Tento typ zóny může být provozován na všech systémech, které podporují operační
systém Oracle Solaris 11.3 \cite{oracle:solaris:zones:brands}.
\subsubsection{Kernel zóna}
\label{chapter:zones:kernel}
Druhým typem zóny, který virtualizační technika Solaris Zones umožňuje vytvářet, je kernel zóna nebo také tlustá zóna. Tento
typ zóny obsahuje vlastní jádro operačního systému a na rozdíl od nativní zóny ho nesdílí s globální zónou. Kernel zóna
tedy může být provozována na jiné verzi jádra než globální zóna. V důsledku toho kernel zóna podporuje funkcionalitu,
které nelze pomocí nativní zóny dosáhnout.

Stejně jako v případě nativní zóny i kernel zóna obsahuje vlastní databázi instalovaných softwarových balíčků. Jelikož i 
kernel zóna je neglobální, nelze z ní žádným způsobem vidět balíčky ostatních zón. Na rozdíl od nativní zóny administrátor
může provádět aktualizaci všech balíčků, protože kernel zóna nesdílí nic s globální zónou. Žádné balíčky tedy nejsou závislé
na balíčkách globální zóny.

V případě síťových rozhraní kernel zóny podporují pouze rozhraní typu \textit{exclusive-ip} a neumožňuje sdílet síťovou adresu
s globální zónou.

Na rozdíl od nativní zóny se kernel zóny mohou chovat jako globální zóny uvnitř hostitelské globální zóny. Jinými slovy je
možné uvnitř kernel zóny vytvářet další neglobální zóny a vytvářet tak hierarchickou strukturu virtuální strojů. Je na zvážení
administrátora, jestli daný scénář vyžaduje tuto strukturu.
\subsubsection*{Požadavky}
Jelikož provoz kernel zóny se liší od provozu standardní nativní zóny, liší se i požadavky na hostitelský systéme. Požadavky
se liší v závislosti na platformě. Pro jednoduchost si uvedeme pouze požadavky pro systémy s architekturou x86 a procesorem 
intel. Podle specifikace \cite{oracle:solaris:zones:kernel_zones_requiremets} se na hostitelský systém kladou následující
požadavky.

\begin{itemize}
 \item Procesor musí být typu Nehalem nebo novější
 \item Virtualizace CPU (VT-x) 
 \item Podpora virtualizace paměti (RVI, EPT)
 \item Ochrana paměti
\end{itemize}

Výše zmíněné požadavky kladou nároky na HW vybavení hostitelského systému. Spolu s těmito požadavky musí být v globální zóně
nainstalovaný balíček \textit{brand/brand-solaris-kz}, který umožňuje vytváření kernel zón. Administrátor může pomocí příkazu
\verb|virtinfo(1)| zjistit, jaký typ virtualizace je v globální zóně podporován. Výpis programu na virtualizované platformě
VMvare, kde jsou splněné výše zmíněné požadavky může vypadat následovně.

\begin{verbatim}
zadmin@shost:~$ virtinfo
NAME            CLASS     
vmware          current   
non-global-zone supported
kernel-zone     supported
\end{verbatim}


\subsubsection{Branded zóna}
\label{chapter:zones:branded}
Branded zóny byly vytvořeny pro zpětné zajištění kompatibility se staršími verzemi operačního sytému Solaris. Díky technologii
\textit{BrandZ} \cite{oracle:solaris:zones:brands} umožňují spouštění aplikací určených pro operační systém Solaris 10 na 
systému s OS Solaris 11. Aplikace mohou běžet v nezměněné formě v bezpečném prostředí, které je zajištěno neglobální zónou.

Z pohledu administrátora se tento typ zóny chová stejně jako nativní zóna a má stejné vlastnosti, které jsou popsané v
podkapitole \ref{chapter:zones:native}.
\subsubsection{Shrnutí}
\label{chapter:zones:summary}
Virtualizační technologie Solaris Zones umožňuje vytvářet neglobální zóny uvnitř primární globální zóny. Neglobální zóny
poskytují izolované prostředí pro nezávislý a bezpečný běh aplikací. Zóny jsou izolovány na úrovni počítačové sítě,
souborového systému a běžících procesů, čímž je zajištěno, že se vzájemně nemohou přímo ovlivňovat. Jediný způsob komunikace
procesů z jiných zón je pomocí počítačové sítě. 

Neglobální zóna může být několika druhů, které poskytují různé vlastnosti. Tabulka \ref{table:zone_comparison} poskytuje stručný
přehled základních vlastností jednotlivých druhů zón.
\begin{table}
  \centering  
  \caption[Porovnání typů zón a jejich vlastností]{Porovnání typů zón a jejich vlastností}
  \label{table:zone_comparison}
\end{table}
\section{Administrace}
\label{chapter:zones:administration}

\section{Konfigurace}

\section{Instalace}

\section{Zálohování}