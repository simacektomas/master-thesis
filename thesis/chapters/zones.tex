% Detailed description of Solaris Zones configuration, installation, backup etc.
V úvodu následující kapitoly jsou popsány základní principy a struktura virtualizační techniky Solaris Zones od
firmy Oracle. Dále se tato kapitola zabývá popisem datových struktur, postupů a nástrojů, které slouží ke správě a manipulaci
se zónami. Detailnější popis je věnován především administrátorským rutinám pro konfiguraci, instalaci a zálohování zón.
\section{Virtualizační technika}
Oracle Solaris Zones je virtualizační technika, která umožňuje běh více virtuálních strojů na jednom fyzickém stroji. Jak již
bylo zmíněno v kapitole \ref{chapter:solaris}, tato technika je standardní součástí operačního systému Oracle Solaris od 
verze systému Solaris 10. Aktuální verze je Solaris 11.3 a přináší novou funkcionalitu v podobě podpory starších verzí 
operačního systému Solaris.

Pokud bychom chtěli zařadit Solaris Zones do klasifikace virtuálních strojů popsané v kapitole \ref{section:clasification}, 
pak bychom jí zařadili do sekce \textit{virtualizace na úrovni OS}. Jinak řečeno, tato technika rozděluje zdroje hostitelského
operačního systému jako CPU, paměť nebo I/O zařízení mezi běžící virtuální stroje a zajišťuje izolaci na úrovni procesů,
souborového systému a sítě. Zónu pak můžeme definovat jako virtuální kontejner běžící v hostitelském operačním systému, který
využívá zdrojů hostitelského operačního systému a je izolovaný od ostatních zón.

Standardně se po instalaci operačního systému Solaris nachází v systému jedna zóna. Je to vlastní instance operačního systému
a nazývá se \textbf{globální} zóna. Jinými slovy, je to zóna, která běží přímo na hardwaru počítače nebo ve virtualizovaném
prostředí. Tato zóna má dvě hlavní funkce. Za prvé tato zóna plní funkci hlavního operačního systému a přebírá kontrolu
nad fyzickými prostředky po startu systému. Za druhé je hlavním centrálním prvkem pro administraci celého systému a ostatních
zón. Globální zóna poskytuje globální pohled na celý systém a má přehled o všech systémových zdrojích a aktivitách ostatních
zón. Její role v rámci Solaris Zones je zásadní a její chyba může zapříčinit pád ostatních zón. Z tohoto důvodu je doporučené
používat globální zónu pouze pro účely administrace systému a managementu ostatních zón.

Zóny, které jsou spouštěny v rámci globální zóny nazýváme \textbf{neglobální} zóny. Tyto zóny jsou navzájem izolované na
několika úrovních. První úroveň izolace je izolace na úrovni sítě. Každá neglobální zóna může mít svůj vlastní logický síťový
adaptér, který je vytvořený nad nějakým fyzickým síťovým rozhraním a je dostupný pouze pro konkrétní zónu. Z jiné neglobální
zóny tento adaptér není přístupný. Takto vytvořený síťový adaptér může být spravován pouze z globální zóny a ze zóny, ke které
byl přiřazen v průběhu jejího vytváření.

Druhou úrovní izolace zón je souborový systém. Globální zóna spravuje svůj vlastní souborový systém ve kterém se nachází
standardní adresářová struktura operačních systému typu UNIX. Můžeme v něm nalézt adresář \textit{/etc} sloužící pro globální
systémovou konfiguraci, adresář \textit{/bin} obsahující uživatelsky spustitelné programy nebo například adresář \textit{/sbin},
který uchovává programy spustitelné pod privilegovaným uživatelem. Každá neglobální zóna potřebuje pro svůj běh velmi podobné
prostředí, a proto má svůj vlastní souborový systém, který se nachází někde v hierarchii souborového systému globální zóny.
Podle typu zón popsaných v kapitole \ref{chapter:zones:types} může být tento souborový systém částečně sdílený se souborovým
systémem globální zóny a nebo může obsahovat kompletně nezávislý obraz zóny. Při spouštění zóny pak dojde pomocí příkazu
\verb|chroot(1)| k přepnutí kořenu souborového systému a zóna pracuje pouze se svojí částí souborového systému. V případě 
sdílené části souborového systému jsou tyto části připojeny v režimu read-only \cite{virt1}. Tím je zajištěno, že souborové
systémy jednotlivých zón jsou vzájemně izolované a nemohou se vzájemně ovlivnit. Správu těchto souborových systémů je opět
možné provádět pouze z konkrétní zóny a nebo ze zóny globální.

Mimo izolace na úrovni sítě a souborového systému Solaris Zones implementuje ještě izolaci na úrovni procesů. Každá neglobální
zóna má svůj plánovač a může spouštět svoje vlastní procesy. Procesy běžící v jedné neglobální zóně nejsou žádným způsobem 
viditelné ani přímo ovlivnitelné z jiných neglobálních zón. Pokud chce proces z jedné zóny komunikovat s procesem z jiné zóny,
nemůže k tomu využít mezi procesovou komunikaci, ale musí použít počítačové sítě. Naopak procesy, které běží v rámci jedné
zóny spolu mohou komunikovat pomocí signálů, sdílené paměti a tak podobně. Všechny procesy běžící v systému mohou být
spravovány z globální zóny. Globální zóna tedy nabízí globální přehled všech procesů, které jsou spuštěny ve všech běžících
zónách v systému. Výstup příkazu \verb|ps(1)| v globální zóně zobrazí všechny procesy, zatímco v neglobální zóně budou zobrazeny
pouze procesy příslušící dané zóně.
\subsection{Typy zón}
\label{chapter:zones:types}
Jak bylo zmíněno výše, virtualizační technika Solaris Zones umožňuje v rámci jedné globální zóny spouštět mnoho neglobálních
zón. Každá neglobální zóna má vlastnost zvanou \textit{brand}, která určuje typ neglobální zóny. Tato vlastnost se specifikuje
při konfiguraci zóny v globální zóny a dle přehledu \cite{oracle:solaris:zones:brands} může typ zóny mít následující hodnoty:

\begin{itemize}
 \item \textit{solaris}
 \item \textit{solaris-kz}
 \item \textit{solaris10}
\end{itemize}

\textit{Brand} neboli typ zóny určuje jakým způsobem se zóna bude po spuštění chovat. Implicitním typem zóny v Solaris Zones
je \textit{solaris}, kterému se také jinak přezdívá nativní zóna nebo také tenká zóna. Dalším typem zóny je \textit{solaris-kz},
kde zkratka za pomlčkou v názvu odpovídá slovnímu spojením kernel zone. Jak název napovídá, tato zóna má vlastní jádro
operačního systému a někdy se jí také přezdívá plná nebo tlustá zóna. Posledním typem zón, kterou Solaris Zones umí vytvářet
je \textit{solaris10}. Hlavním úkolem této zóny je zajišťovat zpětnou kompatibilitu s operačním systémem Solaris 10 a umožňuje
uvnitř této zóny spouštět aplikace určené pro tento systém.

V následujících podkapitolách jsou podrobněji popsány výše zmíněné typy zón.
\subsubsection{Nativní zóna}
\label{chapter:zones:native}
Nativní neboli tenká zóna umožňuje administrátorovi vytvořit zónu, která má sdílené jádro operačního systému s globální zónou.
Jinými slovy verze jádra operačního systému musí být stejná jako v globální zóně. Tento typ zóny je izolovaný pouze nad svým
souborovým systémem a nemá standardně nemá k dispozici informace o žádném fyzickém zařízení systému. Souborové systémy ostatních
zón jsou nedostupné a konkrétní neglobální zóna o nich nemá žádné informace. Z jejího pohledu existuje pouze její kořenový
souborový systém. Jak již bylo popsáno výše, kořenový sytém nativní zóny může být sdílený se souborovým systémem globální
zóny a sdílet tak základní systémové nástroje. Pokud chceme této zóně delegovat nějaký typ zařízení, musíme tak učinit při
konfiguraci zóny v globální zóně. Tímto způsobem můžeme nativní zóně zpřístupnit souborové systémy, ZFS pool nebo
ZFS dataset. Takto definované prostředky pak po instalaci zóny můžeme z této zóny využívat.

Dále tento typ zóny má svoji vlastní databázi produktů, která obsahuje informace o všech nainstalovaných softwarových
komponentech v konkrétní globální zóně. Opět platí, že konkrétní neglobální zóna vidí pouze své balíčky. Díky tomu je možné
instalovat dodatečné softwarové balíčky do neglobálních zón, které nemusí být nainstalované v globální zóně
\cite{oracle:solaris:zones:brands}. Některé softwarové balíčky jsou ale společné s globální zónou (kernel) a nelze tedy
provádět kompletní aktualizaci bez zásahu do globální zóny. 

Nativní zóna podporuje dva typy síťových rozhraní, které mohou být zóně při konfiguraci přiřazeny. Prvním typem je sdílená 
adresa neboli \textit{shared-ip}. Tento typ síťového rozhraní sdílí ip adresu s nějakým fyzickým rozhraním globální zóny. 
Pokud chce neglobální zóna komunikovat s okolím, bude v hlavičce paketu ip adresa globální zóny a při obdržení odpovědi
globální zóna přesměruje paket na virtuální síťové rozhraní konkrétní globální zóny. Zde můžeme pozorovat podobnost s
technikou NAT v sítích. Jako druhý typ rozhraní můžeme použít exkluzivní rozhraní nebo také \textit{exclusive-ip}, které
nesdílí ip adresu s globální zónou, ale má svoji vlastní. V tomto případě veškerý síťový provoz generovaný touto zónou bude
mít v hlavičce jinou ip adresu než zóna globální.

Tento typ zóny nepodporuje vytváření další neglobálních zón. Jinými slovy se nativní neglobální zóna nemůže chovat jako
globální zóna a vytvářet nové zóny uvnitř sebe. Stejně tak z nativní zóny nemůžeme vytvářet ani spravovat jiné neglobální zóny.

Nativní zóna je implicitní typ zóny v Solaris Zones a pokud administrátor nespecifikuje jinak při vytváření zóny, bude nově
vytvořená zóna právě typu \textit{solaris}. Tento typ zóny může být provozován na všech systémech, které podporují operační
systém Oracle Solaris 11.3 \cite{oracle:solaris:zones:brands}.
\subsubsection{Kernel zóna}
\label{chapter:zones:kernel}
Druhým typem zóny, který virtualizační technika Solaris Zones umožňuje vytvářet, je kernel zóna nebo také tlustá zóna. Tento
typ zóny obsahuje vlastní jádro operačního systému a na rozdíl od nativní zóny ho nesdílí s globální zónou. Kernel zóna
tedy může být provozována na jiné verzi jádra než globální zóna. V důsledku toho kernel zóna podporuje funkcionalitu,
které nelze pomocí nativní zóny dosáhnout.

Stejně jako v případě nativní zóny i kernel zóna obsahuje vlastní databázi instalovaných softwarových balíčků. Jelikož i 
kernel zóna je neglobální, nelze z ní žádným způsobem vidět balíčky ostatních zón. Na rozdíl od nativní zóny administrátor
může provádět aktualizaci všech balíčků, protože kernel zóna nesdílí nic s globální zónou. Žádné balíčky tedy nejsou závislé
na balíčkách globální zóny.

V případě síťových rozhraní kernel zóny podporují pouze rozhraní typu \textit{exclusive-ip} a neumožňuje sdílet síťovou adresu
s globální zónou.

Na rozdíl od nativní zóny se kernel zóny mohou chovat jako globální zóny uvnitř hostitelské globální zóny. Jinými slovy je
možné uvnitř kernel zóny vytvářet další neglobální zóny a vytvářet tak hierarchickou strukturu virtuální strojů. Je na zvážení
administrátora, jestli daný scénář vyžaduje tuto strukturu.
\subsubsection*{Požadavky}
Jelikož provoz kernel zóny se liší od provozu standardní nativní zóny, liší se i požadavky na hostitelský systéme. Požadavky
se liší v závislosti na platformě. Pro jednoduchost si uvedeme pouze požadavky pro systémy s architekturou x86 a procesorem 
intel. Podle specifikace \cite{oracle:solaris:zones:kernel_zones_requiremets} se na hostitelský systém kladou následující
požadavky.

\begin{itemize}
 \item Procesor musí být typu Nehalem nebo novější
 \item Virtualizace CPU (VT-x) 
 \item Podpora virtualizace paměti (RVI, EPT)
 \item Ochrana paměti
\end{itemize}

Výše zmíněné požadavky kladou nároky na HW vybavení hostitelského systému. Spolu s těmito požadavky musí být v globální zóně
nainstalovaný balíček \textit{brand/brand-solaris-kz}, který umožňuje vytváření kernel zón. Administrátor může pomocí příkazu
\verb|virtinfo(1)| zjistit, jaký typ virtualizace je v globální zóně podporován. Výpis programu na virtualizované platformě
VMvare, kde jsou splněné výše zmíněné požadavky může vypadat následovně.

\begin{verbatim}
zadmin@shost:~$ virtinfo
NAME            CLASS     
vmware          current   
non-global-zone supported
kernel-zone     supported
\end{verbatim}


\subsubsection{Branded zóna}
\label{chapter:zones:branded}
Branded zóny byly vytvořeny pro zpětné zajištění kompatibility se staršími verzemi operačního sytému Solaris. Díky technologii
\textit{BrandZ} \cite{oracle:solaris:zones:brands} umožňují spouštění aplikací určených pro operační systém Solaris 10 na 
systému s OS Solaris 11. Aplikace mohou běžet v nezměněné formě v bezpečném prostředí, které je zajištěno neglobální zónou.

Z pohledu administrátora se tento typ zóny chová stejně jako nativní zóna a má stejné vlastnosti, které jsou popsané v
podkapitole \ref{chapter:zones:native}.
\subsubsection{Shrnutí}
\label{chapter:zones:summary}
Virtualizační technologie Solaris Zones umožňuje vytvářet neglobální zóny uvnitř primární globální zóny. Neglobální zóny
poskytují izolované prostředí pro nezávislý a bezpečný běh aplikací. Zóny jsou izolovány na úrovni počítačové sítě,
souborového systému a běžících procesů, čímž je zajištěno, že se vzájemně nemohou přímo ovlivňovat. Jediný způsob komunikace
procesů z jiných zón je pomocí počítačové sítě. 

Neglobální zóna může být několika druhů, které poskytují různé vlastnosti. Tabulka \ref{table:zone_comparison} poskytuje stručný
přehled základních vlastností jednotlivých druhů zón.
\begin{table}
  \centering  
  \caption[Porovnání typů zón a jejich vlastností]{Porovnání typů zón a jejich vlastností}
  \label{table:zone_comparison}
\end{table}
\section{Administrace}
\label{chapter:zones:administration}
Globální zóna slouží hlavně pro účely správy hostitelského systému a všech neglobálních zón. Poskytuje nainstalovaným zónám
prostředky pro jejich běh a spravuje informace o jejich stavu. Je to klíčové místo pro správu celého systému a správných chod
neglobálních zón vyžaduje bezproblémový chod globální zóny. Administrátor takového systému by tento fakt vzít v úvahu a
nepoužívat globální zónu jako zdroj pro spouštění uživatelských aplikací.

Proces vytváření zón se skládá ze dvou částí. První částí je konfigurace neglobální zóny, kdy administrátor specifikuje jaké
parametry má zóna mít a jaké prostředky bude moci využívat. Tento proces zavede zónu do databáze globální zóny a od této
chvíle je registrovaná v systému. K tomuto účelu nabízí Solaris Zones nástroj \verb|zonecfg(1)|. Více o tomto nástroji a
možnostech konfigurace zón je popsáno v kapitole \ref{chapter:zones:configuration}.

Z předchozího procesu vznikne jakýsi recept na to, jak danou neglobální zónu vytvořit. Nyní je třeba vytvořit souborový systém
zóny se všemi balíčky, které zóna bude potřebovat ke svému běhu. Této fázi se říká instalace zóny a v jejím průběhu se do
kořenového souborového systému nahrávají určené balíčky a nastavuje se konfigurační profil zón. Pro tento účel slouží nástroj
\verb|zoneadm(1)|. Výstupem tohoto procesu je nainstalovaná zóna připravená ke spuštění. Detailněji nástroj pro instalaci zón
a proces instalace popisuje kapitola \ref{chapter:zones:instalation}.

Čerstvě nainstalovaná zóna může být opět pomocí nástroje \verb|zoneadm(1)| spuštěna a následně se do ní může privilegovaný 
uživatel přihlásit pomocí nástroje \verb|zlogin(1)|.
\subsection{Administrátor}
\label{chapter:zones:administration:administrator}
Jelikož neglobálních zón může být v systému provozováno velké množství, může být pro administrátora globální zóny spravovat 
celou globální zónu a přitom se starat o všechny neglobální zóny. Pro tento účel může být vytvořen uživatel, který se bude
starat výhradně jenom o neglobální zóny. Tento uživatel bude mít práva na správu všech neglobálních zón.

Pokud by i tak bylo neglobálních zón mnoho, je možné jednotlivým neglobálním zónám přiřadit vlastního administrátora. Ten
se bude moc do zóny přihlásit pomocí příkazu \verb|zlogin(1)| a provádět údržbu a správu systému.
\subsection{Stavový model zón}
\label{chapter:zones:administration:states}
V Solaris Zones má neglobální zóna definovaný stavový model. Jsou to stavy, ve kterých se zóna během jejího životního cyklu
může nacházet. Zónu můžeme převést z jednoho stavu do druhého pouze nějakou kombinací nástrojů \verb|zonecfg(1)| a
\verb|zoneadm(1)|. Podle specifikace \cite{oracle:solaris:zones:states} se neglobální zóna může nacházet v jednom z
následujících sedmi stavů.
\begin{itemize}
 \item \textit{Configured}
 \item \textit{Incomplete}
 \item \textit{Unavailable}
 \item \textit{Installed}
 \item \textit{Ready}
 \item \textit{Running}
 \item \textit{Shutting down/Down}
\end{itemize}
V každém z těchto stavů může administrátor používat pouze nějakou podmnožinu příkazů, které zónu ovládají. Pro příklad
můžeme uvést, že nelze zónu spustit pokud se nachází například ve stavu \textit{configured}. Pro spuštění se zóna musí nacházet
ve stavu \textit{installed}. V následujících odstavcích je stručně shrnut význam jednotlivých stavů.
\subsubsection{Configured}
\label{chapter:zones:administration:states:configured}
Stav \textit{configured} značí, že konfigurace zóny je hotová a uložená na perzistentním úložišti. V tuto chvíli se zónou
ještě nebyl asociován žádný diskový obraz a tedy nemá připojený kořenový souborový systém. Tento stav je v pořadí první stav,
ve kterém se zóna může od svého vzniku nacházet. Nachází se v něm buď bezprostředně po vytvoření konfigurace pomocí \verb|zonecfg(1)|
nebo když je zóna odinstalována nebo odpojena.
\subsubsection{Incomplete}
\label{chapter:zones:administration:states:incomplete}
Zóna se nachází ve stavu \textit{incomplete} během procesu instalace a odinstalace. Je to přechodný stav, ale v případě poškození
nainstalované zóny může být v tomto stavu stále. V případě úspěchu procesu instalace pomocí nástroje \verb|zonecfg(1)| je 
stav zóny změněn na stav \textit{installed}. Pokud uspěje proces odinstalování zóny pomocí stejného nástroje, je stav
změněn na stav \textit{configured}.
\subsubsection{Unavailable}
\label{chapter:zones:administration:states:unavailable}
Ve stavu \textit{unavailable} se zóna nachází v případě, kdy zóna byla v minulosti nainstalována, ale momentálně nemůže být
spuštěna, přesunuta nebo její validace vrací chybu. Tento stav může mít několik příčin. Jednou z nich může být nedostupnost
zdrojového souborového systému zóny. Souborový systém může být nedostupný chybou administrátora nebo například chybou diskového
zařízení. Další příčinou může být nekompatibilita softwarového vybavení globální zóny a neglobální zóny. To se může stát například
ve chvíli kdy migrujeme neglobální nativní zónu z jednoho systému na druhý a tyto dva systémy mají odlišnou verzi jádra.
\subsubsection{Installed}
\label{chapter:zones:administration:states:installed}
Stav \textit{installed} signalizuje, že zóna s danou konfigurací je nainstalovaná ve svém kořenovém souborovém systému,
ale nemá alokovanou žádnou virtuální platformu pro svůj běh. Jinými slovy ještě nemůže být přímo spuštěna. Zóna ve stavu
\textit{installed} již může být zálohována nebo migrována mezi různými hosty. 
\subsubsection{Ready}
\label{chapter:zones:administration:states:ready}
Zóna se nachází ve stavu \textit{ready}, právě když je pro ni alokována virtuální platforma pro její běh. To znamená, že jádro
hostitelského operačního systému Solaris vytvořilo proces \verb|zsched|, vytvořilo virtuální síťové rozhraní a zpřístupnilo
je neglobální zóně. Obecně jádro inicializovalo všechny prostředky specifikované v konfiguraci zóny a zpřístupnilo je dané
neglobální zóně. V tomto stavu ještě nebyl spuštěn žádný uživatelský proces asociovaný s konkrétní zónou \cite{oracle:solaris:zones:states}.
Tento stav je tranzitní a nastává v okamžiku kdy zahajujeme boot zóny pomocí příkazu \verb|zoneadm(1)|.
\subsubsection{Running}
\label{chapter:zones:administration:states:running}
Ve stavu \textit{running} se zóna nachází když je spuštěn první uživatelský proces. Většinou se jedná o proces \verb|init(1)|,
který inicializuje celou zónu a umožňuje spouštění procesů uvnitř dané neglobální zóny. Zóna ve stavu \textit{running} má tedy
alokovanou virtuální platformu v jádru hostitelského operačního systému, inicializované všechny zařízení a spuštěné uživatelské
procesy.
\subsubsection{Shutting down/Down}
\label{chapter:zones:administration:states:down}
Posledním stavem respektive dvojicí stavů, ve kterých se může neglobální zóna nacházet jsou stavy \textit{shutting down} resp. 
\textit{down}. Tyto stavy jsou tranzitní a nastávají ve chvíli, kdy daná zóna zastavuje svůj běh. V případě, kdy nelze zónu
z nějakého důvodu zastavit, může daná zóna setrvat v některém z těchto dvou stavů \cite{oracle:solaris:zones:states}. 
\subsubsection{Doplňkové stavy kernel zón}
\label{chapter:zones:administration:states:kernel_zones}
Výše zmíněné stavy jsou společné pro všechny typy zón. Nastávají tedy u nativních zón, kernel zón i branded zón. Pro kernel
zóny existují ještě další stavy, které zmíníme jenom v rychlosti, jelikož nejsou tolik podstatné. Jedná se o stavy
\textit{suspended}, \textit{debugging}, \textit{panicked}, \textit{migrating-out} a \textit{migrating-in}. Jména stavů jsou
samovysvětlující, a proto nemá smysl je podrobněji popisovat.
\section{Konfigurace}
\label{chapter:zones:configuration}
Prvním krokem k vytvoření zóny je zaregistrování její konfigurace do systému Solaris Zones. K tomuto účelu se používá nástroj
\verb|zonecfg(1)|, který umožňuje vytváře, měnit nebo mazat konfigurace jednotlivých neglobálních zón. Dle manuálových stránek 
\cite{oracle:manpages:zonecfg} Tento nástroj umožňuje administrátorovi zadávat konfiguraci ve třech následujících režimech.
\begin{itemize}
 \item Interaktivně
 \item Dávkově
 \item Pomocí souboru s příkazy
\end{itemize}
První režim zadávání konfigurace vyžaduje aktivní účast uživatele a umožňuje interaktivně zadávat příkazy, které definují
konfiguraci dané zón. Další dva režimy načítají příkazy k definici konfigurace z jiného zdroje než je interaktivní vstup uživatele.
V případě dávkového režimu se jedná o vstup příkazů na příkazové řádce, kdy jsou příkazy zřetězeny za sebe a předány nástroji
\verb|zonecfg(1)| jako parametr. V druhém případě jsou příkazy načteny ze souboru, kde je každý příkaz na samostatné řádce.

Všechny režimy mají jednu věc společnou. Tím jsou příkazy, kterými předávají nástroji informace o definici konfigurace zóny.
Nástroj tedy očekává přesně definovanou syntaxi příkazů, které umí zpracovávat. Konfigurace zóny se skládá z konfigurace 
globálních atributů a prostředků. Prostředky mohou reprezentovat síťové rozhraní nebo jiný typ zařízení a mají svoje vlastní
lokální atributy. Popis všech příkazů, které nástroj \verb|zonecfg(1)| umí zpracovávat je zbytečný, a proto si syntaxi
ukážeme na následující výpisu \ref{code:zonecfg} konfiguračního souboru zóny.
\begin{lstlisting}[caption={Ukázka konfigurace zóny}, float, label={code:zonecfg}]
zadmin@shost:~$ cat /var/tmp/rzone-test_hu67.zonecfg
create -b
set brand=solaris
set zonepath=/system/zones/rzone-test
set autoboot=false
set autoshutdown=shutdown
set ip-type=exclusive
add anet
set linkname=net0
set lower-link=auto
set configure-allowed-address=true
set link-protection=mac-nospoof
set mac-address=auto
end
\end{lstlisting}
V ukázce \ref{code:zonecfg} je patrné, že každá řádka začíná příkazem. Příkaz \verb|create| vytvoří v paměti reprezentaci
konfigurace, která se po dokončení procesu konfigurace uloží do souboru. Následuje sekvence příkazů \verb|set|, které nastavují
nějaké globální atributy zóny. Více o nich v následující podkapitole \ref{chapter:zones:configuration:global_attributes}. Dále
můžeme v konfiguračním souboru vidět příkaz \verb|add|, který reprezentuje přidání zdroje k zóně. V tomto případě se jedná
o přidání síťového rozhraní s automatickou konfigurací. Základní typy jednotlivých prostředků popisuje podkapitola 
\ref{chapter:zones:configuration:global_attributes}. Následuje opět sekvence příkazů \verb|set|, která se ale nyní váže
k předchozímu příkazu \verb|add| a nastavuje tak lokální atributy daného síťového rozhraní. Celý konfigurační soubor je ukončený
příkazem \verb|end|, který reprezentuje výstup z konfigurace prostředku.

Nástroj \verb|zonecfg(1)| umožňuje dva módy editace konfigurace zóny. Prvním mód je editace konfigurace uložené v souborovém
systému. Změna konfigurace v tomto módu, žádným způsobem neovlivní běžící zónu. Druhý způsobem je editace konfigurace v takzvaném
živém módu, která umí ovlivňovat nastavení zóny ve stavu \textit{running}. V tomto případě je například možné dočasně přidat
běžící zóně síťové rozhraní.
\subsection{Globální atributy}
\label{chapter:zones:configuration:global_attributes}
Globální atributy popisují globální vlastnosti zóny jako celku. Neváží se tedy ke konkrétnímu prostředku ale k zóně jako takové.
Podle manuálových stránek \cite{oracle:manpages:zonecfg} umožňuje příkaz \verb|zonecfg(1)| konfigurovat třináct globálních
atributů zóny. V této kapitole nebudeme popisovat všechny, ale zaměříme se na nejdůležitější z nich. Důraz bude kladen především
na atributy, které jsou nezbytné pro vytvoření zóny.
\subsubsection{Jméno zóny}
\label{chapter:zones:configuration:global_attributes:zonename}
Jedním z hlavních atributů zóny je její jméno. Tento atribut je hlavním identifikátorem zóny v rámci systému a používá se
pro její specifikaci v rámci nástrojů \verb|zonecfg(1)| a \verb|zoneadm(1)|. Nastavuje se pomocí vlastnosti \textit{zonename},
nemá žádnou implicitní hodnotu a je povinným atributem pro vytvoření neglobální zóny.
\subsubsection{Cesta k souborovému systému zóny}
\label{chapter:zones:configuration:global_attributes:zonepath}
Klíčovým atributem zóny je cesta k adresáři, kde je připojený kořenový souborový systém neglobální zóny. Tento adresář obsahuje
všechny nezbytné softwarové balíčky pro běh zóny. V operačním systému Solaris 11 se pro kořenové souborové systémy zón používá
souborový systém ZFS. Tato skutečnost umožňuje využívat pokročilé funkce ZFS jako například snapshot nebo klonování při správě
a instalaci neglobálních zón. Tento atribut se nastavuje pomocí vlastnosti \textit{zonepath} a je povinným atributem pro 
vytvoření zóny. V jeho definici je možné používat proměnou \textit{zonename} a jeho implicitní hodnota je nastavena na
\verb|/system/zones/%{zonename}|.
\subsubsection{Typ zóny}
\label{chapter:zones:configuration:global_attributes:brand}
Jak již bylo zmíněno v kapitole \ref{chapter:zones:types}, typ neglobální zóny určuje jak s ní bude globální zóna zacházet.
Konfigurace typu zóny se provádí pomocí atributu \textit{brand}, který je povinným atributem pro vytvoření zóny. Může nabývat
hodnot \textit{solaris}, \textit{solaris-kz} nebo \textit{solaris10} a jeho implicitní hodnota je \textit{solaris}.
\subsubsection{Typ IP adresy}
\label{chapter:zones:configuration:global_attributes:ip-type}
Stejně jako předchozí atribut byl i atribut určující typ síťové adresy již zmíněn v kapitole \ref{chapter:zones:types}. V
krátkosti pouze naznačíme, že tento atribut určuje jestli bude síťová adresa sdílená s adresou globální zóny či nikoli. Tento
atribut se nastavuje pomocí atributu \textit{ip-type} a může mít hodnoty \textit{shared} a \textit{exclusive}, což je zároveň
implicitní hodnota.
\subsubsection{Automatické spouštění a vypínání}
\label{chapter:zones:configuration:global_attributes:autoboot}
Jako poslední zmíníme dva atributy, které souvisí se spouštěním a vypínáním neglobálních zón. Atribut \textit{autoboot} vyjadřuje
jestli se má daná neglobální zóna spustit při startu zóny globální. Tento atribut může nabývat dvou hodnot. Jestliže chceme
aby se zóna spustila při startu globální zóny, musíme nastavit hodnotu tohoto atributu na \textit{true}. V opačném případě
nastavíme hodnotu \textit{false}, která je implicitní hodnotou tohoto atributu. O automatické spouštění neglobálních zón se
stará systémová služba \verb|svc:/system/zones:default|, a proto je nutné, aby byla spuštěna \cite{oracle:manpages:zonecfg}.

Druhým atributem je atribut \textit{autoshutdown}, který se uplatňuje při vypínání globální zóny a naznačuje co se má stát
s danou neglobální zónou. Jeho hodnoty mohou být \textit{shutdown} pro korektní vypnutí zóny, \textit{halt} a nebo \textit{suspend}.
Implicitně se při vypínání globální zóny používá korektní vypnutí neglobální zóny.
\subsection{Zdroje}
\label{chapter:zones:configuration:resources}
Mimo generických atributů umožňuje nástroj \verb|zonecfg(1)| přidávat do konfigurace zóny takzvané zdroje. Zdroj je objekt
nějakého typu, který má svoje lokální atributy a do konfigurace zóny se přidává pomocí příkazu \verb|add|. Zdroj reprezentuje
většinou nějaké zařízení, souborový systém, prostředky pro přidělování zdrojů zónám nebo například specifikuje uživatele,
který může zónu administrovat. Některé zdroje mohou být do konfigurace zóny přidány vícekrát. V takovém případě jim \verb|zonecfg(1)|
automaticky přidělí číselný identifikátor, který daný zdroj jednoznačně určuje. Podle manuálových stránek \cite{oracle:manpages:zonecfg}
umožňuje konfigurace zón přidávat až jednadvacet různých typů zdrojů. Stejně jako v případě globálních atributu si zde
popíšeme jenom zdroje, které jsou nejdůležitější pro správný běh neglobální zóny.
\subsubsection{Zařízení}
\label{chapter:zones:configuration:resources:device}
Prvním typem zdroje, který může být delegován neglobální zóně, je obecné zařízení. Takovým zařízením může být například disk
nebo disková partition. V případě operačního systému se jedná o takzvané \textit{slice}, které jsou alternativou k diskovým 
partition. 

Použití tohoto zdroje se liší v závislosti na typu neglobální zóny, ke které ho chceme přiřadit. V případě nativní zóny má tento
zdroj následující  atributy.
\begin{itemize}
 \item \textit{match}
 \item \textit{allow-partition}
 \item \textit{allow-raw-io}
\end{itemize}
Atribut \textit{match} odpovídá jménu zařízení, které chceme delegovat zóně. Hodnotou může být absolutní cesta k zařízení nebo
regulární výraz, který může specifikovat více zařízení najednou. Druhý atribut \textit{allow-partition} určuje, jestli zóna 
bude moci používat nástroj \verb|format(1)|, který umožňuje rozdělování disku na jednotlivé partition. Přímý přístup na disk může
být povolen pomocí atributu \textit{allow-raw-io}.

V případě kernel zóny je podle manuálových stránek \cite{oracle:manpages:solaris-kz} povinné přidat alespoň jedno diskové zařízení,
které bude sloužit jako hlavní disk. Implicitně je pro kernel zónu vytvořen souborový systém ZFS, který je vyexportovaný jako
zařízení. Toto zařízení se přidá v průběhu konfigurace a nastaví se mu atribut \textit{bootpri} na hodnotu 0 (primární disk).
Část konfigurace specifikující úložné zařízení kernel zóny je naznačena v kódu \ref{code:zonecfg:device}
\begin{lstlisting}[caption={Konfigurace zařízení kernel zóny}, float, label={code:zonecfg:device}]
add device
set storage=/dev/zvol/dsk/rpool/VARSHARE/zones/z1/disk
set bootpri=0
set id=0
end
\end{lstlisting}
Podle doporučení v \cite{oracle:manpages:zonecfg} může být nebezpečné delegovat zónám obecné zařízení a povolit na ně přímý
přístup. Tento krok může vést k nestabilitě a ohrožení globální zóny, jelikož uživatelské procesy neglobálních zón mohou
přímo ovlivňovat delegovaná zařízení.
\subsubsection{Síťové rozhraní}
\label{chapter:zones:configuration:resources:network}
Jelikož spolu neglobální zóny nemohou komunikovat jinak než pomocí počítačové sítě, nástroj \verb|zonecfg(1)| umožňuje delegovat
neglobálním zónám následující dva typy sítových rozhraní. 
\begin{itemize}
 \item Fyzické síťové rozhraní - \textit{net}
 \item Automatické síťové rozhraní - \textit{anet}
\end{itemize}
Fyzické síťové rozhraní neboli zdroj \textit{net} umožňuje delegovat do neglobální zóny existující fyzické síťové rozhraní
z globální zóny. Tento typ zdroje má několik lokálních atributů, které definují jeho zdroj a chování. Hlavním atributem je
\textit{physical}, který reprezentuje jméno fyzického rozhraní v globální zóně. Jedno fyzické rozhraní nemůže být sdíleno
napříč více neglobálními zónami. Pomocí další atributů je možné specifikovat například ip adresy, které se mohou k danému 
rozhraní připojovat, gateway nebo ip adresu rozhraní. 

Automatické síťové rozhraní neboli \textit{anet} je druhým síťového zdroje, který může být neglobální zóně přiřazen. Rozdíl
oproti předchozímu typu je v tom, že tento typ rozhraní nemusí v globální síti existovat a je vytvořeno jako virtuální síťové
rozhraní. Jelikož je toto zařízení virtuální, umožňuje administrátorovi nastavovat mnohem větší škálu atributů. Těchto
atributů je podle manuálových stránek \cite{oracle:manpages:zonecfg} opravdu velké množství a opět si popíšeme jenom některé
z nich. Nejdůležitějšími atributy jsou \textit{linkname} a \textit{lower-link}. Atribut \textit{linkname} určuje jméno
síťového rozhraní tak, jak se bude jevit v neglobální zóně. Pomocí tohoto jména je možné toto rozhraní konfigurovat.
Atribut \textit{lower-link} je v podstatě jméno fyzického nebo virtuálního síťového rozhraní v globální zóně. Jinými slovy
přes toto zařízení bude proudit provoz generovaný vytvořeným síťovým rozhraním. Další atributy již nebudeme podrobně zmiňovat.
Ostatní parametry slouží například k nastavování MAC adresy, ochrany linkové vrstvy, VLAN a mnohého dalšího. V ukázce 
\ref{code:zonecfg:anet} je naznačeno jak by mohla vypadat konfigurace automatického síťového rozhraní pomocí nástroje 
\verb|zonecfg(1)|.
\begin{lstlisting}[caption={Konfigurace síťového rozhraní}, float, label={code:zonecfg:anet}]
add anet
set lower-link=auto
set configure-allowed-address=true
set link-protection=mac-nospoof
set mac-address=auto
end
\end{lstlisting}
\subsubsection{Řízení zdrojů}
\label{chapter:zones:configuration:resources:resource_control}
Jak již bylo zmíněno výše, neglobální zóny využívají fyzických prostředků hostitelského systému (globální zóny). Aby bylo
možné zajistit kontrolu na přidělováním prostředků jednotlivým neglobálním zónám, umožňuje konfigurace specifikovat některé
omezení na využívání zdrojů. Tyto omezení se do konfigurace přidávají jako kterékoli jiné zdroje. Podle manuálových stránek
\cite{oracle:manpages:zonecfg} existují následující typy zdrojů, které umožňují řízení přidělování fyzických zdrojů.
\begin{itemize}
 \item Exkluzivní CPU - \textit{dedicated-cpu}
 \item Omezení CPU - \textit{capped-cpu}
 \item Omezení paměti - \textit{capped-memory}
\end{itemize}
První ze zdrojů s názvem \textit{dedicated-cpu} slouží pro alokování určitého počtu procesorů a procesorových jader exkluzivně
pro použití danou neglobální zónou. V systému dojde k vytvoření množiny procesorů, která v době běhu dané neglobální zóny
může být využívána pouze danou neglobální zónou. Dokonce ani globální zóna nemůže tyto procesory a jádra využívat. Tento typ
zdroje umožňuje specifikovat výpočetní zdroje s různou granularitou. Administrátor může zóně přiřadit prostředky na úrovni 
procesorových jader, procesorů nebo dokonce celých soketů s několika procesory.

Zdroj \textit{capped-cpu} reprezentuje množství procesorového času, které může být danou zónou využíváno. Tento zdroj má pouze
jeden numerický atribut, kterým je desetinné číslo určující možné procentuální využití jednoho procesoru. Hodnotou 1 se tedy
myslí, že daná zóna může využít 100\% procesorového času.

Posledním zdrojem, který si zmíníme v rámci řízení zdrojů zón je \textit{capped-memory}. Tento zdroj se podobá \textit{capped-cpu}
ale řídí využití paměti. Tento zdroj má tři atributy \textit{physical}, \textit{swap} a \textit{locked}, které se týkají určitého
druhu paměti. Atribut \textit{physical} souvisí s hlavní operační pamětí počítače a umožňuje administrátorovi specifikovat,
kolik hlavní paměti může daná zóna využít. Ostatní atributy mají stejný význam, ale týkají se jiné části paměti. Všechny 
atributy je možné specifikovat v jednotkách kilobyte (K), megabyte (M), gigabyte (G) nebo terabyte (T).
\subsubsection{ZFS Dataset}
\label{chapter:zones:configuration:resources:dataset}
Standardně mají neglobální zóny ponětí pouze o svém kořenovém souborovém systému a nevidí žádný jiný. Pokud chce zóna využívat
nějaké další úložiště, musíme jí delegovat buď celý disk nebo můžeme použít zdroj \textit{dataset}. Tento zdroj reprezentuje
existující souborový systém ZFS v globální zóně, který je delegován do neglobální zóny jako virtuální ZFS pool. Tento zdroj
má dva atributy specifikující jméno souborového systému v globální zóně a alias pod kterým bude vytvořen virtuální ZFS pool
v neglobální zóně.
\subsubsection{Administrátor zóny}
\label{chapter:zones:configuration:resources:admin}
Jak již bylo zmíněno v kapitole \ref{chapter:zones:administration:administrator}, administrátor globální zóny může delegovat 
administraci neglobální zóny na jiného uživatele. K tomuto účelu se požívá zdroj \textit{admin}, který reprezentuje administrátora
dané neglobální zóny. Tomuto administrátorovi lze přiřadit různé privilegia, které mu umožňují nějakým způsobem spravovat zónu.
Podle manuálových stránek \cite{oracle:manpages:zonecfg} jsou možná privilegia následující.
\begin{itemize}
 \item \textit{login}
 \item \textit{manage}
 \item \textit{copyfrom}
 \item \textit{config}
 \item \textit{liveconfig}
\end{itemize}
Názvy jednotlivých privilegií odpovídají akcím, které může daný uživatel se zónou provádět. Za zmínění snad stojí jenom 
privilegium \textit{copyfrom}, které umožňuje administrátorovi vytvářet nové zóny jako klony dané neglobální zóny. V kódu
\ref{code:zonecfg:admin} je ukázána konfigurace která umožňuje uživateli \textit{zadmin} přihlašování a základní správu dané
zóny.
\begin{lstlisting}[caption={Delegace administrace jinému uživateli}, float, label={code:zonecfg:admin}]
add admin
set user=zadmin
set auths=login,manage
end
\end{lstlisting}
\subsection{Vytvoření konfigurace}
\label{chapter:zones:configuration:creating}
První krok v procesu vytváření neglobální zóny je vytvoření její konfigurace. Konfigurace zóny se vytváří pomocí nástroje 
\verb|zonecfg(1)| a jeho příkazu \verb|create|. Tento příkaz vytvoří datovou reprezentaci konfigurace v paměti počítače a
po dokončení konfigurace jí uloží do souboru ve formátu XML v systémovém adresáři \verb|/etc/zones|. Teoreticky jediné co
administrátor potřebuje k vytvoření standardní zóny je její jméno. K vytvoření konfigurace může administrátor použít
následující způsoby.
\begin{itemize}
 \item Přímá konfigurace
 \item Konfigurace ze šablony
 \item Konfigurace z archivu
\end{itemize}
Přímá konfigurace již byla popsána v kapitole \ref{chapter:zones:configuration} a využívá přímé zadávání příkazů, které obsahují
definici atributů a zdrojů zóny. Tento proces může probíhat buď interaktivně postupným zadáváním příkazů nebo dávkově na
příkazové řádce.

Druhým způsobem je použití takzvané šablony. Šablona není nic jiného než konfigurace zóny, která je již zaregistrovaná v 
systému. Pokud tedy administrátor již vytvořil předem nějaké neglobální zóny a chce jejich konfiguraci znovu využít, stačí
specifikovat jejich jméno jako argument příkazu \verb|create|. Tím vznikne úplně nová konfigurace zóny, která má ale stejné
atributy jako zóna zdrojová. Standardní instalace operačního systému Solaris již obsahuje standardní šablony pro rychlou tvorbu
neglobálních zón. Tyto šablony se nachází společně s konfiguracemi ostatní zón v adresáři \verb|/etc/zones|. Zvídavý uživatel
může zjistit, že adresář obsahuje například šablony \textit{SYSdefault.xml}, \textit{SYSsolaris10.xml} nebo
\textit{SYSsolaris-kz.xml}, které odpovídají jednotlivým typů neglobálních zón. V ukázce kódu \ref{code:zonecfg:create:template}
je viditelné rychlé vytvoření konfigurace zóny se jménem \textit{z1}, které využívá systémovou šablonu \textit{SYSdefault}.
\begin{lstlisting}[caption={Vytvorení zóny ze systémové šablony}, label={code:zonecfg:create:template}]
zadmin@shost:~$ zonecfg -z z1 create -t SYSdefault
zadmin@shost:~$ zonecfg -z z1 export
create -b
set brand=solaris
set zonepath=/system/zones/%{zonename}
set autoboot=false
set autoshutdown=shutdown
set ip-type=exclusive
add anet
set linkname=net0
set lower-link=auto
set configure-allowed-address=true
set link-protection=mac-nospoof
set mac-address=auto
end
\end{lstlisting}
Z ukázky je patrné, že šablona \textit{SYSdefault} nastavuje základní atributy zón s použitím implicitních hodnot a přidává
jedeno síťové rozhraní pro připojení k síti.

Poslední možností pro vytvoření konfigurace zóny je využití archivu. Archiv musí být typu \textit{Unified archive}, což je
výstup systémového nástroje pro zálohování a archivaci zón nebo celých systémů. Více o tomto archivu je popsáno v kapitole
\ref{chapter:zones:backup}, která se zabývá zálohováním. Druhou možností archivu je složka se kořenovým souborovým systémem
zóny, která byla odpojena pomocí příkazu \verb|zoneadm detach|. V tomto případě je konfigurace odpojované zóny nahrána do 
kořenového adresáře a nástroj \verb|zonecfg(1)| si jí převezme.
\subsection{Změna konfigurace}
\label{chapter:zones:configuration:editing}
Dalším administrátorským úkonem ve správě Solaris Zones může být změna existující konfigurace zóny. Administrátor může editovat
konfiguraci zóny jak ve stavech \textit{configured} a \textit{configured} tak i ve stavu \textit{running}. Pokud zóna není spuštěná
administrátor může editovat pouze konfiguraci, která je uložená v adresáři \verb|/etc/zones|. V případě, že konfigurovaná zóna běží
může si administrátor vybrat jestli změny chce propagovat do uložené konfigurace nebo do živé konfigurace běžící zóny. 
Změny, které jsou provedeny pouze v živé konfiguraci zóny, jsou po vypnutí dané zóny ztraceny. Při opětovném startu si zóna
totiž načte konfiguraci z příslušného souboru.

Editace globálních atributů zóny probíhá stejným způsobem jako při jejich vytváření. Pomocí příkazu \verb|set| nástroje
\verb|zonecfg(1)| administrátor pouze přepíše danou hodnotu atributu. Pokud chce uživatel měnit lokální atributy nějakého zdroje,
například síťového rozhraní, musí použít příkaz \verb|set| pro jeho výběr jak je naznačeno v \cite{oracle:solaris:zones:modify}
Dále už může postupovat stejným způsobem jako při nastavování globálních parametrů.

Při ukončování editace konfigurace zóny je nutné použít příkaz \verb|commit|, který dané změny propaguje do perzistentního 
úložiště nebo do živé konfigurace zóny.
\subsection{Smazání konfigurace}
\label{chapter:zones:configuration:deleting}
Posledním typem operace, která se dá s konfigurací zóny provádět, je mazání. K tomuto účelu slouží příkaz \verb|delete| nástroje
\verb|zonecfg(1)|. Tato akce nemůže být vrácena, a proto by si administrátor měl tento úkon řádně promyslet. Po provedení
příkazu je konfigurace odstraněna jak z paměti počítače tak ze souborového systému. Tento příkaz neodstraní zdrojový souborový
systém zóny.
\section{Instalace}
\label{chapter:zones:instalation}
Dalším krokem na cestě za funkční neglobální zónou je proces její instalace. Tento proces vyžaduje, aby daná zóna měla v
systému vytvořenou konfiguraci. Účelem procesu instalace je vytvoření kořenového souborového systému zóny, který obsahuje
softwarové balíky nezbytné pro její správný chod.

Instalace zóny se provádí pomocí nástroje \verb|zoneadm(1)|, který umožňuje několik poskytuje administrátorovi několik způsobů
pro vytvoření kořenového souborového systému zóny. Dle specifikace \cite{oracle:solaris:zones:installation} a manuálových stránek
\cite{oracle:manpages:zoneadm} jsou podporovány následující typy instalace.
\begin{itemize}
 \item Instalace z repozitáře
 \item Instalace pomocí archivu
 \item Klonování zóny
\end{itemize}
Výše zmíněné typy instalace se liší hlavně ve způsobu získávání a vytváření zdrojové souborového systému zón. Kritériem pro
výběr může být například doba trvání instalace, protože mezi jednotlivými typy můžeme pozorovat zásadní rozdíl. Administrátor
musí při výběru druhu instalace brát v úvahu dostupné prostředky systému a časový interval trvání instalace. Druhy instalace
budou popsány z pohledu nativní neglobální zóny a rozdíly v instalaci pro kernel zóny budou upřesněny. 
\subsection{Instalace z repozitáře}
\label{chapter:zones:instalation:repozitory}
Prvním typem instalace neglobální zóny je instalace z repozitáře. Repozitář je databáze softwarových balíčků, které jsou pro 
operační systém Solaris poskytovány. Standardní poskytovatel této databáze je společnost Oracle, která hlavní databázi balíčku
poskytuje jako webovou službu dostupnou z \textit{pkg.oracle.com/solaris/release}. Nastavení této služby v systému se provádí
pomocí nástroje \verb|pkg(1)|, který slouží jako správce softwarových balíků v operačním systému Solaris 11.

Instalace touto metodou se provádí pomocí příkazu \verb|install| nástroje \verb|zoneadm(1)|. Jediný povinný argument je jméno
zóny, pro kterou chce administrátor nainstalovat systém. Tato zóna musí nacházet ve stavu \textit{configured}. Po puštění 
příkazu začne instalátor stahovat potřebné balíčky z repozitáře a instalovat je do kořenového souborového systému zóny. V případě
nativní zóny se implicitně instaluje softwarový balík \textit{pkg:/group/system/solaris-small-server}, je virtuální
a obsahuje pouze závislosti. Z jeho názvu je patrné, že balíky budou umožňovat provozovat zónu jako malý server. Jelikož nativní
zóna sdílí jádro operačního systému s globální zónou, balíček s jádrem se do nativní zóny neinstaluje. V případě kernel zóny
zde nastává rozdíl a balíček s jádrem se nainstaluje. Podle článku \cite{oracle:solaris:zones:kernel_version} je možné tímto
typem instalace nainstalovat pouze stejnou verzi jádra jako používá globální zóna.

Administrátor může předat příkazu \verb|install| další dva volitelné parametry. Prvním z nich je tzv. \textit{manifest}, který
předává instalátoru informace o tom, jaké softwarové balíčky má nainstalovat do kořenového souborového systému. Podrobněji 
je manifest popsán v kapitole \ref{chapter:zones:instalation:repozitory:manifest}. Druhý parametr odpovídá cestě k tzv. 
systémovému profilu, který specifikuje systémové nastavení. Tento parametr je společný pro všechny typy instalace a je podrobněji
popsán v kapitole \ref{chapter:zones:instalation:profile}.

Po úspěšném dokončení instalace jsou všechny softwarové balíčky definované v manifestu nainstalované a připravené k použití.
Čerstvě nainstalovaná zóna se nachází ve stavu \textit{installed} a čeká na první spuštění. Tento typ instalace trvá nejdelší
dobu. Je to zapříčiněno tím, že se všechny balíčky musí stáhnou pomocí počítačové sítě.
\subsubsection{Manifest}
\label{chapter:zones:instalation:repozitory:manifest}
Manifest je soubor obsahující definici všech softwarových balíčků, které má instalátor nainstalovat do souborového systému zóny.
Tento soubor je ve formátu XML a je povinnou součástí instalace. V minulé kapitole je řečeno, že tento parametr je volitelný.
V případě, kdy uživatel nespecifikuje cestu k souboru explicitně, je použit předdefinovaný soubor. Podle specifikace
\cite{oracle:solaris:zones:manifest} se soubor nachází v adresáři \textit{/usr/share/auto\_install/manifest} a jmenuje
se \textit{zone\_default.xml}. Právě v tomto souboru je definované, že standardní výbavou každé zóny je softwarový balík 
\textit{pkg:/group/system/solaris-small-server}.

Administrátor má tedy možnost si vytvořit vlastní XML soubory s definicemi softwarových balíčků. Tyto soubory pak může používat
při instalaci nových zón. V ukázce kódu \ref{code:manifest} je naznačeno, jak by mohl takový uživatelský manifest vypadat.
Pro přehlednost je zobrazena pouze část s definicí balíků a zbytek XML dokumentu je vynechán.
\begin{lstlisting}[language={XML}, caption={Ukázkový manifest}, label={code:manifest}]
<software_data action="install">    
    <name>pkg:/group/system/solaris-small-server</name>
    <name>pkg:/developer/versioning/mercurial</name>
    <name>pkg:/developer/versioning/git</name>    
</software_data>
\end{lstlisting}
\subsection{Instalace z archivu}
\label{chapter:zones:instalation:archive}
Pokud nechceme instalovat zónu přímo z repozitáře můžeme využít možnosti instalace z archivu. Archiv je soubor, který obsahuje
zdrojový souborový systém zóny neboli diskový obraz. V tomto případě se nemusí balíčky stahovat přes počítačovou síť z repozitáře,
ale jsou zkopírovány ze zdrojového archivu. V předchozím případě instalátor zajistí, že se z repozitáře stáhnou správné verze
balíčků a administrátor se o to nemusel starat. V tomto případě může nastat problém s nekompatibilitou balíčků neglobální a 
globální zóny. Pokud se jedná o nativní zónu pak musí souhlasit verze jádra instalované zóny s verzí jádra globální zóny. 
Administrátor může instalátoru nastavit, aby při instalaci provedl potřebnou aktualizaci všech balíčků. Tento krok lze provést
jenom v případě, že verze hosta je vyšší než verze neglobální zóny. V opačném případě se nepodaří zónu z archivu nainstalovat.

Instalace z archivu se provádí opět pomocí příkazu \verb|install| nástroje \verb|zoneadm(1)|, kde administrátor specifikuje
cestu k danému archivu. Jelikož archiv již obsahuje nainstalované softwarové balíky, není možné použít manifest pro další specifikaci.
Stejně jako při minulém typu instalace je možné použít systémový profil pro konfiguraci nainstalované zóny. Instalace z 
archivu je rychlejší než instalace z repozitáře, ale nekompatibilita globální zóny a archivu může vyústit v neúspěch.
\subsection{Klonování}
\label{chapter:zones:instalation:cloning}
Posledním podporovaným druhem instalace je klonování zóny. Vstupem klonování je již existující nakonfigurovaná a nainstalovaná
zóna. Tento proces instalace využívá pokročilé techniky souborového systému ZFS, který umožňuje vytváření klonů z existujících
souborových systémů. Klon je read-write kopie zdrojového souborového systému. Ve skutečnosti se nevytváří úplná kopie
souborového systému, ale data se kopírují až v okamžiku kdy se klon změní. Jak říká specifikace \cite{oracle:solaris:zones:clonning},
vytvoření klonu trvá zlomek času klasické instalace. 

Klonování zón se spouští pomocí příkazu \verb|clone| nástroje \verb|zoneadm(1)|, který bere zdrojovou zónu jako parametr. Pro efektivní
využívání tohoto typu instalace je třeba dodrže následující požadavek. Aby bylo opravdu použito funkce klonování souborového systému
ZFS musí se zdrojová i cílová zóna nacházet ve stejném ZFS poolu. V opačném případě je zdrojový souborový systém opravu zkopírovaná a
není využito této techniky. Dále si administrátor musí dát pozor při konfigurování cílové zóny a musí změnit atributy, které
nemohou zůstat stejné. Především se jedná o atribut \textit{zonepath}, který musí být unikátní.

Obrovskou výhodou tohoto typu vytváření zón je rychlost instalace a celková úspora diskového místa systému. Navíc podle specifikace
\cite{oracle:solaris:zones:clonning} se všechny aktualizace balíčků, které jsou provedeny ve zdrojové zóně automaticky objeví i ve
všech klonech.
\subsection{Systémový profil}
\label{chapter:zones:instalation:profile}
Všechny výše zmíněné typy instalace umožňují specifikovat systémový profil, který slouží pro konfiguraci systému. Profil je soubor
ve formátu XML (podobně jako manifest), který podle specifikace \cite{oracle:solaris:zones:profile} umožňuje konfigurovat
kteroukoli systémovou službu v rámci SMF. Tento soubor se tedy skládá z popisů konfigurace jednotlivých systémových služeb
operačního systému Solaris. Prostřednictvím profilu se například specifikuje počáteční heslo uživatele root, konfigurace 
síťových rozhraní nebo časová zóna systému. Příklad konfigurace uživatele root je naznačen v ukázce kódu \ref{code:profile}.
Pro přehlednost jsou vynechány ostatní nepodstatné části souboru.
\begin{lstlisting}[language={XML}, caption={Konfigurace uživatele root}, label={code:profile}]
<service name="system/config-user" version="1" type="service">
    <instance name="default" enabled="true">
      <property_group type="application" name="root_account">
        <propval type="astring" name="type" value="%{type}"/>
        <propval type="astring" name="login" value="root"/>
        <propval type="astring" name="password" value="%{password}"/>
        <propval name="expire" value="%{expire}"/>
      </property_group>>
    </instance>
</service>
\end{lstlisting}
V ukázce je vidět, že konfigurace uživatele root umožňuje specifikovat jeho počáteční heslo, platnost hesla a typ uživatele.

Konfigurace samostatných služeb je možné vygenerovat pomocí nástroje \verb|sysconfig(1)|, který slouží pro konfiguraci systému.
Jak je zmíněno v manuálových stránkách \cite{oracle:manpages:sysconfig}, pomocí parametru \textit{grouping} lze stanovit typ
služby, pro kterou chcete vygenerovat systémový profil. Pokud administrátor při instalaci zóny neuvede soubor se systémovým
profilem, je po prvním startu zóny spuštěn právě nástroj \verb|sysconfig(1)|. Tento nástroj pak interaktivně nastaví konfiguraci
požadovaných služeb.
\section{Správa}
\label{chapter:zones:instalation}
\section{Zálohování}
\label{chapter:zones:backup}
\section{Migrace}
\label{chapter:zones:backup}