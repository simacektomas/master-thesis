\label{chapter:implementation}
V následující kapitole je představena implementace nástroje pro automatickou správu Solaris Zones. Hlavní částí
je popis knihovny, modulu Solaris Zones a klientské aplikace. Důraz je kladen na popis funkcí jednotlivých funkčních
bloků aplikace a jejich vzájemné komunikace.
\section{Programovací jazyk}
\label{chapter:implementation:language}
Prvním krokem  při implementaci bylo zvolení vhodného programovacího jazyka. Požadavky stanovené v kapitole 
\ref{chapter:design:demands} vyžadují od zvoleného programovacího jazyka následující dvě podmínky.
\begin{itemize}
 \item Operační systém Solaris
 \item Možnost tvorby grafického rozhraní
\end{itemize}
Nástroj pro automatickou správu virtualizačního kontejneru Solaris Zones bude stavět hlavně nad využíváním
nástrojů na příkazové řádce a zpracovávání jejich výstupu. Pro tento účel bude vhodné zvolit interpretovaný
programovací jazyk, který umožní jednoduše nástroje spustit a následně snadno zpracovat jejich výstup. Na základě
výstupu se pak nástroj rozhodne o dalším průběhu zpracování uživatelského příkazu. První podmínka není ve volbě
tolik omezující. Pro operační systém Solaris totiž existuje implementace standardního kompilátoru \verb|gcc(1)|
pro jazyk C a stejně tak i implementace virtuálního stroje JVM pro jazyk Java. Většina interpretovaných jazyků
staví svůj překladač právě nad jedním z těchto základních programovacích jazyků.

Z výše uvedených důvodů je nutné při volbě jazyka dbát hlavně na dostupnost grafických knihoven pro operační systém
Solaris. Standardním skriptovacím jazykem pro většinu operačních systému typu UNIX je \verb|shell|. Tento program
interpretuje uživatelské příkazy na příkazové řádce a následně je provádí. Tato volba by splňovala podmínku platformy,
ale těžko by se s pomocí tohoto jazyka vytvářelo grafické uživatelské rozhraní. Z tohoto důvodu zvolíme programovací
jazyk Ruby \cite{ruby}, který umožní splnit obě dvě stanovené podmínky.
\subsection{Ruby}
\label{chapter:implementation:language:ruby}
Ruby je objektově orientovaný programovací jazyk, který má mnoho možností pro využití. Jedním z využití může být
právě spouštění příkazů na příkazové řádce a tvorba uživatelského rozhraní. Objektová povaha tohoto jazyka umožňuje
programátorovi využívat všech výhod objektově orientovaného programování. Podle dokumentu \cite{ruby:implementation}
existuje několik implementací interpretu jazyka Ruby, z nichž nejpoužívanější jsou YARV \cite{ruby:implementation:yarv} 
a JRuby \cite{ruby:implementation:jruby}. Obě tyto implementace jsou dostupné i pro operační systém Solaris.

Pokud chce programátor využívat grafické rozhraní pomocí programovacího jazyka Ruby, je nutné aby byly v systému
nainstalované potřebné grafické knihovny. Standardní knihovny pro programovací jazyk Ruby ale nejsou na na operačním
systému Solaris podporované. Z toho důvodu bude nutné využít grafické rozhraní, které nabízí implementace JRuby.
Tato implementace je postavená nad virtuálním strojem JVM a může využívat grafické knihovny v něm implementované.
Navrhovaný nástroj pro automatickou správu virtualizačního kontejneru Solaris Zones bude tedy využívat programovacího
jazyka Ruby. Pokud bude chtít uživatel nástroje využívat grafického rozhraní bude muset nástroj spouštět pomocí
interpretu JRuby. Zbytek nástroje bude nezávislý na použitém interpretu programovacího jazyka Ruby.
\section{Knihovna}
\label{chapter:implementation:library}
Hlavním centrálním prvkem implementace bude knihovna, která bude zprostředkovávat komunikaci mezi implementovanými moduly
a klientskými aplikacemi. Knihovna je navržená tak, aby se v budoucnosti dala lehce rozšířit o další moduly, které budou
poskytovat funkce pro správu jiných virtualizačních technologií. Jedním z takových rozšíření by mohl být například modul
pro podporu virtualizační technologie Virtualbox. Výsledná implementace bude obsahovat pouze modul pro podporu automatické
správy virtualizačního kontejneru Solaris Zones, který bude dále podrobně popsán v kapitole \ref{chapter:implementation:szones}.

Knihovna bude poskytovat hlavní rozhraní, pomocí kterého bude moci klient využívat funkcí jednotlivých modulů. Jednotlivé
moduly tedy budou sloužit jako hlavní zdroj funkcionality pro knihovnu.

Mimo zprostředkovávání komunikace mezi moduly a klientem bude knihovna sloužit k validaci šablon, které mají specifikovat
konkrétní virtuální stroj. V případě šablon bude knihovna sloužit jako vstupní bod, který umí šablonu načíst a provést
prvotní validaci. Spouštění těchto operací a jejich výsledky bude knihovna zprostředkovávat klientovi.

Jelikož moduly budou implementovat různé typy operací pomocí různých technologií, je nutné ponechat vývojářům velkou volnost
v možnostech jejich implementace. Pro účel zajištění jednotné komunikace s moduly je nutné, aby každý implementovaný modul
splňoval určité rozhraní. Toto rozhraní zajistí, aby všechny moduly mohli jednotně komunikovat s knihovnou a také aby knihovna
mohla zprostředkovávat jejich funkce klientovi. Definice rozhraní modulu je podrobně popsána v kapitole 
\ref{chapter:implementation:library:interface}.

Poslední funkcí knihovny bude udržování hlavní konfigurace. V této konfiguraci bude například uchováván seznam implementovaných
modulů, kořenový adresář knihovny nebo například jméno knihovny. Klientská aplikace pak bude mít možnost tuto konfiguraci
změnit a docílit jiného chování knihovny.
\subsection{Rozhraní modulu}
\label{chapter:implementation:library:interface}
Povinné rozhraní modulu bude sloužit především ke komunikaci mezi knihovnou a samotným modulem. Funkcionalitu, kterou modul
musí poskytovat můžeme shrnout do následujících tří bodů.
\begin{itemize}
 \item Inicializační rutina
 \item Rozhraní poskytované klientům
 \item Funkce pro validaci šablon
\end{itemize}

Prvním požadavkem na rozhraní modulu je existence inicializační rutiny. Pomocí této rutiny je do modulu předána hlavní
konfigurace knihovny, která umožňuje modulu zjistit kořenový adresář aplikace a další volitelné parametry. Hlavním smyslem
této rutiny je inicializace daného modulu. Hlavní knihovna v rámci inicializační smyčky spustí tuto rutinu pro každý
registrovaný modul. V rámci této rutiny modul může provádět inicializaci vlastních datových struktur nebo vytvoření potřebné
adresářové struktury. Dále může modul využít hlavní konfiguraci knihovny k doplnění vlastní lokální konfigurace. Tímto způsobem
je zajištěno, že všechny registrované moduly knihovny obdrží globální konfiguraci a dojde k jejich inicializaci.

Dalším nutnou částí rozhraní modulu jsou funkce, které mají být poskytovány klientovi. K tomuto účelu musí modul poskytovat
třídu, která bude tyto funkce implementovat nebo je bude pouze zprostředkovávat pomocí jiných tříd modulu. Tato třída je tedy
hlavním funkčním rozhraním modulu, které klientské aplikace mohou využívat. V rámci inicializace celé knihovny dojde nejprve
k inicializaci jednotlivých modulů. Po této akci knihovna provede registraci těchto tříd a v udržuje si jejich seznam.

Posledním požadavkem na rozhraní modulu je existence funkcí pro validaci šablon. Tyto funkce musí umožňovat validovat šablony,
které se týkají konkrétního modulu knihovny. Jinými slovy modul, který podporuje správu virtualizačního kontejneru, musí poskytovat
funkce pro validaci šablon specifikující neglobální zóny.

Vlastní implementace modulu není nijak jinak omezena. Jediným logickým omezením je fakt, že tento modul musí být napsaný
v programovacím jazyku Ruby. Pokud modul splní výše zmíněné požadavky, může být jednoduše registrován do knihovny a klientské
aplikace ho můžou bezprostředně po inicializaci knihovny využívat. Na obrázku \ref{module:interface} je názorně zobrazeno 
jak knihovna využívá rozhraní modulu a jakým způsobem je modul poskytován klientské aplikaci.
\begin{figure}
    \centering    
    \caption{Rozhraní modulu}
    \label{module:interface}
\end{figure}
\subsection{Přesměrování požadavků}
\label{chapter:implementation:library:routing}
Mimo inicializace modulů je hlavní funkcí knihovny přesměrovávat požadavky klientských aplikací na funkční rozhraní 
implementovaných modulů. K tomuto účelu obsahuje knihovna hlavní třídu, která virtuálně reprezentuje rozhraní všech modulů
knihovny. Tato třída se nazývá hlavní rozhraní. Jak již bylo zmíněno v kapitole \ref{chapter:implementation:library:interface},
knihovna si udržuje odkazy na hlavní třídy modulů, které reprezentují jejich funkční rozhraní.

V okamžiku, kdy klientská aplikace vznese požadavek na zavolání nějaké funkce, knihovna v době běhu zjistí jakému modulu daná
funkce přísluší a vyvolá ji. Pokud neexistuje žádný modul, který umí danou funkci provést, dojde k vyvolání výjimky a aplikace
se ukončí. Díky tomuto chování může dojít ke kolizi jmen funkcí. V takovém případě by knihovna vyvolala takovou funkci, kterou
by našla jako první v pořadí. Z tohoto důvodu je nutné se vyvarovat opakování jmen funkcí a nejlépe používat pro funkce určitého
modulu prefix, který daný modul jasně identifikuje. Například jeho jméno.

Toto směrování za běhu aplikace je umožněno díky programovacímu jazyku Ruby a jeho možnosti dynamického volání funkcí za běhu
programu. Směrování požadavků ke konkrétním modulům knihovny je názorně ukázáno na obrázku \ref{module:interface}.
\subsection{Generická šablona}
\label{chapter:implementation:library:generic}
Poslední funkcionalitou knihovny je definice generické šablony, která má za úkol specifikovat virtuální stroj. Hlavním úkolem
generické šablony bude specifikace typu virtuálního stroje. Tento atribut bude určovat, který modul knihovny je zodpovědný
za zpracování a validaci. Generická šablona by dále měla obsahovat jméno, které bude nějakým způsobem vystihovat a popisovat
specifikovaný virtuální stroj. Knihovna bude tedy zajišťovat validaci těchto dvou atributů a v případě úspěchu předá
šablonu zodpovědnému modulu.

Knihovna tedy bude klientským aplikacím poskytovat funkce pro načítání a validaci šablon. V případě úspěšného načtení šablony
bude knihovna vracet objekt, který bude moci být použit v rámci konkrétního modulu. Validace šablony bude provedena ve dvou
krocích. Knihovna nejprve zjistí jestli daná šablona obsahuje atribut jména a typu. Podle typu šablony se knihovna rozhodne
jakému modulu ji předá na druhý krok validace. Způsob validace je popsaný v kapitole \ref{chapter:implementation:library:generic:validation}.
\subsubsection{Struktura šablony}
\label{chapter:implementation:library:generic:structure}
Aby mohla být šablona opakovaně používána pro tvorbu virtuálních strojů, musí být perzistentně uložena v souborovém systému.
Pro tento účel bude použit datový formát JSON \cite{json}, který bude sloužit jako reprezentace šablony. Hlavním důvodem využití tohoto
formátu je relativně dobrá uživatelská čitelnost a především snadné zpracování pomocí programovacího jazyka Ruby. Uživatel
může pro konstrukci šablony použít jednoduchý textový editor nebo grafický editor, který bude součástí uživatelského rozhraní
klientské aplikace. 

Struktura šablony se bude skládat ze dvou částí. První částí bude název a definice typu šablony. Tato hlavička bude určovat
způsob zacházení s danou šablonou. V ukázce kódu \ref{code:generic_template} je naznačeno, jak by mohla taková šablona vypadat.
Atribut \textit{type} určuje o jaký typ virtuálního stroje se jedná a k jakému modulu knihovny přísluší. Druhou povinnou položkou
v šabloně je atribut \textit{name}, který má za úkol popsat funkcionalitu virtuálního stroje. Tečky v ukázce
\ref{code:generic_template} reprezentují atributy specifické pro konkrétní typ šablony. Z důvodu úspory místa byly tyto atributy
v ukázce vynechány.
\begin{lstlisting}[language=json, caption={Generická šablona}, float,label={code:generic_template}]  
{
  "name": "template_webserver",
  "type": "szones",
  ...
}
\end{lstlisting}
\subsubsection{Validace šablony}
\label{chapter:implementation:library:generic:validation}
Šablona musí poskytovat validní definici virtuálního stroje, aby z ní bylo možné konkrétní virtuální stroj zkonstruovat.
Pro tento účel je nutné zavést validaci šablon a jejich atributů. Jelikož je pro ukládání šablon použitý datový formát JSON,
může být pro validaci šablony použité tvz. JSON schéma definované v \cite{json:schema}. Tento dokument je opět ve formátu
JSON, ale neslouží pro ukládání dat. Jeho funkcí je definovat formát jiného dokumentu JSON. Jinými slovy pomocí tohoto schématu
je možné specifikovat atributy a typy jejich hodnot, které má konkrétní typ dokumentu obsahovat.

Pomocí tohoto nástroje je tedy možné definovat, jaké atributy může konkrétní typ virtuálního stroje mít. Pokud uživatel sestrojí
nevalidní šablonu virtuálního stroje, knihovna skrze validaci JSON dokumentu pozná, že se jedná o neplatnou konfiguraci. Knihovna
bude implementovat základní schéma, které bude sloužit pro základní validaci šablon. Jak je vidět v ukázce \ref{code:generic_tmplate:validation},
toto schéma vyžaduje, aby v dokumentu byly přítomné atributy \textit{name} a \textit{type}. Podle atributu \textit{type} je 
rozhodnuto, kterému modulu bude konkrétní šablona předána. Moduly aplikace musí implementovat podrobnější schéma, které bude
definovat konkrétní typ virtuálního stroje.

Aby bylo možné v programovacím jazyku Ruby validovat JSON dokumenty, je nutné použít použít knihovnu, která bude implementovat
JSON schéma. Pro tento účel bude aplikace využívat volně dostupné řešení json-schema \cite{json:schema:ruby}, které implementuje
funkce validace dokumentů typu JSON pomocí schémat.
\begin{lstlisting}[language=json, caption={Schéma generické šablony}, float, label={code:generic_tmplate:validation}]  
{
  {
  "title": "general-vm-template",
  "description": "Used for general template distinction",
  "type": "object",
  "properties": {
    "name": {
      "type": "string",
      "description": "Name of the vm template"
    },
    "type": {
      "enum": [ "szones", "vbox" ],
      "description": "Type of the vm template"
    }

  },
  "required": [ "name", "type" ],
  "additionalProperties": true
}
\end{lstlisting}
\section{Modul Solaris Zones}
\label{chapter:implementation:szones}
Modul Solaris Zones bude hlavním stavebním kamenem celé implementace výsledného nástroje. Tento modul bude zařazen do knihovny
popsané v kapitole \ref{chapter:implementation:library} a mimo jiné bude poskytovat základní rutiny pro správu virtualizačního
kontejneru Solaris Zones. Aby mohl být tento modul využíván knihovnou, musí implementovat rozhraní definované v kapitole \ref{chapter:implementation:library:interface}.
Takto podmínka zahrnuje především implementaci tříd pro zpracovávání šablon virtuálních strojů. V rámci tohoto modulu se bude
jednat o šablony neglobálních zón. 

Pro jednoduchou orientaci a nezávislost bude modul rozdělen do následujících vrstev, které budou poskytovat různou funkcionalitu
a budou se vzájemně využívat. 
\begin{itemize}
 \item Management šablon
 \item Nástroje pro správu Solaris Zones
 \item Administrátorské rutiny
 \item Funkční rozhraní modulu
\end{itemize}
Architekturu vrstev modulu je také možné pozorovat na obrázku \ref{image:implemetation:szones}. V následujících kapitolách
bude podrobně popsána funkce jednotlivých vrstev a jejich interakce.
\begin{figure}
    \centering    
    \caption{Modul Solaris Zones}
    \label{image:implemetation:szones}
\end{figure}
\subsection{Šablona Solaris Zones}
\label{chapter:implementation:szones:template}
Nástroje pro správu Solaris Zones neposkytují možnost vytvářet zóny pomocí jednoho předpisu. Jak bylo popsáno v kapitole
\ref{chapter:zones}, k úspěšnému vytvoření neglobální zóny se využívají tři soubory. Prvním a povinným parametrem instalace zóny
je její konfigurace. Není podmínkou aby konfigurace zóny byla v podobě souboru, ale pro automatizaci tohoto procesu je to výhodné.
Dále je nutné instalátoru předat definici softwarových balíků, které má nainstalovat. Posledním nepovinným parametrem instalace
je konfigurace systémových služeb. Aby bylo možné nainstalovat zónu pomocí jednoho souboru, musí tato šablona šablona kombinovat
vlastnosti výše zmíněných souborů. 

Kostra šablony pro neglobální zónu je naznačena v ukázce kódu \ref{code:szones_template}. Z ukázky je patrné, že šablona obsahuje
tři sekce, které korespondují s jednotlivými soubory vyžadovanými při instalaci. Pro úsporu místa jsou z ukázky vynechány konkrétní
atributy zón, které jsou nahrazeny třemi symboly tečky. Modul tedy musí implementovat JSON schéma, které bude využívat pro
validaci šablony a mechanizmy pro její zpracování. Jinými slovy modul musí být ze šablony schopný reprodukovat soubor s 
konfigurací, manifestem a systémovým profilem. Tyto soubory pak nástroj bude využívat pro instalaci neglobální zóny s parametry,
které jsou určené v dané šabloně.
\begin{lstlisting}[language=json, caption={Kostra šablony neglobální zóny}, float,label={code:szones_template}]  
{  
  "name": "template_webserver",
  "type": "szones",
  "configuration": {...},
  "manifest": {
    "packages": [...]
   },
   "profile": {...} 
}
\end{lstlisting}
Pokud bude modul zpracovávat rutinu využívající šablonu, v prvním kroku dojde k rozdělení šablony do třech zmíněných částí.
V následujícím kroku budou tyto části převedeny do vnitřní reprezentace a následně zpracovány.
\subsubsection{Zpracování konfigurace}
\label{chapter:implementation:szones:template:configuration}
Konfigurační sekce šablony se bude skládat z definice globálních atributů zóny popsaných v kapitole \ref{chapter:zones:configuration:global_attributes}
a z definice zdrojů zóny, které jsou popsané v kapitole \ref{chapter:zones:configuration:resources}. Jednoduché globální atributy
budou v šabloně specifikovány přímo pomocí jejich jména a hodnoty. Pro globální atribut typu zóny by mohla definice vypadat 
takto následovně \lstinline[language=json]{"brand" : "solaris"}.

Dále bude tato sekce šablony obsahovat definici zdrojů zóny, které mají složitější strukturu a obsahují několik atributů. Z tohoto
důvodu bude šablona obsahovat speciální atribut \lstinline[language=json]{resources}, který bude typu pole a bude obsahovat definici
všech zdrojů zóny. V rámci jednotlivého zdroje bude použita stejná technika definice atributu jako ve výše zmíněném případě.

Cílem zpracování této části šablony je vygenerovat soubor s konfigurací, který má přesně definovanou syntaxi. Pomocí načtené
šablony uchované v asociativním poli jsou globální atributy přetransformované do podoby, kterou vyžaduje nástroj \verb|zonecfg(1)|.
V případě zdrojů je proveden stejný postup s tím rozdílem, že se před každý zdroj přidá příkaz \verb|add| a jeho definice
se ukončí příkazem \verb|end|. Takto přetransformovaná konfigurace je připravena k použití v nástroji \verb|zonecfg(1)|.
\subsubsection{Zpracování manifestu}
\label{chapter:implementation:szones:template:manifest}
V rámci sekce  \lstinline[language=json]{manifest} šablony může uživatel specifikovat softwarové balíky, které má zóna obsahovat.
Pro tento účel obsahuje tato část šablony atribut \lstinline[language=json]{packages}, který je typu pole. Hodnotou každého
prvku pole je obyčejný textový řetězec, který obsahuje jméno balíčku. V tomto atributu může uživatel specifikovat libovolné
množství balíčků. 

Cílem zpracování této sekce šablony je vytvořit manifest popsaný v kapitole \ref{chapter:zones:instalation:repozitory:manifest}.
K tomuto účelu si modul drží kopii tohoto souboru, která nese název \textit{manifest\_template.xml} a nachází se v kořenovém 
adresáři aplikace ve složce \textit{szones/manifest}. Při zpracovávání manifestu jsou tedy jednotlivé balíčky načteny ze šablony
a ve správném formátu vloženy do kopie tohoto souboru.
\subsubsection{Zpracování systémového profilu}
\label{chapter:implementation:szones:template:profile}
Poslední částí zpracovávání šablony je transformace systémového profilu. Tato sekce slouží k nastavení systémových služeb
neglobální zóny a bude mít stejnou strukturu jako sekce s konfigurací zóny. Na rozdíl od způsobu zpracování konfigurace je
však v tomto případě vyžadován jiný výstup. Cílem této transformace má být soubor v XML formátu popsaný v kapitole \ref{chapter:zones:instalation:profile}.

Pro každou službu bude existovat korespondující soubor obsahující potřebnou část výsledného XML souboru. Tyto soubory budou
uloženy v kořenovém adresáři aplikace ve složce \textit{szones/profile}. V průběhu zpracování šablony pak budou jednotlivé soubory
načítány a vyplňovány hodnotami ze šablony. Tímto způsobem se zkonstruuje celý soubor, který může být předán instalátoru.
\subsection{Nástroje}
\label{chapter:implementation:szones:commands}
Základní základním stavebním kamenem modulu bude vrstva, která bude zajišťovat vykonávání potřebných příkazů na příkazové řádce.
Tato vrstva bude poskytovat vyšším vrstvám modulu možnost vykonávání základní administračních příkazů pro správu Solaris Zones
a souborového systému ZFS. Konkrétní příkazy jsou vyjmenovány v kapitole \ref{chapter:design:architecture:szones}.

Pro účel vykonávání příkazů bude tato vrstva implementovat třídu, která bude umožňovat provádění příkazů jak na lokálním tak i
na vzdáleném serveru. Tato třída nebude vykonávat příkaz okamžitě, ale umožní vyšším vrstvám aplikace vykonání mohly odložit.
Tento požadavek je na třídu kladen zejména z důvodu efektivity využívání vytvořených SSH spojení. Dále bude třída umožňovat
definovat automatické chování v případě chyby prováděného příkazu. Pro každý nástroj využívaný aplikací bude definována sada
pravidel, které budou reagovat na chybové výstupy nástrojů a vyvolávat příslušné výjimky. Vyšší vrstvy aplikace musí tyto výjimky
odchytávat a adekvátně na ně reagovat.

Pomocí výše zmíněné třídy bude tato vrstva modulu umožňovat volání jednotlivých nástrojů s požadovanými parametry a argumenty.
Například bude umožňovat spouštět nástroj \verb|zonecfg(1)| se jménem konfigurované zóny a cestou k souboru s konfigurací. Vyšším
vrstvám pak bude poskytnutý standardní výstup a standardní chybový výstup daného nástroje.
\subsection{Administrátorské rutiny}
\label{chapter:implementation:szones:routines}
Hlavní částí modulu bude vrstva tříd, které budou implementovat požadované rutiny pro správu Solaris Zones. Rutiny budou využívat
nástrojů implementovaných v nižší vrstvě a budou pomocí nich vytvářet sekvence příkazů nutné k vykonání požadované činnosti.
Tato vrstva bude poskytovat pokročilejší rutiny pro vytváření neglobálních zón, zálohu, obnovu a migraci. Mimo jiné bude také
poskytovat základní funkce pro správu zón. 
\subsubsection{Transakce}
\label{chapter:implementation:szones:routines:transaction}
Jednotlivé rutiny budou implementované jako transakce. Transakce se bude skládat z jednotlivých příkazů, které budou tvořit
celek. V případě úspěchu všech příkazů v transakci bude celá transakce označena za úspěšnou. Jestliže jakýkoliv příkaz v průběhu
transakce selže, selže i celá transakce.

V průběhu některých rutin dochází k vytváření dočasných souborů. Tyto soubory souží například pro přenos konfigurace zóny
na vzdálený server nebo pro dočasné uložení diskového obrazu zóny. Soubory, které jsou dočasně vytvořené v průběhu transakce jsou
zaznamenávány a na konci transakce dojde k jejich smazaní. Status transakce nemám na vykonání tohoto procesu vliv. V případě
Dále některé transakce vytvářejí nové konfigurace zón, snapshoty konkrétních souborových systémů nebo celé diskové obrazy zón. 
Některé z těchto entit mají být například výsledkem transakce. Příkladem může být rutina pro vytváření zóny. V tomto případě 
má být konfigurace zóny a její diskový obraz výsledkem transakce. Tento případ je nutné rozlišit a tyto entity smazat pouze 
v případě neúspěchu transakce. Tuto funkcionalitu v rámci modulu zajišťuje třída \verb|Cleanuper|.

Během zálohovacích a migračních rutin je třeba provádět akce, které nějakým způsobem ovlivní existující zóny. Tyto akce se opět
zaznamenávají a v případě neúspěchu transakce se ovlivněné zóny vrátí do původního stavu. Příkladem může být zastavení běžící
zóny z důvodu její zálohy nebo migrace. Tuto funkcionalitu bude v rámci knihovního modulu zajišťovat třída \verb|Rollbacker|.

Rutiny modulu tedy budou zachovávat stav existujících zón v případě neúspěchu transakce a zajišťovat tak konzistentní stav. Stejně
tak budou zajišťovat, že všechny dočasně vytvořené entity budou po ukončení transakce smazány.
\subsubsection{Vytváření zón}
\label{chapter:implementation:szones:routines:creation}
Hlavní součástí administrátorských rutin budou funkce pro vytváření zón. Tyto rutiny budou umožňovat vytváření vzdálených i
lokálních neglobálních zón z následujících zdrojů.
\begin{itemize}
 \item Ze standardní souborů (konfigurace, manifest, profil)
 \item Ze šablon virtuálních strojů
 \item Z jiné existující zóny (vzdálené i lokální)
 \item Z archivu zóny
\end{itemize}

Prvním způsobem vytvoření zóny je pomocí standardních tří souborů obsahující konfiguraci zóny, manifest a nastavení systémových
služeb. V případě vytváření zóny na vzdáleném serveru dojde v první řadě ke zkopírování zdrojových souborů na cílový server.
Dále se vytvoří konfigurace zóny pomocí nástroje \verb|zonecfg(1)| ze zdrojového konfiguračního souboru. Následuje spuštění
instalace zóny z repozitáře popsané v kapitole \ref{chapter:zones:instalation:repozitory}, kde se jako parametry předají
cesty k souborům s manifestem a systémovým profilem. Tato sekvence příkazů se vykoná lokálně nebo v rámci jednoho SSH spojení
s cílovým serverem.

Dále poskytnuta podpora pro vytváření zón specifikovaných v šablonách popsaných v kapitole \ref{chapter:implementation:szones:template}. 
V průběhu této rutiny nejprve dojde k transformaci šablony, popsané ve stejné kapitole, na standardní soubory. Tato transformace proběhne na 
vždy na lokálním serveru. Dále rutina pokračuje stejně jako v případě instalace ze standardních souborů popsané výše.

Výše zmíněné rutiny pro vytváření zóny neměli k dispozici diskový obraz zón a instalace zóny musela vždy probíhat z repozitáře.
Následující dva způsoby tvorby neglobálních zóny využívají jako zdroj již existující diskový obraz. První z těchto dvou postupů
využívá diskový obraz již existující zóny. Kroky této rutiny můžeme rozdělit na části získání diskového obrazu a samotné instalace
zóny. Obě tyto části mohou být prováděny buď na lokální serveru nebo na vzdáleném. Sever odkud je zóna získávána se označuje
jako \textbf{zdrojový} a server kde je vytvářená nová zóna se nazývá \textbf{cílový}. Před získáváním diskového obrazu se 
nástroj nejprve musí ujistit jestli je zóna v konzistentním stavu a tedy jestli není spuštěná. Pokud je, nástroj ji dočasně zastaví.
Následuje vytvoření archivu zdrojové zóny pomocí techniky ZFS popsané v kapitole \ref{chapter:zones:backup:zfs}. Vytvořený archiv je
následně společně s konfigurací zdrojové zóny přesunut na cílový server a následuje druhá část instalace zóny. Tato část probíhá na
cílovém serveru a zde se nejprve nakonfiguruje cílová zóna s pomocí konfigurace zóny zdrojové. Následuje samotná část připojení
diskového obrazu z poskytnutého archivu. Po tomto procesu je cílová zóna nainstalována na cílový server. Technika klonování
popsaná v kapitole \ref{chapter:zones:instalation:cloning} se použije pouze v případě, že zdrojový a cílový server jsou 
identické stroje.

Poslední podporovanou rutinou pro instalaci neglobálních zón je vytváření z archivu. Tato rutina předpokládá existenci archivu,
který může být typu ZFS nebo UAR. Tvorba obou těchto archivů je popsána v kapitole \ref{chapter:zones:backup}. V případě tvorby
zóny na vzdáleném serveru se nejprve zkopíruje archiv do dočasného adresáře na cílovém serveru. Na cílovém serveru se z archivu
extrahuje konfigurační soubor a pomocí něj se zóna nakonfiguruje. Následuje proces vytvoření kořenového souborového systému cílové
zóny z archivu. Po dokončení tohoto procesu je cílová zóna úspěšně nainstalována na cílovém serveru.

Všechny výše zmíněné typy rutin mají několik společných parametrů, které specifikují chování v krajních situacích. Prvním
takovým parametrem je \textit{force}. Tento parametr nabírá na váze v případě, kdy již existuje zóna se jménem zóny, kterou
chce rutina vytvořit. V případě, že je tento parametr zapnutý, rutina danou existující zóny smaže a na místo ní nainstaluje
zónu novou. V opačném případě skončí rutina s chybou, že se nepodařilo zónu nainstalovat. Druhým parametrem rutin je \textit{boot}.
Tento parametr určuje jestli se vytvořená zóna má rovnou spustit. Implicitní hodnota obou parametrů je nastavená na \textit{false}.
Všechny rutiny pro vytváření neglobálních zón jsou implementované v rámci třídy \verb|DeploymentRoutines|.
\subsubsection{Záloha a obnova zón}
\label{chapter:implementation:szones:routines:backup}
Záloha zón může podle kapitoly \ref{chapter:zones:backup} probíhat dvojím způsobem. Tyto dva způsoby se liší především v technologii,
která vytváří danou zálohu. V obou případech se jedná o vytvoření archivu kořenového souborového systému neglobální zóny. Jeden
způsob používá standardní techniku systémové archivace pomocí UAR archivu a druhý způsob používá přímo nástroje souborového
systému ZFS. Rutiny pro zálohu budou podporovat oba tyto způsoby a budou se lišit pouze v technice vytvoření daného archivu.

Oba typy zálohovacích rutin bude možné provádět na lokálních i vzdálených serverech. V prvním kroku je nutné uvést danou neglobální
zónu do konzistentního stavu a v případně ji zastavit. Následuje proces vytváření archivu, který se liší v závislosti na
použité technice. Po dokončení archivace je záloha hotová. 

V obou případech zálohovacích rutin je možné použít volitelný parametr \textit{archive\_destination} určující, na který server 
se má záloha zkopírovat. Implicitně se záloha vytváří na serveru, kde se daná zóna nachází.

Obnova zóny předpokládá existenci její zálohy. Jelikož se v podstatě jedná o vytvoření zóny z archivu, je tento proces stejný s
procesem vytváření zón z archivu popsaným v předchozí kapitole. Všechny zálohovací rutiny jsou v modulu implementované ve třídě
\verb|MigrationRoutines|.
\subsubsection{Migrace zón}
\label{chapter:implementation:szones:routines:migration}
Migrace v rámci Solaris Zones je přesun neglobální zóny z jednoho virtualizačního serveru na druhý. Jedná se jak o přesun
konfigurace zóny tak i o přesun diskového obrazu. Tato administrační rutina se v mnoha ohledech podobá rutině pro vytvoření
neglobální zóny z jiné již existující zóny. Podstatou je vytvoření archivu zóny na zdrojovém serveru a přesun tohoto archivu
na server cílový. Jak již bylo popsané v následující kapitole, vytvoření archivu je možné provést dvojím způsobem. Migrační rutiny
implementovaného nástroje budou umožňovat migraci zón jak s použitím archivu ZFS tak i s použitím archivu UAR.

Hlavním rozdílem migrace oproti vytváření zóny z již existující zóny je v tom, že původní zóna se v případě úspěchu transakce smaže.
Zóna na zdrojovém server se musí opravdu smazat až v případě, kdy je zóna kompletně nakonfigurována a zdárně nainstalována na
cílovém serveru. Dřívější smazání zdrojové zóny by mohlo vést ke ztrátě dat.

Další rozdíl je v typu používaných příkazů. Před procesem vytváření archivu jakéhokoli typu dojde na zdrojovém serveru k 
použití nástroje \verb|zoneadm(1)| a jeho příkazu \verb|detach|, který bezpečně odpojí diskový obraz zóny od její konfigurace.
Po tomto kroku následuje vytvoření archivu a jeho přesun na cílový server. Jakmile se dokončí přenos archivu je zóna na cílovém
server nakonfigurována a následně je její obraz připojen pomocí příkazu \verb|attach|, který má jako argument vytvořený archiv.
V případě úspěchu může být konfigurace zóny i jejího obrazu smazána ze zdrojového serveru.

Migrační rutiny budou poskytovat ještě jeden typ přenosu diskového obrazu zóny a to přímo pomocí příkazu \verb|zfs send|
a \verb|zfs recv|. Spuštění prvního příkazu na zdrojovém server a druhého příkazu na druhém serveru v rámci SSH spojení
zajistí přenos zdrojového souborového systému z jednoho serveru na druhý. Příkaz \verb|attach| poté již nemá skoro žádnou práci
protože nemusí souborový systém extrahovat z archivu.

Všechny výše zmíněné funkce budou umožňovat migraci mezi všemi hosty dané infrastruktury. Jinými slovy bude umožněno migrovat 
lokální zónu na vzdálený server a naopak a také migrace zóny mezi dvěma vzdálenými hosty. V rámci migračních rutin je možné specifikovat,
že zóna na cílovém serveru se může jmenovat jinak než na zdrojovém. Migrační rutiny jsou v implementovaném modulu zahrnuty
ve třídě s názvem \verb|MigrationRoutines|
\subsubsection{Rutiny pro správu zón}
\label{chapter:implementation:szones:routines:management}
Modul Solaris Zones bude mimo výše zmíněných pokročilejších rutin poskytovat i základní rutiny pro manipulaci a správu 
neglobálních zón. V těchto rutinách bude zahrnuto spouštění, násilné i nenásilné vypnutí, restart nebo úplné odstranění zóny
ze systému. Všechny tyto akce bude možné provádět jak na lokálním tak i na vzdáleném serveru. Třída poskytující tyto
rutiny v rámci implementovaného modulu se nazývá \verb|BasicRoutines|.
\subsection{Funkční rozhraní modulu}
\label{chapter:implementation:szones:api}
Poslední částí modulu pro správu virtualizačního kontejneru Solaris Zones bude rozhraní, které bude nabízet klientským aplikacím
prostřednictvím knihovny. Pro tyto účely bude vytvořena speciální třída, která bude zprostředkovávat administrátorské rutiny popsané 
v minulé kapitole. Princip rozhraní bude fungovat stejně jako v případě knihovny. Toto rozhraní bude mít zaregistrované všechny 
třídy jejichž metody chce veřejně poskytovat knihovně a klientským aplikacím. Pokud knihovna obdrží požadavek na volání nějaké
rutiny, přesměruje tento požadavek právě na toto rozhraní. Rozhraní vyhledá v rámci zaregistrovaných tříd zadali umí obsloužit
konkrétní požadavek. Pokud nějaká z tříd modulu umí danou rutinu provést rozhraní vrátí její návratovou hodnotu. V opačném případě
je vyvolána výjimka, kterou knihovna odchytí a případně bude vyhledávat v ostatních implementovaných modulech.

Ve výsledné implementaci je toto rozhraní reprezentováno třídou \verb|SZONESAPI|, která má zaregistrované třídy modulu korespondující
s rutinami popsanými v kapitole \ref{chapter:implementation:szones:routines}.
\section{Klientská aplikace}
\section{Editor}