\label{chapter:implementation}
V následující kapitole je představena implementace nástroje pro automatickou správu Solaris Zones. Hlavní částí
je popis knihovny, modulu Solaris Zones a klientské aplikace. Důraz je kladen na popis funkcí jednotlivých funkčních
bloků aplikace a jejich vzájemné komunikace.
\section{Programovací jazyk}
\label{chapter:implementation:language}
Prvním krokem  při implementaci bylo zvolení vhodného programovacího jazyka. Požadavky stanovené v kapitole 
\ref{chapter:design:demands} vyžadují od zvoleného programovacího jazyka následující dvě podmínky.
\begin{itemize}
 \item Operační systém Solaris
 \item Možnost tvorby grafického rozhraní
\end{itemize}
Nástroj pro automatickou správu virtualizačního kontejneru Solaris Zones bude stavět hlavně nad využíváním
nástrojů na příkazové řádce a zpracovávání jejich výstupu. Pro tento účel bude vhodné zvolit interpretovaný
programovací jazyk, který umožní jednoduše nástroje spustit a následně snadno zpracovat jejich výstup. Na základě
výstupu se pak nástroj rozhodne o dalším průběhu zpracování uživatelského příkazu. První podmínka není ve volbě
tolik omezující. Pro operační systém Solaris totiž existuje implementace standardního kompilátoru \verb|gcc(1)|
pro jazyk C a stejně tak i implementace virtuálního stroje JVM pro jazyk Java. Většina interpretovaných jazyků
staví svůj překladač právě nad jedním z těchto základních programovacích jazyků.

Z výše uvedených důvodů je nutné při volbě jazyka dbát hlavně na dostupnost grafických knihoven pro operační systém
Solaris. Standardním skriptovacím jazykem pro většinu operačních systému typu UNIX je \verb|shell|. Tento program
interpretuje uživatelské příkazy na příkazové řádce a následně je provádí. Tato volba by splňovala podmínku platformy,
ale těžko by se s pomocí tohoto jazyka vytvářelo grafické uživatelské rozhraní. Z tohoto důvodu zvolíme programovací
jazyk Ruby \cite{ruby}, který umožní splnit obě dvě stanovené podmínky.
\subsection{Ruby}
\label{chapter:implementation:language:ruby}
Ruby je objektově orientovaný programovací jazyk, který má mnoho možností pro využití. Jedním z využití může být
právě spouštění příkazů na příkazové řádce a tvorba uživatelského rozhraní. Objektová povaha tohoto jazyka umožňuje
programátorovi využívat všech výhod objektově orientovaného programování. Podle dokumentu \cite{ruby:implementation}
existuje několik implementací interpretu jazyka Ruby, z nichž nejpoužívanější jsou YARV \cite{ruby:implementation:yarv} 
a JRuby \cite{ruby:implementation:jruby}. Obě tyto implementace jsou dostupné i pro operační systém Solaris.

Pokud chce programátor využívat grafické rozhraní pomocí programovacího jazyka Ruby, je nutné aby byly v systému
nainstalované potřebné grafické knihovny. Standardní knihovny pro programovací jazyk Ruby ale nejsou na na operačním
systému Solaris podporované. Z toho důvodu bude nutné využít grafické rozhraní, které nabízí implementace JRuby.
Tato implementace je postavená nad virtuálním strojem JVM a může využívat grafické knihovny v něm implementované.
Navrhovaný nástroj pro automatickou správu virtualizačního kontejneru Solaris Zones bude tedy využívat programovacího
jazyka Ruby. Pokud bude chtít uživatel nástroje využívat grafického rozhraní bude muset nástroj spouštět pomocí
interpretu JRuby. Zbytek nástroje bude nezávislý na použitém interpretu programovacího jazyka Ruby.
\section{Knihovna}
\label{chapter:implementation:library}
Hlavním centrálním prvkem implementace bude knihovna, která bude zprostředkovávat komunikaci mezi implementovanými moduly
a klientskými aplikacemi. Knihovna je navržená tak, aby se v budoucnosti dala lehce rozšířit o další moduly, které budou
poskytovat funkce pro správu jiných virtualizačních technologií. Jedním z takových rozšíření by mohl být například modul
pro podporu virtualizační technologie Virtualbox. Výsledná implementace bude obsahovat pouze modul pro podporu automatické
správy virtualizačního kontejneru Solaris Zones, který bude dále podrobně popsán v kapitole \ref{chapter:implementation:szones}.

Knihovna bude poskytovat hlavní rozhraní, pomocí kterého bude moci klient využívat funkcí jednotlivých modulů. Jednotlivé
moduly tedy budou sloužit jako hlavní zdroj funkcionality pro knihovnu.

Mimo zprostředkovávání komunikace mezi moduly a klientem bude knihovna sloužit k validaci šablon, které mají specifikovat
konkrétní virtuální stroj. V případě šablon bude knihovna sloužit jako vstupní bod, který umí šablonu načíst a provést
prvotní validaci. Spouštění těchto operací a jejich výsledky bude knihovna zprostředkovávat klientovi.

Jelikož moduly budou implementovat různé typy operací pomocí různých technologií, je nutné ponechat vývojářům velkou volnost
v možnostech jejich implementace. Pro účel zajištění jednotné komunikace s moduly je nutné, aby každý implementovaný modul
splňoval určité rozhraní. Toto rozhraní zajistí, aby všechny moduly mohli jednotně komunikovat s knihovnou a také aby knihovna
mohla zprostředkovávat jejich funkce klientovi. Definice rozhraní modulu je podrobně popsána v kapitole 
\ref{chapter:implementation:library:interface}.

Poslední funkcí knihovny bude udržování hlavní konfigurace. V této konfiguraci bude například uchováván seznam implementovaných
modulů, kořenový adresář knihovny nebo například jméno knihovny. Klientská aplikace pak bude mít možnost tuto konfiguraci
změnit a docílit jiného chování knihovny.
\subsection{Rozhraní modulu}
\label{chapter:implementation:library:interface}
Povinné rozhraní modulu bude sloužit především ke komunikaci mezi knihovnou a samotným modulem. Funkcionalitu, kterou modul
musí poskytovat můžeme shrnout do následujících tří bodů.
\begin{itemize}
 \item Inicializační rutina
 \item Rozhraní poskytované klientům
 \item Funkce pro validaci šablon
\end{itemize}

Prvním požadavkem na rozhraní modulu je existence inicializační rutiny. Pomocí této rutiny je do modulu předána hlavní
konfigurace knihovny, která umožňuje modulu zjistit kořenový adresář aplikace a další volitelné parametry. Hlavním smyslem
této rutiny je inicializace daného modulu. Hlavní knihovna v rámci inicializační smyčky spustí tuto rutinu pro každý
registrovaný modul. V rámci této rutiny modul může provádět inicializaci vlastních datových struktur nebo vytvoření potřebné
adresářové struktury. Dále může modul využít hlavní konfiguraci knihovny k doplnění vlastní lokální konfigurace. Tímto způsobem
je zajištěno, že všechny registrované moduly knihovny obdrží globální konfiguraci a dojde k jejich inicializaci.

Dalším nutnou částí rozhraní modulu jsou funkce, které mají být poskytovány klientovi. K tomuto účelu musí modul poskytovat
třídu, která bude tyto funkce implementovat nebo je bude pouze zprostředkovávat pomocí jiných tříd modulu. Tato třída je tedy
hlavním funkčním rozhraním modulu, které klientské aplikace mohou využívat. V rámci inicializace celé knihovny dojde nejprve
k inicializaci jednotlivých modulů. Po této akci knihovna provede registraci těchto tříd a v udržuje si jejich seznam.

Posledním požadavkem na rozhraní modulu je existence funkcí pro validaci šablon. Tyto funkce musí umožňovat validovat šablony,
které se týkají konkrétního modulu knihovny. Jinými slovy modul, který podporuje správu virtualizačního kontejneru, musí poskytovat
funkce pro validaci šablon specifikující neglobální zóny.

Vlastní implementace modulu není nijak jinak omezena. Jediným logickým omezením je fakt, že tento modul musí být napsaný
v programovacím jazyku Ruby. Pokud modul splní výše zmíněné požadavky, může být jednoduše registrován do knihovny a klientské
aplikace ho můžou bezprostředně po inicializaci knihovny využívat. Na obrázku \ref{module:interface} je názorně zobrazeno 
jak knihovna využívá rozhraní modulu a jakým způsobem je modul poskytován klientské aplikaci.
\begin{figure}
    \centering    
    \caption{Rozhraní modulu}
    \label{module:interface}
\end{figure}
\subsection{Přesměrování požadavků}
\label{chapter:implementation:library:interface}
Mimo inicializace modulů je hlavní funkcí knihovny přesměrovávat požadavky klientských aplikací na funkční rozhraní 
implementovaných modulů. K tomuto účelu obsahuje knihovna hlavní třídu, která virtuálně reprezentuje rozhraní všech modulů
knihovny. Tato třída se nazývá hlavní rozhraní. Jak již bylo zmíněno v kapitole \ref{chapter:implementation:library:interface},
knihovna si udržuje odkazy na hlavní třídy modulů, které reprezentují jejich funkční rozhraní.

V okamžiku, kdy klientská aplikace vznese požadavek na zavolání nějaké funkce, knihovna v době běhu zjistí jakému modulu daná
funkce přísluší a vyvolá ji. Pokud neexistuje žádný modul, který umí danou funkci provést, dojde k vyvolání výjimky a aplikace
se ukončí. Díky tomuto chování může dojít ke kolizi jmen funkcí. V takovém případě by knihovna vyvolala takovou funkci, kterou
by našla jako první v pořadí. Z tohoto důvodu je nutné se vyvarovat opakování jmen funkcí a nejlépe používat pro funkce určitého
modulu prefix, který daný modul jasně identifikuje. Například jeho jméno.

Toto směrování za běhu aplikace je umožněno díky programovacímu jazyku Ruby a jeho možnosti dynamického volání funkcí za běhu
programu. Směrování požadavků ke konkrétním modulům knihovny je názorně ukázáno na obrázku \ref{module:interface}.
\subsection{Generická šablona}
\label{chapter:implementation:library:generic}
Poslední funkcionalitou knihovny je definice generické šablony, která má za úkol specifikovat virtuální stroj. Hlavním úkolem
generické šablony bude specifikace typu virtuálního stroje. Tento atribut bude určovat, který modul knihovny je zodpovědný
za zpracování a validaci. Generická šablona by dále měla obsahovat jméno, které bude nějakým způsobem vystihovat a popisovat
specifikovaný virtuální stroj. Knihovna bude tedy zajišťovat validaci těchto dvou atributů a v případě úspěchu předá
šablonu zodpovědnému modulu.

Knihovna tedy bude klientským aplikacím poskytovat funkce pro načítání a validaci šablon. V případě úspěšného načtení šablony
bude knihovna vracet objekt, který bude moci být použit v rámci konkrétního modulu. Validace šablony bude provedena ve dvou
krocích. Knihovna nejprve zjistí jestli daná šablona obsahuje atribut jména a typu. Podle typu šablony se knihovna rozhodne
jakému modulu ji předá na druhý krok validace. Způsob validace je popsaný v kapitole \ref{chapter:implementation:library:generic:validation}.
\subsubsection{Struktura šablony}
\label{chapter:implementation:library:generic:structure}
Aby mohla být šablona opakovaně používána pro tvorbu virtuálních strojů, musí být perzistentně uložena v souborovém systému.
Pro tento účel bude použit datový formát JSON \cite{json}, který bude sloužit jako reprezentace šablony. Hlavním důvodem využití tohoto
formátu je relativně dobrá uživatelská čitelnost a především snadné zpracování pomocí programovacího jazyka Ruby. Uživatel
může pro konstrukci šablony použít jednoduchý textový editor nebo grafický editor, který bude součástí uživatelského rozhraní
klientské aplikace. 

Struktura šablony se bude skládat ze dvou částí. První částí bude název a definice typu šablony. Tato hlavička bude určovat
způsob zacházení s danou šablonou. V ukázce kódu \ref{code:generic_template} je naznačeno, jak by mohla taková šablona vypadat.
Atribut \textit{type} určuje o jaký typ virtuálního stroje se jedná a k jakému modulu knihovny přísluší. Druhou povinnou položkou
v šabloně je atribut \textit{name}, který má za úkol popsat funkcionalitu virtuálního stroje. Tečky v ukázce
\ref{code:generic_template} reprezentují atributy specifické pro konkrétní typ šablony. Z důvodu úspory místa byly tyto atributy
v ukázce vynechány.
\begin{lstlisting}[language=json, caption={Generická šablona}, float,label={code:generic_template}]  
{
  "name": "template_webserver",
  "type": "szones",
  ...
}
\end{lstlisting}
\subsubsection{Validace šablony}
\label{chapter:implementation:library:generic:validation}
Šablona musí poskytovat validní definici virtuálního stroje, aby z ní bylo možné konkrétní virtuální stroj zkonstruovat.
Pro tento účel je nutné zavést validaci šablon a jejich atributů. Jelikož je pro ukládání šablon použitý datový formát JSON,
může být pro validaci šablony použité tvz. JSON schéma definované v \cite{json:schema}. Tento dokument je opět ve formátu
JSON, ale neslouží pro ukládání dat. Jeho funkcí je definovat formát jiného dokumentu JSON. Jinými slovy pomocí tohoto schématu
je možné specifikovat atributy a typy jejich hodnot, které má konkrétní typ dokumentu obsahovat.

Pomocí tohoto nástroje je tedy možné definovat, jaké atributy může konkrétní typ virtuálního stroje mít. Pokud uživatel sestrojí
nevalidní šablonu virtuálního stroje, knihovna skrze validaci JSON dokumentu pozná, že se jedná o neplatnou konfiguraci. Knihovna
bude implementovat základní schéma, které bude sloužit pro základní validaci šablon. Jak je vidět v ukázce \ref{code:generic_tmplate:validation},
toto schéma vyžaduje, aby v dokumentu byly přítomné atributy \textit{name} a \textit{type}. Podle atributu \textit{type} je 
rozhodnuto, kterému modulu bude konkrétní šablona předána. Moduly aplikace musí implementovat podrobnější schéma, které bude
definovat konkrétní typ virtuálního stroje.

Aby bylo možné v programovacím jazyku Ruby validovat JSON dokumenty, je nutné použít použít knihovnu, která bude implementovat
JSON schéma. Pro tento účel bude aplikace využívat volně dostupné řešení json-schema \cite{json:schema:ruby}, které implementuje
funkce validace dokumentů typu JSON pomocí schémat.
\begin{lstlisting}[language=json, caption={Schéma generické šablony}, float, label={code:generic_tmplate:validation}]  
{
  {
  "title": "general-vm-template",
  "description": "Used for general template distinction",
  "type": "object",
  "properties": {
    "name": {
      "type": "string",
      "description": "Name of the vm template"
    },
    "type": {
      "enum": [ "szones", "vbox" ],
      "description": "Type of the vm template"
    }

  },
  "required": [ "name", "type" ],
  "additionalProperties": true
}
\end{lstlisting}
\section{Modul Solaris Zones}
\label{chapter:implementation:szones}


\section{Klientská aplikace}
\section{Editor}