% Virtualization chapter about concept, types, examples and implementation
V dnešní době existuje mnoho různých typů virtualizace, jak již bylo zmíněno v úvodu. Následující kapitola zběžně představuje několik nejznámější typů virtualizace v IT a dále se věnuje hlavně
tématu virtualizace serverů. V kapitole jsou popsány některé scénáře pro přechod z klasické infrastruktury na virtuální. Ve zbytku kapitoly jsou popsány obecné principy virtualizace a základy 
virtualizace CPU, paměti a I/O zařízení.

\section{Využití virtualizace v sítích}
\section{Virtualizace desktopu}
\section{Virtualizace serverů}

\section{Definice pojmů}
Pro potřeby popisu virtualizačních technik si definujeme základní pojmy a entity, které se ve virtuální infrastruktuře vyskytují.

\textit{Fyzické prostředky} TODO

\textit{Virtualizační monitor}, \textit{Virtual Machine Monitor - VMM} nebo také \textit{Hypervisor} je softwarová vrstva, která virtualizuje HW prostředky fyzického počítače a přiděluje je
virtuálním strojům.

\textit{Host} je HW a SW platforma, která poskytuje virtuálním strojům výpočetní výkon, paměť, úložiště, síťové připojení a další fyzické prostředky. Softwarové vybavení hosta obsahuje virtualizační
monitor a v některých případech může obsahovat i operační systém hosta tzv. \textit{Host OS}

\textit{Virtuální stroj} nebo \textit{Virtual Machine - VM} je virtualizované prostředí vytvořené virtualizačním monitorem, ve kterém běží operační systém virtuálního počítače nazývaný \textit{Guest OS}.

\section{Nasazení virtuální infrastruktury}
Pro přechod k virtuální infrastruktuře serverů existuje v dnešní době několik dobrých důvodů. Jedním z hlavních benefitů virtualizace pro dnešní firmy a organizace je značná finanční úspora. Tato úspora se
projevuje především ve snížení nákladů organizace na pořizování a provoz fyzických zařízení. 

Mezi další benefity virtualizace patří především efektivní využití výpočetních zdrojů, vysoká dostupnost běžících aplikací nebo vytvoření oddělených a nezávislých prostředí pro vývoj, testování a nasazení software.

Výhody zavedení virtuální infrastruktury jsou podrobněji popsány v následujících podkapitolách, které se zabývají základními scénáři pro nasazení virtuální infrastruktury.

\subsection{Konsolidace}

Konsolidace serverů je proces sjednocování více fyzických serverů na jeden fyzický server, který pro tyto servery poskytne virtuální prostředí pro jejich běh. Vstupem tohoto procesu je tedy několik fyzických serverů,
na kterých běží různé aplikace. Vstup procesu je naznačen na obrázku \ref{consolidation} vlevo. Výstupem konsolidace je jeden fyzický server s dostatečnými prostředky, na kterém konsolidované servery běží jako virtuální počítače.
Výstup můžeme vidět na obrázku \ref{consolidation} vpravo.

\begin{figure}
    \centering    
    \caption{Konsolidace serverů}
    \label{consolidation}
\end{figure}

\subsubsection*{Využití scénáře}

Dnes je zcela běžnou praktikou provozovat jednu aplikaci na jednom dedikovaném serveru. Pokud aplikace využívá jen malé procento výpočetních zdrojů daného serveru, může administrátor sjednotit více takovýchto serverů
do jednoho. Pro organizaci, která vlastní tisíce takovýchto serverů může konsolidace výrazně zmenšit požadavky na prostor, spotřebu energie a provoz fyzických serverů. Správnou konsolidací serverů může společnost docílit
efektivního využití dostupných prostředků a tím výrazně snížit vynaložené finanční prostředky. \cite{reasons}

Rychlý vývoj technologií v oblasti hadware zapříčiňuje rychlé stárnutí některých systémů a přechod ze staršího na nový může být složitý. Obzvláště v případě, kdy systém potřebuje ke svému běhu speciální hadware.
Aby bylo možné provozovat služby poskytované těmito zastaralými systémy, můžeme je spustit jako virtuální počítač na modernějším hadware. Systém se bude chovat stejně jako kdyby běžel na zastaralém hadware, zatímco
výkonost služby může těžit z novější a výkonnější hadwarové vrstvy. \cite{reasons}

\subsection{Izolace}

Dalším ze scénářů využití virtualizované infrastruktury je izolace aplikací. Proces izolace aplikací spočívá v oddělení dvou a více kritických aplikací běžících na jednom systému do nezávislých virtuálních prostředí.
Vstupem je jeden systém s aplikacemi, které se mohou negativně ovlivňovat. Vstup izolačního scénáře je naznačen na obrázku \ref{izolation} vlevo. Výstupem je několik nezávislých virtuálních počítačů, ve kterých běží
jednotlivé aplikace. Výstup izolace aplikací je ukázán na obrázku \ref{izolation} vpravo.

\begin{figure}
    \centering    
    \caption{Izolace aplikací}
    \label{izolation}
\end{figure}

\subsubsection*{Využití scénáře}

V dnešní době jsou útoky na aplikace vystavené do internetu běžnou záležitostí. Pokud útočník využije nějaké zranitelnosti aplikace, může v některých případech získat kontrolu nad celým systémem. V takovém případě jsou
ohroženy všechny data a aplikace, které na daném systému běží. Vhodným krokem v tomto případě je proto využití virtualizace a rozdělení aplikací do nezávislých prostředí.

Jedním z příkladů ohrožení systému může být útok na výpočetní zdroje. Podstatou útoku je vyčerpání fyzických zdrojů systému, což má za následek nedostupnost jeho služeb a v některých případech i pád celého systému.
Ve virtualizovaném prostředí lze přidělit každému VM pouze určitou část prostředků a tím chránit celý systém před jejich vyčerpáním. V případě napadení jednoho VM sice dojde k jeho vyřazení, ale ostatní VM a jejich služby
mohou dále pokračovat v běhu.

\subsection{Migrace}

\subsubsection*{Migrace virtuálního stroje}

\begin{figure}
    \centering    
    \caption{Migrace virtuálního stroje}
    \label{migration1}
\end{figure}

\subsubsection*{Migrace fyzického na virtuální stroj}

\begin{figure}
    \centering    
    \caption{Migrace fyzického na virtuální stroj}
    \label{migration2}
\end{figure}

\subsubsection*{Využití scénáře}

fail-over / vysoká dostupnost

eliminace legacy hw






\section{Virtualizační monitor}
\section{Techniky virtualizace}
\section{Virtualizace CPU}
\section{Virtualizace Paměti}
\section{Virtualizace I/O zařízení}


