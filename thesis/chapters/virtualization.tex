% Virtualization chapter about concept, types, examples and implementation
Přechod na virtualizovanou infrastrukturu může mít hned několik důvodů. Úvod následující kapitoly představuje důvody, které přechod několik základních scénářů, kdy je vhodné tento přechod učinit.
Ve zbytku kapitoly jsou popsány jednotlivé typy virtualizace a základní virtualizační techniky.
\section{Definice pojmů}
Pro potřeby popisu virtualizačních technik si definujeme základní pojmy a entity, které se ve virtuální infrastruktuře vyskytují.

\textit{Virtualizační monitor}, \textit{Virtual Machine Monitor - VMM} nebo také \textit{Hypervisor} je softwarová vrstva, která virtualizuje HW prostředky fyzického počítače a přiděluje je
virtuálním strojům.

\textit{Host} je HW a SW platforma, která poskytuje virtuálním strojům výpočetní výkon, paměť, úložiště, síťové připojení a další fyzické prostředky. Softwarové vybavení hosta obsahuje virtualizační
monitor a v některých případech může obsahovat i operační systém hosta tzv. \textit{Host OS}

\textit{Virtuální stroj} nebo \textit{Virtual Machine - VM} je virtualizované prostředí vytvořené virtualizačním monitorem, ve kterém běží operační systém virtuálního počítače nazývaný \textit{Guest OS}.

\section{Nasazení virtuální infrastruktury}

\subsection{Izolace}
\subsection{Konsolidace}
\subsection{Migrace}

\section{Virtualizace sítí}
\section{Virtualizace desktopu}
\section{Virtualizace serverů}
\section{Techniky virtualizace}
\section{Virtualizační monitor}