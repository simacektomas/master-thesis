% Virtualization chapter about concept, types, examples and implementation
V dnešní době existuje mnoho různých typů virtualizace, jak již bylo zmíněno v úvodu. Následující kapitola zběžně představuje několik nejznámější typů virtualizace v IT a dále se věnuje hlavně
tématu virtualizace serverů. V kapitole jsou popsány některé scénáře pro přechod z klasické infrastruktury na virtuální. Ve zbytku kapitoly jsou popsány obecné principy virtualizace a základy 
virtualizace CPU, paměti a I/O zařízení.

TODO general definition of virtualization

\section{Definice pojmů}
Pro potřeby popisu virtualizačních technik si definujeme základní pojmy a entity, které se ve virtuální infrastruktuře vyskytují.

\textit{Fyzické prostředky} počítače jsou TODO

\textit{Virtualizační monitor}, \textit{Virtual Machine Monitor - VMM} nebo také \textit{Hypervisor} je softwarová vrstva, která virtualizuje HW prostředky fyzického počítače a přiděluje je
virtuálním strojům.

\textit{Virtuální prostředky} TODO

\textit{Host} je HW a SW platforma, která poskytuje virtuálním strojům své fyzické prostředky. Softwarové vybavení hosta obsahuje virtualizační
monitor a v některých případech může obsahovat i operační systém hosta tzv. \textit{Host OS}

\textit{Virtuální stroj} nebo \textit{Virtual Machine - VM} je virtualizované prostředí vytvořené virtualizačním monitorem, ve kterém běží operační systém virtuálního počítače nazývaný \textit{Guest OS}.

\section{Virtualizace serverové infrastruktury}

První a pravděpodobně nejvýznamnější oblastí využití virtualizace, je oblast výpočetních serverů. Výpočetní server je fyzický počítač s operačním systémem a výpočetními prostředky, který může poskytovat nějaké služby svému okolí.
V dnešní době se s těmito servery setkáváme každý den a to v podobě webových serverů, serverů poskytujících službu DNS a mnohých dalších.

Proces virtualizace serverů spočívá v přenesení systémů a jejich služeb z fyzických serverů do virtuálního prostředí, které je nezávislé na nižších vrstvách. Toto virtuální prostředí je vytvořeno virtualizačním monitorem běžícím na jiném fyzickém stroji a následně poskytováno virtuálním strojům.
Virtualizační monitor nebo také VMM je detailněji popsán v kapitole \ref{vmm}, kde jsou představeny jeho typy a funkce. Serverová architektura využívající virtualizace se skládá především z tzv. virtualizačních serverů, které slouží jako zdroj fyzických
prostředků pro virtuální počítače. Tyto servery se vyznačují především velkým množstvím operační paměti a vysokým výpočetním výkonem, který je díky VMM rozdělován mezi hostované virtuální stroje. Výhody nasazení virtuální infrastruktury jsou dále
popsány v kapitole \ref{vm_deployment}

Tato práce se zaměřuje právě na techniky virtualizace, které jsou v dnešní době aktuální a využívají se k virtualizaci serverů. Práce podrobně představuje virtualizační techniku Solaris Zones of firmy Oracle, která slouží pro
vyváření virtuálních strojů (zón), které sdílejí jedno jádro OS.

\section{Využití virtualizace v sítích}

Oblast komunikačních sítí je další neméně významnou oblastí pro využití virtualizačních technik. Bez síťové infrastruktury by mezi sebou počítače nemohli komunikovat, a tudíž by jejich využití nemělo takový potenciál. V dnešní době
je tato infrastruktura značně rozsáhlá a to její správu ztěžuje. S virtualizací přichází do sítí možnost dynamické konfigurace sítě a to i její topologie. To vše lze uskutečnit z jednoho místa a bez nutnosti zasahovat do fyzických zařízení sítě.


Virtualizace sítí je koncept, který se v mnoha ohledech podobná virtualizaci serverů. V případě serverů, se VMM stará o reprodukci vlastností fyzických prostředků v SW. Podobně je to tomu i v případě virtualizace sítí, kde existuje funkční ekvivalent VMM,
který reprodukuje síťové komponenty v SW. Administrátor má tak možnost za chodu vytvářet virtuální síťové komponenty jako je switch, router, firewall nebo load balancer a to vše v rámci desítek sekund. Tento síťový hypervisor
také umožňuje spravovat nové virtuální sítě, které zahrnují všechny standardní síťové služby a kvalitu služeb.\cite{vn}

\section{Virtualizace desktopu}

Společně s virtualizací serverů a sítí je virtualizace desktopu posledním typem virtualizace, která stojí za zmínku. Pod pojmem desktop si představme klasický stolní počítač, který má obrazovku, myš a klávesnici.

S desktopem je klasicky spojeno grafické uživatelské prostředí, pomocí kterého uživatel ovládá počítač, instaluje aplikace nebo přizpůsobuje prostředí. Bez využití virtualizace nebo další podpůrných systémů jsou všechny informace o uživatelském nastavení uloženy
na desktopu a uživatel se k nim dostane pouze z toho samého stroje. Virtualizací desktopu je rozuměno oddělení uživatelského prostředí a nastavení od fyzického stroje. Jednou z možností je přesunutí tohoto prostředí do virtuálního stroje, který je centrálně spravován a
spouštěn, když uživatel potřebuje. Tento koncept umožňuje uživateli přístup ke svému prostředí téměř bez ohledu na lokalitu nebo platformu. Mezi další benefity zavedení virtualizovaného desktopu patří zvýšení bezpečnosti a zjednodušení správy
celého systému. Tyto výhody pramení především z centralizaci tohoto řešení.

\section{Nasazení virtuální infrastruktury}
\label{vm_deployment}

Pro přechod k virtuální infrastruktuře serverů existuje v dnešní době několik dobrých důvodů. Jedním z hlavních benefitů virtualizace pro dnešní firmy a organizace je značná finanční úspora. Tato úspora se
projevuje především ve snížení nákladů organizace na pořizování a provoz fyzických zařízení. 

Mezi další benefity virtualizace patří především efektivní využití výpočetních zdrojů, vysoká dostupnost běžících aplikací nebo vytvoření oddělených a nezávislých prostředí pro vývoj, testování a nasazení software.

Výhody zavedení virtuální infrastruktury jsou podrobněji popsány v následujících podkapitolách, které se zabývají základními scénáři pro nasazení virtuální infrastruktury.

\subsection{Konsolidace}
\label{consolidation}

Konsolidace serverů je proces sjednocování systémů z více fyzických serverů na jeden fyzický server, který pro tyto systémy poskytne virtuální prostředí pro jejich běh. Vstupem tohoto procesu je tedy několik systémů na fyzických serverech,
na kterých běží různé aplikace. Vstup procesu je naznačen na obrázku \ref{consolidation_img} vlevo. Výstupem konsolidace je jeden fyzický server s dostatečnými prostředky, na kterém konsolidované systémy běží jako virtuální počítače.
Výstup můžeme vidět na obrázku \ref{consolidation_img} vpravo.

\begin{figure}
    \centering    
    \caption{Konsolidace serverů}
    \label{consolidation_img}
\end{figure}

\subsubsection*{Využití scénáře}

Dnes je zcela běžnou praktikou provozovat jednu aplikaci na jednom dedikovaném serveru. Pokud aplikace využívá jen malé procento výpočetních zdrojů daného serveru, může administrátor sjednotit více takovýchto serverů
do jednoho. Pro organizaci, která vlastní tisíce takovýchto serverů může konsolidace výrazně zmenšit požadavky na prostor, spotřebu energie a provoz fyzických serverů. Správnou konsolidací serverů může společnost docílit
efektivního využití dostupných prostředků a tím výrazně snížit vynaložené finanční prostředky \cite{reasons}.

Rychlý vývoj technologií v oblasti hadware zapříčiňuje rychlé stárnutí některých systémů a přechod ze staršího na nový může být složitý. Obzvláště v případě, kdy systém potřebuje ke svému běhu speciální hadware.
Aby bylo možné provozovat služby poskytované těmito zastaralými systémy, můžeme je spustit jako virtuální počítač na modernějším hadware. Systém se bude chovat stejně jako kdyby běžel na zastaralém hadware, zatímco
výkonost služby může těžit z novější a výkonnější hadwarové vrstvy  \cite{reasons}.

\subsection{Izolace}

Dalším ze scénářů využití virtualizované infrastruktury je izolace aplikací. Proces izolace aplikací spočívá v oddělení dvou a více kritických aplikací běžících na jednom systému do nezávislých virtuálních prostředí.
Vstupem je jeden systém s aplikacemi, které se mohou negativně ovlivňovat. Vstup izolačního scénáře je naznačen na obrázku \ref{izolation} vlevo. Výstupem je několik nezávislých virtuálních počítačů, ve kterých běží
jednotlivé aplikace. Výstup izolace aplikací je ukázán na obrázku \ref{izolation} vpravo.

\begin{figure}
    \centering    
    \caption{Izolace aplikací}
    \label{izolation}
\end{figure}

\subsubsection*{Využití scénáře}

V dnešní době jsou útoky na aplikace vystavené do internetu běžnou záležitostí. Pokud útočník využije nějaké zranitelnosti aplikace, může v některých případech získat kontrolu nad celým systémem. V takovém případě jsou
ohroženy všechny data a aplikace, které na daném systému běží. Vhodným krokem v tomto případě je proto využití virtualizace a rozdělení aplikací do nezávislých prostředí.

Jedním z příkladů ohrožení systému může být útok na výpočetní zdroje. Podstatou útoku je vyčerpání fyzických zdrojů systému, což má za následek nedostupnost jeho služeb a v některých případech i pád celého systému.
Ve virtualizovaném prostředí lze přidělit každému VM pouze určitou část prostředků a tím chránit celý systém před jejich vyčerpáním. V případě napadení jednoho VM sice dojde k jeho vyřazení, ale ostatní VM a jejich služby
mohou dále pokračovat v běhu.

\subsection{Migrace}

Posledním diskutovaným scénářem nasazení virtualizované infrastruktury je migrace. Jedná se o proces přesunutí systému z jednoho počítače na druhý. V rámci virtualizace se budeme bavit o přesouvání systému na počítač s běžícím VMM,
který zprostředkovává virtuální prostředí. Výstupem procesu je systém, který do něj zároveň vstupuje. Rozdíl je v tom, že daný systém na konci procesu běží ve virtuálním prostředí nějakého VMM. Dle typu migrovaného systému můžeme
rozdělit scénář na následující typy.

\subsubsection*{Migrace VM}

Migrací virtuálního stroje se rozumí přesun VM mezi dvěma různými fyzickými stroji s VMM. Tento přesun byl dříve možný pouze v případě když oba stroje měli stejný HW, operační systém a procesor \cite{reasons}.
Tato možnost administrátorovi umožňuje přesouvat virtualizované systémy na výkonnější hosty a tím umožňuje dynamicky regulovat využití fyzických prostředků v závislosti na aktuální zátěži systému.

Další výhodou zavedení virtualizované architektury je zajištění vysoké dostupnosti služeb. Virtualizace umožňuje zajistit redundanci ve smyslu spuštění služby na více serverech najednou. Ve virtualizované architektuře můžou
nastat dva typy selhání. Prvním typem je selhání VM uvnitř VMM. Pokud dojde k selhání některé VM, jiná VM převezme obsluhu požadavků a v minimálním čase dojde k obnovení služby. Druhým typem je selhání celého VMM nebo hosta.
V tomto případě je nutné provozovat více redundantních hostů pro VM, které v případě HW chyby převezmou obsluhu služby.

Proces migrace VM je představen na obrázku \ref{migration1}, kde můžeme vidět konfiguraci před migrací (vlevo) a po provedení migrace VM (vpravo).

\begin{figure}
    \centering    
    \caption{Migrace virtuálního stroje}
    \label{migration1}
\end{figure}

\subsubsection*{Migrace fyzického stoje na VM}

Migrace fyzického stroje na virtuální stroj je proces, kdy dochází k přesunu a virtualizaci systému z fyzického stroje. Vstupem je tedy systém běžící na stroji bez VMM, jak je naznačeno na obrázku \ref{migration2} vlevo. 
Výstupem tohoto procesu je opět virtuální stroj běžící ve virtuálním prostředí VMM.

Virtualizací serverů dochází k uvolňování fyzického HW a stejně jako v případě konsolidace tak společnost může značně ušetřit na nákladech nutných k provozu a správě fyzických serverů. Obecně lze říct, že tento scénář přináší 
podobné benefity jako v případě konsolidace popsané v kapitole \ref{consolidation}.

\begin{figure}
    \centering    
    \caption{Migrace fyzického na virtuální stroj}
    \label{migration2}
\end{figure}

\section{Virtualizační monitor}
\label{vmm}



\section{Techniky virtualizace}



\section{Virtualizace CPU}
\section{Virtualizace Paměti}
\section{Virtualizace I/O zařízení}


