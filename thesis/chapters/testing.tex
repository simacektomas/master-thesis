\label{chapter:testing}
Poslední kapitola této diplomové práce popisuje testování funkcionality nástroje pro podporu automatické správy virtualizačního
kontejneru Solaris Zones, jehož implementace je popsána v kapitole \ref{chapter:implementation}. Zaměřuje se především testování
jednotlivých scénářů použití nástroje a zkoumá jeho chování. Na začátku této kapitoly je definováno prostředí, ve kterém byly
testy prováděny. Následuje série testů, které zkoumají funkčnost nástroje v konkrétních případech použití. Kapitola je zakončena
měřením, které zkoumá dobu trvání některých funkcí nástroje.
\section{Definice testovacího prostředí}
\label{chapter:testing:environment}
Pro účely testování výše zmíněného nástroje bylo nutné vytvořit prostředí odpovídající jeho cílové platformě. Toto prostředí
obsahuje několik virtualizačních serverů s operačním systémem Solaris, který bude poskytovat své prostředky neglobálním zónám.
Tyto servery jsou propojeny počítačovou sítí, pomocí které je lze ovládat. Tato infrastruktura virtualizovaně vytvořena na 
fyzickém systému s následujícími parametry.
\begin{itemize}
 \item Procesor Intel(R) Xeon(R) CPU E3-1230 v3 (3.30Ghz)
 \item RAM 16GB
 \item Operační systém Windows 10 (64-bit)
\end{itemize}
Virtualizace architektury byla docílena pomocí virtualizační technologie Virtualbox, která umožňuje spouštění virtuálních 
počítačů v rámci jiného operačního systému. Pomocí této technologie byly vytvořeny tři virtuální stroje s operačním systémem
Solaris ve verzi 11.3. Tyto stroje byly propojeny pomocí virtuální počítačové sítě a nakonfigurovány tak, aby se na ně dalo 
připojovat pomocí SSH. Dále byly jednotlivým strojů přiřazeny doménové jména \textit{shost}, \textit{shost1} a \textit{shost2},
a přidány korespondující řádky do souboru \textit{/etc/hosts}. Toto nastavení umožňuje používat specifikované doménové jména
místo IP adres a zjednoduší tak identifikaci strojů v testovacích ukázkách. 

Jelikož provozování virtualizační technologie Solaris Zones vyžaduje nemalé množství výpočetní prostředků, bylo nutné
dostupné prostředky fyzického systému rozdělit mezi virtuální stroje. Z tohoto důvodu byly každému virtuálnímu počítači
přiřazeny následující výpočetní prostředky.
\begin{itemize}
 \item Jedno jádro fyzického procesoru
 \item RAM 3 GB 
 \item Virtuální disk 50 GB (HDD)
\end{itemize}

Na virtuální počítač s doménovým jménem \textit{shost} byl nainstalován interpret programovacího jazyka Ruby ve verzi 2.4.2.
Dále byla na stejný počítač nainstalována Java ve verzi 1.8.0\_60 a následně druhý interpret programovacího jazyka Ruby
tentokrát ve verzi 2.3.3 a implementaci JRuby. Pokud nebude uvedeno jinak, testovaný nástroj bude vždy spouštěn z virtuálního
počítače s doménovým jménem \textit{shost}.

Posledním učiněným krokem byla konfigurace uživatele \textit{zadmin}, který má práva na vykonávání příkazů nutných k správnému
chodu implementovaného nástroje. Tyto nástroje byly vyjmenovány v kapitole \ref{chapter:design:architecture:szones}. Uživatel 
byl pomocí nástroje RBAC vytvořen a nakonfigurován na všech vytvořených virtuálních počítačích.
\section{Testování scénářů použití}
\label{chapter:testing:scenario}
V následujících kapitolách je popsáno akceptační testování některých scénářů použití nástroje pro podporu automatické správy
Solaris Zones.  Pro testování aplikace bylo vždy použito popsané prostředí, pokud není uvedeno jinak. Na začátku každého scénáře
je stanoven cíl, který se by uživatel chtěl pomocí implementovaného nástroje dosáhnout. Následně je popsán stav prostředí, ve
kterém se sytém nachází před provedením konkrétní akce. Dále proveden korespondující příkaz v uživatelském rozhraní nástroje,
který má splnit stanovený cíl. Výsledek tohoto kroku je ověřen pomocí systémových příkazů a v závěru je rozhodnuto, zda bylo
dosaženo stanoveného cíle.
\subsection{Vytvoření neglobálních ze šablony}
\label{chapter:testing:scenario:deploy_template}
Pro komplexní otestování funkcionality implementovaného nástroje byl zvolen scénář vytvoření několika neglobálních zón pomocí
šablony. Hlavní důvod pro výběr tohoto scénáře je, že se do tohoto procesu se zapojují téměř všechny části nástroje.

Cílem tohoto scénáře je vytvoření několika neglobálních zón na různých hostech v rámci dané infrastruktury. Jako zdroj byla 
použita šablona popsaná v kapitole \ref{chapter:implementation:szones:template}. Z šablony byly vybrány některé důležité 
vlastnosti, které mají vytvořené zóny mít. Typ zóny byl stanoven jako \textit{solaris} s exkluzivní IP adresou. Dále má 
nainstalovaná zóna obsahovat softwarové balíčky pro správu zdrojového kódu. Tyto balíčky obsahují nástroje \verb|hg| a 
\verb|git|. Šablona také definuje počáteční heslo uživatele \textit{root} a nastavuje typ tohoto uživatelského účtu na roli.
Vedle uživatele \textit{root} je v šabloně definován počáteční systémový uživatel, který má být zároveň systémový administrátor.
Vytvářené neglobální zóny mají mít jedno síťové rozhraní se jménem \textit{net0}, které bude konfigurováno automaticky pomocí
DHCP. Takto definovaná šablona je uložena na severu s doménovým jménem \textit{shost}.

Příkaz pro vytvoření čtyř zón pomocí uživatelského rozhraní nástroje je v ukázce kódu \ref{code:test:deployment} na první řádce.
Tento příkaz říká, že mají být vytvořeny zóny \textit{zdev} a \textit{zdev1} na lokálním serveru a zóna \textit{zdev} na vzdálených
serverech \textit{shost1} a \textit{shost2}. Jako parametr \textit{specification} je udána cesta k šabloně s výše popsanými 
vlastnostmi. Dále byl příkazu předán parametr \textit{boot}, který rovnou spustí vytvořené zóny.
\begin{lstlisting}[basicstyle=\scriptsize\ttfamily, caption={Vytvoření neglobálních zón ze šablony}, float,label={code:test:deployment}]  
# szmgmt_cli deploy -b zdev zdev zdev:shost1 zdev:shost2 -s ~/zdev.json 
Solaris zones deployment from virtual machine specification initialized.
  ---------------------------------------------------------
  Options:
                Boot zones: enable
    Rewrite existing zones: disable
                    Source: specification </export/home/zadmin/zdev.json>
  ---------------------------------------------------------
  Loading virtual machine specification.
  Virtual machine specification loaded.
  ---------------------------------------------------------
  Connecting concurrently to hosts 'localhost, shost1, shost2'.
  Processing zone 'zdev' deployment on host 'localhost'.
  Processing zone 'zdev1' deployment on host 'localhost'.
  Processing zone 'zdev' deployment on host 'shost1'.
  Processing zone 'zdev' deployment on host 'shost2'.
  ---------------------------------------------------------
  Deployment finished.
    Status:
      localhost:
        zdev: success
        zdev1: success
      shost1:
        zdev: success
      shost2:
        zdev: success
\end{lstlisting}

Z konce standardního výstupu nástroje v ukázce kódu \ref{code:test:deployment} je patrné, že vytvoření všech neglobálních zón
proběhlo v pořádku. Jelikož se pro vytváření zón používá stejná rutina a stejná šablona, musí mít všechny stejné parametry.
Pro otestování korektnosti práce nástroje byla použita neglobální zóna \textit{zdev} na vzdáleném serveru \textit{shost2}.
Korektní vytvoření a spuštění zóny bylo ověřeno pomocí nástroje \verb|zlogin(1)|, který umožňuje připojení ke konzoly dané
zóny. Úspěšné přihlášení v ukázce kódu \ref{code:test:deployment:result} signalizovalo hned několik věcí. Za prvé se zdárně
podařilo vytvořit a spustit danou zónu a za druhé byly správně nakonfigurovány uživatelské systémové služby pomocí atributů ze šablony.
Dále je ukázky \ref{code:test:deployment:result} patrné, že došlo k vytvoření síťového adaptérů \textit{net0} a jeho automatické
konfigurace pomocí služby DHCP. Daná neglobální zóna tak byla okamžitě po vytvoření dostupná ze sítě. Přítomnost softwarových
balíčků byla otestována pomocí jejich rozhraní na příkazové řádce jak je vidět v ukázce \ref{code:test:deployment:result}.
Posledním kritériem úspěchu bylo správné nakonfigurování počátečního systémového uživatele. Jméno a heslo bylo ověřeno již
při přihlašování do zóny. Zbývalo tedy ověřit jestli uživatel má práva systémového administrátora, což bylo provedeno pomocí
příkazu \verb|profile|. Výpis na ukázce \ref{code:test:deployment:result} je zkrácený, ale obsahuje profil \textit{System Administrator}.
\begin{lstlisting}[basicstyle=\scriptsize\ttfamily, caption={Ověření správného vytvoření zóny}, float,label={code:test:deployment:result}]  
zadmin@shost2:~$ zlogin -C  zdev
[Connected to zone 'zweb' console]
Hostname: solaris
solaris console login: admin
Password:
admin@solaris:~$ ifconfig net0
net0: flags=100001000843<UP,BROADCAST,RUNNING,MULTICAST,IPv4,PHYSRUNNING>
    inet 10.164.85.13 netmask ff000000 broadcast 10.255.255.255
admin@solaris:~$ git --version
git version 1.7.9.2
admin@solaris:~$ hg --version
Mercurial Distributed SCM (version 3.4)
simactom@solaris:~$ profiles
System Administrator
...
\end{lstlisting}

Pomocí výše zmíněných testů bylo ověřeno, že se daná zóna vytvořila, spustila a že měla vlastnosti specifikované v použité šabloně.
Stejným způsobem byly ověřeny i ostatní vytvářené zóny. Jelikož tyto zóny vykazovaly stejné chování a vlastnosti, byl tento
test uzavřen a konstatován jako splněný.
\subsection{Využití uživatelského žurnálu}
\label{chapter:testing:scenario:journal}
Dalším využitím implementovaného nástroje může být využití uživatelského žurnálu. Cílem následujícího scénáře je kontrola 
funkcionality uživatelského žurnálu a jeho schopnosti informovat uživatele o změnách neglobálních zón v rámci infrastruktury.
K tomuto účelu byl využitý stav, ve kterém se systém nacházel po testování předchozího scénáře popsaného v kapitole
\ref{chapter:testing:scenario:deploy_template}. Součástí předchozího scénáře bylo vytvoření čtyř zón pomocí implementovaného
nástroje. Před tímto vytvořením se v systému nenacházely žádné jiné neglobální zóny. V tomto stavu by měl uživatelský žurnál 
obsahovat čtyři zóny ve stavu \textit{running}. Jak je vidět z ukážky kódu \ref{code:test:journal}, součástí uživatelského
žurnálu byly opravdu čtyři zóny ve stavu \textit{running} a výpis neobsahoval žádné jiné nesledované neglobální zóny.
Toto zjištění indikovalo, že nástroj opravdu aktualizuje uživatelský nástroj provedenou akcí.
\begin{lstlisting}[basicstyle=\scriptsize\ttfamily, caption={Uživatelské žurnál po vytvoření zón}, float,label={code:test:journal}]  
zadmin@shost:~$ szmgm_cli journal status
Tracked zones:
  Host localhost
      zdev1:localhost
           Zone type: solaris
          Zone state: running
           Zone path: /system/zones/zdev1
      zdev:localhost
           Zone type: solaris
          Zone state: running
           Zone path: /system/zones/zdev
  Host shost2
      zdev:shost2
           Zone type: solaris
          Zone state: running                      
           Zone path: /system/zones/zdev
  Host shost1
      zdev:shost1
           Zone type: solaris
          Zone state: running
           Zone path: /system/zones/zdev
\end{lstlisting}
Následně byla simulována situace, kdy jiný uživatel změní nějakým způsobem stav sledované zóny. Konkrétně byla bez pomoci 
implementovaného nástroje přeinstalována zóna \textit{zdev} na vzdáleném serveru \textit{shost2} a její stav byl změněn z původního
\textit{running} na \textit{installed}. Dále byla vytvořena nová zóna \textit{zdev-clone} na stejném vzdáleném počítači. V tomto
případě by měl uživatel při dalším vypsání uživatelského žurnálu zjistit, že se změnil stav a diskový obraz dané zóny 
\textit{zdev} změnil. Součástí výpisu by měla být i informace o nově vytvořené zóny v rámci infrastruktury. Z ukázky kódu 
\ref{code:test:journal:change} je vidět, že uživatelský žurnál opravdu informuje uživatele o změně sledované zóny a na konci
výpisu je zobrazena informace o nově vytvořené zóně. Pro úsporu místa byly ostatní zóny z výpisu vynechány.
\begin{lstlisting}[basicstyle=\scriptsize\ttfamily, caption={Uživatelské žurnál po změně}, float,label={code:test:journal:change}]  
zadmin@shost:~$ szmgm_cli journal status
Tracked zones:
  ...
  Host shost2
      zweb:shost2
           Zone type: solaris
          Zone state: running
                      MISMATCH - Fresh zone state property is installed.
           Zone path: /system/zones/zweb
                      MISMATCH - Fresh zone UUID mismatch.
Untracked zones:
  Host shost2
      zdev-colne:shost2
          Zone type: solaris
         Zone state: installed
          Zone path: /system/zones/zdev-colne
\end{lstlisting}

Pomocí implementovaného nástroje může uživatel neglobální zóny vytvářet, mazat nebo měnit jejich stav. Jak bylo zjištěno
na začátku této kapitoly, nástroj aktualizuje uživatelský žurnál a jeho konkrétní záznam pokud danou zónu vytváří. Podobným
způsobem bylo ověřeno, že uživatelský žurnál je aktualizován i při mazání a změně stavu. Dále tento scénář ověřil funkcionalitu
žurnálu, která má informovat uživatele v případě, kdy dojde ke změně stavu sledované zóny nebo vytvoření nové zóny v rámci
infrastruktury. Z výše uvedených důvodů bylo testování využití uživatelského žurnálu úspěšné.
\subsection{Migrace neglobálních zón}
\label{chapter:testing:scenario:migration}
\subsection{Záloha a obnova zón}
\label{chapter:testing:scenario:migration}

\section{Měření}
\label{chapter:measurement}