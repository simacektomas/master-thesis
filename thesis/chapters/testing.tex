\label{chapter:testing}
Poslední kapitola této diplomové práce popisuje testování funkcionality nástroje pro podporu automatické správy virtualizačního
kontejneru Solaris Zones, jehož implementace je popsána v kapitole \ref{chapter:implementation}. Zaměřuje se především testování
jednotlivých scénářů použití nástroje a zkoumá jeho chování. Na začátku této kapitoly je definováno prostředí, ve kterém byly
testy prováděny. Následuje série testů, které zkoumají funkčnost nástroje v konkrétních případech použití. Kapitola je zakončena
měřením, které zkoumá dobu trvání některých funkcí nástroje.
\section{Definice testovacího prostředí}
\label{chapter:testing}
Pro účely testování výše zmíněného nástroje bylo nutné vytvořit prostředí odpovídající jeho cílové platformě. Toto prostředí
obsahuje několik virtualizačních serverů s operačním systémem Solaris, který bude poskytovat své prostředky neglobálním zónám.
Tyto servery jsou propojeny počítačovou sítí, pomocí které je lze ovládat. Tato infrastruktura virtualizovaně vytvořena na 
fyzickém systému s následujícími parametry.
\begin{itemize}
 \item Procesor Intel(R) Xeon(R) CPU E3-1230 v3 (3.30Ghz)
 \item RAM 16GB
 \item Operační systém Windows 10 (64-bit)
\end{itemize}
Virtualizace architektury byla docílena pomocí virtualizační technologie Virtualbox, která umožňuje spouštění virtuálních 
počítačů v rámci jiného operačního systému. Pomocí této technologie byly vytvořeny tři virtuální stroje s operačním systémem
Solaris ve verzi 11.3. Tyto stroje byly propojeny pomocí virtuální počítačové sítě a nakonfigurovány tak, aby se na ně dalo 
připojovat pomocí SSH. Dále byly jednotlivým strojů přiřazeny doménové jména \textit{shost}, \textit{shost1} a \textit{shost2},
a přidány korespondující řádky do souboru \textit{/etc/hosts}. Toto nastavení umožňuje používat specifikované doménové jména
místo IP adres a zjednoduší tak identifikaci strojů v testovacích ukázkách. 

Jelikož provozování virtualizační technologie Solaris Zones vyžaduje nemalé množství výpočetní prostředků, bylo nutné
dostupné prostředky fyzického systému rozdělit mezi virtuální stroje. Z tohoto důvodu byly každému virtuálnímu počítači
přiřazeny následující výpočetní prostředky.
\begin{itemize}
 \item Jedno jádro fyzického procesoru
 \item RAM 3 GB 
 \item Virtuální disk 50 GB (HDD)
\end{itemize}

Na virtuální počítač s doménovým jménem \textit{shost} byl nainstalován interpret programovacího jazyka Ruby ve verzi 2.4.2.
Dále byla na stejný počítač nainstalována Java ve verzi 1.8.0\_60 a následně druhý interpret programovacího jazyka Ruby
tentokrát ve verzi 2.3.3 a implementaci JRuby. Pokud nebude uvedeno jinak, testovaný nástroj bude vždy spouštěn z virtuálního
počítače s doménovým jménem \textit{shost}.

Posledním učiněným krokem byla konfigurace uživatele \textit{zadmin}, který má práva na vykonávání příkazů nutných k správnému
chodu implementovaného nástroje. Tyto nástroje byly vyjmenovány v kapitole \ref{chapter:design:architecture:szones}. Uživatel 
byl pomocí nástroje RBAC vytvořen a nakonfigurován na všech vytvořených virtuálních počítačích.
\section{Testování scénářů použití}
\section{Měření}