\label{chapter:testing}
Poslední kapitola této diplomové práce popisuje testování funkcionality nástroje pro podporu automatické správy virtualizačního
kontejneru Solaris Zones, jehož implementace je popsána v kapitole \ref{chapter:implementation}. Zaměřuje se především testování
jednotlivých scénářů použití nástroje a zkoumá jeho chování. Na začátku této kapitoly je definováno prostředí, ve kterém byly
testy prováděny. Následuje série testů, které zkoumají funkčnost nástroje v konkrétních případech použití. Kapitola je zakončena
měřením, které zkoumá dobu trvání některých funkcí nástroje.
\section{Definice testovacího prostředí}
\label{chapter:testing:environment}
Pro účely testování výše zmíněného nástroje bylo nutné vytvořit prostředí odpovídající jeho cílové platformě. Toto prostředí
obsahuje několik virtualizačních serverů s operačním systémem Solaris, který bude poskytovat své prostředky neglobálním zónám.
Tyto servery jsou propojeny počítačovou sítí, pomocí které je lze ovládat. Tato infrastruktura virtualizovaně vytvořena na 
fyzickém systému s následujícími parametry.
\begin{itemize}
 \item Procesor Intel(R) Xeon(R) CPU E3-1230 v3 (3.30Ghz)
 \item RAM 16GB
 \item Operační systém Windows 10 (64-bit)
\end{itemize}
Virtualizace architektury byla docílena pomocí virtualizační technologie Virtualbox, která umožňuje spouštění virtuálních 
počítačů v rámci jiného operačního systému. Pomocí této technologie byly vytvořeny tři virtuální stroje s operačním systémem
Solaris ve verzi 11.3. Tyto stroje byly propojeny pomocí virtuální počítačové sítě a nakonfigurovány tak, aby se na ně dalo 
připojovat pomocí SSH. Dále byly jednotlivým strojů přiřazeny doménové jména \textit{shost}, \textit{shost1} a \textit{shost2},
a přidány korespondující řádky do souboru \textit{/etc/hosts}. Toto nastavení umožňuje používat specifikované doménové jména
místo IP adres a zjednoduší tak identifikaci strojů v testovacích ukázkách. 

Jelikož provozování virtualizační technologie Solaris Zones vyžaduje nemalé množství výpočetní prostředků, bylo nutné
dostupné prostředky fyzického systému rozdělit mezi virtuální stroje. Z tohoto důvodu byly každému virtuálnímu počítači
přiřazeny následující výpočetní prostředky.
\begin{itemize}
 \item Jedno jádro fyzického procesoru
 \item RAM 3 GB 
 \item Virtuální disk 50 GB (HDD)
\end{itemize}

Na virtuální počítač s doménovým jménem \textit{shost} byl nainstalován interpret programovacího jazyka Ruby ve verzi 2.4.2.
Dále byla na stejný počítač nainstalována Java ve verzi 1.8.0\_60 a následně druhý interpret programovacího jazyka Ruby
tentokrát ve verzi 2.3.3 a implementaci JRuby. Pokud nebude uvedeno jinak, testovaný nástroj bude vždy spouštěn z virtuálního
počítače s doménovým jménem \textit{shost}.

Posledním učiněným krokem byla konfigurace uživatele \textit{zadmin}, který má práva na vykonávání příkazů nutných k správnému
chodu implementovaného nástroje. Tyto nástroje byly vyjmenovány v kapitole \ref{chapter:design:architecture:szones}. Uživatel 
byl pomocí nástroje RBAC vytvořen a nakonfigurován na všech vytvořených virtuálních počítačích.
\section{Testování scénářů použití}
\label{chapter:testing:scenario}
V následujících kapitolách je popsáno akceptační testování některých scénářů použití nástroje pro podporu automatické správy
Solaris Zones.  Pro testování aplikace bylo vždy použito popsané prostředí, pokud není uvedeno jinak. Na začátku každého scénáře
je stanoven cíl, který se by uživatel chtěl pomocí implementovaného nástroje dosáhnout. Následně je popsán stav prostředí, ve
kterém se sytém nachází před provedením konkrétní akce. Dále proveden korespondující příkaz v uživatelském rozhraní nástroje,
který má splnit stanovený cíl. Výsledek tohoto kroku je ověřen pomocí systémových příkazů a v závěru je rozhodnuto, zda bylo
dosaženo stanoveného cíle.
\subsection{Vytvoření neglobálních ze šablony}
\label{chapter:testing:scenario:deploy_template}
Pro komplexní otestování funkcionality implementovaného nástroje byl zvolen scénář vytvoření několika neglobálních zón pomocí
šablony. Hlavní důvod pro výběr tohoto scénáře je, že se do tohoto procesu se zapojují téměř všechny části nástroje.

Cílem tohoto scénáře je vytvoření několika neglobálních zón na různých hostech v rámci dané infrastruktury. Jako zdroj byla 
použita šablona popsaná v kapitole \ref{chapter:implementation:szones:template}. Z šablony byly vybrány některé důležité 
vlastnosti, které mají vytvořené zóny mít. Typ zóny byl stanoven jako \textit{solaris} s exkluzivní IP adresou. Dále má 
nainstalovaná zóna obsahovat softwarové balíčky pro správu zdrojového kódu. Tyto balíčky obsahují nástroje \verb|hg| a 
\verb|git|. Šablona také definuje počáteční heslo uživatele \textit{root} a nastavuje typ tohoto uživatelského účtu na roli.
Vedle uživatele \textit{root} je v šabloně definován počáteční systémový uživatel, který má být zároveň systémový administrátor.
Vytvářené neglobální zóny mají mít jedno síťové rozhraní se jménem \textit{net0}, které bude konfigurováno automaticky pomocí
DHCP. Takto definovaná šablona je uložena na severu s doménovým jménem \textit{shost}.

Příkaz pro vytvoření čtyř zón pomocí uživatelského rozhraní nástroje je v ukázce kódu \ref{code:test:deployment} na první řádce.
Tento příkaz říká, že mají být vytvořeny zóny \textit{zdev} a \textit{zdev1} na lokálním serveru a zóna \textit{zdev} na vzdálených
serverech \textit{shost1} a \textit{shost2}. Jako parametr \textit{specification} je udána cesta k šabloně s výše popsanými 
vlastnostmi. Dále byl příkazu předán parametr \textit{boot}, který rovnou spustí vytvořené zóny.


Z konce standardního výstupu nástroje v ukázce kódu \ref{code:test:deployment} je patrné, že vytvoření všech neglobálních zón
proběhlo v pořádku. Jelikož se pro vytváření zón používá stejná rutina a stejná šablona, musí mít všechny stejné parametry.
Pro otestování korektnosti práce nástroje byla použita neglobální zóna \textit{zdev} na vzdáleném serveru \textit{shost2}.
Korektní vytvoření a spuštění zóny bylo ověřeno pomocí nástroje \verb|zlogin(1)|, který umožňuje připojení ke konzoly dané
zóny. Úspěšné přihlášení v ukázce kódu \ref{code:test:deployment:result} signalizovalo hned několik věcí. Za prvé se zdárně
podařilo vytvořit a spustit danou zónu a za druhé byly správně nakonfigurovány uživatelské systémové služby pomocí atributů ze šablony.
Dále je ukázky \ref{code:test:deployment:result} patrné, že došlo k vytvoření síťového adaptérů \textit{net0} a jeho automatické
konfigurace pomocí služby DHCP. Daná neglobální zóna tak byla okamžitě po vytvoření dostupná ze sítě. Přítomnost softwarových
balíčků byla otestována pomocí jejich rozhraní na příkazové řádce jak je vidět v ukázce \ref{code:test:deployment:result}.
Posledním kritériem úspěchu bylo správné nakonfigurování počátečního systémového uživatele. Jméno a heslo bylo ověřeno již
při přihlašování do zóny. Zbývalo tedy ověřit jestli uživatel má práva systémového administrátora, což bylo provedeno pomocí
příkazu \verb|profile|. Výpis na ukázce \ref{code:test:deployment:result} je zkrácený, ale obsahuje profil \textit{System Administrator}.


Pomocí výše zmíněných testů bylo ověřeno, že se daná zóna vytvořila, spustila a že měla vlastnosti specifikované v použité šabloně.
Stejným způsobem byly ověřeny i ostatní vytvářené zóny. Jelikož tyto zóny vykazovaly stejné chování a vlastnosti, byl tento
test uzavřen a konstatován jako splněný.
\subsection{Využití uživatelského žurnálu}
\label{chapter:testing:scenario:journal}
Dalším využitím implementovaného nástroje může být využití uživatelského žurnálu. Cílem následujícího scénáře je kontrola 
funkcionality uživatelského žurnálu a jeho schopnosti informovat uživatele o změnách neglobálních zón v rámci infrastruktury.
K tomuto účelu byl využitý stav, ve kterém se systém nacházel po testování předchozího scénáře popsaného v kapitole
\ref{chapter:testing:scenario:deploy_template}. Součástí předchozího scénáře bylo vytvoření čtyř zón pomocí implementovaného
nástroje. Před tímto vytvořením se v systému nenacházely žádné jiné neglobální zóny. V tomto stavu by měl uživatelský žurnál 
obsahovat čtyři zóny ve stavu \textit{running}. Jak je vidět z ukážky kódu \ref{code:test:journal}, součástí uživatelského
žurnálu byly opravdu čtyři zóny ve stavu \textit{running} a výpis neobsahoval žádné jiné nesledované neglobální zóny.
Toto zjištění indikovalo, že nástroj opravdu aktualizuje uživatelský nástroj provedenou akcí.

Následně byla simulována situace, kdy jiný uživatel změní nějakým způsobem stav sledované zóny. Konkrétně byla bez pomoci 
implementovaného nástroje přeinstalována zóna \textit{zdev} na vzdáleném serveru \textit{shost2} a její stav byl změněn z původního
\textit{running} na \textit{installed}. Dále byla vytvořena nová zóna \textit{zdev-clone} na stejném vzdáleném počítači. V tomto
případě by měl uživatel při dalším vypsání uživatelského žurnálu zjistit, že se změnil stav a diskový obraz dané zóny 
\textit{zdev} změnil. Součástí výpisu by měla být i informace o nově vytvořené zóny v rámci infrastruktury. Z ukázky kódu 
\ref{code:test:journal:change} je vidět, že uživatelský žurnál opravdu informuje uživatele o změně sledované zóny a na konci
výpisu je zobrazena informace o nově vytvořené zóně. Pro úsporu místa byly ostatní zóny z výpisu vynechány.


Pomocí implementovaného nástroje může uživatel neglobální zóny vytvářet, mazat nebo měnit jejich stav. Jak bylo zjištěno
na začátku této kapitoly, nástroj aktualizuje uživatelský žurnál a jeho konkrétní záznam pokud danou zónu vytváří. Podobným
způsobem bylo ověřeno, že uživatelský žurnál je aktualizován i při mazání a změně stavu. Dále tento scénář ověřil funkcionalitu
žurnálu, která má informovat uživatele v případě, kdy dojde ke změně stavu sledované zóny nebo vytvoření nové zóny v rámci
infrastruktury. Z výše uvedených důvodů bylo testování využití uživatelského žurnálu úspěšné.
\subsection{Migrace neglobálních zón}
\label{chapter:testing:scenario:migration}
V rámci testování nástroje pro podporu automatické správy virtualizačního kontejneru Solaris Zones byl otestován scénář, který
zahrnuje migraci neglobálních zón mezi dvěma hosty. Jelikož nástroj umožňuje migraci zón jak z lokálního tak ze vzdálených serverů,
bylo nutné vybrat komplexní scénář pokrývající tyto možnosti. Pro komplexní ověření této funkcionality nástroje bylo na hostech 
\textit{shost} a \textit{shost1} vytvořeno několik neglobálních zón, které byly migrovány na cílový vzdálený server \textit{shost2}. 
Pro účely tohoto scénáře byla využita technika přímého posílání souborového systému, kterou implementovaný nástroj poskytuje.
Scénář migrace je graficky znázorněn na obrázku \ref{image:testing:migration}.
\begin{figure}
    \centering    
    \caption{Migrační scénář}
    \label{image:testing:migration}
\end{figure}

Cílem tohoto scénáře je přesunutí neglobálních zón ze zdrojových serverů na cílový server. Na obou zdrojových serverech se na
začátku testování nacházely následující neglobální zóny.
\begin{itemize}
 \item \textit{zmigr:localhost}
 \item \textit{zmigr1:localhost}
 \item \textit{zmigr2:shost1}
 \item \textit{zmigr3:shost1}
\end{itemize}
Po provedení migrace by se tyto zóny měli nacházet na cílovém serveru \textit{shost2} a na zdrojových serverech by se neměly
nacházet žádné neglobální zóny. Uživatelský žurnál zobrazený v ukázce \ref{code:test:migration:before} tuto skutečnost potvrzuje.
 
Pro migraci neglobálních zón slouží příkaz \verb|migrate|, který je možné využívat skrze uživatelské rozhraní implementovaného
nástroje. Právě tento příkaz byl použitý pro testování tohoto scénáře a v ukázce \ref{code:test:migration} je možné vidět jeho
výstup. Jak je z konce výstupu patrné, všechny zóny byly podle nástroje úspěšně přesunuty na cílový server \textit{shost2}. 
Pro ověření funkcionality bylo nutné ověřit jestli se zóny opravdu nachází na cílovém serveru, jestli jsou zóny smazány
ze zdrojových serverů a také jestli nástroj aktualizoval uživatelský žurnál. 

Pomocí nástroje \verb|zoneadm(1)| a příkazu \verb|list| bylo ověřeno, že na vzdáleném serveru se opravdu nachází přesunuté zóny.
Na zdrojových serverech byl spuštěn stejný příkaz pro ověření, že se na nich nenachází žádné globální zóny. Výstup těchto kroků
je zobrazen v ukázce \ref{code:test:migration:list}. Těmito kroky bylo ověřeno, že nástroj správně pracuje se zónami a umožňuje
jejich přesun v rámci jednotlivých serverů infrastruktury. Dále bylo nutné ověřit, zda nástroj správně aktualizuje konkrétní záznamy
v uživatelském žurnálu. Stav v ukázce \ref{code:test:migration:before} by se tedy měl změnit tak, že přesouvané zóny budou
zobrazeny pod hostem \textit{shost2}. Tato skutečnost je potvrzena ukázkou \ref{code:test:migration:after}, která ukazuje očekávaný
výstup z uživatelského žurnálu.

Výše popsaný testovací scénář dokazuje správné chování implementovaného nástroje v případě, kdy uživatel využívá administračních
funkcí pro migraci zón. Test potvrzuje, že uživatel je schopný přesouvat neglobální zóny z lokálního i ze vzdáleného serveru
na cílový server. Migrace je provedena najednou a nástroj po jejím úspěšném provedení adekvátně aktualizuje uživatelský žurnál.
Z výše popsaných důvodů bylo testování migračního scénáře označeno za úspěšné.
\subsection{Záloha a obnova zón}
\label{chapter:testing:scenario:backup_recovery}
Posledním testovaným scénářem bylo vytvoření zálohy a následná obnova. Jelikož se jedná o dvě samostatné funkce nástroje, byl
tento test rozdělený do dvou scénářů. V prvním scénáři použití bylo otestováno vytvoření zálohy několika zón. Následující scénář
potom využil vytvořené zálohy k obnovení zón.
\subsubsection{Záloha neglobálních zón}
\label{chapter:testing:scenario:backup_recovery:backup}
Cílem tohoto scénáře bylo ověření funkcionality nástroje ve vytváření záloh neglobálních zón. Při procesu vytváření zálohy má
nástroj za úkol vytvořit archiv souborového systému zálohované zón. Nástroj umožňuje vytvoření zálohy pomocí dvou typů archivů.
V případě tohoto scénáře byl použit archiv typu UAR. Dále nástroj umožňuje specifikovat vzdálený server a cestu, kam chceme
zálohu uložit. Pomocí tohoto scénáře bylo ověřeno, zda nástroj tuto funkcionalitu opravdu umožňuje. Pro kompletní ověření
funkcionality bude sloužit následující scénář, který z vytvořených záloh bude zóny obnovovat. Pro účely tohoto scénáře byly v infrastruktuře vytvořeny následující 
zóny.
\begin{itemize}
 \item \textit{zback:shost2}
 \item \textit{zback1:shost2}
 \item \textit{zback2:shost1}
 \item \textit{zback3:shost1}
\end{itemize}
Dále byl na lokálním serveru \textit{shost} vytvořen zálohovací adresář \textit{/zonepool/backup}, ve kterém se před zahájením
testu nenacházely žádné soubory.

Nástroj pro zálohování zón nabízí příkaz \verb|backup|, který je možný využít v uživatelském rozhraní. Pomocí tohoto příkazu bylo
spuštěna záloha výše zmíněných zón. Výstup tohoto příkazu je zobrazen v ukázce \ref{code:test:backup} a ukazuje, že záloha
zón proběhla úspěšně. Zálohy by se podle výstupu programu měly nacházet v složce \textit{/zonepool/backup} na lokálním serveru.

Pomocí standardního nástroje \verb|ls(1)| bylo ověřeno, že se zálohy opravdu vytvořili a že se nachází v určeném adresáři na
lokálním serveru. Tento testovací scénář potvrzuje, že nástroj je schopný najednou vytvářet zálohy více neglobálních zón
runě umístěných v infrastruktuře serverů. Dále potvrzuje, že nástroj umí stáhnout zálohy to konkrétního adresáře na předem
určeném serveru. Z výše zmíněných důvodů bylo testování vytváření zálohy neglobálních zón úspěšné.
\subsubsection{Obnova neglobálních zón}
\label{chapter:testing:scenario:backup_recovery:recovery}
Posledním testovaným scénářem použití nástroje byla obnova neglobálních zón z archivu. Cílem testování tohoto scénáře bylo 
ověření, zda nástroj umí obnovit (vytvořit) neglobální zóny pomocí dříve vytvořených záloh. Jako zdrojové archivy byly použity
zálohy vytvořené během předchozího testování. Pro simulaci ztráty dat byly všechny zóny z minulého testování odstraněny a 
zachovány byly pouze jejich archivy ve složce \textit{/zonepool/backup} na lokálním serveru. Výsledkem obnovy by mělo být
vytvoření všech zón ze zálohy a jejich umístění na původní servery. 

Ve složce  na začátku testování nacházely čtyři archivy typu UAR, které obsahovaly diskové obrazy a konfigurace jednotlivých
zón. Nástroj pro podporu automatické správy Solaris Zones poskytuje funkcionalitu pro obnovu zón skrze příkaz \verb|recovery|.
Pomocí tohoto nástroje byla spuštěna obnova z výše zmíněných archivů. Tento proces je vyznačen v ukázce \ref{code:test:recovery}
, kde je vidět i výstup tohoto příkazu. Z příkazu je patrné, že pořadí specifikace identifikátorů zón musí odpovídat pořadí
zadaných archivů. Pokud by toto pořadí nesouhlasilo, došlo by k prohození diskových obrazů daných zón. Podle výstupu nástroje
proběhla obnova zón v pořádku a zóny by měly být nainstalované na své původní hosty.

Stejně jako v ostatních případech vytváření zón bylo pomocí nástroje \verb|zoneadm(1)| a příkazu \verb|list| ověřeno, že se
obnovené zóny opravdu vytvořily na daných vzdálených serverech. Testování tohoto scénáře potvrdilo, že nástroj umí obnovit
neglobální zóny ze sady záloh. Tyto zálohy musí mít tvar definovaný v kapitole \ref{chapter:implementation:szones:routines:backup}.
V případě tohoto scénáře se jednalo o zálohy typu UAR, které byly vytvořeny v rámci předchozího testování. Závěrem je možné říct,
že testování tohoto scénáře využití bylo úspěšné.
\section{Měření}
\label{chapter:measurement}
V rámci testování nástroje pro podporu automatické správy virtualizačního kontejneru Solaris Zones bylo provedeno měření doby
běhu některých funkčních prvků. Konkrétně se jednalo o měření doby běhu vytváření a migrace neglobálních zón v rámci popsané
infrastruktury. V obou případech se přihlíželo k jiným parametrům měření, které jsou detailněji popsány v následujících dvou
kapitolách. K měření bylo opět použito prostředí definované v kapitole \ref{chapter:testing:environment}.
\subsection{Vytváření zón}
\label{chapter:measurement:}
Prvním objektem měření bylo sledování doby běhu nástroje při vytváření zón v závislosti na počtu vytvářených zón.

\subsection{}
\label{chapter:measurement:}

\subsection{Závěr testování}
\label{chapter:testing:scenario:conclusion}
Všechny testované scénáře byly prováděny v rámci prostředí definovaném v kapitole \ref{chapter:testing:environment}. Vybrané 
scénáře použití by měly pokrývat hlavní funkcionalitu nástroje a jejich splnění by mělo vypovídat funkčnosti celého nástroje.
Jelikož jejich testování proběhlo bez chyb a s očekávanými výsledky, je možné konstatovat, že funkcionalita splňuje požadavky
stanovené v kapitole \ref{chapter:design:demands}.