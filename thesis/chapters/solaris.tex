% Brief description of Solaris operating system
Solaris nebo dříve SunOS je operační systém původně vytvořený firmou Sun Microsystems. V~současné době je vyvíjený a podporovaný
firmou Oracle. Je to komplexní unixový operační systém, který v~sobě integruje pokročilé technologie pro virtualizaci, moderní
souborový systém ZFS, vlastní systém pro~instalaci a správu SW a v~neposlední řadě také podporu cloudu. Díky integraci těchto
technologií poskytuje Solaris stabilní a rychlé prostředí pro různé scénáře nasazení aplikací a navíc tato integrace vytváří
pohodlné rozhraní pro~správu tohoto OS. 
\section{Verze Solarisu}
\label{chapter:solaris:version}
Nejaktuálnější stabilní verze operačního systému Solaris je verze s~označením 11.3. Ke dni 30. ledna 2018 byla uvolněna beta
verze 11.4 \cite{oracle:solaris:beta}. Pro účely popisu operačního systému Solaris a jeho služeb, zejména služby Solaris Zones,
bude použita stabilní verze 11.3. Existují i starší verze 11.2 a 11.1, které ale nebudou předmětem zkoumání.
\section{Podporované architektury}
\label{chapter:solaris:support}
Operační systém Solaris v~současné době podporuje následující dvě HW architektury počítačových systémů:
\begin{itemize}
 \item x86,
 \item SPARC.
\end{itemize}
Jelikož architektura SPARC není běžně dostupná, bude pro účely této diplomové práce použita architektura x86, přesněji
její 64 bitová verze.
\subsection{Architektura x86}
\label{chapter:solaris:support:x86}
První počítačovou architekturou podporovanou operačním systémem Solaris je x86. Tato architektura je v~dnešní době velmi rozšířená
především v~oblasti osobních počítačů a je podporována nejznámějšími OS jako Windows, Linux a Mac. Solaris tuto architekturu
podporuje jak v~32 bitové verzi \textit{x86-32} tak i~v~64 bitové verzi \textit{x86-64}.
\subsection{Architektura SPARC}
\label{chapter:solaris:support:sparc}
Scalable Processor Architecture neboli SPARC je z~pohledu operačního systému Solaris domovská architektura. Architektura SPARC
byla stejně jako Solaris původně navržena společností Sun Microsystems a nyní ji spravuje společnost Oracle. Tato architektura
je tedy od začátku své existence úzce spojena s~operačním systémem Solaris, který se snaží využívat všechny její výhody.
Uplatnění této architektury je především v~komerčním sektoru, který klade vysoké nároky na přizpůsobivost a možnosti škálování
potřebného řešení.
\section{Služby}
\label{chapter:solaris:services}
Hlavní předností operačního systému je kvantita a kvalita jeho služeb. Tyto služby umožňují nasazení tohoto OS i ve scénářích,
kdy by ostatní OS selhaly nebo by nemohly být vůbec použity.  
\subsection{Service Management Facility}
\label{chapter:solaris:smf}
Service Management Facility neboli SMF je systém, který v~operačním systému Solaris spravuje systémové služby. Nahrazuje tím
tradiční způsob spravování služeb pomocí tzv. \textit{init} skriptů, který byl běžný v~ostatních unixových operačních systémech
a dokonce i v~dřívějších verzích OS Solaris. Hlavním rozdílem oproti staršímu způsobu je možnost u~služby definovat závislosti
na~ostatních službách. Na~rozdíl od sériového spouštění \textit{init} skriptů z~adresáře je díky tomuto zlepšení možné při startu
systému paralelně spouštět více nezávislých služeb najednou, a tím urychlit start systému \cite{cvut:biadu:sysstart}. Pro účel
startu jsou v~systému definovány speciální služby tzv. \textit{milestone}. Tyto služby mají definovaný
pouze seznam závislostí, který určuje jaké služby se mají spustit. Při startu se určí, do kterého \textit{milestone} má systém
nastartovat a~tím je přesně určeno, které služby se mají spustit.
\subsection{Souborový systém ZettaByte}
\label{chapter:solaris:zfs}
Pro ukládání na disk používá Solaris souborový systém ZettaByte neboli ZFS. Je to pokročilý systém, který byl vyvinut společností
Sun Microsystems a integrován do operačního systému Solaris. ZFS dokáže spravovat velké množství dat díky své 128-bitové 
architektuře \cite{cvut:thesis:mythesis}. Mezi hlavní funkce ZFS patří ověřování integrity dat, vlastní softwarový RAID nebo
šifrování dat. Díky principu \textit{copy on write} dokáže udržet data neustále konzistentní, což některé souborové systémy
nedokážou nebo tento problém řeší složitě. Architektura tohoto souborového systému umožňuje klonování jednotlivých svazků nebo
rychlou a elegantní tvorbu obrazů disku tzv. \textit{snapshot}, které z~počátku zabírají minimální místo na disku. Datové bloky
jsou totiž zduplikovány až v~okamžiku, kdy se zdrojový blok nebo jeho klon změní. Tento způsob uchovávání dat společně s~možností
\textit{deduplikace} stejných datových bloků značně snižuje nároky na diskové místo.

Principů a funkcí ZFS hojně využívají další služby operačního systému Solaris. Příkladem může být uvedena virtualizační technika
Solaris Zones, která je hlavním tématem této diplomové práce.
\subsection{Virtualizace}
\label{chapter:solaris:virtualization}
Dle specifikace \cite{oracle:solaris:virtualization} nabízí operační systém Solaris ve verzi 11.3 následující techniky virtualizace:
\begin{itemize}
 \item virtualizace na úrovni OS,
 \item virtuální počítače,
 \item hardware partitions.
\end{itemize}
Tyto modely se liší zejména ve způsobu izolace virtualizovaných prostředí a~ve~flexibilitě přidělování prostředků těmto
prostředím. Čím více model izoluje prostředí od sebe, tím nabízí menší flexibilitu v~přidělování prostředků.
\subsubsection{Solaris Zones}
\label{chapter:solaris:virtualization:szones}
Jedním z~modelů virtualizace nabízené operačním systémem Solaris je \textit{virtualizace na úrovni OS}. Tento model umožňuje
vytvořit jedno nebo více izolovaných prostředí (zón) pro běh programů v~rámci jedné instance OS. Takto vytvořená prostředí
jsou izolována na úrovni procesů, souborového systému a síťových rozhraní. Každá zóna má vlastní lokální pohled na systémové
prostředky, které mohou být dále virtualizované operačním systémem. Virtualizace na úrovni operačního systému poskytuje vysoký
výkon a flexibilitu, protože nezanechává tak velkou stopu na disku, paměti nebo CPU na rozdíl od ostatních dvou modelů virtualizace. 

Operační systém Solaris poskytuje tento model virtualizace skrze službu Solaris Zones, která je přímo integrována to jádra OS.
\subsubsection{Virtuální počítače}
\label{chapter:solaris:virtualization:vm}
Model \textit{virtuálních počítačů} popsaný v~kapitole \ref{chapter:virtualization:clasification:system_vm:virtual_computer} 
umožňuje souběžný běh více instancí operačního systému na jednom fyzickém stroji. Každý virtuální počítač má svojí instanci
operačního systému, který nemusí být stejný ve všech virtuálních strojích. Tento typ virtualizace je umožněn díky virtualizačnímu
monitoru, který vytváří pro operační systémy iluzi, která izoluje jednotlivé virtuální počítače. Virtuální počítače poskytují na rozdíl od
virtualizace na úrovni OS menší flexibilitu rozdělování prostředků, ale naopak poskytuje větší úroveň izolace.

Tento typ virtualizace je v~OS Solaris 11.3 podporován produkty Oracle VM Server for x86, Oracle VM Server for SPARC a Oracle
VM VirtualBox \cite{oracle:virtualization:technologies}. Každá z~těchto implementací se zaměřuje na jinou architekturu nebo
používá jiný typ virtualizačního monitoru.
\subsubsection{Hardware partitions}
\label{chapter:solaris:virtualization:hw_partition}
Posledním modelem, který je nepřímo podporovaný operačním systémem Solaris, jsou tzv. \textit{hardware partitions}. Je to technika,
která fyzicky odděluje běh OS na oddělených částech fyzických prostředků. Tohoto způsobu virtualizace je dosaženo
bez pomocí virtualizačního monitoru, a proto tato technika poskytuje reálný výkon systému. \textit{Hardware partitions} je technika
poskytující běžícím operačním systémům největší izolaci, ale není tak flexibilní v~konfiguraci prostředků jako výše
zmíněné modely.

Tento model virtualizace není z~logických důvodů poskytován operačním systémem Solaris, jelikož se jedná o~virtualizaci na HW úrovni.
Pro~účely nasazení tohoto OS s~touto virtualizační technikou používá Oracle speciální servery SPARC M-Series 
\cite{oracle:virtualization:technologies}.