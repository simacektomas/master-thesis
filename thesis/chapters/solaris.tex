% Brief description of Solaris operating system
Solaris nebo dříve SunOS je operační systém původně vytvořený firmou Sun Microsystems, ale v současné době vyvíjený a podporovaný firmou Oracle. Je to komplexní unixový operační systém, který v sobě integruje pokročilé technologie
pro virtualizaci, moderní souborový systém ZFS, vlastní systém pro instalaci a správu SW a v neposlední řadě také podporu cloudu. Díky integraci těchto technologií poskytuje Solaris stabilní a rychlé prostředí pro různé scénáře nasazení
aplikací a navíc tato integrace vytváří pohodlné rozhraní pro správu tohoto OS. 

\section{Verze Solarisu}

V době psaní této diplomové práce je nejaktuálnější stabilní verze operačního systému Solaris verze s označením 11.3, avšak ke dni 30. ledna 2018 byla do světa vypuštěna beta verze 11.4 \cite{solaris:beta_release}. Pro účely popisu operačního
systému Solaris a jeho služeb, zejména služby Solaris Zones, bude použita stabilní verze 11.3. Existují i starší verze 11.2 a 11.1, které ale nebudou předmětem zkoumání.

\section{Podporované architektury}

Operační systém Solaris v současné době podporuje následující dvě HW architektury počítačových systémů.

\begin{itemize}
 \item x86
 \item SPARC
\end{itemize}

Jelikož pořízení architektury SPARC by bylo složitě, bude pro účely této diplomové práce použita architektura x86 přesněji její 64 bitová verze.

\subsection{x86}

První počítačovou architekturou podporovanou operačním systémem Solaris je x86. Tato architektura je v dnešní době velmi rozšířená především v oblasti osobních počítačů a je podporována nejznámějšími OS jako Windows, Linux a Mac.
Solaris tuto architekturu podporuje jak v 32 bitové verzi x86-32 tak i v 64 bitové verzi x86-64.

\subsection{SPARC}

Scalable Processor Architecture neboli SPARC je z pohledu Solarisu domovská architektura. Architektura SPARC byla stejně jako Soalris původně navržena společností Sun Microsystems a nyní ji spravuje společnost Oracle. Tato architektura
je tedy od začátku své existence úzce spojena s operačním systémem Solaris, který se snaží využívat všechny její výhody. Uplatnění této architektury je především v komerčním sektoru, který klade vysoké nároky na přizpůsobivost a škálovatelnost řešení.

\section{Služby}

Hlavní předností operačního systému je kvalita ale i kvantita jeho služeb. Tyto služby umožňují nasazení toho OS i ve scénářích, kdy by ostatní OS selhaly nebo by nemohly být vůbec použity.  

\subsection{Service Management Facility}

Service Management Facility neboli SMF je systém, který v operačním systému Solaris spravuje systémové služby. Nahrazuje tím tradiční způsob spravování služeb pomocí tzv. \textit{init} skriptů, který byl běžný v ostatních unixových operačních systémech
a dokonce i v dřívějších verzích OS Solaris. Hlavním rozdílem oproti staršímu způsobu je možnost u služby definovat závislosti na ostatních službách. Na rozdíl od sériového spouštění \textit{init} skriptů z adresáře je díky tomuto zlepšení možné při startu systému
paralelně spouštět více nezávislých služeb najednou, a tím urychlit start systému \cite{cvut:biadu:sysstart}. Pro účel startu jsou v systému definovány speciální služby tzv. \textit{milestone}, které ve skutečnosti nic nedělají. Mají definovaný pouze seznam závislostí,
který určuje jaké služby se mají spustit. Při startu se určí do kterého \textit{milestone} má systém nastartovat a tím je přesně určeno, které služby se mají spustit.

\subsection{Souborový systém ZettaByte}

Pro ukládání na disk používá Solaris souborový systém ZettaByte neboli ZFS. Je to pokročilý systém, který byl vyvinut společností Sun Microsystems a integrován do operačního systému Solaris. ZFS dokáže spravovat velké množství dat díky své 128-bitové 
architektuře \cite{cvut:thesis:mythesis}. Mezi hlavní funkce ZFS patří ověřování integrity dat, vlastní softwarový RAID nebo šifrování dat. Díky principu \textit{copy on write} dokáže udržet data neustále konzistentní, což některé souborové systémy nedokážou nebo tento problém řeší složitě.
Architektura tohoto souborového systému umožňuje klonování jednotlivých svazků nebo rychlou a elegantní tvorbu obrazů disku tzv. \textit{snapshot}, které z počátku nezabírají téměř žádné místo na disku. Datové bloky jsou totiž zduplikovány až v okamžiku,
kdy se zdrojový blok nebo jeho klon změní. Tento způsob uchovávání dat společně s možností \textit{deduplikace} stejných datových bloků značně snižuje nároky na diskové místo.

Principů a funkcí ZFS hojně využívají další služby operačního systému Solaris. Jako příklad můžeme uvést virtualizační techniku Solaris Zones, která je hlavním tématem této diplomové práce.

\subsection{Virtualizace}

Dle specifikace \cite{oracle:solaris:virtualization} nabízí operační systém Solaris ve verzi 11.3 následující techniky virtualizace.

\begin{itemize}
 \item Virtualizace na úrovni OS
 \item Virtuální počítače
 \item Hardware partitions
\end{itemize}

Tyto modely se liší zejména ve způsobu izolace virtualizovaných prostředí a ve flexibilitě přidělování prostředků
těmto prostředím. Čím více model izoluje prostředí od sebe, tím nabízí menší flexibilitu v přidělování prostředků.

\subsubsection{Solaris Zones}

Jedním z modelů virtualizace nabízené operačním systémem Solaris je \textit{virtualizace na úrovni OS}. Tento model umožňuje vytvořit jedno nebo více izolovaných prostředí (zón) pro běh programů v rámci jedné instance OS. Takto vytvořené prostředí jsou
izolovány na úrovni procesů, souborového systému a síťových rozhraní. Každá zóna má vlastní lokální pohled na systémové prostředky, které mohou být dále virtualizované operačním systémem. Virtualizace na úrovni operačního systému poskytuje vysoký výkon a flexibilitu,
protože nezanechává tak velkou stopu na diku, paměti nebo CPU na rozdíl od zbylých dvou modelů. 

Operační systém Solaris poskytuje tento model virtualizace skrz službu Solaris Zones, která je přímo integrována to jádra OS.

\subsubsection{Virtuální počítače}

Model \textit{virtuálních počítačů} popsaný v kapitole \ref{section:vm} umožňuje souběžný běh více instancí operačního systému na jednom fyzickém stroji. Konkrétně každý virtuální počítač má svojí instanci operačního systému, který nemusí být stejný ve všech virtuálních strojích. Tento typ virtualizace je
umožněn díky virtualizačnímu monitoru, který vytváří pro operační systémy iluzi, že jsou na fyzickém stroji samy. Virtuální počítače poskytují na rozdíl od virtualizace na úrovni OS menší flexibilitu rozdělování prostředků, ale naopak poskytuje větší úroveň izolace.

Tento typ virtualizace je v OS Solaris 11.3 podporován produkty Oracle VM Server for x86, Oracle VM Server for SPARC a Oracle VM VirtualBox. Každá z těchto implementací se zaměřuje na jinou architekturu nebo používá jiný typ virtualizačního monitoru.

\subsubsection{Hardware partitions}

Posledním modelem, který je ne přímo podporovaný operačním systémem Solaris, jsou tzv. \textit{hardware partitions}. Je to technika, která fyzicky odděluje běh OS na oddělených částech fyzických prostředků. Tohoto způsobu virtualizace je dosaženo
bez pomocí hypervisoru, a tudíž tato technika poskytuje reálný výkon systému. Hardware partitions je technika poskytující běžícím operačním systémům největší izolaci, ale na druhou stranu není tak flexibilní v konfiguraci prostředků jako výše zmíněné modely.

Tato technika není z logických důvodů poskytována OS Solaris, jelikož se jedná o virtualizaci na HW úrovni. Pro účely nasazení tohoto OS s touto virtualizační technikou používá Oracle speciální servery  SPARC M-Series.