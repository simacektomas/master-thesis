% Brief description of Solaris operating system
Solaris nebo dříve SunOS je operační systém původně vytvořený firmou Sun Microsystems, ale v současné době vyvíjený a podporovaný firmou Oracle. Je to komplexní unixový operační systém, který v sobě integruje pokročilé technologie
pro virtualizaci, moderní souborový systém ZFS, vlastní systém pro instalaci a správu SW a v neposlední řadě také podporu cloudu. Díky integraci těchto technologií poskytuje Solaris stabilní a rychlé prostředí pro různé scénáře nasazení
aplikací a navíc tato integrace vytváří pohodlné rozhraní pro správu tohoto OS. 

\section{Verze Solarisu}

V době psaní této diplomové práce je nejaktuálnější stabilní verze operačního systému Solaris verze s označením 11.3, avšak ke dni 30. ledna 2018 byla do světa vypuštěna beta verze 11.4 \cite{solaris:beta_release}. Pro účely popisu operačního
systému Solaris a jeho služeb, zejména služby Solaris Zones, bude použita stabilní verze 11.3. Existují i starší verze 11.2 a 11.1, které ale nebudou předmětem zkoumání.

\section{Podporované architektury}

Operační systém Solaris v současné době podporuje následující dvě HW architektury počítačových systémů.

\begin{itemize}
 \item x86
 \item SPARC
\end{itemize}

Jelikož pořízení architektury SPARC by bylo složitě, bude pro účely této diplomové práce použita architektura x86 přesněji její 64 bitová verze.

\subsection{x86}

První počítačovou architekturou podporovanou operačním systémem Solaris je x86. Tato architektura je v dnešní době velmi rozšířená především v oblasti osobních počítačů a je podporována nejznámějšími OS jako Windows, Linux a Mac.
Solaris tuto architekturu podporuje jak v 32 bitové verzi x86-32 tak i v 64 bitové verzi x86-64.

\subsection{SPARC}

Scalable Processor Architecture neboli SPARC je z pohledu Solarisu domovská architektura. Architektura SPARC byla stejně jako Soalris původně navržena společností Sun Microsystems a nyní ji spravuje společnost Oracle. Tato architektura
je tedy od začátku své existence úzce spojena s operačním systémem Solaris, který se snaží využívat všechny její výhody. Uplatnění této architektury je především v komerčním sektoru, který klade vysoké nároky na přizpůsobivost a škálovatelnost řešení.

\section{Služby}

\subsection{Souborový systém ZettaByte}

\subsection{Virtualizace}