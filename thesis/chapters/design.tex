\label{chapter:design}
Následující kapitola popisuje návrh aplikace pro podporu automatické správy virtualizačního kontejneru Solaris Zones na
platformě Solaris. Zaměřuje se především na popis funkcionality a požadavků, které aplikace musí splňovat. V závěrečné části
kapitoly jsou rozebrána bezpečnost a požadavky na uživatele, který aplikaci bude moci využívat.
\section{Požadavky na aplikaci}
\label{chapter:design:demands}
Hlavním cílem této práce je vytvořit aplikaci, která bude administrátorovi operačního systému Solaris ulehčovat správu
většího množství neglobálních zón. Na základě účelu aplikace je nutné vytvořit požadavky, které bude muset výsledná implementace
aplikace splňovat. Pokud vytvořená aplikace splní stanovené požadavky, bude moci být cíl práce označen za splněný.

Jak bylo zmíněno v kapitole \label{chapter:zones}, virtualizační technika Solaris Zones je exkluzivním produktem pro operační
systém Solaris. Tomuto faktu musí být přizpůsoben výběr technologií, které budou použity při implementaci výsledné aplikace.
Operační systém Solaris není standardní platformou, i když je v dnešní době podporován na platformě \textit{x86}. Hlavní důraz
musí být kladen na kompatibilitu programovacího jazyka a jeho knihoven s operačním systémem Solaris. Z výše zmíněných důvodů
je možné vyvodit první požadavek na administrační nástroj, který je podpora na operačním systému \textbf{Solaris}.

Účelem nástroje má být podpora automatické správy neglobálních zón. Pod pojmem správa je myšlena podpora základních administračních
postupů a technik, které jsou z velké části popsány v kapitole \label{chapter:zones}. Mezi tyto postupy patří především 
vytváření neglobálních zón, ale také podpora jejich správy, zálohování nebo migrace. Automatickou správou se myslí hlavně
automatizace procesů vytváření zóny, zálohy nebo migrace, které se skládají z několika kroků. Aplikace by tedy měla administrátorovi
poskytovat funkce, které umožní provedení výše zmíněných procesů pomocí jednoho příkazu. Požadavky na aplikaci vyplývající
z účelu nástroje je možné pojmenovat jako \textbf{podpora správy} Solaris Zones a \textbf{automatizace procesů} administrace.

Virtualizační technika Solaris Zones poskytuje administrátorovi skrze příkazy \verb|zonecfg(1)| a \verb|zoneadm(1)| způsob,
jak spravovat lokální neglobální zóny. Zcela zde ale chybí podpora pro správu zón na vzdálených serverech. V dnešních infrastrukturách
počítačových systémů využívající virtualizaci serverů se nachází mnoho serverů, které poskytují své výpočetní prostředky
virtuálním strojů. Z tohoto pohledu je tedy žádoucí, aby implementovaná aplikace umožňovala správu neglobálních zón, které
se nacházejí na \textbf{vzdálených} serverech.

Následující požadavek se vztahuje k požadavku na automatizaci administračních procesů. Jelikož definice zóny se skládá z její
konfigurace, softwarového vybavení a systémového nastavení, aplikace by měla umožňovat specifikaci této definice nějakým 
jednotným způsobem. Jinými slovy aplikace musí poskytovat administrátorovi systém pro vytváření definic zón, které bude možné
používat pro jejich vytváření. Tento požadavek lze pojmenovat jako podpora vytváření \textbf{šablon}.

Uživatel musí mít možnost jak ovládat nástroj pro podporu automatické správy neglobálních zón. To znamená, že aplikace bude
poskytovat uživateli své funkce pomocí \textbf{uživatelského rozhraní}. Toto uživatelské rozhraní musí být přehledné a poskytovat
uživateli všechny informace potřebné pro využívání jeho funkcí. Pomocí toho rozhraní bude uživatel zadávat příkazy, které
aplikace bude vykonávat. Rozhraní by mělo nabízet izolovaný pohled pro každého uživatele, který bude aplikaci využívat.

Posledním požadavkem, který musí aplikace splňovat, je \textbf{bezpečnost}. Na bezpečnost používání aplikace se musí dbát především
proto, že nesprávným a neopatrným používáním virtualizační techniky Solaris Zones může dojít k nestabilitě celého systému.
K takovým případům dochází především ve chvílích, kdy neglobální zóny vyčerpají všechny fyzické prostředky systému a tím
znemožní správný běh globální zóny.

Kompletní požadavky na aplikaci pro podporu automatické správy Solaris Zones můžeme shrnout do následujících bodů.
\begin{itemize}
 \item \textbf{Operační systém Solaris}
 \item \textbf{Lokální a vzdálená správa}
 \item \textbf{Automatizace administračních procesů}
 \item \textbf{Šablony}
 \item \textbf{Uživatelské rozhraní}
 \item \textbf{Bezpečnost}
\end{itemize}
Splněním těchto požadavků by mělo vést k značnému zjednodušení správy virtualizačního kontejneru Solaris Zones. Dále by výsledná
aplikace měla zajistit přehled o neglobálních zónách, které se nacházejí na vzdálených serverech a také umožňovat jejich správu. 
\section{Automatizace}
\label{chapter:design:automation}
\section{Šablony}
\label{chapter:design:templates}
\section{Vzdálený správa}
\label{chapter:design:remote}
\section{Architektura aplikace}
\label{chapter:design:architecture}
\section{Uživatelské rozhraní}
\label{chapter:design:ui}
\section{Bezpečnost}
\label{chapter:design:security}
